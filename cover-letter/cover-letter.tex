\documentclass[11pt]{letter}

%\usepackage{pxfonts}
%\usepackage{color}
%\hyphenation{ma-nu-script}
\usepackage[colorlinks=true,urlcolor=blue, hyperindex,breaklinks=true] {hyperref}

\usepackage{graphicx}
\usepackage[a4paper,left=2cm,right=2cm,bottom=2cm]{geometry}

\usepackage{fancyhdr}
%\usepackage{marvosym}

% nice font.
\renewcommand*\rmdefault{ppl}

\usepackage[table,xcdraw,dvipsnames]{xcolor}
\newcommand{\nick}[1]{\textcolor{red}{[#1]}}
\newcommand{\ddd}[1]{\textcolor{blue}{[#1]}}

\pagestyle{empty}

\begin{document}


\hfill\begin{minipage}{6cm}
	\raggedright
	\vspace{-2.5cm}
	{\footnotesize\itshape Nikolaos Koukoulekidis}\\
	{\footnotesize\itshape Department of Physics}\\
	{\footnotesize\itshape Imperial College London}\\
	{\footnotesize\itshape London SW7 2AZ, UK}\\
	{\footnotesize\itshape %\MVAt~ 
		\verb|nk2314@imperial.ac.uk|}\\
	~\\
		\footnotesize\today
\end{minipage}

\vspace{-3.5cm}
\includegraphics[scale=.1]{icllogo.png}
\vspace{3.5cm}

\vspace{-1.5cm}
Dear Editors of PRX Quantum,

\nick{this is supposed to be a ``100-word concise and compelling argument''. Currently: 203 words}

\vspace{.3cm}
	
Please find enclosed a copy of our manuscript \emph{``Constraints on magic state protocols from the statistical mechanics of Wigner negativity''}, which we would like to submit for review in PRX Quantum. 
We believe our work fits naturally into the scope of the journal. 

Our work establishes a connection between two fields within quantum information: quantum thermodynamics and quasi-probability representations of quantum physics.
This allows for important advances in the subfield of magic theory, an increasingly important area of quantum computation.

Primarily, we develop new techniques that provide analytical bounds on general magic state distillation processes. We show that our bounds are tighter than previously defined bounds, and importantly can be expressed in terms of the physical symmetries of the circuit in question.
We therefore provide novel and powerful constraints on the design of magic protocols.

The implications of our analysis are also important foundationally.
We develop a novel theory of statistical mechanics on quasi-distributions and we further construct entropic functions which attain negative values, a property which acts as a fundamental witness of non-classicality.

For the above reasons, we believe that our manuscript is of relevance to a wide audience of quantum physicists, and so is suitable for publication in Phys. Rev. X.

\vspace{1cm}
\hspace{8cm}
\begin{minipage}{9cm}
\flushleft
Sincerely Yours,\\

Nikolaos Koukoulekidis, David Jennings.
\end{minipage}
\newpage


\textbf{Popular Summary}

\nick{this is supposed to be 250 words long. Currently: 243 words}

Quantum computers promise astronomical speed-ups for important computational tasks.
A practically necessary requirement to achieve such speed-ups for useful computational tasks is establishing a framework that ensures fault-tolerance, which is the ability to correct computational errors in the circuit successfully and reliably.
It is well-known that restricting to this framework, it is impossible to achieve quantum speed-ups, therefore we need to initiate the circuit with some low-noise input state that contains enough resources which can be used by the circuit to perform the desired computational task.

Creating this low-noise, highly resourceful state, called a ``magic state'', is a costly process, requiring in principle a quantum process that manipulates many highly noisy copies of the magic state in order to distill one with very low noise.
Our work addresses the question regarding the highest possible rate of low-noise magic states one could extract, given some copies of highly noisy magic states.
We provide fundamental bounds on this rate, based on the thermodynamic properties of the quantum process, which constraint the efficiency of any general quantum processes attempting to distill magic.
As our bounds depend on the thermodynamic properties of the process, they notably provide a way of assessing how physical properties affect the distillation process.

On our way to addressing this question, we develop a novel physical framework of thermodynamics, where \nick{microstates of a system can be quasi-probabilistic} and where entropies can attain negative values, which are fully meaningful as indicators of quantum computational speed-ups.




\end{document}
