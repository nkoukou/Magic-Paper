\documentclass[11pt]{letter}

\usepackage[colorlinks=true,urlcolor=blue, hyperindex,breaklinks=true] {hyperref}

\usepackage{graphicx}
\usepackage[a4paper,left=2cm,right=2cm,bottom=2cm]{geometry}
\usepackage{fancyhdr}
\usepackage[table,xcdraw,dvipsnames]{xcolor}

\usepackage{amsmath,amsfonts, amssymb, amsthm}
\usepackage{bm, bbm, physics, mathtools}

\renewcommand*\rmdefault{ppl}

\newcommand{\nick}[1]{\textcolor{red}{[#1]}}
\newcommand{\ddd}[1]{\textcolor{blue}{[#1]}}

\pagestyle{empty}
\renewcommand{\cal}[1]{\mathcal{#1}}

\newcommand{\reals}{\mathbb{R}}
\newcommand{\id}{\mathbbm{1}}
\newcommand{\idc}{1_{\rm{C}}}
\newcommand{\supf}{\mathfrak{c}}
\renewcommand{\tr}{{\rm{tr}}}
\renewcommand{\det}{{\rm{det}}}
\newcommand{\floor}[1]{\left\lfloor #1 \right\rfloor}
\newcommand{\ent}[2]{S\left( #1 \middle\vert\middle\vert #2 \right)}
\newcommand{\ents}{{\ent{\frac{m}{n}}{p}}}

\def\dummy{\ell}
\def\NN{n}
\def\mmf{i}
\def\nnf{n}
\def\mlt{m}
\def\ii{i}
\def\jj{j}
\def\kk{k}
\def\II{I}
\def\nn{n}
\def\tt{n'}
\def\mm{a}
\def\wp{u}
\def\wn{v}

\newcommand{\too}[1]{^{\otimes #1}}
\newcommand{\noisys}{\rho_{\rm{S}}}
\newcommand{\noisysn}{\rho_{\rm{S}}(\epsilon)^{\otimes \nn}}
\newcommand{\noisysN}{\rho_{\rm{S}}(\epsilon)^{\otimes \NN}}

\newcommand{\spanv}[1]{
    {{\rm{span}}\left\{#1\right\}}
}
\newcommand{\conv}[1]{
    {{\rm{conv}}#1}
}
\newcommand{\orb}[1]{
    {{\rm{orb}}(#1)}
}
\newcommand{\sn}[1]{
    {{\rm{sn}}\left(#1\right)}
}
\newcommand{\mana}[1]{
    {{\rm{mana}}\left(#1\right)}
}
\newcommand{\lc}[2]{
	{{\rm{L}}_{#1|#2}}
}

\newcommand{\bmx}{\bm{x}}
\newcommand{\bmy}{\bm{y}}
\newcommand{\bmz}{\bm{z}}
\newcommand{\bmu}{\bm{u}}
\newcommand{\bmw}{\bm{w}}
\newcommand{\bmo}{\bm{0}}
\newcommand{\bmd}{\bm{d}}
\newcommand{\bma}{\bm{a}}
\newcommand{\bmxi}{\bm{\xi}}
\newcommand{\bmg}{\bm{g}}

\newcommand{\zd}[1][]{
    \ifthenelse{\isempty{#1}}{
    {\mathbb{Z}_d} }{
    {\mathbb{Z}_{#1}}}
}
\newcommand{\hd}[1][]{
    \ifthenelse{\isempty{#1}}{
    {\cal{H}_d} }{
    {\cal{H}_{#1}}}
}
\newcommand{\pd}[1][]{
    \ifthenelse{\isempty{#1}}{
    {\cal{P}_d} }{
    {\cal{P}_{#1}}}
}
\newcommand{\cd}[1][]{
    \ifthenelse{\isempty{#1}}{
    {\cal{C}_d} }{
    {\cal{C}_{#1}}}
}
\newcommand{\spd}[1][]{
    \ifthenelse{\isempty{#1}}{
    {{\rm{Sp}}(2, \zd)} }{
    {{\rm{Sp}}(2, \zd[#1])}}
}
\newcommand{\gp}[1][]{
    \ifthenelse{\isempty{#1}}{
    {\rm{GP}_d} }{
    {\rm{GP}_{#1}}}
}
\newcommand{\stoch}[1][]{
    \ifthenelse{\isempty{#1}}{
    {{\rm{S}}_d(\bmd)} }{
    {{\rm{S}}_d(#1)}}
}
\newcommand{\stochw}[1][]{
    \ifthenelse{\isempty{#1}}{
    {{\rm{S}}_{d^2}(\W{\sigma})} }{
    {{\rm{S}}_{d^2}(#1)}}
}
\makeatletter
\def\W{\@ifnextchar[{\@with}{\@without}}
\def\@with[#1]#2{ 
    {{\rm{W}}_{#2}\left(#1\right)} }
\def\@without#1{ 
    {{\rm{W}}_{#1}} }
\makeatother

\newcommand{\T}{\cal{T}}
\newcommand{\Z}{\cal{Z}}

\newcommand{\C}{\cal{C}}
\newcommand{\E}{\cal{E}}
\newcommand{\J}{\cal{J}}
\newcommand{\R}{\cal{R}}
\newcommand{\D}{\cal{D}}
\newcommand{\F}{\cal{F}}
\renewcommand{\O}{\cal{O}}
\newcommand{\M}{\cal{M}}

\newcommand{\Fmax}{\F_{\rm{max}}}
\newcommand{\Omax}{\O_{\rm{}max}}
\newcommand{\Rmax}{\R_{\rm{}max}}
\newcommand{\Pis}{\Pi_{\rm{s}}}
\newcommand{\Pio}{\Pi_{\rm{o}}}

\newcommand{\cptp}{{\rm{CPTP}}}
\newcommand{\cpos}{{\rm{CP}}}
\newcommand{\so}{{\rm{SO}}}
\newcommand{\stab}{{\rm{STAB}}}
\newcommand{\spo}{{\rm{SPO}}}
\newcommand{\cspo}{{\rm{CSPO}}}
\newcommand{\rcu}{{\rm{RCU}}}
\newcommand{\tho}{{\rm{TO}}}
\newcommand{\cpwp}{{\rm{CPWPO}}}
\newcommand{\ru}{{\rm{RU}}}


\begin{document}

Dear Prof. Sven Rogge,

Thank you for passing on the Reviewers' recommendations.
In this letter, we address the revisions suggested by both Reviewers, by responding to questions raised and describing the resultant changes which can be found highlighted in the current version of the manuscript.\\

\textbf{\large{Response to Reviewer 1}}\\

\textit{In this work, the authors have derived new constraints on the rate of magic state distillation routines for qudits of odd prime dimension, making use of the discrete Wigner function representation. One of the main reasons qudit magic state distillation is of theoretical interest is that it appears possible to develop a powerful resource theory for fault-tolerant quantum computation in odd dimensions, largely due to the discrete Wigner function representation, which only exists for odd-dimensional qudits.}

\textit{Their essential technique is to apply the theory of majorization, which they review in the paper, to the discrete Wigner function, which is a quasi-probability distribution. Essentially one quasi-distribution p can be transformed to another quasi-probability distribution q by a stochastic transformation if and only if p majorizes q (this can be taken as the definition of majorization, but there are other equivalent definitions). One can show that p majorizes q if and only if the Lorenz curve of p is greater than or equal to that of q everywhere. These types of results were previously known for probability distributions, but the authors show that these arguments can also be applied to quasi-probability distributions (assuming the reference distribution is strictly positive).}

\textit{They apply their results to qutrit magic state distillation, mainly for distillation of the qutrit strange state, which is the simplest and most important magic state to consider. The bounds they derive are clearly much stronger than previously known bounds on magic state distillation in the literature, based on mana and thauma. In fact, the monotonicity of mana as a resource follows as a special case of their majorization arguments.}

\textit{They also discuss the case of finite temperature, which shows that their method can be generalized to a variety of situations.}

\textit{They also show that one can define Renyi entropies for quasi-probability distributions, and derive an elegant interpretation of the mana as the residue of a pole quantifying the divergence of the Renyi entropy as it approaches the Shannon entropy.}

\textit{Overall, the results of the paper appear to me to be novel, topical, and mathematically elegant. I also found the paper to be well-written. I therefore strongly recommend it for publication.}

We thank the Reviewer for the positive comments.

\textit{However, I have one suggestion that could be included in a minor revision:}

\textit{The main result of the authors seems to rest on Theorem 1, which in turn relies on the unnumbered result on page 5, after equation (15), which states roughly that “relative majorization holds between pairs of probability distributions holds if the Lorenz curve of one is greater than or equal to the Lorenz curve of the other, for all x.” The authors refer to [103] for a proof of this fact. However, I think it would substantially enhance accessibility and readability if the authors could give a proof of this result within the present paper, perhaps in an appendix.}

We are grateful for suggestion on making our results more accessible.
To this end, we have added a proof of the result in Appendix~B1, relying on the assumption that the result is clear for simple majorization (i.e. when $\r = \r'= (1/N,1/N,\dots,1/N)$) as shown in~[81]. 
In general, we consider statements about simple majorization (presented in Marshall \& Olkin [81]) as well-established in the quantum information community and do not prove them in our paper.
As a side remark, the original proof of this result in the context of relative majorization can be found in [111], not [103].

\textit{A side remark: As I understand, Lorenz curves were initially defined in economics to describe the income distribution of a population. It seems possible for some individuals to lose money during the year, so the distribution of incomes in a population is in general, more analogous to a quasi-probability distribution rather than a probability distribution. So, perhaps some of the basic results of the paper may also be applicable, in principle, to a wider context in economics.}

\nick{David, thoughts on bounding wealth distillation?}\\

\textbf{\large{Response to Reviewer 2}}\\

\textit{A universal gate set for quantum computing is a set of unitary operators such that it can approximate any unitary to arbitrary accuracy. As quantum circuits will employ error correcting codes, we require universal gate set such that gates can be performed fault tolerantly, i.e. can be implemented on qubits encoded in an error correcting code without spreading errors. Sets of universal gates that can be performed on a quantum error correcting code often fall into a natural separation (based on the choice of code). There are gates that are "naturally" fault tolerant, and gates that are difficult to perform in a fault-tolerant manner. In most popular choices of error correcting code to encode qubits, the former corresponds to unitary operators in the Clifford group, and the latter to unitary operator not generated by the Clifford group. These gates outside the orbit of the Clifford group are a central object of study. The closely associated states, sometimes also referred to as magic
states, are notoriously hard to prepare. It is an important question in quantum computation to understand their structure.}

\textit{Authors have made an important fundamental contribution by illuminating the relationship between these states. They have shown that these states obey a partial order based on majorization. Majorization is a partial order on vectors and is often studied when these vectors are normalized probability distributions. A probability distribution p majorizes a distribution q if we can obtain q from p by a permutation or a convex combination of permutations. Majorization is an important way of characterizing other quantities in quantum information, most famously in entanglement transformation under LOCC. However, it is present in other contexts as well such as quantum thermodynamics.}

\textit{The framework authors use is that of a resource theory. This classifies (quantum) operations into easy (free operations) and difficult (resource operations/states). The induced partial order determines whether we can transform one magic state into another and simultaneously leave a particular state fixed. Each magic state defines a "Lorenz curve". To be precise, each magic state is associated with a quasi-probability distribution called the discrete Wigner function discovered by Wootters and later by Gross. By "balancing" potentially negative values of the Wigner function with the Wigner function of some fixed, positive distribution, we can everywhere deal with positive distributions. It is then possible to sensibly define a partial order with respect to transformations under stochastic maps and majorization as commonly defined.}

\textit{Interestingly, previous metrics, such as mana proposed by Veitch et al., can be obtained from this more fundamental quantity. Authors then proceed to show that this realization can be used to upper bound the rate of magic transformation in the context of strange state, a particular magic state for qutrits. They also study limits of magic transformation by providing each qudit with a Hamiltonian and understanding temperature dependent transformations.}

\textit{This is a strong result. However, before accepting for publication, I would like authors to address some comments.
Questions/comments for the authors:}


\textit{1. In section 3C, an upper bound on the rate of distillation for qutrit magic states is presented. However, we do not see an explicit scheme that can actually achieve this bound. Some comments on whether this bound is achievable would be appreciated.}

\nick{ANSWER}

\textit{In the conclusions, short discussion on lower bounds can be lengthened. Does this require explicit error correcting codes for distillation?}

\nick{David, do you want to take this one? -- your previous notes are accessible on GitHub for copy-pasting}

\nick{I haven't added lines on this in the main text}

\textit{2. It was shown by Bengtsson et al. that qudit magic states for prime-dimensional qudits, where the prime p is 1 mod 3, there exist distinct magic states under distinct orbits of the Clifford group. In other words, there exist magic states such that we cannot convert from one to other using only Clifford operations. Can authors reproduce this result using their theory?}

\nick{I haven't had the time to consider this yet. In principle, we would need to find $R_\alpha = 0$ for some $\alpha$ and reference process, given a magic interconversion between Clifford non-equivalent magic states -- seems tough and numerically doesn't look to be the case for the Strange and Norrell states ($d=3$, does not fall under Bengtsson et al's results)}

\textit{3. Perhaps authors have already mentioned this, but Clifford operations can also be supplemented by Pauli measurements and adaptive gates. In the main text, as well as in conclusions (comments about G-majorization) this is not mentioned. Would this lead to distinct/ refined bounds?}

This is a very interesting question from the lens of implementing our results on actual distillation protocols.
According to Jozsa and Van den Nest's work\footnote{Jozsa R and Van den Nest M (2013) Classical simulation complexity of extended Clifford circuits, arXiv:1305.6190}, an $N$-qudit circuit consisting of intermediate measurements and adaptive Clifford gates is always equivalent to an $(N+k)$-qudit circuit ($k \in \mathbb{N}$) consisting of a Clifford gate.
Therefore, our bounds are directly applicable on the equivalent circuit.

In more general scenarios, suppose we project an $n$-copy state $\rho$ with projector $P$, which has success probabilty $p$. This can be written as
\begin{equation}
	\cal{E}(\rho) = p\ \tr_{i=1,\dots,n-1} [P\rho P] \otimes \ketbra{0} + (1-p)\ \tr[(\id-P)\rho] \sigma \otimes \ketbra{1},
\end{equation}
where the last subsystem is labelled $0$ when the projection is successful and 1 when it is not.
Then, $\E$ admits a stochastic Wigner representation and in principle our results can provide bounds on the distillation ratio $\frac{pm}{n}$.
However, one complication arises due to this operational representation of projective measurements, namely that states are no longer product states and the entropic functions are not guaranteed to factor out nicely (Theorem 11 property 3).

\nick{I haven't added lines on this in the main text}

\textit{4. Is it possible that the bounds in a more refined form will depend on the prime dimension of the qudit?}

Yes, this is already clear in Eq.(52) obtained by restricting to unital channels.
\nick{What about more refined, unrestricted bounds? I am not sure this is a well-defined question, as the bounds implicitly depend on the dimension via the states considered. The question seems motivated from the context of real distillation protocols, where the overhead scaling $\gamma$ tends to depend on the dimension e.g. Krishna and Tillich (2019) Towards Low Overhead Magic State Distillation.}

We further note that the numerically optimal bounds presented in Fig.(1), for which we do not have explicit expressions, depend on the numbers of copies $n,m$ and therefore are not constant for a given ratio $m/n$.

\textit{5. In optics, the Wigner function is not the only quasi-probability distribution. Indeed, there are the P and Q distributions and potentially a continuous spectrum of quasiprobability distributions. In the discrete setting the Wigner function occupies a special role because of Gross' work showing that stabilizer states can be represented positively for prime-power dimensions and other useful properties like Clifford covariance. However is it possible that the other distributions offer extra information not captured by the Wigner distribution? Can we use a discrete analogue of the P and Q distributions?}

We particularly welcome this question as it relates to some of the authors' current research.
Firstly, we stress that in odd prime dimensions Gross' Wigner function is the only non-contextual quasi-probability representation of stabilizer theories (see~[101]), and contextuality has been shown to be equivalent to magic (see~[44]), as understood in the resource-theoretic formalism.
Therefore, using different distributions, would in general represent negatively states and channels that are operationally ``free'', while simultaneously representing positively states and operations that are not operationally free, thus shifting the role of our bounds as fundamental bounds on \emph{magic distillation} protocols.

Having noted the above, we are claiming (in upcoming work) that using different distributions offers useful additional information in the context of classical simulability of general quantum circuits.
Indeed, the P and Q quasi-distributions as well as the entire spectrum of s-ordered quasi-distributions introduced by Cahill and Glauber\footnote{Cahill K E and Glauber R J (1969) Ordered expansions in boson amplitude operators, Phys. Rev. 177 1857} find discrete analogues in Ruzzi \textit{et al.}'s work\footnote{Ruzzi M \textit{et al.} (2005) Extended Cahill-Glauber formalism for finite dimensional spaces: I. Fundamentals, J. Phys. A: Math. Gen. 38 6239}.
In our work we come up with different infinite families of quasi-probability representations, but the core idea is the same: the probability outcome of a general quantum circuit can be estimated via Monte Carlo sampling\footnote{Pashayan H (2015) Estimating Outcome Probabilities of Quantum Circuits Using Quasiprobabilities, Phys. Rev. Lett. 115, 070501}, and using different quasi-distributions for representing different circuit components can reduce the sampling overhead.
More specifically, the number of samples required for a probability estimate, which converges well to the actual circuit outcome, scales as a function of the negativities of the distributions that represent the circuit components. Therefore, different quasi-distribution representations can result in greater reduction of the negativity overhead compared to the Wigner representation.



















\end{document}
