\documentclass[12pt]{letter}


\usepackage[a4paper,left=2.5cm,right=2.5cm,bottom=2cm]{geometry}
\usepackage{graphicx}

\RequirePackage[fixed]{fontawesome5}
\RequirePackage[scaled=.96,osf]{XCharter}
\linespread{1.1}

\usepackage{color}
\definecolor{slateblue}{rgb}{0.17,0.22,0.64}

\usepackage{hyperref}
\hypersetup{
  colorlinks = true, %Colours links instead of ugly boxes
  urlcolor   = slateblue, %Colour for external hyperlinks
  linkcolor  = blue, %Colour of internal links
  citecolor  = blue %Colour of citations
}

\renewcommand*\rmdefault{ppl} % nice font
\pagestyle{empty}

\begin{document}

\hfill\begin{minipage}{6cm}
	\raggedright
	\vspace{-2.5cm}
	{\footnotesize\itshape Nick Koukoulekidis}\\
	{\footnotesize\itshape (corresponding author)}\\
	{\footnotesize\itshape PhD, Quantum Theory Group}\\
	{\footnotesize\itshape Imperial College London}\\
	{\footnotesize\itshape %\MVAt~ 
		\verb|nk2314@imperial.ac.uk|}\\
	%~\\
	%	\footnotesize\today
\end{minipage}

\vspace{-3.6cm}
\hspace{-0.5cm}\includegraphics[scale=.12]{icllogo.png}
\vspace{1.5cm}

A fundamental problem is to understand how quantum physics admits higher degrees of order than possible in classical physics. 
One signature of this non-classical order is the existence of `negative probabilities' in the Wigner representation of quantum systems. 
In this work, we construct a well-defined framework of statistical mechanics for Wigner negativity, and then apply it to quantify the order in magic states.
Specifically, we provide fundamental restrictions on magic state distillation, a leading model of fault-tolerant quantum computing. 
The restrictions are more powerful than previous approaches and they are novel in their ability to include experimental hardware limitations.

Statistical mechanics cannot be directly applied to quantify order on quasi-probability distributions, since the Shannon entropy is not well-defined when Wigner negativity arises.
In order to establish our statistical framework, we utilise recent research on the thermodynamics of small quantum systems and extend it to Wigner negativity and quasi-probability distributions.
Thermodynamic entropic theories are underlied by majorization orderings between probability distributions that represent the thermodynamic states of a system.
In this sense, majorization is a more fundamental notion than entropy and we show that it is possible to define majorization ordering among quasi-distributions.
We further define a sub-set of classical (R\'{e}nyi) entropies which remain meaningful on quasi-distributions and can attain negative values, acting as witnesses of non-classicality, and specifically magic in the context of Wigner negativity.
Finally, we find that the prominent magic measure of mana naturally arises as the divergence of these R\'{e}nyi entropies as one approaches the Shannon entropy for Wigner distributions, allowing for interpreting magic injection in fault-tolerant schemes as a phase transition.
This use of majorization for quasi-distributions could find application in other studies of non-classicality.

Our framework has a profound impact on the quantification of magic in fault-tolerant schemes, which we now review.
Quantum operations that are robust under errors can be implemented experimentally with high precision, but are not sufficient to achieve quantum speed-ups.
Magic states are key resources injected in fault-tolerant schemes to realize universal quantum computation and contain negativities in their Wigner representation.
They can only be implemented probabilistically within a fault-tolerant framework, hence only injection of noisy magic states is possible.
However, one can perform magic state distillation which aims at converting copies of a noisy magic state into fewer copies of a less noisy, higher quality magic state.
Several distillation protocols exist, each effectively providing a low bound on the optimal achievable distillation rate, i.e. the ratio of final to initial magic state copies. 
On the other hand, upper bounds on distillation protocols express restrictions on our ability to distill magic states. 
Some upper bounds do exist, but they are based on abstract information-theoretic measures, which take into account properties of magic states, but are completely agnostic of the protocol properties. 

Our novel statistical framework of Wigner negativity naturally leads to the derivation of an infinite family of entropic upper bounds.
We may encode the physics of a distillation protocol by considering its action on a general reference stabilizer state.
This allows us to impose a majorization ordering between the distribution pairs of the magic and reference stabilizer state, before and after they pass through the protocol, thus obtaining entropic bounds, which encode both the properties of the magic states and of the protocol.
This new family of bounds outperforms existing monotone-based upper bounds, and, to our knowledge, provides the first general framework to incorporate dependence on hardware physics. 
In particular, our bounds attain different optimal values depending on the specific quantum computer architectures. 
The architectural properties that dictate optimality can, for example, be some temperature threshold or noise bias of the device or a fixed-point structure associated with a restriction in the allowed quantum gates. 
We provide explicit demonstrations of how our bounds may depend on the temperature threshold or the Hamiltonian describing the device.
As such, this approach is relevant to the challenge of designing and building quantum computers.














\end{document}
