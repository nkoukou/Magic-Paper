\documentclass[11pt]{letter}

%\usepackage{pxfonts}
%\usepackage{color}
%\hyphenation{ma-nu-script}
\usepackage[colorlinks=true,urlcolor=blue, hyperindex,breaklinks=true] {hyperref}

\usepackage{graphicx}
\usepackage[a4paper,left=2cm,right=2cm,bottom=2cm]{geometry}

\usepackage{fancyhdr}
%\usepackage{marvosym}

% nice font.
\renewcommand*\rmdefault{ppl}

\usepackage[table,xcdraw,dvipsnames]{xcolor}
\newcommand{\nick}[1]{\textcolor{red}{[#1]}}
\newcommand{\ddd}[1]{\textcolor{blue}{[#1]}}

\pagestyle{empty}

\begin{document}


\hfill\begin{minipage}{6cm}
	\raggedright
	\vspace{-2.5cm}
	{\footnotesize\itshape Nikolaos Koukoulekidis}\\
	{\footnotesize\itshape Department of Physics}\\
	{\footnotesize\itshape Imperial College London}\\
	{\footnotesize\itshape London SW7 2AZ, UK}\\
	{\footnotesize\itshape %\MVAt~ 
		\verb|nk2314@imperial.ac.uk|}\\
	~\\
		\footnotesize\today
\end{minipage}

\vspace{-3.5cm}
\includegraphics[scale=.1]{icllogo.png}
\vspace{3.5cm}

\vspace{-1.5cm}
Dear Editors of PRX Quantum,

\vspace{.3cm}
	
Please find enclosed a copy of our manuscript \emph{``Constraints on magic state protocols from the statistical mechanics of Wigner negativity''}, which we would like to submit for review in PRX Quantum. 
We believe our work fits naturally into the scope of the journal. 

A fundamental problem is to understand how quantum physics admits higher degrees of order than possible in classical physics. One signature of this non-classical order is the appearance of `negative probabilities' in the Wigner representation of quantum systems. However, statistical mechanics cannot be directly applied to quantify this order since the Boltzmann entropy is not well-defined when Wigner negativity arises. In this work we do two main things: we construct a well-defined framework of statistical mechanics for Wigner negativity, and then apply it to quantify the order in `magic states', which are the key sources of non-classicality in leading quantum computing models. 

It is a major goal to properly understand the processing and `distillation' of magic states for quantum technologies. Upper bounds on magic state distillation protocols do exist and are based on abstract information-theoretic measures. Our methods outperform existing upper bounds in the literature, and, to our knowledge, provide the first general framework to incorporate dependence on hardware physics, such as system Hamiltonians and temperatures. As such, this approach is relevant to the challenge of designing and building quantum computers.

Our results are also of interest beyond quantum technologies. We show that recent research on the thermodynamics of small quantum systems can be extended to Wigner negativity and quasi-probability distributions. To our knowledge, we provide the first application of majorization (a fundamental concept in classical statistics) on quasi-distributions, and so open up the possibility of new tools to analyse non-classicality more generally. We show that a sub-set of classical (R\'{e}nyi) entropies remain meaningful on quasi-distributions and that the appearance of negative entropies is a witness of non-classicality in quantum systems. Finally, our analysis also raises novel questions relevant to work on statistical mechanics.

For the above reasons, we believe that our manuscript is of relevance to a wide audience of quantum physicists, and so is suitable for publication in PRX Quantum.

\vspace{1cm}
\hspace{8cm}
\begin{minipage}{9cm}
\flushleft
Sincerely Yours,\\

Nikolaos Koukoulekidis, David Jennings.
\end{minipage}




\end{document}
