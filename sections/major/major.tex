\section{Stochastic structure of magic theories}
\label{sec:struc}

\subsection{Fragments}\label{sec:frag}

Magic is commonly quantified via monotones \nick{CITE}. 
A monotone is a projection from the $d$-dimensional set of density states of the theory to the real line. 
It is monotonically decreasing under free operations, 
\begin{equation}\label{eq:monotonicity}
    \cal{M}(\rho_1) \leq \cal{M}(\rho_2)\ \text{if}\ \rho_1 \prec_{\R} \rho_2,
\end{equation}
reflecting the no resource generating property of free operations and thus respecting the pre-order $\prec$ of the theory.
A popular magic monotone is the \emph{mana} of a state \nick{CITE}, defined as
\begin{equation}
    \mana{\rho} \coloneqq \log{\left(\sum\limits_{\bmz \in \pd} \abs{\W[\bmz]{\rho}}\right)}.
\end{equation}
However, a single monotone does not provide information on whether two states are incomparable in the theory.
Motivated by this restrictive nature of the monotone, we introduce the idea of a \emph{fragment} as a less restrictive way of comparing the resource of states in an arbitrary theory $\R = (\F, \O)$. \nick{explain exactly why fragments $>$ monotones}
Let $\R' = (\F', \O')$ be a subtheory of $\R$ so that $\F' \subseteq \F$ and $\O' \subseteq \O$. 
Then, we define a fragment as follows.
\begin{definition}[\textbf{Resource projection}]\label{def:fragment}
    Let a resource theory $\R = (\F, \O)$ have pre-order $\prec_{\R}$ and operational composition rule $\circ_{\R}$. 
    We call a resource fragment of $\R$ any theory $\R' = (\F', \O')$ with pre-order $\prec_{\R'}$ and operational composition rule $\circ_{\R'}$, if there exists a surjective projection $\Pi \equiv (\Pis, \Pio): \R \mapsto \R'$ that satisfies the following two conditions.
    \begin{enumerate}
        \item $\Pis: \F \mapsto \F'$ and $\Pis(\rho_1) \prec_{\R'} \Pis(\rho_2)$ whenever $\rho_1 \prec_{\R} \rho_2$ for any states $\rho_1, \rho_2 \in \F$;
        \item $\Pio: \O \mapsto \O'$ and $\Pio(\E_1) \circ_{\R'} \Pio(\E_2) = \Pio(\E_1 \circ_{\R} \E_2)$ for any free operations $\E_1, \E_2 \in \O$. 
    \end{enumerate}
    We call $\Pi$ a resource projection.
\end{definition}
The subtheory $\R'$ is essentially the image of the surjection $\Pi$.
The point of constructing a resource fragment with such a surjective map would be to achieve a more convenient set of free states or operations, or importantly a more convenient pre-order.

\begin{figure}
    \centering
    \includegraphics[height=3cm]{sections/major/fragments.pdf}
    \caption{Fragments \nick{Split into subfigures}
    }
    \label{fig:fragments}
\end{figure}

Any monotone is a trivial example of a resource fragment in the sense of the following~\cref{thm:monofrag}.
\begin{proposition}\label{thm:monofrag}
	Any monotone $\cal{M}$ of a resource theory $\R$ is a resource projection that maps the set of free states to $0$ and the pre-order $\prec_\R$ to a total order.
\end{proposition}
\begin{proof}
	Consider a monotone $\cal{M}$ in the context of a general resource theory $\R = (\F, \O)$.
	Let $\R' = (\F', \O')$, where $\F' \equiv \mathbb{R}_{\geq 0}$ is the set of non-negative real numbers and $\O'$ is the set of non-increasing real functions mapping $\mathbb{R}_{\geq 0}$ to $\mathbb{R}_{\geq 0}$. 
	Naturally, we also set $\prec_{\R'}$ as the usual total order $\leq$ and $\circ_{\R'}$ as the usual function composition.
	
	The defining property of a monotone, given in~\cref{eq:monotonicity} ensures condition 1 of~\cref{def:fragment}.
	
	We can then project any $\E \in \O$ onto a function $f \in \O'$, so that $f$ maps $\cal{M}(\rho)$ onto $\cal{M}(\E(\rho))$ for all states $\rho$.
	
	\nick{expand}
\end{proof}
Mapping free states to real numbers and free operations to simple addition, the two conditions in the definition are equivalent to monotonicity and additivity respectively.
The ``free state'' for real numbers is 0 and all free states of the resource theory are mapped onto it.

The most general magic theory, the theory of Wigner negativity \nick{or positivity?}, can be expressed as $\R = (\F, \O)$ with
\begin{align}
    \F &= \{ \rho: \W[\bmz]{\rho} \geq 0\}\ \text{and} \\
    \O &= \{ \E: \W[\bmy|\bmx]{\E} \geq 0\}
\end{align}
for all points $\bmx, \bmy, \bmz \in \pd$.
Therefore, any free state $\sigma$ corresponds to a $d^2$-dimensional probability distribution $\W{\sigma}$ and any free operation $\E: \cal{B}(\hd) \mapsto \cal{B}(\hd)$ corresponds to a $d^2 \times d^2$ stochastic matrix (or conditional probability distribution) $\W{\E}$.
Note that these mappings are one-to-one due to the orthogonality of the phase-point operators as an operator basis.

Our goal is to give insight into magic state transformations.
We first break up the theory into fragments that allow for the incorporation of $\bmd$-majorization in its framework.
\begin{definition}[\textbf{$\boldsymbol\sigma$-fragment}]\label{def:sigmafrag}
    A subtheory $\R'$ of the Wigner negativity theory $\R = (\F, \O)$ is called a \emph{$\sigma$-fragment} iff $\R' = (\F, \O_\sigma)$, where the free operations are restricted to the ones that preserve $\sigma$,
    \begin{equation}
        \O_\sigma \coloneqq \{ \E \in \O: \E(\sigma) = \sigma \}.
    \end{equation}
\end{definition}

State $\sigma$ is thus a fixed point of all operations in $\O_\sigma$ and we provide the following theorem which characterises the $\sigma$-fragments as well as how they make up the whole set of free operations. \nick{move below}

\subsection{Majorization}\label{sec:major}

Majorization is a fundamental tool that has recently found many applications in quantum information theory \nick{CITE}.
It describes the \nick{disorder / non-uniformity} of vectors via stochastic transformations that are possible between them.

In order to discuss such stochastic transformations, we first denote by $\stoch$ the set of $(d \times d)$ stochastic matrices that preserve vector $\bmg$.
Specifically, any $S \in \stoch$ satisfies:
\begin{enumerate}
    \item $S_{ij} \geq 0$ for all $i, j \in \zd$;
    \item $\sum\limits_{j=1}^n S_{ij} = 1$ for all $i \in \zd$;
    \item $S\bmg = \bmg$.
\end{enumerate}
It forms a group under matrix multiplication because it contains the identity and $S^{-1} \in \stoch$ for any $S \in \stoch$.

We can motivate majorization very well for our purposes via its application on quantum thermodynamics in the absence of coherence.
At any given temperature $\beta$, the thermal state $\gamma_\beta$ is intuitively the most ordered state. 
Thermal operations are defined as operations that cannot extract energy from the Gibbs state, $\E(\gamma_\beta) = \gamma_\beta$.
Convertibility between states via thermal operations is equivalent to a stochasticity condition on the energy level populations of the states \nick{CITE}.
Roughly, the statement is that there exists a thermal operation $\E$ such that $\tau = \E(\rho)$ if and only if there exists a a matrix $S \in \stoch[\bmg]$ such that $\bm{q} = S\bm{p}$, where $\bm{q}, \bm{p}$ and $\bmg$ and the energy level population vectors of $\tau, \rho, \gamma_\beta$ respectively. \nick{expand or remove}

We can define majorization based on this definition.
\begin{definition}\label{def:dmajor}
    Given $\bmx, \bmy, \bmg \in \reals^d$, such that the components of $\bmg$ are positive, $\bmy$ is said to $\bmg$-majorize $\bmx$, iff there exists a matrix $S \in \stoch$ such that $\bmx = S\bmy$.
    
    We denote this partial order by $\bmx \prec_{\bmg} \bmy$.
\end{definition}
If $\bmg = \frac{1}{d}\bm{1}$, the $d$-dimensional uniform distribution, then $\stoch$ is the set of bistochastic matrices and we retrieve the familiar notion of majorization in entanglement theory. \nick{CITE}

A visual representation of $\bmg$-majorization is provided by the Lorenz curve of a vector $\bmz \in \reals^d$.
Let the vector $\bmz^\downarrow$ denote $\bmz$, but with its components arranged in non-increasing order.
\begin{definition}
    Let $\bmz \in \reals^n$.
    Let $\bmg \in \reals^d$ be a vector with positive components, $\pi$ a permutation mapping $(z_i/g_i) \mapsto (z_i/g_i)^\downarrow$ for all $i=1,\dots,d$ and $D = \sum_{i=1}^d g_i$.
    The Lorenz curve $L(\bmz|\bmg)$ of vector $\bmz$ is the piecewise linear curve obtained by joining the points 
\begin{equation}\label{eq:lorenz}
    \left\{ L_k(\bmz|\bmg) \coloneqq \left( \frac{1}{D}\sum_{i=1}^k g_{\pi(i)}, \sum_{i=1}^k z_{\pi(i)} \right) \in \mathbb{R}^2: k = 1,\dots,d \right\}.
\end{equation}
\end{definition}
\emph{Remark 1.} The origin $L_0(\bmz|\bmg) \coloneqq (0,0)$ is usually included in the curve.

\emph{Remark 2.} The first components of $L_k(\bmz|\bmg)$ are rescaled by $D$ so that comparison of curves with unequal dimensions is possible.
In fact, the Lorenz curves $L(\bmz|\bmg)$ and $L(\bmz \otimes \bmg|\bmg \otimes \bmg)$, where $\otimes$ denotes the Kronecker product, are identical.

\emph{Remark 3.} Lorenz curves are always concave.

\emph{Remark 4.} If $L_d(\bmz|\bmg) = 1$ and for all $k$, $L_k(\bmz|\bmg) \leq 1$, then $\bmz$ is a probability distribution.
Lorenz curves of quasi-probability distributions in principle reach above 1.

An example of comparison between different Lorenz curves is illustrated in~\cref{fig:lctoy}.
\begin{figure}
    \centering
    \includegraphics[height=5cm]{sections/major/lctoy.pdf}
    \caption{Example of different Lorenz curves for quasi-probability vectors under $\bmg$-majorization.
    Vectors $\bmy$ and $\bmg$ are simply probability distributions.
    The curve corresponding to vector $\bmg$ is always the straight line connecting $(0,0)$ and $(1,1)$, so that any other Lorenz curve lies above it, for example $\bmx \prec_{\bmg} \bmg$.
    Curves $L_k(\bmx|\bmg)$ and $L_k(\bmy|\bmg)$ intersect, so neither $\bmx \prec_{\bmg} \bmy$ nor $\bmy \prec_{\bmg} \bmx$.
    }
    \label{fig:lctoy}
\end{figure}

We now present the important majorization result needed in our analysis of magic.
\begin{theorem}\label{thm:dmajor}
Given $\bmx, \bmy, \bmg \in \reals^d$, such that the components of $\bmg$ are positive, the following statements are equivalent:
 \begin{enumerate}%[label=\enlabel{TM}{\arabic*}]
	\item\label{en:tm1} $\bmx \prec_{\bmg} \bmy$;
	\item\label{en:tm5} $L_k(\bmx|\bmg) \leq L_k(\bmy|\bmg)$ for all $k=1,2,\dots, d-1$ and $L_d(\bmx|\bmg) = L_d(\bmy|\bmg)$.
 \end{enumerate}
\end{theorem}
A restatement of the theorem including more equivalent conditions, along with a proof is provided in the \nick{appendix}.

\subsection{Magic fragments}\label{sec:magfrag}

\begin{theorem}\label{thm:frag}
    In the resource theory of Wigner negativity $\R = (\O, \F)$, the following statements hold:
    \begin{enumerate}
        \item The $\sigma$-fragment $\O_\sigma$ is equal to the set of all $\cptp$ operations with stochastic Wigner distributions that leave $\W{\sigma}$ invariant. \nick{Assuming maximal resource theory}
        \item Every free operation leaves at least one free state invariant such that $\O = \bigcup\limits_{\sigma \in \F} \O_\sigma$.
        \item If a free operation leaves two states invariant, then it also leaves every mixture of them invariant, $\O_{\sigma} \cap \O_{\sigma'} \subseteq \O_{p\sigma + (1-p)\sigma'}$ for all $p \in [0,1]$.
    \end{enumerate}
\end{theorem}
\begin{proof}
    \begin{enumerate}
    \item Let $\O_\sigma' \coloneqq \{ \E \in \cptp: \W{\E} \in \stochw \}$ be the described set of operations.
    
    Suppose $\E$ is in $\O_\sigma$, then $\E \in \cptp$ and $\W{\E} \in \stochw$ due to property~\ref{en:wo4} of~\cref{thm:wchannel}, hence $\O_\sigma \subseteq \O_\sigma'$.
    
    Conversely, suppose $\E \in \cptp$ with $\W{\E} \in \stochw$. 
    Then, $\W[\bmy|\bmx]{\E} \geq 0$ for all $\bmx, \bmy$, hence $\E \in \O$.
    Furthermore, $\W{\E} \W{\sigma} = \W{\sigma}$ implies $\E(\sigma) = \sigma$ using~\cref{eq:woperation} defined for any $\cptp$ operation $\E$.
    Hence, $\O_\sigma' \subseteq \O_\sigma$.
    
    \item Suppose $\E$ is in $\O_\sigma$, then it is also in $\O$, hence $\bigcup\limits_{\sigma \in \F} \O_\sigma \subseteq \O$.
    
    Conversely, suppose $\E$ is in $\O$. 
    The free states are mapped one-to-one to a subset $\cal{S}$ of the $(d^2 - 1)$-dimensional probability simplex.
    $\cal{S}$ is convex, since any combination of free states is also free and the Wigner distribution is linear.
    Therefore, $\cal{S}$ is convex and compact as a convex subset of the bounded compact probability simplex. \nick{Need to prove that $\cal{S}$ is closed.}
    Then, $\W{\E}{}$ is a stochastic, thus continuous, mapping from $\cal{S}$ to itself and Brouwer's fixed point theorem \nick{CITE} implies that there exists a probability distribution $g_{\bmz}, \bmz \in \pd$ that is preserved by $\W{\E}{}$.
    This corresponds one-to-one to a state $\sigma \coloneqq \sum_{\bmz \in \pd} g_{\bmz} A_{\bmz}$ and so $\O \subseteq \bigcup\limits_{\sigma \in \F} \O_\sigma$.
    
    \item Let $\E \in \O_{\sigma} \cap \O_{\sigma'}$.
    Then $\E \in \cptp$ and corresponds to stochastic Wigner distribution $\W{\E}$ such that $\W{\E} \W{\sigma} = \W{\sigma}$ and $\W{\E} \W{\sigma'} = \W{\sigma'}$.
    Then, $\W{\E} \W{p\sigma + (1-p)\sigma'} = \W{p\sigma + (1-p)\sigma'}$ for any $p \in [0,1]$ due to the additive property~\ref{en:w4} of the Wigner distribution, implying that state $p\sigma + (1-p)\sigma'$ is also left invariant by $\E$.
    \end{enumerate}
\end{proof}

The structure of the $\sigma$-fragment $(\F, \O_\sigma)$ for any $d$-dimensional $\sigma$ in $\F$ admits the pre-order of $\bmg$-majorization with $\bmg = \W{\sigma}$, a $d^2$-dimensional probability vector. 
If any component of $\W{\sigma}$ is zero, then we can always add some $\epsilon$ amount of unital noise by mixing $\sigma$ with the maximally mixed state $\frac{1}{d}\id$. 
This ensures that all components are strictly positive and $\bmd$-majorization can be used.

\begin{proposition}
    Let $\R = (\O, \F)$ be a theory of magic.
    If the state conversion $\rho \xrightarrow{\O} \tau$ is possible, then there exists a full-rank free state $\sigma \in \F$ such that $\W{\tau} \prec_{\W{\sigma}} \W{\rho}$.
\end{proposition}
\begin{proof}
    Suppose there exists $\E \in \O$ such that $\E(\rho) = \tau$.
    The free operation belongs to a $\sigma$-fragment, $\E \in \O_\sigma$, for some $\sigma \in \F$.
    Then, $\W{\E}\W{\rho} = \W{\tau}$ with $\W{\E} \in \stochw$, or, equivalently, $\W{\tau} \prec_{\W{\sigma}} \W{\rho}$.
\end{proof}
\nick{This proposition needs modification. What about $\E(\rho) = \ketbra{0}$?}

\emph{Remark 1.} Note that free states $\F$ map onto a \emph{strict subset} of the set of probability distributions.
As a counterexample, consider the 9-dimensional probability distribution with non-zero components $\left(\frac{1}{2}, \frac{1}{4}, \frac{1}{4} \right)$. 
It does not correspond to any qutrit Wigner distribution because the component with value $\frac{1}{2}$ does not satisfy the boundedness condition~\ref{en:w3} of~\cref{thm:wstate}.

\emph{Remark 2.} Similarly, any $\O_\sigma$ may be mapped onto a \emph{strict subset} of the set $\stochw$ of stochastic matrices that preserve $\W{\sigma}{}$.

As an example, consider the permutation matrix
\begin{equation}
    \Pi_X = \begin{psmallmatrix}
        0 & 1 & 0 & 0 & 0 \\
        0 & 0 & 0 & 0 & 1 \\
        0 & 0 & 0 & 1 & 0 \\
        1 & 0 & 0 & 0 & 0 \\
        0 & 0 & 1 & 0 & 0
    \end{psmallmatrix} \otimes \begin{psmallmatrix}
        0 & 0 & 1 & 0 & 0 \\
        0 & 0 & 0 & 0 & 1 \\
        0 & 0 & 0 & 1 & 0 \\
        1 & 0 & 0 & 0 & 0 \\
        0 & 1 & 0 & 0 & 0    
    \end{psmallmatrix} \in \stochw,\ d=5.
\end{equation}
It preserves the uniform distribution $\W{\frac{1}{5}\id}{}$, but it does not correspond to any completely positive operation and therefore to any $\E \in \O$, due to~\cref{thm:frag}.

\emph{Remark 3.} No $\sigma$-fragment is empty.
In fact, an example of a stabilizer operation $\E \in \so \cap \O_\sigma$, for any $\sigma$-fragment, is the completely depolarising \nick{replacement} map
\begin{align}
    \E(\rho) &= \sigma \tr[\rho],\ \text{with} \\
    \W[\bmy|\bmx]{\E} &= \W[\bmy]{\sigma} \tr[\rho],
\end{align}
which can be thought as a sequence of tracing out state $\rho$ and preparing the stabilizer $\sigma$.

The zoo of all the operation classes is summarised in ~\cref{fig:zoo}.
Completely positive-Wigner-preserving operations ($\cpwp$) form the largest operation class in the literature.
Therefore, $\sigma$-fragments cover this theory of magic exactly and any magic subtheory is contained within this cover.
For example, the stabilizer-preserving ($\spo$)~\cite{cit:ahmadi} and completely stabilizer-preserving ($\cspo$)~\cite{cit:seddon} operation subclasses follow the hierarchy $\so \subset \cspo \subset \spo \subset \cpwp$ and therefore are described by our $\sigma$-fragments.

\begin{figure}[b]
    \centering
        \includegraphics[scale=0.45]{sections/major/operations.pdf}
    \caption{Zoo of allowed operations for magic resource theories.
    Established theories involve operations within the yellow regions, following the hierarchy $\so \subset \cspo \subset \spo \subset \cpwp$.
    We introduce fragments $\O_\sigma \subset \stochw,\ \sigma \in \F$ that cover $\cpwp$ with each one extending to a set of stochastic maps that allows for $\bmd$-majorization to be used.
    \nick{fragments are the intersections of blue and orange, not the whole bubble - technically $\stochw$ should be replaced by its operational pre-image.}
    }
    \label{fig:zoo}
\end{figure}