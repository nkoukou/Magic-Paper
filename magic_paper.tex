\documentclass[pra,
aps,
twocolumn,
superscriptaddress,
groupedaddress,
nofootinbib,
reprint
]{revtex4-1}

% PACKAGES
\usepackage{amsmath,amsfonts, amssymb, amsthm}
\usepackage{bm, bbm, physics, mathtools}
\usepackage{graphicx, subfigure, multirow, makecell}
\usepackage{xcolor, enumerate}
\usepackage{xifthen, hyperref}
\usepackage[capitalise]{cleveref}

\hypersetup{
	colorlinks=true,  
	linkcolor=blue,   
	citecolor=blue,   
	urlcolor=blue     
}

\newcommand{\crefrangeconjunction}{--}
\creflabelformat{figure}{(#2#1#3)}

% COMMENT NOTATION
\newcommand{\nick}[1]{{\color{red}[#1]}}
\newcommand{\ddd}[1]{\textcolor{blue}{#1}}

% ENVIRONMENTS
\newtheorem{theorem}{Theorem}
\newtheorem{proposition}[theorem]{Proposition}
\newtheorem{lemma}[theorem]{Lemma}
\newtheorem{definition}[theorem]{Definition}
\newtheorem{example}{Example}

% REFERENCES
\iffalse
\renewcommand{\eqref}[1]{Eq.~(\ref{#1})}
\newcommand{\figref}[1]{Fig.~(\ref{#1})}
\newcommand{\tabref}[1]{Tab.~(\ref{#1})}
\newcommand{\secref}[1]{Section~(\ref{#1})}
\newcommand{\appref}[1]{Appendix~(\ref{#1})}
\newcommand{\defref}[1]{Definition~\ref{#1}}
\newcommand{\lemref}[1]{Lemma~\ref{#1}}
\newcommand{\thmref}[1]{Theorem~\ref{#1}}
\fi

% SYMBOL DEFINITIONS
\renewcommand{\cal}[1]{\mathcal{#1}}

\newcommand{\reals}{\mathbb{R}}
\newcommand{\id}{\mathbbm{1}}
\newcommand{\idc}{1_{\rm{C}}}
\newcommand{\supf}{\mathfrak{c}}
\newcommand{\floor}[1]{\left\lfloor #1 \right\rfloor}
\newcommand{\ent}[2]{S\left( #1 \middle\vert\middle\vert #2 \right)}
\newcommand{\ents}{{\ent{\frac{m}{n}}{p}}}
\renewcommand{\tr}{{\rm{tr}}}
\renewcommand{\det}{{\rm{det}}}


\newcommand{\spanv}[1]{
    {{\rm{span}}\left\{#1\right\}}
}
\newcommand{\conv}[1]{
    {{\rm{conv}}#1}
}
\newcommand{\orb}[1]{
    {{\rm{orb}}(#1)}
}
\newcommand{\sn}[1]{
    {{\rm{sn}}\left(#1\right)}
}
\newcommand{\mana}[1]{
    {{\rm{mana}}\left(#1\right)}
}
\newcommand{\lc}[2]{
	{{\rm{L}}_{#1|#2}}
}

\newcommand{\bmx}{\bm{x}}
\newcommand{\bmy}{\bm{y}}
\newcommand{\bmz}{\bm{z}}
\newcommand{\bmu}{\bm{u}}
\newcommand{\bmo}{\bm{0}}
\newcommand{\bmd}{\bm{d}}
\newcommand{\bma}{\bm{a}}
\newcommand{\bmw}{\bm{w}}
\newcommand{\bmm}{\bm{m}}
\newcommand{\bmk}{\bm{k}}
\newcommand{\bmq}{\bm{q}}
\newcommand{\bmg}{\bm{g}}

\newcommand{\spd}[1][]{
    \ifthenelse{\isempty{#1}}{
    {{\rm{Sp}}(2, \zd)} }{
    {{\rm{Sp}}(2, \zd[#1])}}
}
\newcommand{\stoch}[1][]{
    \ifthenelse{\isempty{#1}}{
    {{\rm{S}}_d(\bmd)} }{
    {{\rm{S}}_d(#1)}}
}
\newcommand{\stochw}[1][]{
    \ifthenelse{\isempty{#1}}{
    {{\rm{S}}_{d^2}(\W{\sigma})} }{
    {{\rm{S}}_{d^2}(#1)}}
}
\makeatletter
\def\W{\@ifnextchar[{\@with}{\@without}}
\def\@with[#1]#2{ 
    {{\rm{W}}_{#2}\left(#1\right)} }
\def\@without#1{ 
    {{\rm{W}}_{#1}} }
\makeatother

\newcommand{\T}{\cal{T}}
\newcommand{\Z}{\cal{Z}}
\newcommand{\C}{\cal{C}}
\newcommand{\E}{\cal{E}}
\newcommand{\J}{\cal{J}}
\newcommand{\R}{\cal{R}}
\newcommand{\D}{\cal{D}}
\newcommand{\M}{\cal{M}}
\newcommand{\F}{\cal{F}}
\renewcommand{\O}{\cal{O}}

\newcommand{\Fmax}{\F_{\rm{max}}}
\newcommand{\Omax}{\O_{\rm{max}}}
\newcommand{\Rmax}{\R_{\rm{max}}}
\newcommand{\Pis}{\Pi_{\rm{s}}}
\newcommand{\Pio}{\Pi_{\rm{o}}}

\newcommand{\cptp}{{\rm{CPTP}}}
\newcommand{\cpos}{{\rm{CP}}}
\newcommand{\so}{{\rm{SO}}}
\newcommand{\stab}{{\rm{STAB}}}
\newcommand{\spo}{{\rm{SPO}}}
\newcommand{\cspo}{{\rm{CSPO}}}
\newcommand{\rcu}{{\rm{RCU}}}
\newcommand{\tho}{{\rm{TO}}}
\newcommand{\cpwp}{{\rm{CPWPO}}}
\newcommand{\ru}{{\rm{RU}}}



%%%%%% DJ DEFINITIONS %%%%


\def\>{\rangle}
\def\<{\langle}
\def\K{ {\mathcal K} }
\def\E{ {\mathcal E} }
\def\P{ {\mathcal P} }
\def\H{ {\mathcal H} }
\def\M{ {\mathcal M} }

\def\U {{\mathcal U}}
\def\R {{\mathcal R}}
\def\G {{\mathcal G}}
\def\N{ {\mathcal N} }
\def\F{ {\mathcal F} }
\def\A{ {\mathcal A} }
\def\B{ {\mathcal B} }
\def\O{ {\mathcal O} }
\def\P{ {\mathcal P} }
\def\D{ {\mathcal D} }
\def\T{ {\mathcal T} }
\def\I{ \mathbbm{1} }
\def\tr{ \mbox{tr} }
\def\non{ \nonumber\\}
\def\plus{ |+\> }
\def\minus{|-\> }
\def\plusdag{ \<+| }
\def\minusdag{ \<-| }
\def\diag{ \mathrm{diag}}

\def\r{\boldsymbol{r}}
\def\w{\boldsymbol{w}}
\def\x{\boldsymbol{x}}
\def\y{\boldsymbol{y}}
\def\z{\boldsymbol{z}}
\def\t{\boldsymbol{t}}
\def\p{\boldsymbol{p}}
\def\q{\boldsymbol{q}}


%%%%%%%%%%%%%%%%

\begin{document}

\begin{abstract}
Magic states are essential for achieving universality in fault-tolerant schemes.
Magic resource theories attempt to quantify magic via monotones and thus describe the manipulation of magic states.
Here we introduce the concept of majorization fragments as a more generalised projection of such theories in discrete odd dimensions which allows for more powerful results than what monotones can provide.
Fragments naturally link the symmetries of a gate sequence with conditions on the convertibility between states.
We demonstrate the power of fragments by providing exact conditions for the convertibility of single-copy qutrit magic states as well as stricter distillation bounds than the established mana monotone in any odd dimension.
\end{abstract}

\preprint{APS/123-QED}

\title{Application of single-shot thermodynamics to magic distillation for quantum computation}

\author{Nikolaos Koukoulekidis}
	\affiliation{Department of Physics, Imperial College London, London SW7 2AZ, UK}
	\email{nk2314@imperial.ac.uk}
\author{David Jennings}
	\affiliation{School of Physics and Astronomy, University of Leeds, Leeds, LS2 9JT, UK}
	\affiliation{Department of Physics, Imperial College London, London SW7 2AZ, UK}

\date{\today}
\maketitle

%%%%%%%%%%%%%%%%%%%%%%%%%%%%%%%%%%%%%%%%

\section{Introduction}
\label{sec:intro}

\begin{enumerate}
    \item Fault-tolerance~\cite{cit:raussendorf, cit:gross2, cit:markov, cit:gross, cit:nest, cit:nest2, cit:vidal, cit:fujii, cit:gottesman}
    \item Magic~\cite{cit:veitch, cit:veitch2, cit:wang, cit:wang2, cit:howard, cit:campbell, cit:gross3, cit:gross4}
    \item Bringing in majorization~\cite{cit:cwiklinski, cit:lostaglio2, cit:lostaglio, cit:gour, cit:janzing, cit:brandao2, cit:gour2, cit:marshall, cit:nielsen}
    \item Section breakdown
\end{enumerate}

%%%%%%%%%%%%%%%%%%%%%%%%%%%%%%%%%%%%%%%%

\section{Magic Resource Theories}
\label{sec:magic}

\subsection{Introduction}

Magic states are necessary for achieving universal quantum computation within fault-tolerant schemes.
Identifying magic as a resource for quantum universality has led to several theories which try to provide a framework for its quantification and manipulation \nick{CITE}.
The main question that such a theory attempts to answer is:
\begin{center}
    \emph{Given two magic states $\rho$ and $\rho'$ is there a free operation that can convert $\rho$ to $\rho'$?}
\end{center}

We are interested in all resource theories of magic $\R = (\F, \O)$ in which free operations cannot generate any amount of resource. 
Further denote by $\D$ the set of states considered under the theory, that is the union of free and resource states.
The structure of such theory is described by a partial order \nick{CITE}, hereinafter called a \emph{pre-order}, $\prec_{\R}$ between states.
We write $\rho' \prec_{\R} \rho$ iff there exists $\E \in \O$ such that $\E(\rho) = \rho'$.
Naturally, states may be incomparable under the given theory, meaning that there exists no free operation that converts one to the other.
We further call $\R' = (\F', \O')$ a \emph{subtheory} of $\R$ iff $\F' \subseteq \F$ and $\O' \subseteq \O$. 
The above notation will be used for general resource theories as well.

Formally, the no resource generation condition on the theories translates into two assumptions:
\begin{enumerate}[I]
    \item Free operations send free states into free states, $\E: \F \mapsto \F$, for all $\E \in \O$;
    \item Resource theory $\R$ is a completely free state preserving theory, in the sense that for any $d$-dimensional ancilla system and all free operations $\E$, $(\id_d \otimes \E ) \sigma \in \F$ whenever $\sigma \in \F$.
\end{enumerate}
The first assumption simply states that resources cannot be generated for free and is a minimal requirement for a resource theory. 
An immediate consequence is that if statistical mixing is included in $\O$, then the resource theory is convex.
Convex resource theories have attracted a lot of attention recently \nick{CITE} and include the magic theories discussed in~\cref{sec:prev}.
The second assumption implies that resources cannot be generated even when ancillas are allowed \nick{example of T state generation on Bell state by Campbell}.

Monotones are often used \nick{CITE} to address the question of state convertibility, although such approaches are usually generic.
A monotone of any general resource theory is a projection of the theory onto the non-negative real numbers, collapsing the pre-order of the theory to the total order defined on the real line.
This is the \nick{most naive} non-trivial projection under which the images of incomparable states can be compared.
Our first contribution is the introduction of a generalised notion of \emph{resource projection} which maps a general resource theory onto a subtheory which in principle still retains a partial structure.
Applying this notion on existing magic theories highlights the hidden stochasticity that governs magic state conversions.
We show that a magic theory can be subdivided into \emph{fragments} \nick{expand}

\subsection{Previous work}\label{sec:prev}

The stabilizer theory \nick{CITE} is the first theory to introduce the idea of magic and it is discussed in sufficient detail for our purposes in~\cref{sec:so}. 
It comprises of the so-called ``stabilizer'' states ($\stab$) and operations ($\so$), while non-stabilizer (resource) states are called magic.
The stabilizer operations can be expressed in terms of a Stinespring dilation as 
\begin{equation}
    \E(\rho) = \tr_E [U(\rho \otimes \sigma_E)U^\dagger],
\end{equation} 
for an ancilla stabilizer state $\sigma_E$. 
The motivation of the theory stems from the fact that stabilizer operations are experimentally straightforward to implement and they can be used to detect and correct errors on the stabilizer states due to their construction \nick{CITE}.
The Gottesman-Knill theorem however indicates that stabilizer operations need to be supplemented with magic states in order to achieve universality, justifying the term ``magic''.

Generalisations of the stabilizer theory appear in the literature intending to include broader classes of operations \nick{CITE}.
The class of stabilizer preserving operations ($\spo$) is defined as the set of $\cptp$ maps that send stabilizer states into stabilizer states~\cite{cit:ahmadi}.
An important subclass of $\spo$ is the set of completely stabilizer preserving operations ($\cspo$), which intuitively cannot induce ``non-stabilizerness'' even when applied to only part of a quantum state, i.e. operations $\E$ such that $(\id_d \otimes \E ) \sigma \in \stab$ for all positive dimensions $d$ whenever $\sigma \in \stab$.

Even though non-stabilizerness is a necessary resource for universality, it has been proven insufficient for magic state distillation~\cite{cit:bravyi, cit:campbell}.
In fact, all states with non-negative Wigner distributions have been proven to be efficiently classically simulable in~\cite{cit:mari}, a result that serves as a generalization of the Gottesman-Knill theorem.
The Wigner distribution of a state in odd prime dimensions is discussed rigorously in~\cref{sec:wigner} and arises as the unique quasi-probability representation of quantum theory that identifies non-contextuality exactly with the states that are efficiently classically simulable~\cite{cit:howard2, cit:veitch2}.
In this framework, the stabilizer states are the only pure states represented with non-negative distributions~\cite{cit:gross3}. 
However, there exist mixed states with non-negative Wigner distributions that are not mixtures of stabilizer states~\cite{cit:gross}.
Therefore, stabilizer-preserving theories have been extended to a theory that preserves state ``Wigner positivity''~\cite{cit:wang}, formally defined in~\cref{sec:wigner} for odd prime dimensions.
Informally, it can be considered as the maximal theory of magic $\Rmax = (\Fmax, \Omax)$, where free states have non-negative Wigner distributions and free operations completely preserve this property.

%%%%%%%%%%%%%%%%%%%%%%%%%%%%%%%%%%%%%%%%

\section{Phase Space formalism}
\label{sec:ps}

\subsection{Stabilizer Theory}\label{sec:so}

Let $\{\ket{k}\}_{k \in \zd}$ be the standard computational basis for an arbitrary fault-tolerant scheme, defined over the finite field $\zd = \{ 0, 1, \dots,d-1 \}$, with $d$ an odd prime. 
Since the field has character $d$, addition and multiplication on the field are always considered modulo $d$.
The Hilbert space of any system associated with this scheme is $\hd \coloneqq \spanv{\ket{k}: k \in \zd}$.

The generalised Pauli matrices $X, Z$ can be defined by their respective roles as shift and phase operators,
\begin{align}
    X \ket{k} &= \ket{k + 1} \label{eq:xpauli}\\
	Z \ket{k} &= \omega^k \ket{k}, \label{eq:zpauli}
\end{align}
where $\omega \coloneqq e^{2\pi i/d}$ is the $d$-th root of unity. 

The Hilbert space $\hd$ is associated with a phase space $\pd \coloneqq \zd \times \zd$, where every point $\bmx \coloneqq (x_0, x_1)$ corresponds to a displacement operator, defined as
\begin{equation}\label{eq:ddef}
    D_{\bmx} \coloneqq \tau^{x_0 x_1} X^{x_0} Z^{x_1},\ \bmx  \in \pd.
\end{equation}
The phase factor $\tau \coloneqq -\omega^{1/2}$ ensures unitarity.
For a system with composite Hilbert space, $\hd = \hd[d_A] \otimes \hd[d_B]$, the displacement operators are defined as
\begin{equation}\label{eq:composited}
    D_{\bmx_A \oplus \bmx_B} \coloneqq D_{\bmx_A} \otimes D_{\bmx_B},
\end{equation}
where $\bmx_A \oplus \bmx_B \coloneqq (x_{A0}, x_{B0}, x_{A1}, x_{B1}) \in \pd[d_A] \times \pd[d_B]$.

The displacement operators, form a group under matrix multiplication modulo phases,
\begin{equation}\label{eq:gp}
    \gp \coloneqq \{ \tau^k D_{\bmz}: k \in \zd, \bmz \in \pd\}.
\end{equation}
The Clifford unitaries $\cd$ can then be defined as the normaliser of this group, \nick{Reformulate for copies of qudits: $\gp,\ \cd \rightarrow \gp^n,\ \cd^n$. $C-SUM$s live in $\cd^2$}
\begin{equation}
    \cd \coloneqq \{ U \in {\rm{SU}}(d): U\gp U^\dagger = \gp \}.
\end{equation}
The pure stabilizer states are then the orbit of the Clifford unitaries over a computational basis state,
\begin{equation}\label{eq:stab}
    \stab_{{\rm{pure}}} \coloneqq \{ U\ketbra{0}{0}U^\dagger: U \in \cd \}.
\end{equation}

The free states of the stabilizer theory are mixtures of pure stabilizers, 
\begin{equation}
    \stab = \conv\ \stab_{{\rm{pure}}}.
\end{equation}
The free operations of the stabilizer theory is the set of stabilizer operations $\so$ defined as any composition of:
\begin{enumerate}
    \item Preparation in computational basis;
    \item Random Clifford unitaries $\rcu$, i.e. operations $\E$ such that 
    \begin{equation}
        \E(\rho) = \sum_i p_i U_i \rho U_i^\dagger,\ U_i \in \cd;
\end{equation}
    \item Measurement in computational basis.
\end{enumerate}

\subsection{Wigner Distribution}\label{sec:wigner}

We can define the phase-point operators,
\begin{align}
	A_{\bmx} \coloneqq \frac{1}{d} \sum_{\bmz \in \pd} \omega^{\bmx \wedge \bmz} D_{\bmz},\ \bmx \in \pd. \label{eq:ax}
\end{align}
\nick{$\wedge$ has not be defined} They form an orthogonal Hermitian operator basis.
Therefore, any quantum state $\rho \in \cal{B}(\hd)$ can be expressed as a linear combination of the phase-point operators,
\begin{equation}
    \rho = \sum_{\bmz \in \pd} \W[\bmz]{\rho} A_{\bmz},
\end{equation}
where the coefficient vector $\W{\rho}$ is the Wigner distribution of state $\rho$,
\begin{equation}\label{eq:wstate}
    \W[\bmx]{\rho} \coloneqq \frac{1}{d}\tr[A_{\bmx} \rho].
\end{equation}
It is in fact a real, bounded, $d^2$-dimensional quasi-probability distribution over $\pd$ as shown in~\cref{app:wigner}. 

The Wigner distributions of different types of qutrit states are illustrated in~\cref{fig:wstate_examples}.
\begin{figure}%
    \centering
    \subfigure[][]{%
    \label{fig:maxmix}%
    \includegraphics[height=2cm]{figs/maxmixed.pdf}
    %\caption{Maximally mixed state $\frac{1}{3}\id$}%
    }\hspace{8pt}%
    \subfigure[][]{%
    \label{fig:zero}%
    \includegraphics[height=2cm]{figs/zerostate.pdf}
    %\caption{Zero state $\ketbra{0}{0}$}%
    }\\
    \subfigure[][]{%
    \label{fig:bound}%
    \includegraphics[height=2cm]{figs/boundstate.pdf}
    %\caption{Bound state}%
    }\hspace{8pt}%
    \subfigure[][]{%
    \label{fig:strange}%
    \includegraphics[height=2cm]{figs/strangestate.pdf}
    %\caption{Strange state $\ketbra{S}{S}$}%
    }
    \caption{Wigner distributions for qutrit states of varying magic. 
    \subref{fig:maxmix} Maximally mixed state $\frac{1}{3}\id$; \subref{fig:zero} Stabilizer zero state $\ketbra{0}{0}$; \subref{fig:bound} A non-stabilizer Wigner-positive state; \subref{fig:strange} Magic strange state $\ket{{\rm{S}}} = \frac{1}{\sqrt{2}}(\ket{1} - \ket{2})$.
    \nick{Explain what a magic / bound magic state is in intro}
    }%
    \label{fig:wstate_examples}
\end{figure}

We can exploit the channel-state duality and use the normalised Choi-Jamio\l{}kowski state 
\begin{equation}\label{eq:cj}
    \frac{1}{d_A}\J_\E \coloneqq \frac{1}{d_A}(\id \otimes \E) \sum_{i,j} \ket{ii}\bra{jj}
\end{equation}
to extend the definition of the Wigner distribution to quantum $\cptp$ operations $\E: \cal{B}(\hd[d_A]) \mapsto \cal{B}(\hd[d_B])$, 
\begin{align}\label{eq:woperation}
    \W[\bmy|\bmx]{\E} 
    &\coloneqq d_A^2 \W[\bm{\bar x} \oplus \bmy]{\frac{1}{d_A}\J_\E} \\
    &= \frac{1}{d_B} \tr_B[A_{\bmy} \E(A_{\bmx})],
\end{align}
where $\bm{\bar x} \coloneqq (x_0, -x_1)$.

The specific form of~\cref{eq:woperation} is chosen so that Wigner distributions of operations act as transition matrices for Wigner distributions of states, $\W{\E(\rho)} = \W{\E}\W{\rho}$.
In particular, $\cptp$ operations that map between density operators of equal dimensions and have non-negative Wigner distributions correspond to stochastic matrices, as shown in~\cref{app:wigner}

The single-qudit Hadamard gate $H$ and phase gate $S$ generate the $d$-dimensional Clifford group $\cd$. \nick{CITE}
Their Wigner distributions are given by permutation matrices,
\begin{align}
    H &\coloneqq \frac{1}{\sqrt{d}}\sum_{j,k} \omega^{jk} \ketbra{j}{k}, \W[\bmy|\bmx]{H} = \delta_{y_0, -x_1}\delta_{y_1, x_0};\label{eq:H}\\
    S &\coloneqq \sum_k \tau^{k(k+1)} \ketbra{k}{k}, \W[\bmy|\bmx]{S} = \delta_{y_0, x_0}\delta_{y_1, x_0 + x_1 + 2^{-1}}.\label{eq:S}
\end{align}

%%%%%%%%%%%%%%%%%%%%%%%%%%%%%%%%%%%%%%%%

\section{Stochastic structure of magic theories}
\label{sec:struc}

\subsection{Magic fragments}\label{sec:magfrag}

Equipped with the definitions of the Wigner distribution in odd prime dimensions, we can formally recast the maximal magic theory $\Rmax$ into a stochasticity setting.
The free states correspond to proper probability distributions 
\begin{equation}
    \Fmax \coloneqq \{ \rho: \W[\bmz]{\rho} \geq 0 \text{ for all } \bmz \in \pd\}
\end{equation}

The free operations should send the set of free states $\Fmax$ into itself and completely preserve the non-negativity of the states, in the sense that $\E \in \Omax$ iff $(\id_d \otimes \E ) \sigma \in \stab$ for all odd prime dimensions $d$ whenever $\sigma \in \Fmax$.
It is shown by Wang \textit{et al.}~\cite{cit:wang} that $\Omax$ coincides with the set of operations $\E$ that correspond to stochastic Wigner distributions, 
\begin{equation}
    \Omax = \{ \E: \W[\bmy|\bmx]{\E} \geq 0 \text{ for all } \bmx, \bmy \in \pd\}.
\end{equation}

Any magic theory $\R = (\F, \O)$ is a subtheory of $\Rmax$ as explained in~\cref{sec:intro}, and as such it falls under this new stochasticity setting. For technical simplicity in what follows we assume that $\F$ is a closed set, and note that $\F_{\mbox{\tiny max}}$ is itself a closed set, since it is specified by a finite set of linear constraints of the form $\tr [ L\rho] \ge 0$ with $L \in \mathcal{B}(\mathcal{H})$.


Given this context we now define the following key notion, that is central to our analysis.
\begin{definition}[\textbf{$\boldsymbol\sigma$-fragment}]\label{def:sigmafrag}
   Given a resource theory of magic $\R = (\F, \O)$, the \emph{$\sigma$--fragment of $\R$} is the resource theory $\R_\sigma = (\F, \O_\sigma)$, where the free operations are restricted to the ones that leave $\sigma$ invariant, namely
    \begin{equation}
        \O_\sigma \coloneqq \{ \E \in \O: \E(\sigma) = \sigma \}.
    \end{equation}
\end{definition}

With this basic notion defined, we now show that any resource theory of magic can be faithfully subdivided into $\sigma$--fragments, in such a way that any problem of interconversion in the parent magic theory $\R$ can be analysed across the different fragments.

\begin{theorem}
    Let $\R = (\F, \O)$ be a theory of magic.
    Every operation in $\O$ leaves at least one free state invariant,
  \begin{equation}
\O = \bigcup\limits_{\sigma \in \F} \O_\sigma.
\end{equation}
Therefore, $\rho\longrightarrow \tau$ in $\R$ if and only if $\rho\longrightarrow \tau$ in a $\sigma$--fragment of $\R$.
\end{theorem}
\begin{proof}
    Suppose $\E$ is in $\O_\sigma$, then it is also in $\O$, hence $\bigcup\limits_{\sigma \in \F} \O_\sigma \subseteq \O$. Conversely, suppose $\E$ is in $\O$. 
    The free states are a closed set that is mapped one-to-one to a closed subset $\cal{S}$ of the $(d^2 - 1)$-dimensional probability simplex.
    $\cal{S}$ is convex, since any combination of free states is also free and the Wigner distribution is linear.
    Therefore, $\cal{S}$ is convex and compact as a convex subset of the bounded compact probability simplex.
    Then, $\W{\E}{}$ is a stochastic, continuous, mapping from $\cal{S}$ to itself and so Brouwer's fixed point theorem \nick{CITE} implies that there exists a probability distribution $g_{\bmz}$ for some $ \bmz \in \pd$ that is a fixed point of $\W{\E}{}$.
    This corresponds to an operator $\sigma \coloneqq \sum_{\bmz \in \pd} g_{\bmz} A_{\bmz}$  which is a state in $\F$. Therefore $\E \in \O_\sigma$, and so $\O = \bigcup\limits_{\sigma \in \F} \O_\sigma$. The state interconversion result follows immediately.
\end{proof}

The zoo of all magic operation classes is summarised in ~\cref{fig:zoo}.
Completely positive-Wigner-preserving operations~\cite{cit:wang} form the operation class $\Omax$.
Therefore, $\sigma$--fragments cover this theory of magic exactly and any magic subtheory is contained within this cover.
In particular, the stabilizer operations $\so$ are contained within $\Omax$.
\begin{figure}[t]
    \centering
        \includegraphics[scale=0.47
        ]{figs/operations.pdf}
    \caption{\textbf{Decomposition of a magic theory $\R$ intro $\sigma$--fragments.} 
    Established theories involve operations within the yellow regions, following the hierarchy $\so \subset \cspo \subset \spo \subset \Omax$.
    We introduce fragments $\O_\sigma,\ \sigma \in \F$ that cover $\Omax$ with each one extending to a set of stochastic maps that allows for $\bmd$-majorization to be used.
    }
    \label{fig:zoo}
\end{figure}

The subdivision of magic theories into $\sigma$--fragments is powerful because the pre-order $\prec_{\R'}$ of every $\sigma$--fragment is described by well-behaved majorization tools, as we establish in~\cref{sec:major}. \nick{expand}

\subsection{Majorization in the $\sigma$--fragments}\label{sec:major}

Majorization is a powerful tool that has recently found many applications in quantum information theory \nick{CITE}.
It can describes the \nick{disorder / non-uniformity} of distributions that undergo stochastic transformations.

To formally state majorization results, we first denote by $\stoch$ the set of $(d \times d)$ stochastic matrices that preserve the probability vector $\bmd$. \nick{Should we introduce notation directly in the magic setting?}
Specifically, for any $S \in \stoch$, all matrix elements are non-negative, all rows sum to $1$ and $S\bmd = \bmd$.
\iffalse
\begin{enumerate}
    \item $S_{ij} \geq 0$ for all $i, j \in \zd$;
    \item $\sum\limits_{j=1}^n S_{ij} = 1$ for all $i \in \zd$;
    \item $S\bmd = \bmd$.
\end{enumerate}
\fi
The set $\stoch$ forms a group under matrix multiplication for all $\bmd$ with positive components.

Majorization finds an important application on quantum thermodynamics in the absence of coherence.
The use of majorization in this setting provides useful intuition for our purposes.
At any given temperature $\beta$, the thermal state $\gamma_\beta$ is thermodynamically the most ordered state. 
Thermal operations are defined as operations that cannot extract energy from the Gibbs state, $\E(\gamma_\beta) = \gamma_\beta$.
Convertibility between states via thermal operations is equivalent to a stochasticity condition on the energy level populations of the states \nick{CITE}.
Roughly, the statement is that there exists a thermal operation $\E$ such that $\tau = \E(\rho)$ if and only if there exists a a matrix $S \in \stoch$ such that $\bm{q} = S\bm{p}$, where $\bm{q}, \bm{p}$ and $\bmd$ and the energy level population vectors of $\tau, \rho, \gamma_\beta$ respectively.

Drawing intuition from this setting, we can define majorization as follows.
\begin{definition}[\textbf{$\boldsymbol\bmd$-majorization}]\label{def:dmajor}
    Given $\bmx, \bmy, \bmd \in \reals^d$, such that the components of $\bmd$ are positive, $\bmy$ is said to $\bmd$-majorize $\bmx$, iff there exists a matrix $S \in \stoch$ such that $\bmx = S\bmy$.
\end{definition}
We denote this pre-order by $\bmx \prec_{\bmd} \bmy$.
If $\bmd = \frac{1}{d}\bm{1}$, the $d$-dimensional uniform distribution, then $\stoch$ is the set of doubly stochastic matrices and we retrieve the familiar notion of majorization in entanglement theory. \nick{CITE}

The pre-order $\prec_{\R'}$ of the $\sigma$--fragment $\R' = (\F, \O_\sigma)$ between $d$-dimensional states corresponds to the majorization pre-order $\prec_{\W{\sigma}}$ between their $d^2$-dimensional Wigner distributions.

\begin{theorem}
    Let $\R = (\F, \O)$ be a theory of magic.
    Suppose the state conversion $\rho \xrightarrow{\E \in \O} \tau$ is possible, $\tau \prec_\R \rho$.
    Then, there exists a full-rank free state $\sigma \in \F$ such that  $\W{\tau} \prec_{\W{\sigma}} \W{\rho}$.

\nick{We need to address zeros in the Wigner distribution (I think full-rank states do not have any zeros - there certainly exist non-full-rank states with no zeros).

For example, the replacement operation $\E(\rho) = \ketbra{0}$ is free in $\Rmax$ with $\E \in \O_{\ketbra{0}}$, BUT $W{\ketbra{0}}$-majorization is not defined because $W{\ketbra{0}}$ contains zeros.

In such a case we can always add some $\epsilon$ amount of noise by mixing $\sigma$ with a free full-rank state, e.g. a thermal state $\gamma_\beta$, to get $\sigma' = (1-\epsilon)\sigma + \epsilon\gamma_\beta$.
This ensures that all Wigner components of $\sigma'$ are strictly positive and $\W{\sigma'}$-majorization can be used BUT $\E$ is NOT in $\O_{\sigma'}$ now for any $\epsilon > 0$.

This could be solved if every operation preserves some thermal state.
Otherwise, simply stating in this theorem that ``there exists a full-rank free state $\sigma \in \F$ such that  $\W{\tau} \prec_{\W{\sigma}} \W{\rho}$'' is not accurate and needs reformulation to include the error $\epsilon$.
}
\end{theorem}
\begin{proof}
    Suppose there exists $\E \in \O$ such that $\E(\rho) = \tau$.
    The free operation belongs to a $\sigma$--fragment, $\E \in \O_\sigma$, for some $\sigma \in \F$ so that $\W{\E}\W{\rho} = \W{\tau}$ with $\W{\E} \in \stochw$.
    
    If all components of $\W{\sigma}$ are positive, we directly have $\W{\tau} \prec_{\W{\sigma}} \W{\rho}$.
    
    If $\W{\sigma}$ contains some zero components, we can construct the full-rank state $\sigma' = (1-\epsilon)\sigma + \epsilon\gamma_\beta$ by mixing an arbitrarily small amout $\epsilon>0$ of some thermal state $\gamma_\beta$ \nick{can replace with maximally mixed state}, so that all components of $\W{\sigma'}$ are positive. \nick{but $\W{\tau} \prec_{\W{\sigma'}} \W{\rho}$ is NOT true now}
\end{proof}

A visual representation of $\bmd$-majorization is provided by Lorenz curves.
Let the vector $\bmz^\downarrow$ denote a component permutation of vector $\bmz \in \reals^d$, so that its components are arranged in non-increasing order.
\begin{definition}[\textbf{Lorenz curve}]
    Let $\bmz \in \reals^d$.
    Let $\bmd \in \reals^d$ be a vector with positive components, $\pi$ a permutation mapping $(z_i/d_i) \mapsto (z_i/d_i)^\downarrow$ for all $i=1,\dots,d$ and $D = \sum_{i=1}^d d_i$.
    The Lorenz curve $L(\bmz|\bmd)$ of vector $\bmz$ is the piecewise linear curve obtained by joining the points $\{ (x_k, L_k(\bmz|\bmd)) \}_{k=1,\dots,d}$, where
    \begin{equation}\label{eq:lorenz}
        (x_k, L_k(\bmz|\bmd)) \coloneqq \left( \frac{1}{D}\sum_{i=1}^k d_{\pi(i)}, \sum_{i=1}^k z_{\pi(i)} \right) \in \mathbb{R}^2.
    \end{equation}
\end{definition}
\emph{Remark 1.} The origin $(x_0, L_0(\bmz|\bmd)) \coloneqq (0,0)$ is usually included in the curve.

\emph{Remark 2.} Components $x_k$ are rescaled by $D$ so that comparison of curves with unequal dimensions is possible.
In fact, the Lorenz curves $L(\bmz|\bmd)$ and $L(\bmz \otimes \bmd|\bmd \otimes \bmd)$, where $\otimes$ denotes the Kronecker product, are identical.

\emph{Remark 3.} Lorenz curves are always concave.

\emph{Remark 4.} If $L_d(\bmz|\bmd) = 1$ and for all $k$, $L_k(\bmz|\bmd) \leq 1$, then $\bmz$ is a probability distribution.
Lorenz curves of quasi-probability distributions in principle reach above 1.
\ddd{[It is usually better to convert the Lorenz curve into a \emph{function} $L_{\bmz|\bmd}(x)$ defined on $[0,1]$. Then the majorization simply becomes the functional inequality $L_{\bmz|\bmd}(x) \ge L _{\bmy|\bmd}(x)$ for all $x$.]}

A vector $\bmy$ is said to \emph{$\bmd$--majorize} another vector $\bmx$  if and only if the Lorenz curve $L(\bmy|\bmd)$ lies above Lorenz curve $L(\bmx|\bmd)$.
\begin{theorem}\label{thm:dmajor}
    Let $\bmx, \bmy, \bmd \in \reals^d$, such that the components of $\bmd$ are positive. 
    Then, $\bmx \prec_{\bmd} \bmy$ if and only if $L_k(\bmx|\bmd) \leq L_k(\bmy|\bmd)$ for all $k=1,2,\dots, d-1$ and $L_d(\bmx|\bmd) = L_d(\bmy|\bmd)$.
\end{theorem}
A restatement of the theorem including more equivalent conditions, along with a proof is provided in the \nick{appendix}.

An example of comparison between different Lorenz curves is illustrated in~\cref{fig:lctoy}.
\begin{figure}
    \centering
    \includegraphics[height=5cm]{figs/lctoy.pdf}
    \caption{Example of different Lorenz curves for quasi-probability vectors under $\bmd$-majorization.
    Vectors $\bmy$ and $\bmd$ are simply probability distributions.
    The curve corresponding to vector $\bmd$ is always the line segment connecting $(0,0)$ and $(1,1)$, so that any other Lorenz curve lies above it, for example $\bmx \prec_{\bmd} \bmd$.
    Curves $L_k(\bmx|\bmd)$ and $L_k(\bmy|\bmd)$ intersect, so $\bmx \nprec_{\bmd} \bmy$ as well as $\bmy \nprec_{\bmd} \bmx$.
    \nick{Recast in terms of magic}
    }
    \label{fig:lctoy}
\end{figure}

%%%%%%%%%%%%%%%%%%%%%%%%%%%%%%%%%%%%%%%%

\section{Extension to general quantum resource theories}
\label{sec:frag}

In the previous section we introduced the notion of $\sigma$--fragments for any resource theory of magic. In this section we pause to generalise this concept to an arbitrary resource theory and explain precisely how it connects with resource monotones. The busy reader more focussed on magic may skip this section.

State convertibility within a given resource theory is often a hard question to address due to the intricate structure of the theory.
In general, the structure of a theory $\R$ is described by a pre-order $\prec_\R$ \nick{expand / move to intro}.
Subdividing magic theories into $\sigma$--fragments \nick{helps}.
\subsection{Monotones}\label{sec:mono}

Resource theories are commonly equipped with monotones which quantify the resource\nick{CITE}. 
\begin{definition}[\textbf{Resource monotone}]\label{def:mono}
    Let $\R = (\F, \O)$ be a resource theory.
    A resource monotone $\M$ is a projection from the set of quantum states of the theory onto the real line, so that $\M$ is monotonically decreasing under free operations,
    \begin{equation}
        \M(\rho_1) \leq \M(\rho_2)\ \text{whenever}\ \rho_1 \prec_{\R} \rho_2.
    \end{equation}
\end{definition}
The monotonicity condition reflects the no resource generating property of free operations, so that monotones respect the pre-order $\prec_\R$ of the theory.
Furthermore, if a monotone satisfies the additivity condition,
\begin{equation}
    \M(\rho_1 \otimes \rho_2) = \M(\rho_1) + \M(\rho_2),
\end{equation}
it is of practical importance for resource distillation, which we discuss in~\cref{sec:distill} within the context of magic.

A very commonly used magic monotone is the \emph{mana} of a state \nick{CITE}, defined as
\begin{equation}
    \mana{\rho} \coloneqq \log{\left(\sum\limits_{\bmz \in \pd} \abs{\W[\bmz]{\rho}}\right)}.
\end{equation}
It is a monotone, additive function of the $\ell_1$-norm of negativity~\nick{CITE}.

\subsection{Fragments}

Monotones reduce the structure of the resource theory $\R$ to a \emph{total} order on the real numbers.
Therefore, two states, even if incomparable in $\R$, are always mapped onto ordered real numbers.
We now generalise this idea to any projection of the theory that ensures a form of comparison between states.
\begin{definition}[\textbf{Covariant projection}]\label{def:covproj}
Let $\R = (\F, \O)$ be a resource theory with pre-order $\prec_\R$. 
Then a \emph{covariant resource projection} of $\R$ into a resource theory $\R'$ with pre-order $\prec_{\R'}$, is a pair of mappings $(\Pi_s, \Pi_o)$, where $\Pi_s$ maps quantum states in $\R$ into quantum states in $\R'$, and $\Pi_o$ maps free operations in $\R$ into free operations in $\R'$. 
Moreover, these obey
	\begin{enumerate}
        \item $\Pis(\rho_2) \prec_{\R'} \Pis(\rho_1)$ whenever $\rho_1 \prec_\R \rho_2$;
        \item $\Pio(\E) = \Pio(\E_1) \circ \Pio(\E_2)$ whenever $\E = \E_1 \circ \E_2$.
    \end{enumerate}
We call $\R'$ a \emph{covariant fragment} of $\R$.
\end{definition}

Resource monotones can now be clearly seen as covariant projections of resource theories.
\begin{proposition}\label{thm:monoproj}
	Let $\M$ be a resource monotone of a resource theory $\R$. 
	Then $\M$ is a covariant projection that reduces the pre-order $\prec_\R$ to a total order.
\end{proposition}
\begin{proof}
	Consider a monotone $\M$ in the context of a general resource theory $\R = (\F, \O)$.
	The defining property of a monotone, stated in~\cref{def:mono} ensures order preservation under state projection $\Pis$.	
	Let $\Pio$ project any $\E \in \O$ onto a real non-increasing function $f$ mapping $\M(\rho)$ to $\M(\E(\rho))$ for all states $\rho$.
	Then, the composition of functions is preserved under operation projection $\Pio$.	
	The ordered pair $(\Pis, \Pio)$ is the covariant resource projection which corresponds to monotone $\M$.
\end{proof}

On the other hand, $\sigma$--fragments are an example of reducing the structure of a magic theory $\R$ to a subtheory with a tractable pre-order.
States which are incomparable in $\R$ remain incomparable, but transformations between comparable states may no longer be possible.
We generalise the idea of reducing the structure of a theory into a simpler one, by introducing two types of resource projections.
\begin{definition}[\textbf{Contravariant projection}]\label{def:contraproj}
	Let $\R = (\F, \O)$ be a resource theory with pre-order $\prec_\R$. 
Then a \emph{contravariant resource projection} of $\R$ onto a resource theory $\R'$ with pre-order $\prec_{\R'}$, is a pair of mappings $(\Pi_s, \Pi_o)$, where $\Pi_s$ maps quantum states in $\R$ onto quantum states in $\R'$, and $\Pi_o$ maps free operations in $\R$ onto free operations in $\R'$. 
Moreover, these obey
	\begin{enumerate}
        \item $\rho_1 \prec_\R \rho_2$ whenever $\Pis(\rho_1) \prec_{\R'} \Pis(\rho_2)$;
        \item $\E = \E_1 \circ \E_2$ whenever $\Pio(\E) = \Pio(\E_1) \circ \Pio(\E_2)$.
    \end{enumerate}
We call $\R'$ a \emph{contravariant fragment} of $\R$.
\end{definition}
The use of covariant and contravariant in~\cref{def:covproj,def:contraproj} refers to the direction of implication between the two pre-orders and operation compositions\footnote{Note that strictly these are not projections in the sense of $\Pi^2 = \Pi$, but are instead morphisms. 
Here our use of the term projection is motivated by the idea that one one generally loses information about $\R$ under the mapping.}.

\begin{proposition}\label{thm:sigmaproj}
    Let $\R = (\F, \O)$ be a theory of magic.
    Every $\sigma$--fragment $(\F, \O_\sigma)$ is a contravariant fragment of $\R$.
\end{proposition}
\begin{proof}
    Let the state projection $\Pis$ map all states onto themselves, so order preservation is trivially satisfied.    
    Consider the operation projection $\Pio:\O \mapsto \O_\sigma$, defined as
    \begin{equation}
    \Pio(\E) =
    \begin{cases}
        \E,\ &\E \in \O_\sigma, \\
        1_{\rm{C}},\ &\E \notin \O_\sigma.
    \end{cases}
    \end{equation}
    Since for any $\E_1, \E_2 \in \O_\sigma,$ it is also true that $\E_1 \circ \E_2 \in \O_\sigma$, operational composition is contravariantly preserved.
\end{proof}

\begin{proposition}
	Let $\R = (\F,\O)$ be a resource theory, and let $\D \in \O$ be a free operation, which is reversible by $\D_{\rm{rev}} \in \O$, so that $\D_{\rm{rev}} \circ \D = 1_{\rm{C}}$.
	
	Then, we can define a contravariant projection of $\R$, by acting with $\D$ on all states.
\end{proposition}
\begin{proof}
	We show that the theory $\R' = (\F', \O)$, with $\F' = \{\D(\rho):\rho \in \F \}$, is a contravariant fragment of $\R$.
	Let $\Pis$ map every state $\rho$ to $\D(\rho)$.
	Then, $\D(\rho_1) \prec \D(\rho_2)$ implies that there exists $\E \in \O$ such that $\rho_1 = (\D_{\rm{rev}} \circ \E \circ \D) (\rho_2)$, so the state order is preserved contravariantly.
	
	Finally, let $\Pio$ map every free operation to itself, so that composition of operations is trivially preserved.
	
\nick{If $\D$ is a recovery map, so that $\D \circ \D_{\rm{rev}} = 1_{\rm{C}}$, then this is a covariant projection instead.

If $\D$ is not reversible, this mapping is in general NOT contravariant (consider the replacement map $\D(\rho) = \frac{1}{d}\id$ for a strange state and stabilizer state - surely there is such a counterexample in thermodynamics theory if we consider a highly coherent state and one with the same energy population but no coherences.} 
\end{proof}

Important examples of resource fragments appear in several established resource theories. \nick{Need to check if the thermodynamics example works, include magic theories as fragments of $\Rmax$, include Nielsen's bipartite entanglement.}

\newpage

\subsection{Fragments - OLD}

\nick{OLD VERSION}

Monotones reduce the structure of the resource theory $\R$ into a \emph{total} order on the real numbers.
Therefore, two states, even if incomparable in $\R$, are always mapped onto ordered real numbers.
This is a generic approach that ensures a form of comparison between states and resource quantification.
However, motivated by $\sigma$--fragments in magic theories, it is clear that we can often reduce the theory by a less \nick{information-collapsing} projection into a subtheory $\R'$ which has a tractable pre-order $\R'$, as sketched in~\cref{fig:fragments}.
\begin{figure}[t]
    \centering
    \includegraphics[height=3cm]{figs/fragments.pdf}
    \caption{Fragments \nick{Split into subfigures}
    }
    \label{fig:fragments}
\end{figure}
We define such a resource projection for general resource theories as follows.
\begin{definition}[\textbf{Resource projection}]\label{def:fragment}
    Let a resource theory $\R = (\F, \O)$ have pre-order $\prec_{\R}$ and operational composition rule $\circ_{\R}$. 
    Any subtheory $\R' = (\F', \O')$ with pre-order $\prec_{\R'}$ and operational composition rule $\circ_{\R'}$ is called a resource fragment iff there exists a surjective projection $\Pi \equiv (\Pis, \Pio): \R \mapsto \R'$ that satisfies the following two conditions.
    \begin{enumerate}
        \item $\Pis: \D \mapsto \D$ and $\Pis(\rho_2) \nprec_{\R'} \Pis(\rho_1)$ whenever $\rho_1 \prec_{\R} \rho_2$;
        \item $\Pio: \O \mapsto \O'$ and $\Pio(\E_1) \circ_{\R'} \Pio(\E_2) = \Pio(\E_1 \circ_{\R} \E_2)$ for all free operations $\E_1, \E_2 \in \O$.
    \end{enumerate}
    We call $\Pi$ a resource projection.
\end{definition}

\ddd{[I think the following might be the right way to understand things, using the notion of ``covariant'' and ``contravariant''.]}

Fragment $\R'$ is the image of the projection $\Pi$.
Considering a resource projection is particularly useful when the pre-order of the fragment is tractable.
Note the subtle difference of condition 1 in~\cref{def:fragment} and condition 2 in~\cref{def:mono}, which is due to fragments generally retaining a pre-order, while monotones impose a total order.

\ddd{[If you agree with the covariant-contravariant definitions above then you can update the following results using the new terminology.]}
A monotone is the projection with the simplest tractable order as formally stated in~\cref{thm:monofrag}.
\begin{proposition}\label{thm:monofrag}
	Let $\M$ be a resource monotone of a resource theory $\R$. 
	Then $\M$ is a resource projection that reduces the pre-order $\prec_\R$ to a total order.
\end{proposition}
\begin{proof}
	Consider a monotone $\M$ in the context of a general resource theory $\R = (\F, \O)$.
	Let $\R' = (\F', \O')$, where $\F' \equiv \{0\}$ and $\O'$ is the set of non-increasing real functions mapping the set of non-negative real numbers $\mathbb{R}_{\geq 0}$ to itself. 
	We also set $\prec_{\R'}$ as the usual total order $\leq$ and $\circ_{\R'}$ as the composition of real functions.
	
	Let $\Pis = \M$.
	The defining properties of a monotone, given in~\cref{def:mono}, ensure condition 1 of~\cref{def:fragment}. \nick{expand}
	
	Let $\Pio$ project any $\E \in \O$ onto a function $f \in \O$ which maps $\M(\rho)$ onto $\M(\E(\rho))$ for all states $\rho$.
	
	The ordered pair $(\Pis, \Pio)$ is the resource projection which corresponds to monotone $\M$.
\end{proof}

\begin{example}. As a simple example, we can consider the thermal operations resource theory of thermodynamics, and take $\D$ to be the dephasing map that kills off coherences between energy eigenspaces. Then the resource projection $\R'$ induced by $\D$ is fully described by the thermo-majorization pre-order on distributions over energies.
\end{example}

We now justify the name $\sigma$--fragment for the subdivision of a magic theory $\R = (\F, \O)$, by establishing a resource projection which reduces $\R$ into any subtheory $(\F, \O_\sigma)$.
\begin{proposition}
    Let $\R = (\F, \O)$ be a theory of magic.
    Every $\sigma$--fragment $(\F, \O_\sigma)$ \ddd{defines a contravariant projection} of $\R$.
\end{proposition}
\begin{proof}
    Let the state projection be the identity projection $\Pis:\D\mapsto\D$.
    
    Consider the operation projection $\Pio:\O \mapsto \O_\sigma$, defined as
    \begin{equation}
    \Pio(\E) =
    \begin{cases}
        \E,\ &\sigma \in \O, \\
        1_{\rm{C}},\ &\sigma \notin \O.
    \end{cases}
    \end{equation}
    \nick{$\Pio$ should be different, this here is not right. If we need to project every $\E \in \O$, we may need a different condition on $\Pio$ than $\Pio(\E_1) \circ_{\R'} \Pio(\E_2) = \Pio(\E_1 \circ_{\R} \E_2)$.}
    The ordered pair $(\Pis, \Pio)$ acts as the desired projection.
\end{proof}

An important example of a fragment appears in quantum thermodynamics. 
Consider a projection $\Pi$, such that $\Pis$ dephases all states in the energy eigenbasis, while $\Pio$ maps all free operations to themselves.
Then $\Pi$ describes the theory in the absence of coherences and the new pre-order is simply thermo-majorization, which in fact is fully solvable in the form of entropic conditions~\cite{cit:gour}. \nick{expand/rephrase}

Existing magic theories can be thought of as fragments of $\Rmax$.
\begin{proposition}
    Every theory of magic $\R$ is a fragment of the maximal theory $\Rmax$.
\end{proposition}
\begin{proof}
    \nick{We need a state projection that establishes that some free states are not free any more $+$ an operations projection.}
\end{proof}

\ddd{[Is there a way to view Nielsen's theorem as describing a projection of the full mixed state LOCC entanglement theory? Is really not essential but might be nice to include. Send $\rho_{AB} $ to the most entangled pure state in its eigendecomposition?]}
%%%%%%%%%%%%%%%%%%%%%%%%%%%%%%%%%%%%%%%%

\section{Distillation bounds}
\label{sec:distill}

Consider a general magic state distillation process,
\begin{equation}\label{eq:distprocess}
    \rho^{\otimes k} \xrightarrow{\E \in \O} \tau,
\end{equation}
where $n$ noisy copies of magic state $\rho$ are converted to a single-copy magic state $\tau$.
Identifying a symmetry of the distillation process responsible for leaving a state $\sigma$ invariant is equivalent with restricting the process to the $\sigma$--fragment $\O_\sigma$.

Mana is monotonic and additive in all $\sigma$--fragments as seen in~\cref{app:major}, so it provides a bound for distillation processes
\begin{equation}
    \mana{\rho} \geq \frac{1}{k} \mana{\tau}.
\end{equation}.

A new bound can be obtained in any $\sigma$--fragment by comparing the Lorenz curves of the initial and target states,
\begin{equation}\label{eq:majbound}
    L_k(\rho^{\otimes k} | \sigma) \geq L_k (\tau | \sigma),\ k=1,\dots,d^2,
\end{equation}
If the Lorenz curve of the initial state is below the target curve at any point, the process is not possible.
In general, the Wigner components of a $k$-copy state $\rho^{\otimes k}$ are calculated, along with their multiplicities, by expanding the terms in the multinomial expansion $\left( \sum_{\bm{z} \in \cal{P}_d} W_\rho(\bm{z}) \right)^n$.
This follows from the multiplicativity of the Wigner distribution.

The Strange state $\ketbra{\rm{S}}$ depicted in~\cref{fig:strange} is the simplest to analyse, since it only has two distinct components $\{ -\frac{1}{3}, \frac{1}{6} \}$, the latter with a multiplicity of 8.
Calculating the binomial expansion for the components of $\ket{S}\bra{S}^{\otimes k}$ gives $\{(-1)^j 2^{j-k} 3^{-k}\}_{0 \leq j \leq k}$ with multiplicity $8^{k-j} \binom{k}{j}$ for the $j$-th term.
This allows analytical calculation of all Lorenz curve points, hence the maximum of the $k$-copy state is
\begin{equation}\label{eq:Skmax}
    \max_k{L_k \left( \ketbra{S}^{\otimes k} \bigm| \sigma \right)} = 1 + \left(\frac{4}{3}\right)^k \sum\limits_{j: 1 \leq 2j+1 \leq k} 4^{-(2j+1)} \binom{k}{2j+1}.
\end{equation}

Consider the noisy Strange state,
\begin{equation}
    \rho_{\rm{S}}(\epsilon) = (1 - \epsilon) \ket{\rm{S}}\bra{\rm{S}} + \epsilon \sigma,
\end{equation}
in the $\sigma$--fragment $\O_\sigma$.
At noise level $\epsilon \leq \frac{3}{4}$, the Wigner distribution $\W{\rho_{\rm{S}}(\epsilon)}$ contains negativities and the state can be purified so as to obtain a single-copy state with sufficiently low $\epsilon$.
In~\cref{fig:distill}, we examine the purifying process 
\begin{equation}\label{eq:purify}
    \rho_{\rm{S}}^{\otimes k}(\epsilon_{\rm{th}}) \xrightarrow{\E \in \O_\sigma} \rho_{\rm{S}}(0.05),\ \sigma = (1-p)\ketbra{0} + p \frac{1}{3}\id
\end{equation}
with $\epsilon_{\rm{th}}$ being the noise level threshold that does not prohibit the process for given number of copies $k$ and $\sigma$--fragment, parametrised by $p$ as a mixture of the zero and the maximally mixed states.
\begin{figure}
    \centering
    \includegraphics[scale=0.5]{figs/ratios.pdf}
    \caption{Plot of the noise level threshold ratios between mana and Lorenz curve for the Strange state purifying process in~\cref{eq:purify}.
    The ratios are calculated for different numbers of initial state copies and different $\sigma$--fragments parametrised by $p$ such that $\sigma = (1-p)\ketbra{0} + p \frac{1}{3}\id$.
    Lorenz curve comparison consistently gives stricter bounds as proven in~\cref{thm:bounds}.
    \nick{Highlight physics/invariance}
    }
    \label{fig:distill}
\end{figure}

Thresholds provided by Lorenz curve comparison are always much stricter than mana thresholds \nick{threshold/bound? need to define the notion of a bound precisely}.
In fact, it is clear than this is the case in any general distillation process.
\begin{theorem}\label{thm:bounds}
    Consider the distillation process in~\cref{eq:distprocess}.
    In any $\sigma$--fragment, $\W{\sigma}$-majorization provides a stricter bound than mana.
\end{theorem}
\begin{proof}
    The maximum of the Lorenz curve of a state $\rho$ can be expressed monotonically in terms of mana, independently of the $\sigma$--fragment,
    \begin{equation}
        \max_k{L_k(\rho | \sigma)} = 1 + \sum_{\bmz: \W[\bmz]{\rho}<0} \abs{\W[\bmz]{\rho}} = \frac{1}{2} \left( 1 + e^\mana{\rho} \right).
    \end{equation}
    Therefore, the majorization condition stated in~\cref{eq:majbound} implies that $\mana{\rho^{\otimes k}} \geq \mana{\tau}$. \ddd{[This is a bit too short]}
\end{proof}

%%%%%%%%%%%%%%%%%%%%%%%%%%%%%%%%%%%%%%%%

\section{Conclusion}
\label{sec:conc}

\begin{enumerate}
    \item Introduced fragments
    \item Identify symmetries of the setup
    \item Combined single-shot thermodynamics with magic 
    \item Can we solve other cases exactly? (apart from single qutrit)
\end{enumerate}

%%%%%%%%%%%%%%%%%%%%%%%%%%%%%%%%%%%%%%%%

\bibliography{bib}
%\bibliographystyle{apsrev4-2}

%%%%%%%%%%%%%%%%%%%%%%%%%%%%%%%%%%%%%%%%

\appendix
\newpage
\section{Properties of Wigner distributions}
\label{app:wigner}

Here, we present basic properties of the phase-point operators and the Wigner distribution that are used throughout the paper.

\begin{proposition}\label{thm:aproperties}
    For any dimension $d$, the phase-point operators satisfy:
    \begin{enumerate}
        \item[(i)]\label{en:a1} Hermiticity and unitarity: $A_{\bmx}^\dagger = A_{\bmx} = A_{\bmx}^{-1}$;
	    \item[(ii)]\label{en:a2} Closure under transposition: $A_{(x, p)}^T = A_{(x, -p)}$;
	    \item[(iii)]\label{en:a3} Unit trace for odd $d$: $\tr[A_{\bmx}] = 1$;
	    \item[(iv)]\label{en:a4} Completeness relation: $\sum_{\bmz \in \cal{P}_d} A_{\bmz} = d\id$;
	    \item[(i)]\label{en:a5} Orthogonality: $\tr[A_{\bmx}^\dagger A_{\bm{x'}}] = d \delta_{\bmx,\bm{x'}}$.
	\end{enumerate}
\end{proposition}
All properties follow from the definition in~\cref{eq:ax} along with properties of the displacement operator $D_{\bmx}$ and can be found in the literature, e.g.~\cite{cit:veitch,Vourdas_2004,cit:gross3}

\begin{proposition}\label{thm:wstate}
  The Wigner distribution of a state $\rho \in \cal{B}(\cal{H}_d)$ is
  \begin{enumerate}
    \item[(i)]\label{en:w1} Real valued: $\W{\rho} \in \mathbb{R}^{d^2}$;
    \item[(ii)]\label{en:w2} Normalised: $\sum_{\bmz \in \cal{P}_d} \W[\bmz]{\rho}=1$;
    \item[(iii)]\label{en:w3} Bounded: $\abs{\W[\bmx]{\rho}} \leq \frac{1}{d}$.
    \item[(iv)]\label{en:w4} Additive under mixing: \vspace{2pt}\\
    $\W[\bmx]{p\rho_1 + (1-p)\rho_2} = p\W[\bmx]{\rho_1} + (1-p)\W[\bmx]{\rho_2}$;
    \item[(v)]\label{en:w5} Multiplicative under tensor products: \vspace{2pt}\\
    $\W[\bmx_A \oplus \bmx_B]{\rho_A \otimes \rho_B} = \W[\bmx_A]{\rho_A}\W[\bmx_B]{\rho_B}$.
	\end{enumerate}
\end{proposition}
\begin{proof}
	Proof of all properties can be found in the literature~\cite{cit:veitch,Vourdas_2004,cit:gross3,Wang_2019} except for property (iii) which we prove here.
	
Let $\{\lambda_i\}_{i \in \mathbb{Z}_d}$ be the (non-negative) eigenvalues of $\rho$, summing to 1.
Let $\{\alpha_{\bmx,i}\}_{i \in \mathbb{Z}_d}$ be the eigenvalues of $A_{\bmx}$. For any $\bmx, \alpha_{\bmx,i} \in \{-1, 1\}$, due to the hermiticity and unitarity of the phase-point operators. 
Then,
\begin{align}
	\abs{W_{\rho}(\bmx)} &= \frac{1}{d}\abs{\tr[A_{\bmx} \rho]} \leq \frac{1}{d} \abs{\sum_i \alpha_{\bmx,i} \lambda_i} \nonumber\\ &\leq \frac{1}{d}\sum_i \lambda_i = \frac{1}{d}.
\end{align}
The first inequality follows from Theorem 1 of~\cite{cit:mirsky} for the trace of complex matrices, while the second is the Cauchy-Schwarz inequality.
\end{proof}

\begin{proposition}
    \label{thm:wchannel}
    The Wigner distribution of a $\cptp$ operation $\E: \cal{B}(\cal{H}_{d_A}) \mapsto \cal{B}(\cal{H}_{d_B})$ is
    \begin{enumerate}
        \item[(i)]\label{en:wo1} Real-valued: $\W{\E} \in \mathbb{R}^{d^2} \times \mathbb{R}^{d^2}$;
        \item[(ii)]\label{en:wo2} Normalised: $\sum_{\bmz \in \cal{P}_{d_B}} \W[\bmz|\bmx]{\E} = 1$ \\ 
        for any $\bmx \in \cal{P}_{d_A}$;
        \item[(iii)]\label{en:wo3} Bounded: $\abs{\W[\bmy|\bmx]{\E}} \leq \frac{d_A}{d_B}$;
	    \item[(iv)]\label{en:wo4} Transitive: $\W[\bmy]{\E(\rho)} = \sum_{\bmz \in \cal{P}_{d_A}} \W[\bmy|\bmz]{\E} \W[\bmz]{\rho}$ for any $\bmy \in \cal{P}_{d_B}$.
    \end{enumerate}
\end{proposition}
If $d_A = d_B$, and in particular if operation $\E$ maps a Hilbert space onto itself, then the stochasticity condition $\abs{\W[\bmy|\bmx]{\E}} \leq 1$ is satisfied.
\begin{proof}
	Proof of all properties are provided by Wang \textit{et al.}~\cite{Wang_2019} except for property (iii) which is a direct consequence of the definition of $\W{\E}$ and the corresponding property (iii) in~\cref{thm:wstate}.
\end{proof}

%%%%%%%%%%%%%%%%%%%%%%%%%%%%%%%%%%%%%%%%

\section{Properties of majorization}
\label{app:major}
	
We now give the following equivalent formulations of $d$--majorization.

\begin{proposition}\label{prop:rmajor}
Given $\bmx, \bmy, \r \in \mathbb{R}^n$, such that the components of $\r$ are positive, the following statements are equivalent:
  \begin{enumerate}
    \item[(i)] $\bmy = A\bmx$ and $\r = A\r$ for a stochastic map $A$;
    \item[(ii)]\label{en:tm3} $\sum\limits_{i=1}^n \abs{x_i - r_i t} \leq \sum\limits_{i=1}^n \abs{y_i - r_i t}$ for all $t \in \mathbb{R}$;
    \item[(iii)] $L_{\bmx|\r}(t) \leq L_{\bmy|\r}(t)$ for $t\in [0,1)$ and \vspace{5pt}\\ $L_{\bmx|\r}(1) = L_{\bmy|\r}(1)$.
  \end{enumerate}
\end{proposition}
The proofs for these can be found in~\cite{cit:marshall,cit:bhatia,cit:nielsen,cit:lostaglio} and references therein.

The following result is used in the text to relate relative majorization of quasi-distributions to their Lorenz curves.
\begin{proposition}\label{lemma:Lorenz_linearity}
	Let $\p$ be a quasi-probability distribution and let $\r$ be a probability distribution with strictly non-zero components. Let $a > 0$ and $b \in \mathbb{R}$ then $L_{a\p + b \r | \r} (x) = a L_{\p |\r}(x) + b x$.
\end{proposition}
\begin{proof} 
	The Lorenz curve of $a\p + b \r$ relative to $\r$ passes through $(0,0)$ and the points $(\sum_{i=1}^k{r_{\pi(i)}}, \sum_{i=1}^k(a \p + b \r)_{\pi(i)})$ where $\pi$ is the permutation that puts $(a p_i/r_i + b)$ in non-increasing order. Since $a > 0$, the permutation $\pi$ puts  $(p_i/r_i)$ in non-increasing order too. We thus have
\begin{align*}
&\left( \sum_{i=1}^kr_{\pi(i)}, \sum_{i=1}^k(a \p + b \r)_{\pi(i)} \right) = \\ 
&\left( \sum_{i=1}^k r_{\pi(i)},a \sum_{i=1}^k  p_{\pi(i)} + b\sum_{i=1}^k r_{\pi(i)} \right) \nonumber,
\end{align*}
therefore the value of the Lorenz function at each potential elbow point $x_k = \sum_{i=1} ^kr_{\pi(i)}$ is given by
\begin{align}
&L_{a \p +b \r|\r} (x_k) = a L_{\p|\r} (x_k) + b L_{\r|\r}(x_k) = \nonumber\\
&a L_{\p|\r} (x_k) + b x_k,
\end{align}
so we have $L_{a\p  + b\r|\r} (x) = a L_{\p |\r}(x) + b x$ for any $x \in [0,1]$ due to linearity.
\end{proof}

\begin{theorem*}
	Given a magic state $\rho$, the maximum $L_\star$ of its Lorenz curve $\lc{\rho}{\sigma}(x)$ is independent of the $\sigma$--fragment and equal to $1+\sn{\rho}$. Moreover, the majorization constraint is stronger than mana in every fragment.
\end{theorem*}
\begin{proof}
	We denote the Wigner distributions of the states compactly as vectors $\bmw(\rho) \equiv W_\rho(x)$ and $\bmw(\sigma) \equiv W_\sigma(x)$.
	We choose the component indexing so that the rescaled distribution 
	\begin{equation}
		\widetilde{\bmw}(\rho|\sigma) \coloneqq \left(\frac{w(\rho)_1}{w(\sigma)_1}, \dots, \frac{w(\rho)_{d^2}}{w(\sigma)_{d^2}} \right)^T,
	\end{equation}
	is sorted, $\widetilde{\bmw} = \widetilde{\bmw}^\downarrow$.
	Note that all components of $\bmw(\sigma)$ are positive, so $\widetilde{w}_i \geq 0$ if and only if $w(\rho)_i \geq 0$ for any $i=1,\dots,d^2$.
	
	Let $i_\star$ be the index of the smallest non-negative component of $\widetilde{\bmw}^\downarrow$.
	Then, $w(\rho)_i < 0$ if and only if $i > i_\star$, so the maximum of Lorenz curve $\lc{\rho}{\sigma}(x)$ takes the value 
	\begin{equation}
		\lc{\rho}{\sigma}(x_{i_\star}) = \sum_{i=1}^{i_\star} w(\rho)_i,
	\end{equation}
	and is achieved at
	\begin{equation}\label{eq:maxloc}
		x_{i_\star} \coloneqq \sum_{i=1}^{i_\star} w(\sigma)_i.
	\end{equation}

	The location of the maximum ($x=x_{i_\star}$) varies from fragment to fragment, but its value is independent of $\sigma$,
	\begin{align}
	L_\star &:=	\lc{\rho}{\sigma}(x_{i_\star}) 
		= \sum\limits_{\bmx: \W[\bmx]{\rho} \geq 0} \W[\bmx]{\rho} \nonumber \\
		&= 1 + \sn{\rho}.
	\end{align}
	
Since the magic monotone mana is a monotonic function of sum-negativity, $\rm{mana}(\rho) \coloneqq \ln{(2\hspace{1pt}\sn{\rho}+1)}$, we see that mana corresponds precisely to the peak of the Lorenz curve $L_{\rho|\sigma}(x)$. Therefore, mana is one of $d^{2n}$ constraints, so majorization is strictly a stronger constraint in any fragment.
\end{proof}



%%%%%%%%%%%%%%%%%%%%%%%%%%%%%%%%%%%%%%%%

\section{Technical properties of $\sigma$--fragments}
\label{app:frag}

In this section, we discuss some technical aspects of general $\sigma$--fragments.

We first prove a result on the independence of the Lorenz curve constraints, stated in~\cref{sec:major_frag}.
\begin{proposition}\label{thm:elbows}
	Let $\rho, \tau$ be two quantum states with Lorenz curves $\lc{\rho}{\sigma}(x), \lc{\tau}{\sigma}(x)$ in the $\sigma$--fragment.
	
	Let $t$ be the number of elbows of $\lc{\tau}{\sigma}(x)$ at locations $x_1, \dots, x_t$.
	
	Then, $\lc{\rho}{\sigma}(x) \geq \lc{\tau}{\sigma}(x)$ for all $x \in [0,1]$ iff $\lc{\rho}{\sigma}(x_{i}) \geq \lc{\tau}{\sigma}(x_{i})$ for all $i =1,\dots,t$.
\end{proposition}
\begin{proof}	
	$\lc{\rho}{\sigma}(x) \geq \lc{\tau}{\sigma}(x)$ for all $x \in [0,1]$ trivially implies $\lc{\rho}{\sigma}(x_{i}) \geq \lc{\tau}{\sigma}(x_{i})$ for all $i = 1,\dots,n'$.
	
	Conversely, assume that $\lc{\rho}{\sigma}(x_{i}) \geq \lc{\tau}{\sigma}(x_{i})$ for all $i = 1,\dots,r$.
	First, let $x_0 = 0$ and $x_{n'+1} = 1$, so that $\lc{\rho}{\sigma}(x_0) = \lc{\tau}{\sigma}(x_0) = 0$ and $\lc{\rho}{\sigma}(x_{n'+1}) = \lc{\tau}{\sigma}(x_{n'+1}) = 1$.
	Hence, we can extend the set of elbows $E$ to $E' = E \cup \{x_0, x_{n'+1}\}$.
	
	Pick two consecutive locations $x_{i}, x_{i+1}$ in $E'$ and consider the line segment $\ell_\tau(x)$ connecting points $(x_{i}, \lc{\tau}{\sigma}(x_{i}))$ and $(x_{i+1}, \lc{\tau}{\sigma}(x_{i+1}))$ as well as the line segment $\ell_\rho(x)$ connecting points $(x_{i}, \lc{\rho}{\sigma}(x_{i}))$ and $(x_{i+1}, \lc{\rho}{\sigma}(x_{i+1}))$.
	This is illustrated in~\cref{fig:elbows_proof}.
\begin{figure}[h]
    \centering
    \includegraphics[scale=0.5]{figs/elbows_proof.pdf}
    \caption{\textbf{Illustration of~\cref{thm:elbows}}.
    }
    \label{fig:elbows_proof}
\end{figure}

	Due to concavity of $\lc{\rho}{\sigma}$, it is clear that for all $x \in [x_{i}, x_{i+1}]$, we have $\lc{\rho}{\sigma}(x) \geq \ell_\rho(x) \geq \ell_\tau(x) = \lc{\tau}{\sigma}(x)$.
	This argument can be made in all intervals $[x_{i}, x_{i+1}]$ with $i=0,\dots,n'$, so the proof is complete.
\end{proof}
The above theorem can be of practical importance in reducing the necessary distillation constraints derived via majorization in $\sigma$--fragments.

\begin{proposition}\label{thm:frag_app}
    Let $\R = (\F, \O)$ be a magic theory and $\sigma, \sigma' \in \F$. The following statements hold:
    \begin{enumerate}
        \item No $\sigma$--fragment of $\R$ is empty.
        \item If a free operation leaves two states invariant, then it also leaves their mixtures invariant, 
        \begin{equation*}
            \O_{\sigma} \cap \O_{\sigma'} \subseteq \O_{p\sigma + (1-p)\sigma'}\ \text{for any}\ p \in [0,1].
        \end{equation*}
    \end{enumerate}
\end{proposition}
\begin{proof}$ $\vspace{-12pt}\\

\begin{enumerate}
    \item The identity channel $1_{\rm{C}}$ belongs to every $\sigma$--fragment, as $1_{\rm{C}} \in \O$ and $1_{\rm{C}}\sigma = \sigma$ for all $\sigma \in \F$.
    
    \item Let $\E \in \O_{\sigma} \cap \O_{\sigma'}$.
    Then $\E \in \cptp$ and corresponds to stochastic Wigner distribution $\W{\E}$ such that $\W{\E} \W{\sigma} = \W{\sigma}$ and $\W{\E} \W{\sigma'} = \W{\sigma'}$.
    Then, $\W{\E} \W{p\sigma + (1-p)\sigma'} = \W{p\sigma + (1-p)\sigma'}$ for any $p \in [0,1]$ due to the additive property~\ref{en:w4} of the Wigner distribution, implying that state $p\sigma + (1-p)\sigma'$ is also left invariant by $\E$.
\end{enumerate}
\vspace{-20pt}
\end{proof}

Any free state $\sigma \in \F$ corresponds to a $d^2$--dimensional probability distribution $\W{\sigma}$ and any free operation $\E \in \O$ corresponds to a $d^2 \times d^2$ stochastic matrix (or conditional probability distribution) $\W{\E}$.
Note that these mappings are one-to-one due to the orthogonality of the phase-point operators as an operator basis.
Note further that free states $\F$ are mapped onto a \emph{strict subset} of the set of probability distributions.
As a counterexample, the sharp $d^2$--dimensional probability distribution $(1, 0, \dots, 0)$ does not correspond to any qudit Wigner distribution because of the boundedness condition in~\cref{thm:wstate}.
Similarly, not all stochastic matrices correspond to completely positive operations.
As an example, consider the permutation matrix
\begin{equation}
    \Pi_X = \begin{psmallmatrix}
        0 & 1 & 0 & 0 & 0 \\
        0 & 0 & 0 & 0 & 1 \\
        0 & 0 & 0 & 1 & 0 \\
        1 & 0 & 0 & 0 & 0 \\
        0 & 0 & 1 & 0 & 0
    \end{psmallmatrix} \otimes \begin{psmallmatrix}
        0 & 0 & 1 & 0 & 0 \\
        0 & 0 & 0 & 0 & 1 \\
        0 & 0 & 0 & 1 & 0 \\
        1 & 0 & 0 & 0 & 0 \\
        0 & 1 & 0 & 0 & 0    
    \end{psmallmatrix} \in {\rm{S}}_5({\W{\frac{1}{5}\id}}).
\end{equation}
It preserves the uniform distribution $\W{\frac{1}{5}\id}$, but it does not correspond to any positive (hence quantum) operation.

%%%%%%%%%%%%%%%%%%%%%%%%%%%%%%%%%%%%%%%%

\section{Lorenz curves in the unital fragment}
\label{app:lcsu_technical}

\subsection{Binomial distributions and error bounds}\label{app:phi}
Consider an experiment consisting of $n$ trials of throwing a $p$--coin, that is a coin with probability $p$ of landing on one side and $1-p$ of landing on the other.
We express the sums over an even number $m$ of successful trials $\Phi_+$ and an odd number $m$ of successful trials $\Phi_-$,
\begin{align}	
	\Phi_+(m; n, p) &\coloneqq \sum\limits_{\ell=0}^{m/2} \binom{n}{2\ell} p^{2\ell} (1-p)^{n-2\ell}, \nonumber\\ 
	&\text{for even integers } m\in[0,n], \label{eq:fp_app} \\
	\Phi_-(m; n, p) &\coloneqq \sum\limits_{\ell=1}^{(m-1)/2} \binom{n}{2\ell+1} p^{2\ell+1} (1-p)^{n-(2\ell+1)}, \nonumber\\ 
	&\text{for odd integers }m\in[0,n]. \label{eq:fn_app}
\end{align}
Note that index $m$ only takes even (odd) values when labelling $\Phi_+$ ($\Phi_-$).
In~\cref{app:lcsu_coord}, we will use $\Phi_+$ and $\Phi_-$ to express the elbow coordinates of Lorenz curves in the unital fragment.

We also define the classical entropy of a $p$--coin and the classical relative entropy between a $p$--coin and a $q$--coin,
\begin{align}
	S(p) &\coloneqq -p\log{p} -(1-p)\log{(1-p)}, \label{eq:ent}\\
	\ent{p}{q} &\coloneqq p \log{\frac{p}{q}} + (1-p) \log{\frac{1-p}{1-q}}. \label{eq:ent_rel}
\end{align}
They are symmetric in the sense that $S(p) = S(1-p)$ and $\ent{p}{q} = \ent{1-p}{1-q}$.

A useful result is the entropic bound on a combination~\cite{cit:ash}.
\begin{lemma}\label{lem:comb_bounds}
	For all $\ell\in [1,n-1]$,
	\begin{align}
		&\left[ 8\ell\left(1-\frac{\ell}{n}\right) \right]^{-\frac{1}{2}} 2^{n S\left(\frac{\ell}{n}\right)} \leq \binom{n}{\ell} \leq \\
		&\left[ 2\pi \ell\left(1-\frac{\ell}{n}\right) \right]^{-\frac{1}{2}} 2^{n S\left(\frac{\ell}{n}\right)}.
	\end{align}
\end{lemma}
The proof provided in~\cite{cit:ash} proceeds with direct calculation for the edge cases $\ell = 1,2, n-1, n-2$ and use Stirling's approximation for the remaining cases.
We can use~\cref{lem:comb_bounds} to provide strict upper and lower bounds on the functions $\Phi_+, \Phi_-$.

\subsection{Theory on bounding the core functions}
\nick{Should we really keep this section? It is standard calculus and the bounds are really loose.}
Here we present a more manageable method of bounding the core functions $\Phi_{\pm}$, which however results in looser bounds. 
We can rewrite the functions as
\begin{equation}
	\Phi_{\pm}(m; n, a) = \frac{1}{2}(\Phi(m; n, a) \pm (1+a)^{-n} S(m; n, a)),
\end{equation}
where we have substituted $a = p/(1-p)$. 
$\Phi$ is the standard cumulative function
\begin{equation}
	\Phi(m; n, a) = (1+a)^{-n} \sum_{k=0}^m \binom{n}{k} a^k,
\end{equation}
and the remainder term is
\begin{equation}
	S(m; n, a) \coloneqq \sum_{k=0}^m \binom{n}{k} (-a)^k.
\end{equation}
We have the following asymptotic bounds on the behaviour of $\Phi$~\cite{cit:ash},
\begin{lemma}\label{lem:phi_bounds}
	Given fixed $n>0$ and $p$, $\Phi$ satisfies the following bounds:
	\begin{align*}
		\begin{split}
		&\text{1. } \Phi(m; n, p) \geq \left[ 8m\left(1-\frac{m}{n}\right) \right]^{-\frac{1}{2}} 2^{-n\ent{\frac{m}{n}}{p}}, \\
		&\hspace{14pt} m\in [1,n-1] \\
		&\text{2. } \Phi(m; n, p) \geq 1 - 2^{-n\ent{\frac{m+1}{n}}{p}},\ m\in [np+1,n-2] \\
		&\text{3. } \Phi(m; n, p) \leq 1 - \left[ 8(m+1)\left(1-\frac{m+1}{n}\right) \right]^{-\frac{1}{2}}\times \\
		&\hspace{14pt} 2^{-n\ent{\frac{m+1}{n}}{p}},\ m\in [0,n-2]
		\end{split}
		\\
		&\text{4. } \Phi(m; n, p) \leq 2^{-n\ent{\frac{m}{n}}{p}},\ m\in [0,np]
	\end{align*}
\end{lemma}

We would like some theory that estimates the value of $S(m; n, a)$ for different parameter regimes. 
We can consider the function $f(x) = (1+x)^n$ and note that $S(m; n, a)$ is the $m$'th partial sum of this expansion at the point $x=-a$.

The truncated Maclaurin series of a general function $f(x)$ is
\begin{equation}
	f(x) = f(0) + x f'(0) + \dots \frac{x^m}{m!}f^{(m)}(0) + R_m(x)
\end{equation}
with a remainder term
\begin{align}
	R_m (x)&= \int_{0}^x dt f^{(m+1)}(t) \frac{(x-t)^m}{m!} \\
	&= \frac{x^{m+1}}{(m+1)!} f^{(m+1)}(x_*),
\end{align}
where in the second expression, $x_*$ is an implicit point that lies between $0$ and $x$ that comes from the Mean Value Theorem.

Applying this to the function $f(x) = (1+x)^n$ gives
\begin{equation}
	(1+x)^n = \sum_{k=0}^m \binom{n}{k} x^k + R_m.
\end{equation}
Evaluating at $x=-a$ gives
\begin{equation}
	S(m; n, a) = (1-a)^n - R_m(-a),
\end{equation}
where the key remainder term is given by
\begin{align}
	R_m(-a) &= \int_0^{-a} dt f^{(m+1)}(t) \frac{(-a-t)^m}{m!} \\
&= \frac{(-a)^{m+1}}{(m+1)!} f^{(m+1)}(x_*).
\end{align}
We can also compute the derivative $f^{(m+1)}(x)$ explicitly,
\begin{equation}
	f^{(m+1)}(x) = (m+1)!\binom{n}{m+1}(1+x)^{n-m-1}.
\end{equation}
Therefore, we have that
\begin{align}
	R_m(-a) &= (-1)^{m+1}(m+1)\binom{n}{m+1}\times \nonumber\\
	&\hspace{12pt} \int_{-a}^0 dt (1+t)^{n-m-1}(a+t)^m \\
&= \binom{n}{m+1}(-a)^{m+1}(1+x_*)^{n-m-1},
\end{align}
where in the latter expression $x_* \in [-a,0]$. 
Note that the first integral expression can be estimated via the Cauchy-Schwarz or the H{\"o}lder inequality. 
Therefore, we can either work with an explicit form with an unknown (but bounded) parameter $x_*$, or we can use the integral form and provide concrete estimates on it value.

A very simple estimate, based on $x_*$ lying in the interval $[-a,0]$ gives
\begin{equation}
(1-a)^{n-m-1} \leq \frac{R_m(-a)}{\binom{n}{m+1}(-a)^{m+1}} \leq 1,
\end{equation}
which in turn leads to the following bounds on $\Phi_+(m; n, a)$:
\begin{align}
	\hspace{-2cm}2\Phi_+(m; n, a) \leq\ &\Phi(m; n, a) + (1-a)^n - \frac{(-a)^{m+1}}{m!} \times \nonumber\\
	 &\binom{n}{m+1} (1-a)^{n-m-1} \text{ and} \\
	2\Phi_+(m; n, a) \geq\ &\Phi(m; n, a) + (1-a)^n - \frac{(-a)^{m+1}}{m!} \binom{n}{m+1} .
\end{align}

\subsection{Lorenz curve coordinates in the unital fragment}\label{app:lcsu_coord}
The Wigner distribution of the $n$--copy qutrit maximally mixed state $\left(\id/3\right)^{\otimes n}$ is the uniform probability distribution over the phase space, consisting of $9^n$ components equal to $9^{-n}$.
The Wigner distribution of the 1-copy $\epsilon$--noisy Strange state $\rho_{\rm{S}}(\epsilon)$ in the unital fragment consists of some permutation of a single negative component
\begin{equation}
	- v(\epsilon) \coloneqq - \left( \frac{1}{3} -\frac{4}{9}\epsilon \right),
\end{equation} 
and $8$ positive components
\begin{equation}
	u(\epsilon) \coloneqq \frac{1}{6} -\frac{1}{18}\epsilon.
\end{equation}
where in the unital fragment we need the condition $0 \leq \epsilon < 3/4$, so that the state contains some Wigner negativity ($-v < 0$).
It is also clear that $v \geq u$ in the interval $0 \leq \epsilon \leq 3/7$, while $u > v$ in the interval $3/7 < \epsilon < 3/4$.

The Wigner distribution of the $n$--copy $\epsilon$--noisy Strange state $\rho_{\rm{S}}(\epsilon)^{\otimes n}$ in the unital fragment is given by the convolution $\W{\rho_{\rm{S}}(\epsilon)^{\otimes n}} = W_{\rho_{\rm{S}}(\epsilon)}^{\otimes n}$.
In general, $\rho_{\rm{S}}(\epsilon)^{\otimes n}$ contains $n + 1$ distinct components, labelled $0,\dots, n$.
We present the distinct Wigner components of $\rho_{\rm{S}}(\epsilon)^{\otimes n}$ along with their multiplicites in~\cref{tab:lcsu}.
Note that LHS (RHS) refers to elbow coordinates $i$ on the left of and including (right of) the Lorenz curve maximum, stated precisely as
\begin{align}
&\text{LHS: } 0 \leq i \leq \left\lfloor \frac{n}{2} \right\rfloor \text{ and} \\
&\text{RHS: } \left\lfloor \frac{n}{2} \right\rfloor +1 \leq i \leq n.
\end{align}
\begin{table}[h]
  \def\arraystretch{1.5}
  \centering
  \begin{tabular}{c|c|c|r|r}
    \multicolumn{3}{c|}{Case} & \multicolumn{1}{c}{$m_{i}(n, \epsilon)$} & \multicolumn{1}{|c}{$w_{i}(n, \epsilon)$} \\[0.5ex]\hline
    \multirow{4}{*}{\raisebox{-4ex}{\rotatebox[origin=c]{90}{$0\leq \epsilon < \frac{3}{7}$}}} & \hspace{0.8ex}\multirow{2}{*}{\raisebox{-1ex}{\rotatebox[origin=c]{90}{$n$ even}}}\hspace{0.8ex} & LHS & $8^{2i}\binom{n}{2i}$ & $\left( \frac{1}{6} - \frac{1}{18}\epsilon \right)^{2i}\left( -\frac{1}{3} + \frac{4}{9}\epsilon \right)^{n-2i}$ \\
    & & RHS & $8^{n-2i}\binom{n}{2i}$ & $\left( \frac{1}{6} - \frac{1}{18}\epsilon \right)^{n-2i}\left( -\frac{1}{3} + \frac{4}{9}\epsilon \right)^{2i}$ \\ \cline{2-5}
    & \multirow{2}{*}{\raisebox{-2ex}{\rotatebox[origin=c]{90}{$n$ odd}}} & LHS & $8^{2i+1}\binom{n}{2i+1}$ & $\left( \frac{1}{6} - \frac{1}{18}\epsilon \right)^{2i+1}\left( -\frac{1}{3} + \frac{4}{9}\epsilon \right)^{n-2i-1}$ \\
    & & RHS & $8^{n-2i-1}\binom{n}{2i+1}$ & $\left( \frac{1}{6} - \frac{1}{18}\epsilon \right)^{n-2i-1}\left( -\frac{1}{3} + \frac{4}{9}\epsilon \right)^{2i+1}$ \\ \hline
    \multirow{4}{*}{\raisebox{-4ex}{\rotatebox[origin=c]{90}{$\frac{3}{7}\leq \epsilon < \frac{3}{4}$}}} & \multirow{2}{*}{\raisebox{-1ex}{\rotatebox[origin=c]{90}{$n$ even}}} & LHS & $8^{n-2i}\binom{n}{2i}$ & $\left( \frac{1}{6} - \frac{1}{18}\epsilon \right)^{n-2i}\left( -\frac{1}{3} + \frac{4}{9}\epsilon \right)^{2i}$ \\
    & & RHS & $8^{2i}\binom{n}{2i}$ & $\left( \frac{1}{6} - \frac{1}{18}\epsilon \right)^{2i}\left( -\frac{1}{3} + \frac{4}{9}\epsilon \right)^{n-2i}$ \\ \cline{2-5}
    & \multirow{2}{*}{\raisebox{-2ex}{\rotatebox[origin=c]{90}{$n$ odd}}} & LHS & $8^{n-2i}\binom{n}{2i}$ & $\left( \frac{1}{6} - \frac{1}{18}\epsilon \right)^{n-2i}\left( -\frac{1}{3} + \frac{4}{9}\epsilon \right)^{2i}$ \\
    & & RHS & $8^{2i}\binom{n}{2i}$ & $\left( \frac{1}{6} - \frac{1}{18}\epsilon \right)^{2i}\left( -\frac{1}{3} + \frac{4}{9}\epsilon \right)^{n-2i}$ \\ \hline
  \end{tabular}
  \caption{Wigner components $w_{i}(n, \epsilon)$ of $\rho_{\rm{S}}(\epsilon)^{\otimes n}$ along with their multiplicities $m_{i}(n, \epsilon)$, with $0 \leq i \leq n$.
  The expressions change depending on the noise level $\epsilon$, the parity of the number of copies $n$ and whether the index $i$ is lower or higher than the index of the Lorenz curve maximum (LHS vs RHS).
  Multiplication $2i$ is considered modulo $(n+1)$.}
  \label{tab:lcsu}
\end{table}

Every Lorenz curve in the unital fragment contains $n$ elbows, which, along with the boundary points $(x_{-1}, L_{-1}) = (0,0)$ and $(x_{n}, L_{n}) = (1,1)$, are labelled by 
\begin{equation*}
\{(x_{i}, L_{i})\}_{i=-1,0,\dots,n}.
\end{equation*}
The maximum is the $\lfloor n/2 \rfloor$-th elbow and its coordinates are calculated by collecting all the positive Wigner components,
\begin{align}
	x_{\lfloor n/2 \rfloor} &= \frac{1}{2}\left(1 + \left(\frac{7}{9}\right)^n\right), \\
	L_{\lfloor n/2 \rfloor} &= \frac{1}{2}\left (1 + \left(\frac{15 - 8\epsilon}{9}\right)^n \right).
\end{align}
%\sum_{j: even}^n a^j \binom{n}{j} = \frac{1}{2} [ (1+a)^n + (1-a)^n ]

Expressions for all the elbow coordinates follow from summing up the Wigner components in decreasing order.
In~\cref{tab:lcsu_coord_elb_app}, we present the elbow coordinates of the $n$-copy, $\epsilon$--noisy Strange state Lorenz curve in the unital fragment for any combination of parameters $n, \epsilon$.
\begin{table}[h]
  \def\arraystretch{1.5}
  \centering
  \begin{tabular}{c|c|c|r|r}
\multicolumn{3}{c|}{\multirow{2}{*}{Case}} & \multicolumn{1}{c|}{$x_{i}$} & \multicolumn{1}{c}{$L_{i}$} \\
    \multicolumn{3}{c|}{} & \multicolumn{1}{c|}{$x_{i} - x_{\lfloor n/2 \rfloor}$} & \multicolumn{1}{c}{$L_{i} - L_{\lfloor n/2 \rfloor}$} \\[0.5ex]\hline 
    \multirow{4}{*}{\raisebox{-5ex}{\rotatebox[origin=c]{90}{$0\leq \epsilon < \frac{3}{7}$}}} & \hspace{0.8ex}\multirow{2}{*}{\raisebox{-3ex}{\rotatebox[origin=c]{90}{$n$ even}}}\hspace{0.8ex} & LHS & $\Phi_+\left(2i;n,\frac{8}{9}\right)$ & $\left( \frac{5}{3} - \frac{8}{9}\epsilon\ \right)^n \Phi_+\left(2i;n,\frac{12-4\epsilon}{15-8\epsilon}\right)$ \\
    & & RHS & $\Phi_-\left(2i;n,\frac{1}{9}\right)$ & $- \left( \frac{5}{3} - \frac{8}{9}\epsilon\ \right)^n\Phi_-\left(2i;n,\frac{3-4\epsilon}{15-8\epsilon}\right)$ \\ \cline{2-5}
    & \multirow{2}{*}{\raisebox{-3ex}{\rotatebox[origin=c]{90}{$n$ odd}}} & LHS & $\Phi_-\left(2i;n,\frac{8}{9}\right)$ & $\left( \frac{5}{3} - \frac{8}{9}\epsilon\ \right)^n \Phi_-\left(2i;n,\frac{12-4\epsilon}{15-8\epsilon}\right)$ \\
    & & RHS & $\Phi_-\left(2i;n,\frac{1}{9}\right)$ & $- \left( \frac{5}{3} - \frac{8}{9}\epsilon\ \right)^n\Phi_-\left(2i;n,\frac{3-4\epsilon}{15-8\epsilon}\right)$ \\ \hline
    \multirow{4}{*}{\raisebox{-5ex}{\rotatebox[origin=c]{90}{$\frac{3}{7}\leq \epsilon < \frac{3}{4}$}}} & \multirow{2}{*}{\raisebox{-3ex}{\rotatebox[origin=c]{90}{$n$ even}}} & LHS & $\Phi_+\left(2i;n,\frac{1}{9}\right)$ & $\left( \frac{5}{3} - \frac{8}{9}\epsilon\ \right)^n \Phi_+\left(2i;n,\frac{3-4\epsilon}{15-8\epsilon}\right)$ \\
    & & RHS & $\Phi_-\left(2i;n,\frac{8}{9}\right)$ & $- \left( \frac{5}{3} - \frac{8}{9}\epsilon\ \right)^n\Phi_-\left(2i;n,\frac{12-4\epsilon}{15-8\epsilon}\right)$ \\ \cline{2-5}
    & \multirow{2}{*}{\raisebox{-3ex}{\rotatebox[origin=c]{90}{$n$ odd}}} & LHS & $\Phi_+\left(2i;n,\frac{1}{9}\right)$ & $\left( \frac{5}{3} - \frac{8}{9}\epsilon\ \right)^n \Phi_+\left(2i;n,\frac{3-4\epsilon}{15-8\epsilon}\right)$ \\
    & & RHS & $\Phi_+\left(2i;n,\frac{8}{9}\right)$ & $- \left( \frac{5}{3} - \frac{8}{9}\epsilon\ \right)^n\Phi_+\left(2i;n,\frac{12-4\epsilon}{15-8\epsilon}\right)$ \\ \hline
  \end{tabular}
  \caption{Lorenz curve elbow coordinates in the unital fragment.
  The coordinate expressions depend on the noise level $\epsilon$, the parity of the number of copies $n$ and the location of the elbow relative to the maximum (LHS vs RHS).
  Multiplication $2i$ is considered modulo $(n+1)$.
  For completeness, note that $(x_{-1}, L_{-1}) \coloneqq (0,0)$ is not included in the table.
  }
  \label{tab:lcsu_coord_elb_app}
\end{table}

We can get explicit expressions for all $9^{n}$ points of the Lorenz curve $\lc{\rho_{\rm{S}}(\epsilon)^{\otimes n}}{(\id/3)^{\otimes n}}$, in terms of the elbow coordinates:
\begin{align}
    x_{ij} &= \left( 1-\frac{j}{m_{i}} \right) x_{i-1} + \frac{j}{m_{i}} x_{i}, \label{eq:x}\\
    L_{ij} &= \left( 1-\frac{j}{m_{i}} \right) L_{i-1} + \frac{j}{m_{i}} L_{i} \label{eq:l}
\end{align}
for $j = 1,\dots,m_{i}$ and $i=0,\dots,n$, where multiplicities $m_i = m_i(n, \epsilon)$ are given in~\cref{tab:lcsu}.

Consider the state 
\begin{equation*}
\rho_{\rm{S}}(\epsilon')^{\otimes n'} \otimes \left( \frac{1}{3}\id \right)^{\otimes (n-n')},
\end{equation*}
where tensoring with the maximally mixed state keeps the Lorenz curve unchanged, but increases the resolution of (the uniformly distributed) points.
The new point coordinates are given by:
\begin{align}
    &x_{ijk} = \left( 1-p_{ijk}\right) x_{i-1} + p_{ijk} x_{i} \label{eq:lcsu_xcoord}\\
    &L_{ijk} = \left( 1-p_{ijk} \right) L_{i-1} + p_{ijk} L_{i}, \label{eq:lcsu_lcoord}\\
    &\text{where } p_{ijk} = \frac{k + (j-1)9^{n-n'}}{9^{n-n'} m_{i}} \nonumber\\
    &\text{for } i=0,\dots,n',\ j = 1,\dots,m_{i}(n', \epsilon') \text{ and } k = 1,\dots,9^{n-n'}. \nonumber
\end{align}

We can unify the indices, by introducing a single index
\begin{equation}
    I(i,j,k) \coloneqq k + \left[ (j-1) + \sum_{\ell=0}^{i-1} m_{\ell}(n', \epsilon') \right]9^{n-n'},
\end{equation}
so that $I=1,2,\dots, 9^{n}$.
The elbow coordinates correspond to 
\begin{equation}
	I(i, m_{i}(n', \epsilon'), 9^{n-n'}) = \sum_{\ell=0}^{i} m_{\ell}(n', \epsilon'),\ i= 0,\dots,n'.
\end{equation}
The index function $I$ is bijective, i.e.
\begin{equation}
	(i,j,k) = (i',j',k') \text{ iff } I(i,j,k) = I(i',j',k').
\end{equation}

%%%%%%%%%%%%%%%%%%%%%%%%%%%%%%%%%%%%%%%%

\section{Technical details for the derivation of distillation bounds from Lorenz curves}
\label{app:lcst_technical}

\subsection{First and last elbow constraints}
\label{app:elb_constraints}
Here we prove two simple majorization constraints, one arising by considering only the ascending part of the Lorenz curves between the origin $(0,0)$ and the first elbow and the other by considering only the descending part of the curves between the last elbow and the endpoint $(1,1)$.
\begin{proposition}\label{prop:first_elb}
	Consider a magic state process $\rho \longrightarrow \tau$ with input and output Lorenz curves $\lc{\rho}{\sigma}(x), \lc{\tau}{\sigma}(x)$ in $\R_\sigma$ and denote by $X_0, X'_0$ the first elbow locations of the input and output curves respectively.
	
	Then, given any coordinates $(x_0, L_0)$ and $(x'_0, L'_0)$ on the input and output Lorenz curves, where $x_0 \leq X_0$ and $x'_0 \leq X'_0$, the process is possible in $\R_\sigma$ only if
\begin{equation}\label{eq:first_elb_bound1}
	\frac{L_0}{x_0} \geq \frac{L_0'}{x_0'}.
\end{equation}
\end{proposition}
\begin{proof}
Since both pairs of coordinates are located between $(0,0)$ and the first elbow of their respective curves, we can derive the bound via a simple interpolation on the line segment connecting the origin and the appropriate first elbow.

First assume that $x_0 < x'_0$ and consider the Lorenz curve constraint at $x = x_0$,
\begin{equation}
	\lc{\rho}{\sigma}(x_0) \geq \lc{\tau}{\sigma}(x_0).
\end{equation}
We can find the output state Lorenz curve coordinate $L'_\star$ at location $x = x_0$ by interpolating between the origin and the output state's first elbow, 
\begin{equation}
	L'_\star = \frac{x_0}{x'_0}L'_0.
\end{equation}
The process is then possible only if $L_0 \geq L_\star'$ which is a rearrangement of~\cref{eq:first_elb_bound1}.

If instead, $x_0 \geq x'_0$, consider the Lorenz curve constraint at $x = x'_0$,
\begin{equation}
	\lc{\rho}{\sigma}(x'_0) \geq \lc{\tau}{\sigma}(x'_0).
\end{equation}
We now need to find the input state Lorenz curve coordinate $L_\star$ at location $x = x'_0$ by interpolating between the origin and the input state's first elbow, 
\begin{equation}
	L_\star = \frac{x'_0}{x_0}L_0.
\end{equation}
The process is then possible only if $L_\star \geq L'_0$ which is again a rearrangement of~\cref{eq:first_elb_bound1}.
\end{proof}

\begin{proposition}\label{prop:last_elb}
	Consider a magic state process $\rho \longrightarrow \tau$ with input and output Lorenz curves $\lc{\rho}{\sigma}(x), \lc{\tau}{\sigma}(x)$ in $\R_\sigma$ and denote by $X_E, X'_E$ the last elbow locations of the input and output curves respectively.
	
	Then, given any coordinates $(x_E, L_E)$ and $(x'_E, L'_E)$ on the input and output Lorenz curves, where $x_E \geq X_E$ and $x'_E \geq X'_E$, the process is possible in $\R_\sigma$ only if
\begin{equation}\label{eq:last_elb_bound1}
	\frac{L_E - 1}{1-x_E} \geq \frac{L'_E - 1}{1-x_E'}.
\end{equation}
\end{proposition}
\begin{proof}
Since both pairs of coordinates are located between the last elbow of their respective curves and $(1,1)$, we can derive the bound via a simple interpolation on the line segment connecting the endpoint and the appropriate last elbow.

First assume that $x_E > x'_E$ and consider the Lorenz curve constraint at $x = x_E$,
\begin{equation}
	\lc{\rho}{\sigma}(x_E) \geq \lc{\tau}{\sigma}(x_E).
\end{equation} 
We can find the output state Lorenz curve coordinate $L'_\star$ at location $x = x_E$ by interpolating between the endpoint $(1,1)$ and the output state's last elbow, 
\begin{equation}
	L'_\star = 1 + \frac{1-x_E}{1-x'_E} (L'_E - 1)
\end{equation}
The process is then possible only if $L_E \geq L_\star'$ which is a rearrangement of~\cref{eq:last_elb_bound1}.

If instead, $x_E \leq x'_E$, consider the Lorenz curve constraint at $x = x'_E$,
\begin{equation}
	\lc{\rho}{\sigma}(x'_E) \geq \lc{\tau}{\sigma}(x'_E).
\end{equation} 
We now need to find the input state Lorenz curve coordinate $L_\star$ at location $x = x'_E$ by interpolating between the endpoint $(1,1)$ and the input state's last elbow, 
\begin{equation}
	L_\star = 1 + \frac{1-x'_E}{1-x_E} (L_E - 1).
\end{equation}
The process is then possible only if $L_\star \geq L'_E$ which is again a rearrangement of~\cref{eq:last_elb_bound1}.
\end{proof}

\subsection{Component-multiplicity pairs}
\label{app:cmpairs}
In general, a $1$--copy $d$--dimensional state $\rho$ is described exactly by its $d^2$--dimensional Wigner distribution $\W{\rho}$. 
The distribution $\W{\rho}$ is usually defined on the phase space, but it can be convenient to define it using vector notation. 
In particular, we introduce a component vector $\bmw(\rho) = (w_i)_{i=1,\dots,D}$ and a multiplicity vector $\bmm(\rho) = (m_i)_{i=1,\dots,D}$, where $D \leq d^2$ which together form a set of component-multiplicity pairs $\{(w_i, m_i)\}_{i=1,\dots,D}$.
\begin{definition}
	Consider a distribution $W$ and a positive integer $D \leq {\rm{dim}}\hspace{1pt}W$. 
	We call the set of ordered pairs $\{(w_i, m_i)\}_{i=1,\dots,D}$ a \emph{complete set of component-multiplicity pairs}, if $W$ contains $m_i$ components $w_i$ and $\sum_{i=0}^D m_i = d^2$.
\end{definition}
Therefore, such a set describes each component of $\W{\rho}$ exactly once.
As an example, two complete sets of pairs for the Strange state are $\{( -1/3, 1), ( 1/6, 8)\}$ and $\{(-1/3, 1), (1/6, 2), (1/6, 3), (1/6, 3)\}$.
The latter corresponds to the phase space split in~\cref{fig:pd_split}

Consider two states $\rho_A, \rho_B$ with Wigner distributions $\W{\rho_A}, \W{\rho_B}$ described respectively by complete sets of component-multiplicity pairs 
\begin{equation}
	\{(w_i, m_i)\}_{i=1,\dots,D_A} \text{ and } \{(w_j', m_j')\}_{j=0,\dots,D_B}.
\end{equation}
The multiplicative property of the Wigner distribution over a composite phase space $\cal{P}_{d_A} \times \cal{P}_{d_B}$ shown in~\cref{thm:wstate},
\begin{equation}
	\W[\bmx_A \oplus \bmx_B]{\rho_A \otimes \rho_B} = \W[\bmx_A]{\rho_A}\W[\bmx_B]{\rho_B},
\end{equation}
implies that the distribution $\W{\rho_A \otimes \rho_B}$ is $d_A^2 d_B^2$--dimensional and contains components of the form $w_i w_j'$. 
Therefore, the set $\{(w_i w_j', m_i m_j')\}$ with $i=1,\dots,D_A$ and $j=1,\dots,D_B$ is a complete set of component-multiplicity pairs for the distribution of the composite system $\W{\rho_A \otimes \rho_B}$.
This is true because all components are of the form $w_i w_j'$ and 
\begin{equation*}
	\sum_{i=1}^{D_A}\sum_{j=1}^{D_B} m_i m_j' = \sum_{i=1}^{D_A} m_i \sum_{j=1}^{D_B} m_j' = d_A^2 d_B^2.
\end{equation*}

Note that the rescaled distribution is also multiplicative,
\begin{align}
	&\widetilde{\rm{W}}_{\rho_A \otimes \rho_B | \gamma_A \otimes \gamma_B}(\bmx_A \oplus \bmx_B) = \frac{\W[\bmx_A \oplus \bmx_B]{\rho_A \otimes \rho_B}}{\W[\bmx_A \oplus \bmx_B]{\gamma_A \otimes \gamma_B}} = \nonumber \\
	&\frac{\W[\bmx_A]{\rho_A}\W[\bmx_B]{\rho_B}}{\W[\bmx_A]{\gamma_A}\W[\bmx_B]{\gamma_B}} = \widetilde{\rm{W}}_{\rho_A | \gamma_A}(\bmx_A)\widetilde{\rm{W}}_{\rho_B  | \gamma_B}(\bmx_B),
\end{align}
so a complete set of component-multiplicity pairs can be obtained for this distribution in the same fashion as for usual Wigner distributions.

Given a state $\rho$ and a complete set of component-multiplicity pairs describing its Wigner distribution $\W{\rho}$, we now provide a method of computing the components (and multiplicities) of the $n$--copy distribution $\W{\rho}^{\otimes n}$.
\begin{lemma}\label{lem:ncopycomponents}
	Let $W$ be a distribution defined by a complete set of component-multiplicity pairs $\{(w_i, m_i)\}_{i=1,\dots,D}$ with $D \leq {\rm{dim}}\hspace{1pt}W$ and consider the distribution $W^{\otimes n}$ obtained by taking the $n$-fold (Kronecker) product $W \otimes \dots \otimes W$ between $n$ copies of $W$.
	
	Denote by $C_D^n \coloneqq \{\bmk\}$ the set of all vectors $\bmk \coloneqq (k_1, \dots, k_D)$ with non-negative integer components that sum to $n$, i.e.
	\begin{equation*}
	0 \leq k_1, \dots, k_D \leq n \text{ and } k_1 + \dots + k_D = n.
	\end{equation*}
	
	Then, $W^{\otimes n}$ admits a complete set of component-multiplicity pairs $\{(W_{\bmk}, M_{\bmk})\}_{\bmk \in C_D^n}$, where
\begin{align}
	M_{\bmk} &= \frac{n!}{k_1!\dots k_D!} \prod\limits_{i=1}^D {m_i}^{k_i}, \label{eq:M}\\
	W_{\bmk} &= \prod\limits_{i=1}^D {w_i}^{k_i}. \label{eq:W}
\end{align}
\end{lemma}
\begin{proof}
	We proceed by induction.
	
	Assume $n = 1$.
	Let $\bmk_i$ be the vector with its $i$-th component equal to 1 and 0's elsewhere.
	The set $C_D^1$ consists of all vectors of this form, i.e. 
\begin{equation*}
	C_D^1 = \{ \bmk_i \}_{i=1,\dots,D}
\end{equation*}
	It is also true by direct calculation that
\begin{equation*}
	\left( W_{\bmk_i}, M_{\bmk_i} \right) = (w_i, m_i).
\end{equation*}
Therefore, $\{ (W_{\bmk}, M_{\bmk}) \}_{\bmk \in C_D^1}$ is a complete set of component-multiplicity pairs for $W$.

	Assume that $\{(W_{\bmk}, M_{\bmk})\}_{\bmk \in C_D^n}$ as given in~\cref{eq:M,eq:W} is a complete set of component-multiplicity pairs for the $n$--copy distribution $W^{\otimes n}$.
	By construction, the distribution $W^{\otimes (n+1)} = W^{\otimes n} \otimes W$ is multiplicative, so it admits the complete set of component multiplicity pairs
\begin{equation}
	\{(W_{\bmk} w_i, M_{\bmk} m_i)\},\ \bmk \in C_D^n \text{ and } i=1,\dots,D.
\end{equation}
	
	Consider the component sum of the distribution $W^{\otimes (n+1)}$,
\begin{align*}
	&\sum_{\bmk \in C_D^n}\sum_{i=1}^D M_{\bmk} m_i W_{\bmk} w_i = \sum_{\bmk \in C_D^n} M_{\bmk}W_{\bmk} \sum_{i=1}^D m_i w_i =\\
	&\sum_{\bmk \in C_D^n} \frac{n!}{k_1!\dots k_D!} \prod\limits_{i=1}^D {m_i}^{k_i}{w_i}^{k_i} \sum_{i=1}^D m_i w_i =\\
	&\left( \sum_{i=1}^D m_i w_i \right)^n \left( \sum_{i=1}^D m_i w_i \right) = \left( \sum_{i=1}^D m_i w_i \right)^{n+1} =\\
	&\sum_{\bmq \in C_D^{n+1}} M_{\bmq}W_{\bmq},
\end{align*}
where in the last expression, vectors $\bmq = (q_1, \dots, q_D)$ have non-negative integer components that sum to $(n+1)$ and 
\begin{align*}
	M_{\bmq} &= \frac{(n+1)!}{q_1!\dots q_D!} \prod\limits_{i=1}^D {m_i}^{q_i},\\
	W_{\bmq} &= \prod\limits_{i=1}^D {w_i}^{q_i}.
\end{align*}
We have used the multinomial theorem to proceed between lines 2-3 and lines 3-4 of the derivation.

We have achieved a regrouping of the distribution components.
Every component $W_{\bmq}$ is of the form $W_{\bmk} w_i$ with $q_i = k_i + 1$ and $q_j = k_j$ for $j\neq i$ and 
\begin{align*}
	\sum_{\bmq \in C_D^{n+1}}  \hspace{-6pt} M_{\bmq} =  \hspace{-10pt} \sum_{\bmq \in C_D^{n+1}} \frac{(n+1)!}{q_1!\dots q_D!} \prod\limits_{i=1}^D {m_i}^{q_i} = 
	\left( \sum_{i=1}^D m_i \right)^{n+1} \hspace{-10pt} = d^{n+1},
\end{align*}
which is the dimension of $W^{\otimes (n+1)}$.

Therefore, $\{ (W_{\bmq}, M_{\bmq}) \}_{\bmq \in C_D^{n+1}}$ is a complete set of component-multiplicity pairs for $W^{\otimes n}$, completing the proof.
\end{proof}

%Index vector $\bmk$ has $D-1$ independent components and in the proof of our main theorem in~\cref{sec:stab} we have $D=4$, so we simplify the notation by writing the component and multiplicity vectors of the $n$--copy state distributions as $m_{ijk}, w(\rho_{\rm{S}})_{ijk}, w(\sigma)_{ijk}$ and $w(\rho_{\rm{S}}|\sigma)_{ijk}$, where $i,j,k$ are the 3 independent index components.

\newpage
\subsection{Free energy dependent magic distillation bounds}\label{free-energy-bound-proof}
\label{app:main_proof}
Here we prove~\cref{thm:free-energy}, which bounds magic distillation rates in terms of free energies.

\begin{theorem*}
	Consider a magic distillation protocol on qutrits that transforms $n$ copies of an $\epsilon$--noisy Strange state into $m$ copies of an $\epsilon'$--noisy Strange state, with depolarising errors $\epsilon' \leq \epsilon \leq 3/7$. 
	
	Let $T =(k\beta)^{-1}$ be any temperature for the physical system and let $H= \sum_{k \in \mathbb{Z}_3} E_k |E_k\>\<E_k|$ be the Hamiltonian of each qutrit subsystem in its eigen-decomposition.
Assume that for $n,m \rightarrow \infty$ the protocol's channel generates negligible correlations on the equilibrium state $\tau$, and so $\tau^{\otimes n} \longrightarrow \tau^{\prime \otimes m}$ for $n,m \gg 1$. We write the state $\tau'$ as $\tau' = e^{-\beta H'}/\Z'$ for some Hermitian $H'$.

Given this, there are local Clifford changes of basis $C_1,C_2$, such that $\rho_1 := C_1 \rho_S(\epsilon) C_1^\dagger$ and $\rho_2 := C_2 \rho_S(\epsilon') C_2^\dagger$ for which the protocol gives
\begin{equation}
\rho_1^{\otimes n} \longrightarrow \rho_2^{ \otimes m}.
\end{equation}
Moreover, the magic distillation rate $R = m/n$ for the protocol is bounded by the expression
\begin{equation}\label{eq:rate_bounds_proof}
	R \leq \dfrac{\ln{\big( 1-\frac{4}{3}\epsilon \big)} + \beta (\phi - F)}{\ln{\big( 1-\frac{4}{3}\epsilon' \big)} + \beta (\phi' - F')},
\end{equation}
where $F$ is the free energy of $\tau$,  and 
\begin{equation}
	\phi = -\beta^{-1} \log \zeta
\end{equation}
with $\zeta$ given by the equations
\begin{align}
	\zeta &= \sum_{k\in \mathbb{Z}_3} \alpha_k e^{-\beta E_k}, \nonumber\\
	\alpha_k &= \sum_{r \in \mathbb{Z}_3} \braket{E_k}{-r}\braket{r}{E_k}.
\end{align}
The primed variables are defined similarly for the output system.
\end{theorem*}


\begin{proof}	
For the sake of clarity, we write $\rho_n \coloneqq \rho_S(\epsilon)^{\otimes n}$, $\rho'_m \coloneqq \rho_S(\epsilon')^{\otimes m}$, $\tau_n \coloneqq \tau^{\otimes m}$ and $\tau'_m \coloneqq \tau'^{\otimes m}$. 

To establish the distillation bound we consider the distillation protocol that gives $\E( \rho_n) = \rho'_m$ for the magic states. 
We then consider that the protocol transforms the reference equilibrium state as $\E( \tau_n) = \tau'_m$.
Since $\tau$ and $\tau'$ are assumed to be full rank stabilizer states, they have a strictly positive Wigner distribution, while, in contrast, the input and output magic states generally have quasi-probability Wigner distributions. 
For any such protocol we therefore have that
\begin{equation}
	( W_{\rho_n}(\bmx), W_{\tau_n}(\bmx) ) \succ ( W_{\rho'_m}(\bmx), W_{\tau'_m}(\bmx) ),
\end{equation}
or, equivalently, in terms of the relevant Lorenz curves,
\begin{equation}
	L_{\rho_n |\tau_n}(x) \ge L_{\rho'_m |\tau'_m}(x) \mbox{ for all } x.
\end{equation}

We define the rescaled Wigner distribution $W_{\rho | \tau}(\x) \coloneqq W_\rho(\x)/W_\tau(\x)$, which is always well-defined since $\tau$ is full-rank. 
Due to the multiplicative property of the Wigner distribution, the rescaled distribution is also multiplicative in the sense that
\begin{equation}
	W_{\rho \otimes \rho' | \tau \otimes \tau'} (\x_1 \oplus \x_2) = W_{\rho | \tau}(\x_1)W_{\rho' | \tau'}(\x_2),
\end{equation}
for any states $\rho, \rho'$ and any full-rank stabilizer states $\tau, \tau'$.
Therefore we have that
\begin{align}
	W_{\rho_n |\tau_n} (\x) &= \prod_{i=1}^n W_{\rho|\tau}(\x_i)\\
	W_{\rho_n } (\x) &= \prod_{i=1}^n W_\rho(\x_i)
\end{align}
where $\x = \oplus_{i=1}^n \x_i \in \mathbb{Z}_3^n$ is the phase space point for the full system in terms of those of the individual subsystems.

The points defining the Lorenz curve $L_{\rho_n |\tau_n}(x)$ are obtained from sorting the components of $W_{\rho_n |\tau_n}(\x)$ in non-increasing order and then computing the partial sums of $W_{\rho_n |\tau_n}(\pi(\x))$ where $\pi$ is the permutation that realises the sorting. 
However, similarly to the unital fragment analysis, we aim to use the constraint that is obtained by considering the line segment connecting the origin to the first elbow of both Lorenz curves.

The Wigner distribution of a single noisy Strange state consists of 1 negative component $W_{\rho}(0,0) = -v(\epsilon)$ and 8 positive components $W_{\rho}(\bmx) = u(\epsilon)$ for $\bmx \neq \bmo$.
The Wigner distribution of the full-rank, stabilizer equilibrium state $\tau$ is $W_{\tau}(\bmx) > 0$ for all $\bmx \in \mathbb{Z}_3^2$.

Assume that the smallest component of the distribution $W_\tau(\x)$ is at $x=\x_\star$.
We then make use of the freedom to apply Clifford unitaries in order to perform the  Clifford operation
\begin{equation}
	\tau \longrightarrow X^{-i}Z^{-j}\ \tau\ Z^{j}X^{i},
\end{equation}
which permutes the smallest component to the origin of the phase space, $W_{\tau}(i,j) \longrightarrow W_{\tau}(0,0)$.
This operation would inevitably affect the magic distillation protocol, but since we consider the state distillation modulo Clifford operations, we are allowed to perform the reverse Clifford operation keeping the magic distillation process as is. \nick{statement here is inaccurate - incorporate $C_1, C_2$ in the proof.}

The rescaled distribution components are in general given by
\begin{equation}
	W_{\rho_n|\tau_n} = \left(\frac{-v}{W_{\tau}(0,0)}\right)^{i_{\bmo}} \prod_{\bmx \neq \bmo} \left(\frac{u}{W_{\tau}(\bmx)}\right)^{i_{\bmx}},
\end{equation}
where the (integer) indices obey the following conditions:
\begin{align}
&0 \leq i_{\bmx} \leq n \mbox{ for all } \bmx \in \mathbb{Z}_3^2, \nonumber \\
&\sum_{\bmx \in \mathbb{Z}_3^2} i_{\bmx} = n.
\end{align}
It is now easy to find the largest rescaled component.
Firstly, note that $n$ is even, so we require that $i_{\bmo} \in \{0,2,\dots,n\}$ for the component to be positive.
Then, we have that $v \geq u$ because $\epsilon \leq 3/7$, and we have already ensured that $W_{\tau}(0,0) \leq W_{\tau}(\bmx)$ for all $\bmx \in \mathbb{Z}_3^2$.
Therefore, the largest rescaled component occurs when $i_{\bmo} = n$ and $i_{\bmx} = 0$ for $\bmx \neq \bmo$ and is equal to $(v/W_{\tau}(0,0))^n$.
Accordingly, the coordinates of the first Lorenz curve point after the origin are given by
\begin{equation}
	(x_0, L_0) = ((W_{\tau}(0,0))^n, v^n)
\end{equation}

Given a Hamiltonian decomposition,
\begin{equation}
	H = \sum_{k \in \mathbb{Z}_3} E_k\ketbra{E_k}{E_k},
\end{equation}
we can re-express the coordinate location, by writing
\begin{align}
	W_{\tau}(0,0) &= \frac{1}{3\Z}\tr\left[ A_{0,0}e^{-\beta H} \right] \nonumber\\
	&= \frac{e^{\beta F}}{3} \tr\left[ \sum_{r \in \mathbb{Z}_3} \ketbra{-r}{r}\sum_{k \in \mathbb{Z}_3} e^{-\beta E_k}\ketbra{E_k}{E_k} \right] \\
	&= \frac{e^{\beta F}}{3} \sum_{r \in \mathbb{Z}_3} \alpha_k e^{-\beta E_k}
	= \frac{e^{\beta F}}{3} \zeta \\
	&= \frac{e^{\beta (F - \phi)}}{3},
\end{align}
where $\alpha_k, \zeta$ and $\phi$ are as defined in the statement of the theorem.
Finally, the coordinates can be expressed as
\begin{equation}
	(x_0, L_0) = \left( \frac{e^{n\beta (F - \phi)}}{3^n}, v(\epsilon)^n \right).
\end{equation}
In the same manner, we can derive the first point coordinates of the output Lorenz curve as
\begin{equation}
	(x'_0, L'_0) = \left( \frac{e^{n\beta (F' - \phi')}}{3^n}, v(\epsilon')^n \right).
\end{equation}

If the largest rescaled component of a state is distinct with no multiplicities, then these coordinates correspond to the first elbow of the corresponding Lorenz curve, whereas if it appears multiple times, then the coordinates derived correspond to a point on the interior of the line segment connecting the origin to the first elbow.
In both cases, the distillation bound remains the same, as is clear by its derivation in~\cref{app:elb_constraints}, and it is given by $L_0 / x_0 \geq L'_0 / x'_0$, which can be directly rearranged into the stated bound of the theorem.
\end{proof}
	
\subsection{Deriving distillation bounds from the last elbow}
\label{sec:last_elb}

Using a similar analysis, we can also derive upper bounds from comparison of the last point coordinates of the Lorenz curve, which now corresponds to the smallest rescaled component.

To this end, assume that the largest component of the equilibrium state distribution is $W_{\tau}(i,j)$.
We now perform the Clifford operation
\begin{equation}
	\tau \longrightarrow X^{1-i}Z^{-j}\ \tau\ Z^{j}X^{i-1},
\end{equation}
which permutes the largest component such that $W_{\tau}(i,j) \longrightarrow W_{\tau}(1,0)$.
Since we have that $u \leq v$ because $\epsilon \leq 3/7$, and we have already ensured that $W_{\tau}(1,0) \geq W_{\tau}(\bmx)$ for all $\bmx \in \mathbb{Z}_3^2$.
Therefore, the largest rescaled component occurs when $i_{(1,0)} = n$ and $i_{\bmx} = 0$ for $\bmx \neq \bmo$ and is equal to $(v/W_{\tau}(1,0))^n$.
Accordingly, the coordinates of the first Lorenz curve point after the origin are given by $(x_E, L_E) = ((W_{\tau}(1,0))^n, u(\epsilon)^n)$ for the input state and $(x'_E, L'_E) = ((W_{\tau'}(1,0))^m, u(\epsilon')^m)$ for the output state.

We now redefine the quantities $\alpha_k$, so that again $\phi = -\beta^{-1} \log \zeta$, where $\zeta= \sum_{k\in \mathbb{Z}_3} \alpha_k e^{-\beta E_k}$ and $\alpha_k$ now given by
\begin{align}
	\alpha_k = \sum_{r \in \mathbb{Z}_3} \braket{E_k}{1-r}\braket{1+r}{E_k}.
\end{align}
This allows us to rewrite $W_{\tau}(1,0) = {e^{\beta (F - \phi)}}/{3}$ and by using the last elbow constraint 
\begin{equation*}
	\frac{L_E - 1}{1-x_E} \geq \frac{L'_E - 1}{1-x_E'},
\end{equation*}
derived in~\cref{app:elb_constraints}, we can get a new bound expression.

We note that it is also possible to perform a different Clifford transformation and get a different location $(W_{\tau}(\bmx))^n$, with $\bmx \neq \bmo$, since components $(W_{\rho}(\bmx))^n = u$ are all the same as long as $\bmx \neq \bmo$.
The effect of this is to alter the expressions for the quantities $\alpha_k$, but not the resulting bound expression.

\nick{This section on the last elbow bound is a little weak?}

\subsection{OLD VERSION OF MAIN THEOREM}

\ddd{[Do we need to keep any of the details in this section? If not then delete it entirely.]}
\nick{This proof offers: 1. the exact derivation of our main contour plot, justifying the quantities $\epsilon_\star$ and $\beta_\star$ 2. a method for lifting the ``Clifford change in basis'' assumption - Can be trimmed down to the case where $H=H'$. What do you think?}
\begin{theorem}
	Consider a magic distillation protocol transforming noisy Strange state
	\begin{equation*}
		\rho_S(\epsilon)^{\otimes n} \longrightarrow \E(\rho_S(\epsilon)^{\otimes n})=\rho_S(\epsilon')^{\otimes m} 
	\end{equation*}
	with $n, m \gg 1$.

Let each qutrit have a Hamiltonian $H$ with stabilizer eigenstates and energies $E_0, E_1, E_2$, and define $E_{\rm max} = \max\{E_0, E_1, E_2\}$ and $E_s$ the eigenvalue of the eigenstate that \nick{overlaps} the negative component of $\rho_S(\epsilon)$ in the Wigner representation. Let $T =(k\beta)^{-1}$ be any characteristic temperature for the physical system in the state $\tau^{\otimes n}= (e^{-\beta H}/\Z)^{\otimes n}$, with free energy $F$. Assume that for $n,m \gg 1$ the channel $\E$ generates negligible correlations on $\tau^{\otimes n}$, and so $\E(\tau^{\otimes n}) = \sigma^{\otimes m}$ for some state $\sigma$.

Define $\beta_\star = (k T_\star)^{-1}$ through the relation
\begin{equation}
	E_{\rm max} - E_s \eqqcolon kT_\star \ln 2,
\end{equation}
and define a threshold noise,
\begin{equation}
	\epsilon_{\star}(\beta) \coloneqq 
	\begin{cases}
		3 - \dfrac{9}{4-2^{\beta/\beta_\star - 1}}, &\text{ for } \beta \leq \beta_\star \\
		0, &\text{ for } \beta > \beta_\star.
	\end{cases}
\end{equation}
Then, the distillation rate $R = m/n$ of the magic protocol is bounded as:
\begin{equation}\label{eq:rate_bounds_proof}
	R \leq
	\begin{cases}
		\dfrac{\ln{\big( 1-\frac{4}{3}\epsilon \big)} + \beta (E_s - F)}{\ln{\big( 1-\frac{4}{3}\epsilon' \big)} + \beta (E'_s - F')},\ &\epsilon \leq \epsilon_\star, \epsilon' \leq \epsilon'_\star, \vspace{10pt}\\
		\dfrac{\ln{\big( 1-\frac{4}{3}\epsilon \big)} + \beta (E_s - F)}{\ln{\big( \frac{1}{2}-\frac{1}{6}\epsilon' \big)} + \beta (E'_{\rm{max}} - F')},\ &\epsilon \leq \epsilon_\star, \epsilon' > \epsilon'_\star, \vspace{10pt}\\
		\dfrac{\ln{\big( \frac{1}{2}-\frac{1}{6}\epsilon \big)} + \beta (E_{\rm{max}} - F)}{\ln{\big( 1-\frac{4}{3}\epsilon' \big)} + \beta (E'_s - F')},\ &\epsilon > \epsilon_\star, \epsilon' \leq \epsilon'_\star, \vspace{10pt}\\
		\dfrac{\ln{\big( \frac{1}{2}-\frac{1}{6}\epsilon \big)} + \beta (E_{\rm{max}} - F)}{\ln{\big( \frac{1}{2}-\frac{1}{6}\epsilon' \big)} + \beta (E'_{\rm{max}} - F')},\ &\epsilon > \epsilon_\star, \epsilon' > \epsilon'_\star,
	\end{cases}
\end{equation}
where $F'$ is the free energy of the state $\sigma = e^{-\beta H'}/\Z'$ and other primed quantities are defined in the same way as the corresponding unprimed quantities. 
\end{theorem}
We note, firstly, that the specific numerical factors in $\epsilon_\star$ are a result of our choice of magic state. 
\begin{proof}
To establish the distillation bounds we consider the distillation protocol that gives
\begin{equation}
	\E( \rho_S(\epsilon)^{\otimes n}) = \rho_S(\epsilon')^{\otimes m},
\end{equation}
for the magic state. We then consider how the protocol transforms the reference equilibrium state as
\begin{equation}
	\E( \tau^{\otimes n}) = \sigma^{\otimes m}.
\end{equation}
Since $\tau$ and $\sigma$ are assumed to be full rank stabilizer states they have a strictly positive Wigner distribution, while, in contrast, the input and output magic states will generally have quasi-probability distributions for their Wigner functions. For any such protocol we therefore have that
\begin{equation}
	(W_{\rho_n} (x), W_{\tau_n}(x) ) \succ  (W_{\rho'_m} (x), W_{\sigma_m}(x) ),
\end{equation}
where, for the sake of clarity, we write $\rho_n \coloneqq \rho_S(\epsilon)^{\otimes n}$, $\rho'_m \coloneqq \rho_S(\epsilon')^{\otimes m}$ and $\tau_n \coloneqq \tau^{\otimes m}$, $\sigma_m \coloneqq \sigma^{\otimes m}$. We then have that
\begin{equation}
	L_{\rho_n |\tau_n}(x) \ge L_{\rho'_m |\sigma_m}(x) \mbox{ for all } x.
\end{equation}
We must therefore compute the Lorenz curve data for $\rho_n$ relative to $\tau_n$, and compare with the Lorenz curve of the output state $\rho_m'$ relative to $\sigma_m$.

We define $W_{\rho | \tau}(\x) \coloneqq W_\rho(\x)/W_\tau(\x)$, which is always well-defined since $\tau$ is full-rank. Now we show in~\cref{app:wigner} that
\begin{equation}
W_{\rho_1\otimes \rho_2 | \tau_1 \otimes \tau_2} (\x_1 \oplus \x_2) = W_{\rho_1 | \tau_1}(\x_1)W_{\rho_2 | \tau_2}(\x_2),
\end{equation}
and also,
\begin{equation}
W_{\rho_1\otimes \rho_2} (\x_1 \oplus \x_2) = W_{\rho_1}(\x_1)W_{\rho_2}(\x_2),
\end{equation}
for any states $\rho_1, \rho_2$ and any full-rank stabilizer states $\tau_1, \tau_2$. \ddd{[Notation clashing here. Frame result outside as its own lemma -- we gotta check everything.]} \nick{Will frame in~\cref{app:wigner} and fix notation in the next commit}
Therefore we have that
\begin{align}
W_{\rho_n |\tau_n} (\x) &= \prod_{i=1}^n W_{\rho|\tau}(\x_i)\\
W_{\rho_n } (\x) &= \prod_{i=1}^n W_\rho(\x_i)
\end{align}
where $\x = \oplus_{i=1}^n \x_i$ is the phase space point for the full system in terms of those of the individual subsystems.

The points defining the Lorenz curve $L_{\rho_n |\tau_n}(x)$ are obtained from first sorting the components of $W_{\rho_n |\tau_n}(\x)$ in non-increasing order and then computing the partial sums of $W_{\rho_n |\tau_n}(\pi(\x))$ where $\pi$ is the permutation that realises the sorting. Therefore, we first look at the values of $W_{\rho|\tau}(\x_i)$ for the single-copy case.

The equilibrium state at inverse temperature $\beta$ on a single qutrit is given by $\tau = e^{-\beta H} / \Z$. Moreover we have that $\tau$ is a full-rank stabilizer state, where $\beta \geq 0$ and $H = E_0 \ketbra{\varphi_0} + E_1 \ketbra{\varphi_1} + E_2 \ketbra{\varphi_2}$ is an eigendecomposition of $H$.
The state $\tau$ can now be written as 
\begin{equation}
	\tau = \frac{e^{-\beta E_0}}{\Z} \ketbra{\varphi_0} + \frac{e^{-\beta E_1}}{\Z} \ketbra{\varphi_1} + \frac{e^{-\beta E_2}}{\Z} \ketbra{\varphi_2},
\end{equation}
where the eigenstates $\{\ket{\varphi_k}\<\varphi_k|\}$ are pure, orthonormal stabiliser states, which can be represented in terms of generalized Paulis. To make our analysis simpler, we perform a change of basis that does not affect the Wigner negativity of the problem. We let $C$ be the unitary transforming each $|\varphi_k\>\<\varphi_k|$ to $|k\>\<k|$. Since the Clifford group is the normalizer of the Heisenberg-Weyl group, $C$ is a Clifford unitary. Therefore, $C$ maps $\tau$ to another stabilizer state that is diagonal in the computational basis, and we can assume without loss of generality that $\tau$ is diagonal in $\{|0\>,|1\>, |2\>\}$. However, this choice means that the location of the negative Wigner component $-v(\epsilon)$ of the Strange state will be permuted on the discrete phase space. We denote by $E_s$ the eigenvalue of $H$ where the associated eigenvector has Wigner distribution overlapping the negative component of the magic state $C\rho_S(\epsilon)C^\dagger$.  This is unique, since the eigenstates form an orthonormal basis.

The Wigner distribution of state $\tau$ is then given by
\begin{align}
	\W[\bmx]{\tau} &= \sum\limits_{k=0}^2 \frac{e^{-\beta E_k}}{\Z}W_{\ketbra{k}}(x, p) \nonumber\\
	&= \sum\limits_{k=0}^2 \frac{e^{-\beta E_k}}{\Z} \delta_{x, k} = \frac{e^{-\beta E_x}}{3\Z},
\end{align}
where $x$ labels one of the three vertical lines in the phase space.
The rescaled Wigner distribution $W_{\rho|\tau}(\x)$ is then easily computed. It has $9$ components, but several of these come with multiplicities. In total, there are four distinct values on the phase space, as illustrated in~\cref{fig:pd_split}.
\begin{figure}[h]
    \centering
    \includegraphics[scale=0.45]{figs/pd_split_thermal.pdf}
    \caption{\textbf{Qutrit phase space regions for $W_{\rho | \tau}(\x)$.}
    Here, the negative component of the magic state overlaps the Wigner distribution of $|0\>$. The rescaled distribution attains a single value in each of the four regions, proportional to the value depicted in the region, see~\cref{eq:bmw_rescaled}.
    }
    \label{fig:pd_split}
\end{figure}

We now denote by $\w(\rho), \w(\rho|\tau)$ the unique values occurring in $W_\rho(\x), W_{\rho|\tau}(\x)$ respectively and $\bmm$ the vector of associated multiplicities of each value in $W_\rho(\x)$. The component values and multiplicities of the relevant distributions in the four distinct regions are given by
\begin{align}
	\w(\rho) &\coloneqq (-v, u, u, u), \\
		\bmm &\coloneqq (1,2,3,3), \\
	\bmw(\tau) &\coloneqq \frac{1}{3\Z} \left( e^{-\beta E_0}, e^{-\beta E_0}, e^{-\beta E_1}, e^{-\beta E_2} \right), \\
	\bmw(\rho_{\rm{S}} | \tau) &\coloneqq 3\Z \left( -v e^{\beta E_0}, u e^{\beta E_0}, u e^{\beta E_1}, u e^{\beta E_2} \right). \label{eq:bmw_rescaled}
\end{align}

Using this notation, the values and multiplicities of the $n$--copy distribution $\bmw(\rho_n |\tau_n)$ are computed in~\cref{lem:ncopycomponents} in~\cref{app:cmpairs}. The values are given by 
\begin{align}\label{eq:ncopy_w_rescaled}
	[\w(\rho_n | \tau_n)]_{ijk} &= (3\Z)^{n} (-v)^{n-\alpha} u^{\alpha} e^{\beta (n-\alpha)E_s} e^{\beta ( i E_0 + j E_1 + k E_2 )},
\end{align}
where the indices $i,j,k$ are non-negative integers that obey the constraint $\alpha \coloneqq i+j+k \leq n$.
The multiplicity of this above value is $m_{ijk}$ with
\begin{equation}
	m_{ijk} = \frac{n!}{i!j!k!(n-\alpha)!} 2^i 3^j 3^k.
\end{equation}
The associated components of $\w(\rho_n)$ are given by
\begin{align}
	[\w(\rho_n)]_{ijk} &= (-v)^{n-\alpha} u^{\alpha}, \label{eq:ncopy_wrho}\\
	[\w(\tau)]_{ijk} &= (3\Z)^{-n} e^{-\beta (n-\alpha)E_s} e^{-\beta ( i E_0 + j E_1 + k E_2 )}. \label{eq:ncopy_wsigma}
\end{align}

In order to construct the $n$--copy Lorenz curve $L_{\rho_n|\tau_n}(x)$ we need to order the components of the distribution, $w(\rho_{\rm{S}} | \tau)_{ijk}$ in decreasing order, and identify the sequence of indices that give us $W_{\rho_n}(\pi(\x))$.

Generally this is complex, but in order to obtain distillation bounds it is sufficient to determine the location of the first elbow $(x_0, L_0)$ of $L_{\rho_n|\tau_n}(\x)$. To do so, we compute the largest component 
\begin{equation}
	w_{\rm max} \coloneqq \max_{i,j,k} [\w(\rho_n | \tau_n)]_{ijk},
\end{equation}
and determine the indices at which this occurs.
Putting in the values we obtain
\begin{align}
	&(3\Z)^{-n}w_{\rm max} = \nonumber\\
	&\max\limits_{i,j,k}\Big\{ (-v)^{n-\alpha} u^{\alpha} e^{\beta (n-\alpha)E_s} e^{\beta ( i E_0 + j E_1 + k E_2 )} \Big\}, \label{eq:max_slope}
\end{align}
where $0 \leq i,j,k \leq n$ and $\alpha \coloneqq i+j+k \leq n$.
Now for $0 \leq \epsilon \leq 3/7$, we have $v \geq u$. Since we assume that $n$ is even, we need the sum $\alpha = i+j+k$ to be even too, so that the expression is positive. 

Given an even value for $\alpha$, the term $v^{n-\alpha} u^{\alpha} e^{-\beta (n-\alpha)E_s}$ is fixed, so the expression is maximised by setting the coefficient of the highest energy $E_{\rm{max}}$ equal to $\alpha$.
Hence, we have
\begin{align}
	&w_{\rm max} = \nonumber\\
	&(3\Z)^{n} v^n e^{n\beta E_s}\max\limits_{\substack{\alpha = 0,2, \\ \dots,n-2,n}}{\Big\{ \left( \frac{u}{v} e^{\beta (E_{\rm{max}} - E_s)} \right)^{\alpha} \Big\}}.
\end{align}
If the expression $\frac{u(\epsilon)}{v(\epsilon)} e^{\beta (E_{\rm{max}} - E_s)}$ is less than $1$ then the maximum occurs at $\alpha=0$, otherwise it occurs at $\alpha = n$. For a fixed state $\tau$, this transition is determined by the value of the depolarising noise parameter $\epsilon$ of the noisy magic state. The transition occurs at $\epsilon = \epsilon_\star$ where
\begin{equation}\label{eq:noise_transition}
	\frac{u(\epsilon_\star)}{v(\epsilon_\star)} e^{\beta (E_{\rm{max}} - E_s)} = \frac{3-\epsilon_\star}{6-8\epsilon_\star} e^{\beta (E_{\rm{max}} - E_s)} = 1.
\end{equation}
If $E_{\rm{max}} = E_s$, namely if the state negativity lies in the same phase space region as the highest energy, this threshold is constant in temperature and given by $\epsilon_{\star} = 3/7$. However, the condition that $\epsilon_\star \ge 0$ also implies a constraint on the effective temperature of the stabilizer state. Specifically, there is a threshold temperature value $\beta_\star$ given by
\begin{equation}
	\beta_{\star} \coloneqq \frac{1}{E_{\rm{max}} - E_s} \ln2,
\end{equation}
such that for the regime $0 \leq \beta \leq \beta_\star$ a transition noise level $\epsilon_\star$ exists, and for $\beta > \beta_\star$ no such transition exists, so we choose $\epsilon_\star = 0$. 
Therefore, the transition value for the noise is given by
\begin{equation}
	\epsilon_{\star}(\beta) \coloneqq 
	\begin{cases}
		3 - \dfrac{9}{4-2^{\beta/\beta_\star - 1}}, &\text{ for } \beta \leq \beta_\star \\
		0, &\text{ for } \beta > \beta_\star.
	\end{cases}
\end{equation}
The quantity $w(\rho_{\rm{S}} | \sigma)_{\rm{max}}$ is now given by
\begin{equation*}
w_{\rm max} =
	\begin{cases}
		(3\Z)^{n} v^n e^{n\beta E_s}, &\mbox{if }\epsilon \leq \epsilon_{\star},\ \hspace{3pt}\rm{(C1)}\\
		(3\Z)^{n} u^n e^{n\beta E_{\rm{max}}}, &\mbox{if }\epsilon > \epsilon_{\star}.\ \hspace{5pt}\rm{(C2)} 
	\end{cases}
\end{equation*}
Case $\rm{(C1)}$ corresponds to $(i,j,k) = (0,0,0)$, so the multiplicity is $m_{000} = 1$, while
Case $\rm{(C2)}$ corresponds to
\begin{equation}
	(i,j,k) = 
	\begin{cases}
	(0,n,0), &\text{if } E_{\rm{max}} = E_1, \\
	(0,0,n), &\text{if } E_{\rm{max}} = E_2,
	\end{cases}
\end{equation}
so the multiplicity in both cases is $3^n$.

Using that $F = -\beta^{-1} \log \Z$, the first elbow coordinates in the two cases are now given by
\begin{equation}\label{eq:first_elb_coords}
	(x_0, L_0) =
	\begin{cases}
		\left(\frac{1}{3^n} e^{-n\beta (E_s - F)}, v^n \right), &\epsilon \leq \epsilon_\star \vspace{10pt}\\
		\left( e^{-n\beta (E_{\rm{max}}-F)}, (3u)^n \right). &\epsilon > \epsilon_\star
	\end{cases}
\end{equation}

\ddd{[Hmmm this needs an additional assumption on the output state to re-run the same analysis. Very annoying.]}\nick{What additional assumption beyond that the output state can be written as a Gibbs state for some $H'$?}
Similarly, considering the output magic state with respect to state $\sigma'$, the image of equilibrium state $\sigma$ under the magic protocol, we get output Lorenz curve coordinates,
\begin{equation}\label{eq:transformed_first_elb_coords}
	(x'_0, L'_0) =
	\begin{cases}
		\left(\frac{1}{3^{n'}} e^{-n\beta (E'_s - F')}, v(\epsilon')^{n'} \right), &\epsilon' \leq \epsilon'_\star \vspace{10pt}\\
		\left( e^{-n'\beta (E'_{\rm{max}}\hspace{-2.5pt}-F')}, (3u(\epsilon'))^{n'} \right), &\epsilon' > \epsilon'_\star
	\end{cases}
\end{equation}
There are four combinations of coordinates, depending on the noise parameters $\epsilon, \epsilon'$ for the input and output states.
In each of these combinations, we simply use the first elbow constraint, as described in~\cref{app:elb_constraints}, which leads to the bounds in the statement of the theorem.
\end{proof}






























\end{document}