\documentclass[pra,
aps,
twocolumn,
superscriptaddress,
groupedaddress,
nofootinbib,
reprint
]{revtex4-1}

% PACKAGES
\usepackage{amsmath,amsfonts, amssymb, amsthm}
\usepackage{bm, bbm, physics, mathtools}
\usepackage{graphicx, subfigure, multirow, makecell}
\usepackage{xcolor, enumerate}
\usepackage{xifthen, hyperref}
\usepackage[capitalise]{cleveref}

\hypersetup{
	colorlinks=true,  
	linkcolor=blue,   
	citecolor=blue,   
	urlcolor=blue     
}

\newcommand{\crefrangeconjunction}{--}
\creflabelformat{figure}{(#2#1#3)}

% COMMENT NOTATION
\newcommand{\nick}[1]{{\color{red}[#1]}}
\newcommand{\ddd}[1]{\textcolor{blue}{#1}}

% ENVIRONMENTS
\newtheorem{theorem}{Theorem}
\newtheorem{proposition}[theorem]{Proposition}
\newtheorem{lemma}[theorem]{Lemma}
\newtheorem{definition}[theorem]{Definition}
\newtheorem{example}{Example}

% REFERENCES
\iffalse
\renewcommand{\eqref}[1]{Eq.~(\ref{#1})}
\newcommand{\figref}[1]{Fig.~(\ref{#1})}
\newcommand{\tabref}[1]{Tab.~(\ref{#1})}
\newcommand{\secref}[1]{Section~(\ref{#1})}
\newcommand{\appref}[1]{Appendix~(\ref{#1})}
\newcommand{\defref}[1]{Definition~\ref{#1}}
\newcommand{\lemref}[1]{Lemma~\ref{#1}}
\newcommand{\thmref}[1]{Theorem~\ref{#1}}
\fi

% SYMBOL DEFINITIONS
\renewcommand{\cal}[1]{\mathcal{#1}}

\newcommand{\reals}{\mathbb{R}}
\newcommand{\id}{\mathbbm{1}}
\newcommand{\idc}{1_{\rm{C}}}
\newcommand{\supf}{\mathfrak{c}}
\newcommand{\floor}[1]{\left\lfloor #1 \right\rfloor}
\newcommand{\ent}[2]{S\left( #1 \middle\vert\middle\vert #2 \right)}
\newcommand{\ents}{{\ent{\frac{m}{n}}{p}}}
\renewcommand{\tr}{{\rm{tr}}}
\renewcommand{\det}{{\rm{det}}}


\newcommand{\spanv}[1]{
    {{\rm{span}}\left\{#1\right\}}
}
\newcommand{\conv}[1]{
    {{\rm{conv}}#1}
}
\newcommand{\orb}[1]{
    {{\rm{orb}}(#1)}
}
\newcommand{\sn}[1]{
    {{\rm{sn}}\left(#1\right)}
}
\newcommand{\mana}[1]{
    {{\rm{mana}}\left(#1\right)}
}
\newcommand{\lc}[2]{
	{{\rm{L}}_{#1|#2}}
}

\newcommand{\bmx}{\bm{x}}
\newcommand{\bmy}{\bm{y}}
\newcommand{\bmz}{\bm{z}}
\newcommand{\bmu}{\bm{u}}
\newcommand{\bmo}{\bm{0}}
\newcommand{\bmd}{\bm{d}}
\newcommand{\bma}{\bm{a}}
\newcommand{\bmw}{\bm{w}}
\newcommand{\bmm}{\bm{m}}
\newcommand{\bmk}{\bm{k}}
\newcommand{\bmq}{\bm{q}}
\newcommand{\bmg}{\bm{g}}

\newcommand{\spd}[1][]{
    \ifthenelse{\isempty{#1}}{
    {{\rm{Sp}}(2, \zd)} }{
    {{\rm{Sp}}(2, \zd[#1])}}
}
\newcommand{\stoch}[1][]{
    \ifthenelse{\isempty{#1}}{
    {{\rm{S}}_d(\bmd)} }{
    {{\rm{S}}_d(#1)}}
}
\newcommand{\stochw}[1][]{
    \ifthenelse{\isempty{#1}}{
    {{\rm{S}}_{d^2}(\W{\sigma})} }{
    {{\rm{S}}_{d^2}(#1)}}
}
\makeatletter
\def\W{\@ifnextchar[{\@with}{\@without}}
\def\@with[#1]#2{ 
    {{\rm{W}}_{#2}\left(#1\right)} }
\def\@without#1{ 
    {{\rm{W}}_{#1}} }
\makeatother

\newcommand{\T}{\cal{T}}
\newcommand{\Z}{\cal{Z}}
\newcommand{\C}{\cal{C}}
\newcommand{\E}{\cal{E}}
\newcommand{\J}{\cal{J}}
\newcommand{\R}{\cal{R}}
\newcommand{\D}{\cal{D}}
\newcommand{\M}{\cal{M}}
\newcommand{\F}{\cal{F}}
\renewcommand{\O}{\cal{O}}

\newcommand{\Fmax}{\F_{\rm{max}}}
\newcommand{\Omax}{\O_{\rm{max}}}
\newcommand{\Rmax}{\R_{\rm{max}}}
\newcommand{\Pis}{\Pi_{\rm{s}}}
\newcommand{\Pio}{\Pi_{\rm{o}}}

\newcommand{\cptp}{{\rm{CPTP}}}
\newcommand{\cpos}{{\rm{CP}}}
\newcommand{\so}{{\rm{SO}}}
\newcommand{\stab}{{\rm{STAB}}}
\newcommand{\spo}{{\rm{SPO}}}
\newcommand{\cspo}{{\rm{CSPO}}}
\newcommand{\rcu}{{\rm{RCU}}}
\newcommand{\tho}{{\rm{TO}}}
\newcommand{\cpwp}{{\rm{CPWPO}}}
\newcommand{\ru}{{\rm{RU}}}



%%%%%% DJ DEFINITIONS %%%%


\def\>{\rangle}
\def\<{\langle}
\def\K{ {\mathcal K} }
\def\E{ {\mathcal E} }
\def\P{ {\mathcal P} }
\def\H{ {\mathcal H} }
\def\M{ {\mathcal M} }

\def\U {{\mathcal U}}
\def\R {{\mathcal R}}
\def\G {{\mathcal G}}
\def\N{ {\mathcal N} }
\def\F{ {\mathcal F} }
\def\A{ {\mathcal A} }
\def\B{ {\mathcal B} }
\def\O{ {\mathcal O} }
\def\P{ {\mathcal P} }
\def\D{ {\mathcal D} }
\def\T{ {\mathcal T} }
\def\I{ \mathbbm{1} }
\def\tr{ \mbox{tr} }
\def\non{ \nonumber\\}
\def\plus{ |+\> }
\def\minus{|-\> }
\def\plusdag{ \<+| }
\def\minusdag{ \<-| }
\def\diag{ \mathrm{diag}}

\def\r{\boldsymbol{r}}
\def\w{\boldsymbol{w}}
\def\x{\boldsymbol{x}}
\def\y{\boldsymbol{y}}
\def\z{\boldsymbol{z}}
\def\t{\boldsymbol{t}}
\def\p{\boldsymbol{p}}
\def\q{\boldsymbol{q}}


%%%%%%%%%%%%%%%%

\begin{document}
% Distillation rate vs bound
% Proofs in App. A
% Eq. numbering in last line always
% Sec: App. D
% Sec: General resources
% Sec: Lower bounds
% Sec: Intro
% Sec: Abstract
% Figs consistency (esp. figs 3,4,5,6 and 1,7) + Figure 0 ?
% Spacing (fig positioning, sections, paragraphs, QED symbols)

\begin{abstract}
\ddd{[To be sharpened]} Magic states are key ingredients in schemes to realise universal fault-tolerant quantum computation.
Theories of magic states attempt to quantify this computational element via monotones and determine how these states may be efficiently transformed into useful forms. Here we introduce the concept of `fragments', which generalise the concept of magic monotones and has a natural thermodynamic structure based on majorisation. From this perspective magic can be viewed as a form of free energy within each fragment and is constrained by relative majorisation relations but now on quasi-probability distributions. Notably this approach allows us to incorporate actual physical constraints, for example noise models with particular bias or temperature-dependent features, and study how these constrain general magic distillation protocols. In this context we present general temperature-dependent bounds on distillation rates that any theory of magic must respect. Significantly, this analysis also presents a thermodynamic context which cannot be analysed via traditional methods based on thermodynamic entropies, due to the presence of negativity, and raises novel questions in the context of statistical mechanics.
\end{abstract}

\preprint{APS/123-QED}

\title{Thermodynamic fragments for magic states in quantum computation}

\author{Nikolaos Koukoulekidis}
	\email{nk2314@imperial.ac.uk}
	\affiliation{Department of Physics, Imperial College London, London SW7 2AZ, UK}
\author{David Jennings}
	\affiliation{School of Physics and Astronomy, University of Leeds, Leeds, LS2 9JT, UK}
	\affiliation{Department of Physics, Imperial College London, London SW7 2AZ, UK}

\date{\today}
\maketitle

%%%%%%%%%%%%%%%%%%%%%%%%%%%%%%%%%%%%%%%%

\section{Introduction and background}
\label{sec:intro_prev}

\subsection{Introduction}
\label{sec:intro}

Magic states are necessary for achieving universal quantum computation within fault-tolerant schemes~\cite{cit:veitch, cit:mari, cit:gottesman, cit:bravyi, cit:knill, Campbell_2011}.
Identifying magic as a resource for quantum universality has led to several theories which try to provide a framework for its quantification and manipulation \cite{cit:veitch2, cit:howard, Wang_2019, Seddon_2021}.
The main question that such a theory attempts to answer is:
\begin{center}
    \emph{Given two magic states $\rho$ and $\rho'$ is there a free operation that can convert $\rho$ to $\rho'$?}
\end{center}

We are interested in all resource theories of magic $\R = (\F, \O)$ in which free operations cannot generate any amount of resource. 
Further denote by $\D$ the set of states considered under the theory, that is the union of free and resource states.
The structure of such theory is described by a partial order, hereinafter called a \emph{pre-order}, $\prec_{\R}$ between states.
We write $\rho' \prec_{\R} \rho$ iff there exists $\E \in \O$ such that $\E(\rho) = \rho'$.
Naturally, states may be incomparable under the given theory, meaning that there exists no free operation that converts one to the other.
We further call $\R' = (\F', \O')$ a \emph{subtheory} of $\R$ iff $\F' \subseteq \F$ and $\O' \subseteq \O$. 
The above notation will be used for general resource theories as well.

Formally, the no resource generation condition on the theories translates into two assumptions:
\begin{enumerate}[I]
    \item Free operations send free states into free states, $\E: \F \mapsto \F$, for all $\E \in \O$;
    \item Resource theory $\R$ is a completely free state preserving theory, in the sense that for any $d$--dimensional ancilla system and all free operations $\E$, $(\id_d \otimes \E ) \sigma \in \F$ whenever $\sigma \in \F$.
\end{enumerate}
The first assumption simply states that resources cannot be generated for free and is a minimal requirement for a resource theory. 
An immediate consequence is that if statistical mixing is included in $\O$, then the resource theory is convex.

Monotones are often used to address the question of state convertibility, although such approaches are usually generic.

The monotonicity condition reflects the no resource generating property of free operations, so that monotones respect the pre-order $\prec_\R$ of the theory.
A monotone of any general resource theory is a projection of the theory onto the non-negative real numbers, collapsing the pre-order of the theory to the total order defined on the real line.
Our contribution is the introduction of a generalised notion of \emph{resource projection} which maps a general resource theory onto a subtheory which in principle still retains a partial structure (as opposed to the real line).
Applying this notion on existing magic theories highlights the hidden stochasticity that governs magic state conversions.
We show that a magic theory can be subdivided into \emph{fragments} \nick{FIX AND EXPAND}

\subsection{Previous work}
\label{sec:prev}

The stabiliser theory of magic comprises of the so-called ``stabiliser'' states ($\stab$) and operations ($\so$), while non-stabiliser (resource) states are called magic.
The stabiliser operations can be expressed in terms of a Stinespring dilation as 
\begin{equation}
    \E(\rho) = \tr_E [U(\rho \otimes \sigma_E)U^\dagger],
\end{equation} 
for an ancilla stabiliser state $\sigma_E$. 
The motivation of the theory stems from the fact that stabiliser operations are experimentally straightforward to implement and they can be used to detect and correct errors on the stabiliser states due to their construction \cite{cit:gottesman, cit:gottesman2, cit:gottesman1998}.
The Gottesman-Knill theorem however indicates that stabiliser operations need to be supplemented with magic states in order to achieve universality, justifying the term ``magic''.

Generalisations of the stabiliser theory appear in the literature intending to include broader classes of operations.
The class of stabiliser preserving operations ($\spo$) is defined as the set of $\cptp$ maps that send stabiliser states into stabiliser states~\cite{cit:ahmadi}.
An important subclass of $\spo$ is the set of completely stabiliser preserving operations ($\cspo$)~\cite{cit:seddon}, which intuitively cannot induce ``non-stabiliserness'' even when applied to only part of a quantum state, i.e. operations $\E$ such that $(\id_d \otimes \E ) \sigma \in \stab$ for all positive dimensions $d$ whenever $\sigma \in \stab$.

Even though non-stabiliserness is a necessary resource for universality, it has been proven insufficient for magic state distillation~\cite{cit:bravyi, cit:campbell}.
In fact, all states with non-negative Wigner distributions have been proven to be efficiently classically simulable in~\cite{cit:mari}, a result that serves as a generalization of the Gottesman-Knill theorem.
The Wigner distribution of a state in odd prime dimensions, formally defined in~\cref{sec:wigner}, arises as the unique quasi-probability representation of quantum theory that identifies non-contextuality exactly with the states that are efficiently classically simulable~\cite{Delfosse_2017, cit:howard2, cit:veitch2}.
In this framework, the stabiliser states are the only pure states represented with non-negative distributions~\cite{cit:gross3}. 
However, there exist mixed states with non-negative Wigner distributions that are not mixtures of stabiliser states~\cite{cit:gross}.
Therefore, stabiliser-preserving theories have been extended to a theory that preserves state ``Wigner positivity''~\cite{Wang_2019}.
Informally, it can be considered as the maximal theory of magic $\Rmax = (\Fmax, \Omax)$, where free states have non-negative Wigner distributions and free operations completely preserve this property.\\ 

\ddd{Things we MUST emphasize:
\begin{enumerate}
\item Perhaps a nice lead-in question: ``What happens if we view stabiliser states as thermodynamic equilibrium states and magic as a form of free energy?''
\item We have found a scenario in which it is impossible to describe a thermodynamic structure using any entropic approach!
\item We can tackle more `physicsy' questions like: how much magic can be distilled via available operations with some given fixed-point structure? 
\item This allows a diagnostic on the kind of operations needed to do good distillation. I.e. what fixed point structure should they have?
\item We go beyond the concept of monotones and replace a monotone with a $\sigma$--fragment.
\item We can get both upper and lower bounds on magic distillation.
\end{enumerate}
}
%%%%%%%%%%%%%%%%%%%%%%%%%%%%%%%%%%%%%%%%

\newpage
\section{Phase space representations of quantum states}
\label{sec:ps}

Central to our construction is the representation of any quantum state or quantum operation on a system of dimension $d$ in terms of quasi-probability representations on a discrete phase space \ddd{[CITE the review on quasi-reps-- I think Chris Ferrie has one that is good.]}\nick{which one?}. This construction is a discrete version of Wigner representations in quantum optics.

Consider a $d$--dimensional quantum system with Hilbert space $\H_d$, and let $\{ |0\>, |1\>, \dots , |d-1\>\}$ denote the standard computational basis, defined over $\mathbb{Z}_d = \{ 0, 1, \dots,d-1 \}$. On this space, generalised Pauli matrices $X, Z$ can be defined by their respective roles as shift and phase operators, acting on the basis states as follows,
\begin{align}
    X \ket{k} &= \ket{k + 1} \label{eq:xpauli}\\
	Z \ket{k} &= \omega^k \ket{k}. \label{eq:zpauli}
\end{align}
Here $\omega \coloneqq e^{2\pi i/d}$ is the $d$-th root of unity and addition is taken modulo $d$. From these we can construct a phase space $\cal{P}_d = \mathbb{Z}_d \times \mathbb{Z}_d$ that provides a complete representation of the quantum system. Given a point $\bmx \coloneqq (x, p)$ we define a displacement operator, 
\begin{equation}\label{eq:ddef}
    D_{\bmx} \coloneqq \tau^{x p} X^{x} Z^{p},\ 
\end{equation}
where the phase factor $\tau \coloneqq -\omega^{1/2}$ ensures unitarity. We assume going forward that $d$ is an odd prime, however the case $d=2$ can also be handled, but with some additional technical caveats~\cite{Appleby_2005}. For a composite system with composite dimension $d = d_1 \dots d_n$ we can decompose the Hilbert space as $\cal{H}_d = \cal{H}_{d_1} \otimes \dots \otimes \cal{H}_{d_n}$, and then define displacement operators as
\begin{equation}\label{eq:composited}
    D_{\bmx} \coloneqq D_{(x_1, p_1)} \otimes \dots \otimes D_{(x_n, p_n)},
\end{equation}
where now we have
\begin{align*}
	\bmx \coloneqq (x_1, p_1, x_2, p_2, \dots, x_n, p_n) \in \cal{P}_{d_1} \times \dots \times \cal{P}_{d_n} \eqqcolon  \cal{P}_d,
\end{align*}
to denote the phase space point for the composite system. 
For simplicity going forward, we assume $n$ copies of a $d$--dimensional system $d_1=d_2 = \cdots = d$, and therefore, we have that $\x \in \mathbb{Z}_d^{2n}$.


The displacement operators form the Heisenberg-Weyl group~\cite{Folland_1989, Bengtsson_2006} under matrix multiplication modulo phases,
\begin{equation}\label{eq:gp}
    {\rm{HW}}_d^n \coloneqq \{ \tau^k D_{\bmx}: k \in \mathbb{Z}_d, \bmx \in \cal{P}_d^n\}.
\end{equation}
The Clifford operations $ \cal{C}_d^n $ are then defined as the set of unitaries that normalise the Heisenberg-Weyl group~\cite{Appleby_2005}. We may define the pure stabiliser states as those states obtained by acting on $|0\>$ with Clifford unitaries~\cite{cit:gross3}. Finally, we define $\stab$ as the convex hull of all pure stabiliser states, namely all probabilistic mixtures of states of the form $U|0\>\<0|U^\dagger$ where $U$ is some Clifford unitary. 

\subsection{Wigner representations for quantum states and quantum operations}\label{sec:wigner}

In order to provide a complete decomposition of arbitrary quantum states and quantum operations we now define a complete basis of Hermitian observables that behaves naturally under the action of the Clifford group. At every point $\x \in \P_d$ we define the phase-point operator
\begin{equation}\label{eq:ax}
	A_{\bmx} \coloneqq \frac{1}{d} \sum_{\bmz \in \cal{P}_d} \omega^{\eta(\bmx, \bmz)} D_{\bmz}, 
\end{equation}
where $\eta(\bmx, \bmz)$ is the symplectic inner product between any two points $\x,\z \in \P_d$, and is given explicitly by
\begin{equation}
	\eta(\bmx, \bmz) \coloneqq \bmz^T \begin{pmatrix}
		0  & \id \\ %\mathbbm{O}_n  & \id_n \\
		-\id & 0 \\ %-\id_n & \mathbbm{O}_n \\
	\end{pmatrix} \bmx,
\end{equation}
where $0, \id$ denote the $n\times n$ zero and identity matrices.

The phase-point operators form an orthogonal (with respect to the Hilbert-Schmidt inner product) operator basis as discussed in~\cref{app:wigner}.
Therefore, any quantum state $\rho \in \cal{B}(\cal{H}_d)$ can be expressed as a linear combination of them,
\begin{equation}
    \rho = \sum_{\bmz \in \cal{P}_d} \W[\bmz]{\rho} A_{\bmz},
\end{equation}
where the coefficient vector $\W{\rho}$ is the Wigner distribution of state $\rho$,
\begin{equation}\label{eq:wstate}
    \W[\bmx]{\rho} \coloneqq \frac{1}{d}\tr[A_{\bmx} \rho].
\end{equation}


For any quantum state $\rho$, the Wigner distribution $W_\rho(\bmx)$ is readily seen to be a $d^2$-dimensional quasi-probability distribution over $\cal{P}_d$ (see~\cref{app:wigner} for details). More precisely, $W_\rho(\x)$ is a real-valued function on $\P_d$ with the property that $\sum_{\x} W_\rho(\bmx) = 1$.  In~\cref{fig:wstate_examples}, we show Wigner distributions of different types of qutrit states.


\begin{figure}%
    \centering
    \subfigure[][]{%
    \label{fig:maxmix}%
    \includegraphics[height=2cm]{figs/maxmixed.pdf}
    %\caption{Maximally mixed state $\frac{1}{3}\id$}%
    }\hspace{8pt}%
    \subfigure[][]{%
    \label{fig:zero}%
    \includegraphics[height=2cm]{figs/zerostate.pdf}
    %\caption{Zero state $\ketbra{0}{0}$}%
    }\\
    \subfigure[][]{%
    \label{fig:bound}%
    \includegraphics[height=2cm]{figs/boundstate.pdf}
    %\caption{Bound state}%
    }\hspace{8pt}%
    \subfigure[][]{%
    \label{fig:strange}%
    \includegraphics[height=2cm]{figs/strangestate.pdf}
    %\caption{Strange state $\ketbra{S}{S}$}%
    }
    \caption{\textbf{Qutrit Wigner distributions of varying magic.} 
    \subref{fig:maxmix} Maximally mixed state $\frac{1}{3}\id$; \subref{fig:zero} Stabilizer zero state $\ketbra{0}{0}$; \subref{fig:bound} A non-stabiliser Wigner-positive state; \subref{fig:strange} Magic Strange state $\ket{{\rm{S}}} = \frac{1}{\sqrt{2}}(\ket{1} - \ket{2})$, coined in~\cite{cit:veitch2}.
    }%
    \label{fig:wstate_examples}
\end{figure}


\subsection{Magic theories for quantum computation}
\label{sec:mono}

\nick{There are repetitions with text in introduction, this sections looks more introduction-like I think ?}\ddd{We'll organise things later - better to get key chunks down first.}

The state-injection model for quantum computation~\cite{cit:bravyi} is a key avenue to realising a full-scale, fault-tolerant quantum computer. Within this approach, Clifford unitaries can be done in a robust, fault-tolerant way. However, due to the Eastin-Knill theorem~\cite{Eastin_2009}, it is impossible to have a universal set of transversal gates and so while Clifford unitaries can be realised transversally one must find ways around the Eastin-Knill restriction. The state-injection approach supplements the Clifford gates with noisy quantum states, called `magic states', that are outside the set of stabliser states and, when incorporated into Clifford circuits, provide universality. The obstacle to this is that the states are invariably noisy and so protocols must be employed to purify many copies of the magic states and improve the overall performance of the induced quantum gates. A central question is then: given a number $n$ of noisy magic states how many purified magic states can we obtain within the fault-tolerant set of quantum operations?

To address such questions both concrete distillation protocols have been developed, and there has been analysis of bounds of distillation rates within natural theoretical frameworks. These frameworks provide theories of magic in which one views magic states as ``resource'' states with respect to a natural class of quantum operations that are consider cheap, or ``free''~\cite{Gour_2019}. One very natural class of free operations considered are those obtained from Clifford operations, measurements and the ability to discard quantum systems. However there are several other candidate frameworks. 

For any theory of magic a natural route to bounding the distillation rates obtainable is through the concept of a \emph{magic monotone}. A magic monotone is a real-valued function of any quantum state $\M(\rho)$ that is monotonically non-increasing under the free operations of the magic theory. More precisely $\M(\sigma) \le \M(\rho)$ whenever it is possible to convert $\rho$ into $\sigma$ using at least one of the free operations available (e.g. Clifford operations).

One of the most fundamental and commonly used magic monotones is the \emph{mana} of a state~\cite{cit:veitch2}, defined as
\begin{equation}
    \mana{\rho} \coloneqq \ln{(2\hspace{1pt}\sn{\rho}+1)},
\end{equation}
where the \emph{sum-negativity}~\cite{cit:veitch2} is the sum of the negative components in $\W{\rho}$,
\begin{equation}
    \sn{\rho} \coloneqq \sum\limits_{\bmx: \W[\bmx]{\rho} < 0} \abs{\W[\bmx]{\rho}}.
\end{equation}

Mana is an additive\footnote{It satisfies the condition $\mana{\rho_1 \otimes \rho_2} = \mana{\rho_1} + \mana{\rho_2}$ which is practical in distillation settings.} magic monotone, so it provides an analytical, necessary condition for many-copy magic state interconvertibility.

A range of monotones have been developed in the literature\cite{cit:howard, Wang_2020, Seddon_2021} and each provides a means to upper bound distillation rates within a particular theory of magic. In this work we shall consider a broad framework for magic, with conditions that any theory of magic should satisfy. However, in contrast to prior works we shall not construct magic monotones, but instead develop a generalisation to resource monotones. The approach we take will also allow us to address how distillation rates are constrained by physical properties -- such as a system having biased-noise, some equilibrium structure or operations that manifest invariances, such as axial symmetry around the qubit $Z$-axis.



%%%%%%%%%%%%%%%%%%%%%%%%%%%%%%%%%%%%%%%%

\subsection{Stochastic structure of magic theories}
\label{sec:struc}

Within the Wigner representation all negatively represented states are magic states, although in general the converse is not true. However, it is known that all positively represented states used in Clifford circuits admit an efficient classical description, and therefore it is natural to consider focus on describing those negatively represented magic states. The free states in any magic theory must therefore lie in the set
\begin{equation}
    \Fmax \coloneqq \{ \rho: \W[\bmx]{\rho} \geq 0 \text{ for all } \bmx \in \cal{P}_d\}.
\end{equation}
Our focus is on states with negativity, and so the particular choice of free states is not so important for our analysis. The remaining component that needs addressing is the set of quantum operations that is considered free, or easy. Stabilizer operations are the most natural set of operations as they admit a classically efficient description for uniform quantum circuits. However, other choices of operations have been considered in the literature \ddd{[CITE]}. Common to these theories is that free $\E$  is such that $id \otimes \E (\rho)$ has a positive Wigner distribution whenever $ \rho$ has a positive Wigner distribution. Therefore the condition that $\E$ does not generate Wigner negativity is an essential requirement of any magic theory.


Any quantum channel admits a Wigner representation, in terms of the phase space operators defined earlier. If $\E$ is some quantum channel from a quantum system $A$ to a quantum system $B$, and $\J(\E)$ is its associated Choi state then we can define
\begin{equation}
W_{\E}(\y |\x) := d_A^2 W_{\J(\E)}(\bar{\x} \oplus \y),
\end{equation}
where $\bar{x} =(x_1, -p_1, \dots , x_n, -p_n)$ can be viewed as the time-reversed version of $\x$ in the discrete phase space, where momenta are reflected while coordinates remain unchanged.

With this representation of a quantum channel, we can view $W_{\E}( \y | \x)$ as a transition matrix sending $\x \rightarrow \y$, which is consistent with the representation of quantum states, as
\begin{equation}
W_{\E(\rho)} (\y) = \sum_{\x \in \P_d} W_{\E}( \y | \x) W_\rho(\x),
\end{equation}
for any $\E$ and $\rho$. Now if $\E$ sends any positively represented quantum state to another positively represented quantum state then it can be shown~\cite{Wang_2019} that $W_{\E}(\y |\x)$ must form a stochastic matrix. In particular, all Clifford operations correspond to stochastic matrices in this Wigner representation. However, it is important to note that not all stochastic maps on the phase space correspond to valid quantum operations. In particular, the stochastic maps must respect the symplectic structure of the phase space and so this provides an additional non-trivial constraint.

In what follows we shall therefore assume that we have a magic theory $\R = (\F, \O)$ in which the free states $\F$ form a closed, convex set with each state represented by a positive Wigner function, while the free operations $\O$ are represented by stochastic maps acting on the discrete phase space.


\section{Breaking a magic theory into fragments}
We wish to consider how physical constraints that may be hardware specific, affect magic distillation rates. These are not a priori encoded in the formulation of the magic theory, and constitute additional physical conditions on top of the underlying resource-theoretic accounting. For example, there may be non-trivial thermal noise in the physical set-up or the accessible operations, there might be strong bias in the noise present, or we might be limited in how easily certain gates can be realised. \ddd{[find roughly related cites for these?]}

While we will not consider very model-specific limitations, we still want to inject in some physical constraints. We proceed by considering imposing equilibrium, or fixed-point, restrictions on the actual operations that are accessible to us. While such restrictions could be viewed as the existence of some non-trivial thermodynamic constraint, it could equally be viewed as encoding a high noise-bias or difficulty in realising a particular quantum gate. We can therefore define the following sub-theory of any magic theory $\R$.
\begin{definition}[\textbf{$\boldsymbol\sigma$--fragment}]\label{def:sigmafrag}
   Given a theory of magic $\R = (\F, \O)$, the \emph{$\sigma$--fragment of $\R$} is the sub-theory $\R_\sigma = (\F, \O_\sigma)$, in which the free operations are 
   \begin{equation}
        \O_\sigma \coloneqq \{ \E \in \O: \E(\sigma) = \sigma \},
    \end{equation}
namely those operations that leave the state $\sigma \in \F$ invariant.
\end{definition}
This notion provides a simple way to break up any theory into smaller, more manageable parts that are destinguished by additional physical constraints. Moreover, the union over all fragments returns the parent theory, and so we are not discarding any information about the theory by breaking it up in this way. This can be made precise as follows.
\begin{theorem}\label{thm:frag}
    Let $\R = (\F, \O)$ be a theory of magic.
Then a transformation $\rho \longrightarrow \tau$ is possible in $\R$ if and only if the transformation $\rho \longrightarrow \tau$ is possible in at least one $\sigma$--fragment of $\R$.
\end{theorem}
The proof of this is very straightforward.
\begin{proof}
    Suppose the interconversion is possible in a $\sigma$--fragment via some $\E \in \O_\sigma$. But since $\O_\sigma \subseteq \O$ it is also possible in $\R$. Conversely, suppose the transformation is possible in $\R$ via some $\E$ in $\O$. The free states $\F$ are a closed, convex set and moreover the image of $\F$ under the map $\E$ is in $\F$. By the Brouwer Fixed-Point theorem~\cite{cit:brouwer} this mapping must therefore have a fixed point $\sigma \in \F$. Therefore $\E \in \O_\sigma$ and so the interconversion is possible in this fragment.
\end{proof}

This implies that we can now consider magic distillation rates within a particular $\sigma$--fragment and then see how the bounds on the distillation rates depend on the particular fragment we work within. Therefore, this route can potentially shed light on how the physics of a protocol constrain distillation rates above and beyond global constraints from the parent resource theory. For example, how does temperature affect distillation bounds? Do isotropic maps have better bounds, or does bias towards a pure stabilizer state fixed point provide more freedom?

\begin{figure}[t]
    \centering
        \includegraphics[scale=0.3
        ]{figs/operations.pdf}
    \caption{\textbf{Decomposition of a magic theory $\R$ intro $\sigma$--fragments.} 
	Examples of magic theories ($\so$: Stabilizer operations, $\Omax$: Completely positive-Wigner-preserving operations, $\rcu$: Random Clifford Unitaries -- subclass of $\so$) involve operations denoted by the two yellow regions, with other established magic theories  between them.
    We introduce $\sigma$--fragments $\O_\sigma$ defined for all free states $\sigma$ that cover $\Omax$. 
    Each $\O_\sigma$ is extensible to a set of stochastic maps outside the $\cptp$ operations.
    Within each $\sigma$--fragment $\bmd$--majorisation can be used allowing for tractable approach towards the study of magic state interconversion.
    }
    \label{fig:zoo}
\end{figure}

More importantly, by adding in this additional structure to the magic theory, it actually allows us to say more than would be possible at the level of the full parent theory. In particular, the interconversion structure in each fragment can be analysed using tools from single-shot thermodynamics and majorization theory as we discuss in the next section.

\subsection{Quantifying disorder without entropies}
\label{sec:major}

We can now show that each magic fragment admits constraints that do not take the form of state monotones. The development of any monotone $\M(\rho)$ for a magic theory $\R$ can be viewed abstractly as
\begin{equation}
\M : \mbox{(pre-order of states)} \longrightarrow \mbox{(total order of numerical values).} \nonumber
\end{equation}
More precisely, within the theory we order magic states as $\rho_1 \succ \rho_2$ whenever there is an operation $\E \in \O$ such that $\E(\rho_1) = \rho_2$. However, in general many states will not be interconvertible in either direction, and so this ordering is a pre-order in the sense that incomparable states exist. A magic monotone projects this pre-order of states onto the number line, and so through this lens any two states are directly comparable in terms of the values of $\M(\rho_1)$ and $\M(\rho_2)$. Of course, the price is that the monotone provides only a necessary, but not sufficient, condition for the transformation $\rho_1 \rightarrow \rho_2$.

The restriction of a magic theory $\R$ to some fragment $\R_\sigma$ can now be interpreted as a generalisation of the magic monotone construction: instead of mapping the pre-order of states in $\R$ onto the number line, we instead map this pre-order onto \emph{a simpler pre-order of states.} Therefore, we are retaining more of the original parent theory, while working in a scenario that admits greater structure and simplicity. We can view this as a mapping $\P_\sigma: \R \rightarrow \R_\sigma$ that restricts to a sub-theory (a $\sigma$--fragment) of the parent theory $\R$. Thus,
\begin{equation}
\P_\sigma: \mbox{(pre-order)} \longrightarrow \mbox{(simpler pre-order).}
\end{equation}
By framing the projection onto a fragment in this light, we can view Theorem \ref{thm:frag} as a statement that the collection $\{\P_\sigma\}$ provides a complete set of generalized monotones for the theory.

The question is now whether the sub-theories can be analysed to provide additional insights into magic distillation protocols. We shall do this by working in the Wigner representation of a given $\sigma$--fragment. The problem of magic state distillation then becomes that of processing negativity under stochastic maps $W_{\E}( \y |\x)$ that leave a particular quasi-distribution $W_\sigma(\x)$ invariant. However $\sigma \in \F$ is is positively represented and so the sub-theory has an invariant classical probability distribution as a fixed point. This structure allows us to make use of a range of classical tools, and in particular majorization theory.

Majorisation~\cite{cit:marshall} is a collection of powerful tools that has recently found many applications in quantum information theory~\cite{Nielsen_1999, cit:cwiklinski, cit:lostaglio2, cit:gour, cit:gour2, Horodecki_2003, Vallejos_2021}.
It describes the disorder of distributions that undergo stochastic transformations. In its simplest form it defines a pre-order on probability distributions. Given two distributions $\p= (p_1, \dots, p_n)$ and $\q = (q_1, \dots, q_n)$ over $n$ outcomes, we say that $\p$ majorises $\q$, denoted $\p \succ \q$, if there is a bistochastic map $A=(A_{ij})$ such that $A\p = \q$. Here, bistochastic means that $\sum_i A_{ij} = \sum_j A_{ij} = 1$. A well known result tells us that the condition $ \p \succ \q$ over probability distributions is equivalent to $n-1$ inequalities, and these conditions are easily checked.

However, there is a natural generalisation of this, called \emph{relative majorization} (also known as $d$--majorization, or in the context of thermodynamics, thermo-majorization) that dates back to the 1950s. For a fixed probability distribution $\r = (r_1, \dots, r_n)$ we define majorization relative to $\r$ as follows:
\begin{equation}
\p \succ_{\r} \q,
\end{equation}
if and only if there exists a stochastic map $A$ such that $A\r = \r$ and $A \p = \q$. In order words, the relative majorization pre-order coincides with the image of $\p$ under stochastic maps that have $\r$ as a fixed point. It is readily checked that the original majorization condition between probability distributions corresponds to the case $\r = (1/n, \dots, 1/n)$. The pre-order that results from relative majorization is also easily checked and again corresponds to checking a small number of inequalities.

However, it turns out that relative majorization is equally applicable to \emph{quasi-probability} distributions, and therefore can be applied to the Wigner representation of a given fragment. To our knowledge, this application of majorization to quasi-distributions in  physics has not been considered previously. The caveat here is that relative majorization ranges over all stochastic maps with a given fixed point, while in the case of the Wigner representation there is an additional restriction that the symplectic structure is respected. This implies that any relative majorization conditions provide necessary, but not sufficient, conditions on magic state interconversions in the fragment. However, this observation raises a question that to our knowledge has not been previously considered in the classical statistical mechanics literature: what are the conditions for symplectic majorization of classical statistical mechanics? We shall return to this question later on when we discuss how our approach could be used to construct concrete lower-bounds on distillation rates.

\subsection{The relative majorization conditions on magic interconversion within a fragment}
The pre-order imposed by $\bmd$--majorisation admits a numerically efficient reformulation in terms of Lorenz curves.
Let the vector $\bmu^\downarrow$ denote the vector $\bmu \in \reals^d$ with its components arranged in non-increasing order.
\begin{definition}[\textbf{Lorenz curve}]\label{def:lc}
    Let $\bmw, \bmd \in \reals^d$, where the components of $\bmd$ are positive with $D = \sum_{i=1}^d d_i$ and denote by $\widetilde{\bmw} \coloneqq (w_1/d_1, \dots, w_d/\nick{d_d})^T$ the rescaled vector $\bmw$ by $\bmd$.
    
    Finally, denote by $\pi: \mathbb{Z}_d \mapsto \mathbb{Z}_d$ the permutation that sorts $\widetilde{\bmw}$, $(\widetilde{\bmw}^\downarrow)_i = w_{\pi(i)}$ for all $i=1,\dots,d$.
    
    Consider the piecewise linear curve obtained by joining the points $\{(0,0)\} \cup \{ (x_k, \lc{\bmw}{\bmd}(k)) \}_{k=1,\dots,d}$, where
    \begin{equation}\label{eq:lorenz}
        (x_k, \lc{\bmw}{\bmd}(k)) \coloneqq \left( \frac{1}{D}\sum_{i=1}^k d_{\pi(i)}, \sum_{i=1}^k w_{\pi(i)} \right).
    \end{equation}
    We define the set of points on this curve, $\lc{\bmw}{\bmd}(x),\ x \in [0,1]$, as the \emph{Lorenz curve} of vector $\bmw$ with respect to $\bmd$.
\end{definition}
Components $x_k$ are rescaled by $D$ so that comparison of curves with unequal dimensions is possible.
In fact, the Lorenz curves $\lc{\bmw}{\bmd}$ and $\lc{\bmw \otimes \bmd}{\bmd \otimes \bmd}$, where $\otimes$ denotes the Kronecker product, coincide.
Furthermore, a Lorenz curve $\lc{\bmw}{\bmd}(x)$ is always concave in $x$, since it consists of $d$ line segments each with slope $(\widetilde{\bmw}^\downarrow)_i$ for $i=1,\dots,d$ which by definition is a non-increasing sequence.
Finally, the points on the interior of the Lorenz curve that connect line segments of different slopes are in general non-differentiable and we call them\emph{elbows}.

A vector $\bmy$ is said to \emph{$\bmd$--majorise} another vector $\bmx$ if and only if the Lorenz curve $\lc{\bmy}{\bmd}$ lies above Lorenz curve $\lc{\bmx}{\bmd}$, thus reducing $\bmd$--majorisation into a finite set of inequalities.
\begin{theorem}\label{thm:dmajor}
    Let $\bmx, \bmy, \bmd \in \reals^d$, such that the components of $\bmd$ are positive. 
    Then, $\bmx \prec_{\bmd} \bmy$ if and only if $\lc{\bmx}{\bmd}(x) \leq \lc{\bmy}{\bmd}(x)$ for all $x \in [0,1]$ with strict equality at $x=1$.
\end{theorem}
A restatement of the theorem including more equivalent conditions and a proof are provided in~\cref{app:major}.
	


\subsection{Majorisation of quasi-probabilities in $\sigma$--fragments}\label{sec:major_frag}

We can approach any magic theory through a thermodynamic lens, and in doing so we are provided with valuable insights on the structure of the theory. 
Firstly, any free state, for example a stabiliser state, can be viewed as a thermal state $\gamma_\beta$.
Without loss of generality we can always write a quantum state $\sigma$ as a thermal state, $\sigma = \gamma_\beta \coloneqq \frac{1}{\Z_\beta} e^{-\beta H}$ for some $\beta \geq 0$ and Hamiltonian $H$ (either effective or actual)\footnote{Technicalities arise for the case where $\sigma$ is not full rank, but this can be still described via $ \beta \rightarrow 0$ limiting process.}.
We can also view the set of free operations $\O_\sigma$ as a counterpart of thermal operations, in the sense that any operation $\E$ in $\O_\sigma$ preserves state $\sigma$. 

It is then apparent that the pre-order $\prec_{\R_\sigma}$ between the operations in the $\sigma$--fragment follows the rules of $\bmd$--majorisation as outlined in~\cref{sec:major}.
In particular, the pre-order $\prec_{\R_\sigma}$ of the $\sigma$--fragment $\R_\sigma = (\F, \O_\sigma)$ between $d$--dimensional states corresponds to the majorisation pre-order $\prec_{\W{\sigma}}$ between their $d^2$--dimensional Wigner distributions.
For simplicity we shall merge the notation into $\prec_\sigma$, as there is little risk of confusion.

Note that the Wigner components of an $n$--copy state $\rho^{\otimes n}$ can be calculated directly from $\W{\rho}$ by convolution of the distribution with itself,
\begin{equation}
	\W{\rho^{\otimes n}} = \W{\rho}^{\otimes n},
\end{equation}
where $\otimes$ can be interpreted as the usual Kronecker product in the last expression.
We may use the vector notation $\bmw(\rho) = (w(\rho)_i)_{i=1,\dots,d^2}$ for the Wigner distribution, in which case we can express the Kronecker product as
\begin{equation}
	\W{\rho} \otimes \W{\rho} = (w(\rho)_i w(\rho)_j)_{i,j=1,\dots,d^2}.
\end{equation}
The correspondence between the phase space and pure vector representations of the Wigner distribution is discussed more in~\cref{app:cmpairs}.
The vector notation will be useful in the proof of our main result,~\cref{thm:stab_bounds}

Furthermore, we restrict our analysis to $\sigma$--fragments, where $\sigma$ is full-rank.
This is justified as majorisation is continuous between fragments, in the sense that the pre-orders $\prec_\sigma$ and $\prec_{\sigma'}$ are equivalent for states $\sigma$ and $\sigma'$ which are $\epsilon$--close. \nick{CHECK}
Therefore, majorisation analysis is robust under imperfections in the experimental implementation of quantum operations.

In particular, we can always use common noise effects to approximate any $\sigma$--fragment where $\sigma$ is not full-rank by the $\sigma'$--fragment where $\sigma'$ is a noisy approximation of $\sigma$. 
For example, inducing depolarising noise, we can write $\sigma' = (1-\epsilon)\sigma + \epsilon\frac{1}{d}\id$, for some infinitesimal $\epsilon > 0$, so that $\sigma'$ is full-rank and arbitrarily close to $\sigma$.
Important examples of such $\sigma$--fragments include pure stabiliser states which are rank-$1$, e.g. the zero state depicted in~\cref{fig:zero}. 
Operations in such fragments include important stabiliser operations like the replacement channel, $\E(\rho) = \sigma$ for all states $\rho$.

\begin{theorem}\label{thm:sigmamajor}
    Let $\R = (\F, \O)$ be a theory of magic. If $\rho \longrightarrow \tau$  in $\R$ then  $\W{\tau} \prec_\sigma \W{\rho}$ within at least one $\sigma$--fragment.
\end{theorem}
\begin{proof}
Suppose we can convert $\rho$ into $\tau$ in the magic theory. 
Thus there is some $\O_\sigma$ and some $\E \in \O_\sigma$ such that $\E(\rho) = \tau$, and $\E(\sigma)=\sigma$. 
Therefore, the Wigner distribution of this free operation satisfies $\W{\E} \in \stochw$ and $\W{\E}\W{\rho} = \W{\tau}$. 
Since $\sigma$ is full-rank and free, its Wigner distribution is strictly positive in all components, so it directly follows from~\cref{def:dmajor} that $\W{\tau} \prec_{\sigma} \W{\rho}$.
\end{proof}
The converse is in general not true, since stochastic matrices do not necessarily correspond to valid quantum operations.

The result of~\cref{thm:sigmamajor} can be understood as an extension of the idea of a magic monotone, where we replace $\M(\tau) \leq \M(\rho)$ with $\W{\tau} \prec_\sigma \W{\rho}$. 
The physical difference between the two expressions is that the majorisation ordering occurs in a specific $\sigma$--fragment.
Therefore, majorisation constraints can be used to place upper bounds on magic state distillation in a way that allows one to incorporate the physics of the allowed operations -- specifically, it enables us to bound how much magic can be distilled via quantum operations that, for example, preserve the equilibrium state of the system, or via operations that are symmetric about the $Z$-axis of the Bloch sphere.
We discuss distillation upper bounds in detail in~\cref{sec:unital,sec:stab}

This approach can also provide \emph{lower bounds} on distillation, however now more structure about the specific free operations must be included. 
We briefly discuss distillation lower bounds in~\cref{sec:lower_bounds}.

\subsection{Lorenz curves of quasi-probabilities in $\sigma$--fragments}
\label{sec:lc}

It is straightforward to construct Lorenz curves for Wigner distributions in any $\sigma$--fragment.
As we have seen in~\cref{thm:sigmamajor}, for any full-rank free state $\sigma$ we have that $\W{\sigma}$ is a strictly positive full-rank probability distribution, and so one can define a corresponding notion of $\bmd$--majorisation on \emph{quasi}--distributions.
We write $\lc{\rho}{\sigma}(x)$ for the Lorenz curve of $\W{\rho}$ with respect to $\W{\sigma}$.
The vector of ratios $\widetilde{\bmw}(\rho|\sigma)$ used to construct the curve is
\begin{equation}
	\widetilde{w}(\rho|\sigma)_i \coloneqq \frac{w(\rho)_i}{w(\sigma)_i},
\end{equation}
and is called the \emph{rescaled} distribution of $\rho$ with respect to $\sigma$.

An example of comparison between different Lorenz curves is provided in~\cref{fig:lctoy}.
The curves in the figure are constructed in the \emph{unital fragment}, i.e. the $\sigma$--fragment defined by the maximally mixed state $\sigma = \frac{1}{d}\id$ whose Wigner distribution is the uniform probability distribution.

\begin{figure}
    \centering
    \includegraphics[height=4.5cm]{figs/lc_strange.pdf}
    \caption{\textbf{A `heretical' family of Lorenz curves.} Traditionally, Lorenz curves are monotone increasing cumulant functions that reach a maximum value of $1$. In contrast, Lorenz curves for magic states break through the value of $L(x)=1$ due to the presence of negativity in the associated quasi-probability distributions. The above family of curves correspond to multiple copies of noisy Strange states $\rho_{\rm{S}}(\epsilon)^{\otimes n}$ for $n=1,2,3,4$ within a particular stabilizer fragment. Solid lines represent pure Strange states, while dashed lines represent noisy Strange states with depolarising noise $\epsilon = 0.1$.
    }
    \label{fig:lcs}
\end{figure}

\begin{figure}
    \centering
    \includegraphics[height=4.5cm]{figs/lctoy.pdf}
    \caption{\textbf{Quasi-probability Lorenz curve comparison}.
    The Lorenz curves are constructed by the Wigner distributions illustrated in~\cref{fig:wstate_examples} in the unital fragment $\O_{\id/d}$ for $d=3$.
    The maximally mixed state curve is simply the line connecting $(0,0)$ and $(1,1)$.
    There is no operation in $\O_{\id/d}$ that can convert $\ket{\rm{S}}$ to $\ket{0}$, as their Lorenz curves intersect.\ddd{[Is debatable if this figure gives much...is a bit boring.]}
    }
    \label{fig:lctoy}
\end{figure}


Normalisation of the Wigner distribution ensures that for all quantum states $\rho$, $\lc{\rho}{\sigma}(x) \geq 0$ and $\lc{\rho}{\sigma}(1) = 1$.
We stress that $0 \leq \lc{\rho}{\sigma}(x) \leq 1$ for all $x \in [0,1]$ if and only if $\rho$ is a positive Wigner state.
As a consequence, checking whether a magic state conversion of the form
\begin{equation}\label{eq:sigma_conv}
	\rho \xrightarrow{\E \in \O_\sigma} \tau
\end{equation} 
is not possible, reduces to the set of constraints
\begin{equation}\label{eq:majbound}
    \lc{\rho}{\sigma}(x) \geq \lc{\tau}{\sigma}(x),\ x\in [0,1],
\end{equation}
due to~\cref{thm:sigmamajor,thm:dmajor}.

We can refine the number of independent constraints stemming from this inequality. 
In fact, there are only as many independent constraints as there are elbows in the Lorenz curve of the target state as shown in~\cref{thm:elbows} in~\cref{app:frag}.
However, in principle any one location $x$ provides a valid constraint leading to some upper distillation bound, while optimising over the location would provide the strictest bound.
In particular, we establish the \emph{first elbow constraint} which we use in~\cref{sec:unital,sec:stab} explicitly.
\begin{lemma}
	Consider a magic state interconversion as in~\cref{eq:sigma_conv}, where we denote by $(x_0, L_0)$ and $(x'_0, L'_0)$ the first elbow coordinates of the initial and target states respectively.
	If $x_0 < x'_0$, then it holds that
\begin{equation}\label{eq:first_elb_bound1}
	\frac{L_0}{x_0} \geq \frac{L_0'}{x_0'}.
\end{equation}
\end{lemma}
\begin{proof}
	Consider the Lorenz curve constraint at $x = x_0$,
\begin{equation}
	\lc{\rho}{\sigma}(x_0) \geq \lc{\tau}{\sigma}(x_0).
\end{equation}
Since $x_0 < x'_0$, we can find the target state Lorenz curve coordinate $L'_\star$ at location $x = x_0$ by interpolating between the origin and the target state's first elbow, 
\begin{equation}
	L'_\star = \frac{x_0}{x'_0}L'_0.
\end{equation}
We need $L_0 \geq L_\star'$ and rearranging we get~\cref{eq:first_elb_bound1}.
\end{proof}

We can now turn the attention from the first elbow to the peak of the Lorenz curve to associate it directly with the magic monotone of mana, and equivalently the sum-negativity of a quantum state. 
This holds independently of the particular $\sigma$--fragment one works in and as a result it becomes apparent that mana provides a weaker condition than majorisation for all magic state interconversions.

We first show that the Lorenz curve maximum of state $\rho$ is independent of the $\sigma$--fragment and directly related to its sum-negativity.
\begin{lemma}\label{lem:lcmax}
	Given a quantum state $\rho$, the maximum of its Lorenz curve $\lc{\rho}{\sigma}$ is independent of the $\sigma$--fragment and is equal to $1+\sn{\rho}$.
\end{lemma}
\begin{proof}
	We may use the vector notation of the Wigner distributions $\bmw(\rho)$ and $\bmw(\sigma)$.
	We choose the component indexing so that the rescaled distribution 
	\begin{equation}
		\widetilde{\bm{w}}(\rho|\sigma) \coloneqq \left(\frac{w(\rho)_1}{w(\sigma)_1}, \dots, \frac{w(\rho)_{d^2}}{w(\sigma)_{d^2}} \right)^T,
	\end{equation}
	is sorted, $\widetilde{\bm{w}} = \widetilde{\bm{w}}^\downarrow$.
	Note that all components of $\bmw(\sigma)$ are positive, so $\widetilde{w}_i \geq 0$ if and only if $w(\rho)_i \geq 0$ for any $i=1,\dots,d^2$.
	
	Let $i_\star$ be the index of the smallest non-negative component of $\widetilde{\bm{w}}^\downarrow$.
	Then, $w(\rho)_i < 0$ if and only if $i > i_\star$, so the maximum of Lorenz curve $\lc{\rho}{\sigma}(x)$ takes the value 
	\begin{equation}
		\lc{\rho}{\sigma}(x_{i_\star}) = \sum_{i=1}^{i_\star} w(\rho)_i,
	\end{equation}
	and is achieved at
	\begin{equation}\label{eq:maxloc}
		x_{i_\star} \coloneqq \sum_{i=1}^{i_\star} w(\sigma)_i.
	\end{equation}

	The location of the maximum ($x=x_{i_\star}$) varies from fragment to fragment, but its value is independent of $\sigma$,
	\begin{equation}
		\lc{\rho}{\sigma}(x_{i_\star})
		= \sum\limits_{\bmx: \W[\bmx]{\rho} \geq 0} \W[\bmx]{\rho}
		= 1 + \sn{\rho}.
	\end{equation}
	
\end{proof}

We can therefore view mana as just one feature of the Lorenz curve, namely its maximum value. 
Conversely, it is now clear that the maximum of the Lorenz curve acts as a valid magic monotone.
\begin{theorem}\label{thm:bounds}
    Given a magic state conversion $\rho \longrightarrow \tau$, the majorisation condition is stronger than the mana condition in every $\sigma$--fragment.
\end{theorem}
\begin{proof}
    The maximum of the Lorenz curve of a state $\rho$ is independent of the $\sigma$--fragment due to~\cref{lem:lcmax}, and can be expressed as an increasing function of mana,
    \begin{equation}
        \max_{x\in[0,1]}{\lc{\rho}{\sigma}(x)} = 1 + \sn{\rho} = \frac{1}{2} \left( 1 + e^\mana{\rho} \right).
    \end{equation}
    Therefore, the majorisation bound
    \begin{equation}
    	\lc{\rho}{\sigma}(x) \geq \lc{\tau}{\sigma}(x),\ x\in[0,1]
    \end{equation}
    implies the order $\max_{x\in[0,1]}{\lc{\rho}{\sigma}(x)} \geq \max_{x\in[0,1]}{\lc{\tau}{\sigma}(x)}$, hence the mana condition $\mana{\rho} \geq \mana{\tau}$.
\end{proof}

The area $\A_\sigma(\rho)$ between the curve $L_{\rho|\sigma}$ and the line $y=1$ is also a resource monotone in the $\sigma$--fragment. 
This is clear because for any state conversion like~\cref{eq:sigma_conv}, the Lorenz curve $L_{\tau|\sigma}$ is lower than $L_{\rho|\sigma}$, hence $\A_\sigma(\E(\rho)) \leq \A_\sigma(\rho)$.
It cannot be directly expressed in terms of mana, as it depends on positive Wigner components as well.

In~\cref{sec:unital,sec:stab}, we study majorisation constraints on magic distillation arising in different fragments. 
For this reason, we define the $n$--copy, $\epsilon$--noisy Strange state,
\begin{equation}\label{eq:noisysn}
    \rho_{\rm{S}}(\epsilon)^{\otimes n} \coloneqq \left[ (1 - \epsilon) \ket{\rm{S}}\bra{\rm{S}} + \epsilon \frac{1}{3}\id \right]^{\otimes n},
\end{equation}
in $\O_\sigma$, where the pure magic state $\ket{\rm{S}}$ is induced with depolarising noise. 
Its Wigner distribution is visualised in~\cref{fig:strange}.

We refer to parameter $\epsilon$ as \emph{noise level} to avoid confusion with the error rate $\delta$, a term commonly used in the literature to denote that the distilled state has a marginal overlap of at least $1-\delta$ with the desired magic state.
In our case the error rate of the state in~\cref{eq:noisysn} would be $\delta = \frac{2}{3}\epsilon$.

At $\epsilon=0$, the Strange state is a qutrit magic state of maximal sum-negativity / mana~\cite{cit:veitch2} and therefore acts as an ideal distillation target, analogous to a Bell state in bipartite entanglement theory.
The Strange state is exceptionally symmetric under Clifford transformations and as a result there exists a \emph{twirling} protocol, discussed in detail in~\cite{cit:prakash,cit:prakash2}, allowing for the conversion of any noisy magic state to the form of~\cref{eq:noisysn} via Cliffords.

In~\cref{fig:lcs}, we illustrate the Lorenz curve $\lc{\rho_{\rm{S}}}{\sigma}$\footnote{Hereinafter, we may omit obvious variable dependencies like $\epsilon$ and $n$ for clarity} of pure and noisy $n$--copy Strange states in some thermal $\sigma$--fragment.
Due to~\cref{lem:lcmax}, it is clear that the curves peak at $1 + \sn{\rho_{\rm{S}}(\epsilon)^{\otimes n}}$.


\section{Magic in the unital fragment}
\label{sec:unital}
We shall provide a general analysis across multiple fragments, but before we do that it is useful to illustrate the techniques in the simplest fragment first. The unital fragment encompasses the circuits which preserve the maximally mixed state ($\id/d$) and so it includes many important families of circuits.

MSD circuits in principle consist of bulk sequences of random Clifford unitaries ($\rcu$)~\cite{cit:bravyi}, depicted in~\cref{fig:zoo}.
Operations in $\rcu$ can be expressed as
\begin{equation}
    \E(\rho) = \sum_i p_i U_i \rho U_i^\dagger,\ U_i \in \cal{C}_d.
\end{equation}
Depending on the symmetries of such operations, a Clifford sequence may belong in other $\sigma$--fragments as well.
In such a case, the majorisation condition~(\ref{eq:majbound}) needs to be checked in the $\sigma$--fragments that reflect all symmetries of the operation sequence.

In general, noisy circuits are well-described by the unital fragment.
To see this, consider incorporating noisy channels in the circuit, for example dephasing channels as in~\cref{eq:dephase} defined in different bases.
This process destroys the circuit symmetries, except for the invariance of the maximally mixed state.
Dephasing and bit-flip error channels are examples of the many error-inducing channels that respect the unital symmetry. \nick{Expand on significance of unital fragment}
\null\\

We consider the task of purifying $n$ copies of a noisy Strange state $\rho_{\rm{S}}(\epsilon)^{\otimes n}$ as given in~\cref{eq:noisysn} into a smaller number of copies $n'$ of a less noisy strange state $\rho_{\rm{S}}(\epsilon')^{\otimes n'}$, with $\epsilon' < \epsilon$ and $n' \leq n$,
\begin{equation}\label{eq:sudist}
	\rho_{\rm{S}}(\epsilon)^{\otimes n} \longrightarrow \rho_{\rm{S}}(\epsilon')^{\otimes n'} \otimes \left( \frac{1}{3}\id \right)^{\otimes (n-n')},
\end{equation}
where all copies $n, n', n - n'$ are even.
Since the state $\id / 3$ is free, tensoring in copies of it does not affect the distillation process.
The distillation rate $R \coloneqq n'/n$ for this process will in general depend on the noise levels, $R = R(\epsilon, \epsilon')$ and our task is to provide it with an upper bound.

The Lorenz curve $\lc{\rho_{\rm{S}}}{\id/3}$, for some general noise parameter $\epsilon$ and number of copies $n$, is defined at $9^n$ points between $0$ and $1$.
The exact expressions for the coordinates of these points can take $8$ different forms, depending on whether the noise level $\epsilon$ is greater or less than $\frac{3}{7}$, the parity of the number of copies $n$ is even or odd and the location relative to the curve peak is on the left hand side (LHS -- including the curve peak) or right hand side (RHS) of the curve peak.
The full details for the construction of all Lorenz curve forms are provided in~\cref{app:lcsu_technical}.

Here we focus on the comparison of the first elbow (which lies in the LHS part) of Lorenz curves with even copies $n, n'$ and low noise levels ($\epsilon' < \epsilon \leq 3/7$).

The Wigner distribution of the 1-copy, $\epsilon$--noisy Strange state can be written as 
\begin{equation}
	\W[\bmx]{\rho_{\rm{S}}(\epsilon)} = (1-\epsilon)\W[\bmx]{\ketbra{\rm{S}}} + \epsilon\W[\bmx]{\frac{1}{3}\id},
\end{equation}
so we get positive components
\begin{equation}
	u(\epsilon) \coloneqq \frac{1}{6} -\frac{1}{18}\epsilon
\end{equation}
at the 8 phase space points $\bmx \in \cal{P}_3 \setminus \{\bmo\}$ and a negative component
\begin{equation}
	- v(\epsilon) \coloneqq - \left( \frac{1}{3} -\frac{4}{9}\epsilon \right)
\end{equation}
at the origin $\bmx = \bmo$.

The rescaled distribution in the unital fragment simply is 
\begin{equation}
	\widetilde{\rm{W}}_{\rho_{\rm{S}}|\frac{1}{3}\id}(\bmx) = \frac{\W[\bmx]{\rho_{\rm{S}}}}{\W[\bmx]{\frac{1}{3}\id}} = d\W[\bmx]{\rho_{\rm{S}}},
\end{equation}
so ordering the rescaled distribution is equivalent to ordering $\W{\rho_{\rm{S}}}$.

The component values and multiplicities ($m_i$) in the $n$--copy case are 
\begin{align}
	m_i &= 8^{2i}\binom{n}{2i}, \\
	w(\rho_{\rm{S}})_i &= u^{2i}(-v)^{n-2i}, \\
	w(\id/3)_i &=  \frac{1}{9^n},
\end{align}
where index $i$ runs through $0,\dots,\frac{n}{2}$.
This is derived in~\cref{app:lcsu_coord_elb}.

It is readily seen that the maximum Wigner component is achieved when $i=0$, so the corresponding multiplicity and Wigner component are $m_0 = 1$, $w(\rho_{\rm{S}})_0 = v^n$.
The first elbow coordinates therefore can be expressed as
\begin{equation}\label{eq:first_elb_coords}
	(x_0, L_0) = (m_0\hspace{2pt} w(\id/3)_0,\ m_0\hspace{2pt} w(\rho_{\rm{S}})_0) = \left(\frac{1}{9^n}, v^n \right)
\end{equation}

The first elbow of the target state is located at $x'_0 = \frac{1}{9^{n'}} > x_0$, so we can use the first elbow condition,
\begin{equation}\label{eq:first_elb_bound2}
	\frac{L_0}{x_0} \geq \frac{L_0'}{x_0'},
\end{equation}
to compute analytical distillation bounds for distillation rate $R = R(\epsilon, \epsilon') \coloneqq n'/n$ in the unital fragment.
We substitute coordinates from~\cref{eq:first_elb_coords} appropriately in~\cref{eq:first_elb_bound2} to get
\begin{equation}
	R \leq \frac{\ln{(3-4\epsilon)}}{\ln{(3-4\epsilon')}}.
\end{equation}

Specifically for the problem of distilling pure magic states ($\epsilon'=0$), we obtain an upper bound in the unital fragment given by
\begin{equation}
	R \leq 1 + \frac{\ln (1 - \frac{4}{3} \epsilon)}{\ln 3}.
\end{equation}

Similar upper bounds on distillation rates for qudits of odd prime dimension include the mana bound~\cite{cit:veitch} and the max--thauma bound~\cite{Wang2020} which is defined via a semi-definite program, but possesses properties that allow for an easy comparison with our bound.
The mana bound can be directly calculated as
\begin{equation}
	R \leq \frac{\mana{\rho_{\rm{S}}(\epsilon)}}{\mana{\rho_{\rm{S}}(0)}} = 1 + \frac{\ln \left(1 - \frac{8}{15}\epsilon \right)}{\ln \frac{5}{3}}.
\end{equation}
Using the max--thaum properties of super-additivity and vanishing at free states, we obtain
\begin{align}
	&\theta_{\rm{max}}(\rho_{\rm{S}}(\epsilon)) \geq (1-\epsilon) \theta_{\rm{max}}(\ketbra{\rm{S}}) + \epsilon \theta_{\rm{max}}\left(\frac{1}{3}\id \right) = \nonumber\\
	&(1-\epsilon) \theta_{\rm{max}}(\rho_{\rm{S}}(0)),
\end{align}
so that for a deterministic single-copy distillation process, an initial number of copies $n \geq \theta_{\rm{max}}(\tau) / \theta_{\rm{max}}(\rho) \geq 1-\epsilon$ is required.
Therefore, the max--thauma bound for the pure Strange state distillation process can get as tight as
\begin{equation}
	R = \frac{1}{n} = \frac{\theta_{\rm{max}}(\rho_{\rm{S}}(\epsilon))}{\theta_{\rm{max}}(\rho_{\rm{S}}(0))} \leq 1-\epsilon.
\end{equation}

Both the mana and the max--thauma bounds are looser than the majorization bound we derived via the first elbow constraint as illustrated in~\cref{fig:distill_bounds}.
\begin{figure}[t]
    \centering
    \includegraphics[scale=0.45]{figs/distill_bounds.pdf}
    \caption{\textbf{Distillation bounds in the unital fragment.} Distillation bounds obtained by majorisation, mana and the line $1-\epsilon$ (tightest bound of max--thauma) are plotted for $\epsilon' = 0$, up to noise level $\epsilon = 3/7$.
    Majorisation consistently provides stricter rates than mana and max--thauma.
    }
    \label{fig:distill_bounds}
\end{figure}

\section{Magic bounds in arbitrary stabiliser fragments}
\label{sec:stab}

We now generalise the approach taken in the previous section and consider bounds on magic distillation for an arbitrary stabiliser fragment, $\R_\sigma$ where $\sigma$ is any quantum state $\sigma \in \stab$.
In other words, we consider those bounds on distillation that apply when the free operations have $\sigma$ as a fixed point.

These bounds can be interpreted in two different ways: on one hand they can be viewed as a family of upper bounds parameterized by a stabiliser state $\sigma$, on the other we can associate $\sigma$ to actual hardware limitations or to biased noise models in which it is an equilibrium state of some kind. 
Without loss of generality, we can always write a stabiliser state $\sigma$ as a Gibbs state $\sigma = \gamma_\beta \coloneqq \frac{1}{\Z_\beta} e^{-\beta H}$ for some $\beta \geq 0$ and Hamiltonian $H$ as discussed in~\cref{sec:major_frag}.

We focus on the distillation process
\begin{equation}\label{eq:stdist}
	\rho_{\rm{S}}(\epsilon)^{\otimes n} \longrightarrow \rho_{\rm{S}}(\epsilon')^{\otimes n'} \otimes \sigma^{\otimes (n-n')},
\end{equation}
where the noisy Strange state is given in~\cref{eq:noisysn} and we highlight again that any magic state can be transformed to this form via Clifford operations.
Notice that tensoring in copies of $\gamma_\beta$ does not affect the process, since the state is free.
In the following result, all copies $n, n', n - n'$ are even with $n > n'$, and $\epsilon'=0$, but the bounds are easily generalised to odd numbers of copies and $\epsilon'$ such that $0 < \epsilon' < \epsilon$. 
Finally, we assume a range of initial noise levels, $\epsilon \leq 3/7$, so that the largest Wigner component of the 1-copy noisy magic state is negative.
The numerical value of $3/7$ is higher than corresponding values of relevant existing distillation protocol error thresholds~\cite{cit:bravyi,cit:prakash}.

Given this context, we now provide the following result on bounding the distillation rate $R = R(\epsilon, \epsilon', \beta) \coloneqq \frac{n'}{n}$ of the process in~\cref{eq:stdist}. \nick{I think we should denote distillation rates by $R$ and our distillation bound by $R(\epsilon, \epsilon', \beta)$, since the bound is what really depends on $\epsilon, \epsilon', \beta$.}
We also write $R(\epsilon, \beta) = R(\epsilon, \epsilon'=0, \beta)$.
The bounds depend on the free energy $F_\beta$ of state $\sigma$,
\begin{equation}
	F_\beta \coloneqq \tr[H \sigma] - \beta^{-1}S(\sigma) = -\beta^{-1}\log{\Z_\beta},
\end{equation}
where the von Neumann entropy is $S(\sigma) \coloneqq -\tr[\sigma\log{\sigma}]$.

\begin{figure}[t!]
    \centering
    \includegraphics[scale=0.55]{figs/rate_scatter.pdf}
    \caption{\textbf{Bounds on magic distillation rates $R(\epsilon, \beta)$ within arbitrary stabiliser fragments.}
    The vertical dashed line is the `Landauer-like' temperature threshold $\beta_\star$ and the diagonal dashed curve corresponds to the noise threshold $\epsilon_\star (\beta)$ at every $\beta \leq \beta_\star$. The unital fragment corresponds to the $\beta =0 $ line.
    }
    \label{fig:rate_contour}
\end{figure}

\begin{theorem}\label{thm:stab_bounds}
	Let $\sigma$ be any qutrit stabiliser state, and write the state as $\sigma = \frac{1}{\Z} e^{-\beta H}$ with $H$ having eigenvalues $E_0 \le E_1 \le E_2$, and where $\beta$ is an effective inverse temperature for the state. 
We define $\beta_\star = (1/kT_\star)$ through the relation
\begin{equation}
	E_2 - E_0 =: kT_\star \ln{2}
\end{equation}
and for $\beta \leq \beta_\star$, and define a threshold noise level
\begin{equation}
	\epsilon_{\star}(\beta) \coloneqq 3 - \dfrac{18}{8-e^{(E_2 - E_0)\beta}}.
\end{equation}
Then any distillation rate $R(\epsilon, \beta)$ in the $\sigma$--fragment of a magic theory is bounded as follows:\\
If $\beta \leq \beta_{\star}$ and $\epsilon  \leq \epsilon_{\star}$,
\begin{equation}
	R(\epsilon,\beta) \leq 1 + \frac{\ln{\left( 1 - \frac{4}{3}\epsilon \right)}}{\beta (E_0 - F_\beta)}.
\end{equation}
If $\beta \leq \beta_{\star}$ and $\epsilon_{\star} < \epsilon$, then
\begin{equation}
	R(\epsilon, \beta) \le 1 + \frac{\ln{\left(1-\frac{1}{3}\epsilon \right)} - (E_2 - E_0)(\beta_{\star} - \beta)}{\beta (E_0 - F_\beta)}.
\end{equation}
Otherwise if $\beta > \beta_{\star}$,
\begin{equation}
	R(\epsilon, \beta) \leq  1+ \frac{\ln{\left(1-\frac{1}{3}\epsilon \right)}}{-\ln{2} + \beta (E_2 - F_\beta)}.
\end{equation}
In the above expressions $F_\beta$ is the effective free energy of the stabilizer state.
\end{theorem}
A few comments can be given on this result. 
Firstly, the specific numerical factors in $\epsilon_\star$ are a result of our choice of magic state. 
Secondly, these bounds are derived based on analysis of only a part of the Lorenz curves and can be improved via a finer analysis. 
This is apparent by simple numerical calculations on the entirety of the curves.
Specifically, the existing bounds follow by considering the dominant terms in the rescaled Wigner distribution and do not, for example, take into account the Lorenz curve's peak structure. 
Finally, it is striking that we obtain a Landauer-like condition with a characteristic temperature $kT_\star \ln 2 = E_2 - E_0$, where $kT_\star = \beta_\star^{-1}$. 
It is unclear whether this points to a generic feature that can be directly related to fundamental thermodynamic relations, such as the erasure cost of a single bit being $kT \ln 2$. 
Given that we work with qutrits this seems surprising, but does deserve further study.

\begin{proof}
	We provide the full proof here, with some technical details pushed to~\cref{app:lcst_technical}.\\

Let $\displaystyle \sigma = e^{-\beta H} / \Z_\beta$ be a stabiliser state, where $\beta \geq 0$ and $H = E_0 \ketbra{0} + E_1 \ketbra{1} + E_2 \ketbra{2}$, with $E_k \geq 0$.
Its eigen-decomposition can be written as 
\begin{equation}
	\sigma = \frac{e^{-\beta E_0}}{\Z_\beta} \ketbra{\varphi_0} + \frac{e^{-\beta E_1}}{\Z_\beta} \ketbra{\varphi_1} + \frac{e^{-\beta E_2}}{\Z_\beta} \ketbra{\varphi_2},
\end{equation}
where $\{\ket{\varphi_k}\}$ are pure, orthonormal stabiliser states. These pure states can be represented in terms of Generalized Paulis. Let $C$ be the unitary transforming each $|\varphi_k\>\<\varphi_k|$ to $|k\>\<k|$. Since the Clifford group is the normalizer of the Heisenberg-Weyl group this implies that $C$ must be a Clifford unitary. Therefore, $C$ maps $\sigma$ to $\gamma_\beta$, where 
\begin{equation}
\gamma_\beta = \frac{e^{-\beta E_0}}{\Z_\beta} \ketbra{0} + \frac{e^{-\beta E_1}}{\Z_\beta} \ketbra{1} + \frac{e^{-\beta E_2}}{\Z_\beta} \ketbra{2},
\end{equation}
is a stabilizer state, diagonal in the computational basis. The Clifford operation permutes the Hamiltonian eigenvalues on the phase space, so that the negative component $-v$ of $\rho_S(\epsilon)$ can lie on the same point on the phase space as any of the eigenvalues.
For this reason, we impose no order between the eigenvalues, but simply choose $E_0$ as the eigenvalue that is associated with the state negativity and denote the highest energy by $E_{\rm{max}} \coloneqq \max{\{E_0, E_1, E_2\}}$.

The Wigner distribution of state $\gamma_\beta$ can be seen as the ensemble average of the distributions of the computational basis states,
\begin{align}
	\W[\bmx]{\gamma_\beta} &= \sum\limits_{k=0}^2 \frac{e^{-\beta E_k}}{\Z_\beta}\W[\bmx]{\ketbra{k}} \nonumber\\
	&= \sum\limits_{k=0}^2 \frac{e^{-\beta E_k}}{\Z_\beta} \delta_{x_0, k} = \frac{e^{-\beta E_{x_0}}}{3\Z_\beta},
\end{align}
for all $\bmx \in \cal{P}_3$. 
All Wigner components are strictly positive, therefore the pre-order $\prec_{\gamma_\beta}$ is always well-defined.

Our aim is to obtain a distillation bound for~\cref{eq:stdist} which depends on variables $n, n', \epsilon, \epsilon'$ as well as $\beta$.
In the analysis that follows, we again drop obvious variable dependencies for clarity.

We construct the Strange state rescaled distribution 
\begin{equation}
	\widetilde{\rm{W}}_{\rho_{\rm{S}}|\gamma}(\bmx) = \frac{\W[\bmx]{\rho_{\rm{S}}}}{\W[\bmx]{\gamma}},
\end{equation}
which attains four distinct values on the phase space, naturally splitting it into four regions as illustrated in the~\cref{fig:pd_split_thermal}.
\begin{figure}[h]
    \centering
    \includegraphics[scale=0.45]{figs/pd_split_thermal.pdf}
    \caption{\textbf{Qutrit phase space regions with different rescaled values.}
    The rescaled distribution attains a unique value in each of the four regions, given by $3\Z \times$ the value depicted in the region, according to~\cref{eq:bmw_rescaled}.
    }
    \label{fig:pd_split_thermal}
\end{figure}

The component values and multiplicities of the relevant distributions in the four distinct regions are summarised by the following component and multiplicity vectors,
\begin{align}
	\bmm &\coloneqq (1,2,3,3), \\
	\bmw(\rho_{\rm{S}}) &\coloneqq (-v, u, u, u), \\
	\bmw(\gamma) &\coloneqq \frac{1}{3\Z} \left( e^{-\beta E_0}, e^{-\beta E_0}, e^{-\beta E_1}, e^{-\beta E_2} \right), \\
	\bmw(\rho_{\rm{S}} | \gamma) &\coloneqq 3\Z \left( -v e^{\beta E_0}, u e^{\beta E_0}, u e^{\beta E_1}, u e^{\beta E_2} \right). \label{eq:bmw_rescaled}
\end{align}

Using this notation, the component values and multiplicities of the $n$--copy distributions can be readily provided by~\cref{lem:ncopycomponents} in~\cref{app:cmpairs}.
They are parametrised by three independent components $i,j,k$, with sum $\alpha \coloneqq i+j+k \leq n$.
The $n$--copy multiplicity is given by
\begin{equation}
	m_{ijk} = \frac{n!}{i!j!k!(n-\alpha)!} 2^i 3^j 3^k,
\end{equation}
while the distribution values that correspond to the same index triplet $(i,j,k)$ are given by
\begin{align}
	w(\rho_{\rm{S}})_{ijk} &= (-v)^{n-\alpha} u^{\alpha}, \\
	w(\gamma)_{ijk} &= (3\Z)^{-n} e^{-\beta (n-\alpha)E_0} e^{-\beta ( i E_0 + j E_1 + k E_2 )}, \\
	w(\rho_{\rm{S}} | \gamma)_{ijk} &= (3\Z)^{n} (-v)^{n-\alpha} u^{\alpha} e^{\beta (n-\alpha)E_0} e^{\beta ( i E_0 + j E_1 + k E_2 )}.
\end{align}

In order to construct the $n$--copy Lorenz curve $\lc{\rho_{\rm{S}}^{\otimes n}}{\gamma}$ we need to sort the components of the $n$--copy rescaled distribution, $w(\rho_{\rm{S}} | \gamma)_{ijk}$ in decreasing order.
In particular, to find the coordinates of the first elbow $(x_0, L_0)$, we need to evaluate the maximum rescaled component,
\begin{align}
	&\bmw(\rho_{\rm{S}} | \gamma)_{\rm{max}} \coloneqq (3\Z)^{n}\times \nonumber\\
	&\max\limits_{i,j,k}\Big\{ (-v)^{n-\alpha} u^{\alpha} e^{\beta (n-\alpha)E_0} e^{\beta ( i E_0 + j E_1 + k E_2 )} \Big\}, \label{eq:max_slope}
\end{align}
where $0 \leq i,j,k \leq n$ and $\alpha \coloneqq i+j+k \leq n$.
Notice that for $0 \leq \epsilon \leq 3/7$, we have $v \geq u$. 
We assume that $n$ is even, so that we need the sum $\alpha = i+j+k$ to be even for the expression to be positive. 
The following analysis is similar if $n$ is chosen to be odd.

Given an even value for the sum $\alpha$, the term $v^{n-\alpha} u^{\alpha} e^{-\beta (n-\alpha)E_0}$ is fixed, so the expression is maximised by setting the coefficient of the highest energy $E_{\rm{max}}$ equal to $\alpha$.
Hence, we have
\begin{align}
	&\bmw(\rho_{\rm{S}} | \gamma)_{\rm{max}} = \nonumber\\
	&(3\Z)^{n} v^n e^{n\beta E_0}\max\limits_{\substack{\alpha = 0,2, \\ \dots,n-2,n}}{\Big\{ \left( \frac{u}{v} e^{\beta (E_{\rm{max}} - E_0)} \right)^{\alpha} \Big\}}.
\end{align}
If the expression $\frac{u}{v} e^{\beta (E_{\rm{max}} - E_0)}$ is less than $1$ then the maximum occurs at $\alpha=0$, otherwise the maximum occurs at $\alpha = n$. To determine this transition we set
\begin{equation}\label{eq:noise_transition}
	\frac{u(\epsilon)}{v(\epsilon)} e^{\beta (E_{\rm{max}} - E_0)} \coloneqq \frac{3-\epsilon}{6-8\epsilon} e^{\beta (E_{\rm{max}} - E_0)} = 1.
\end{equation}
We want to find in which cases there exists a threshold noise level $\epsilon_\star$ at which the transition in~\cref{eq:noise_transition} occurs.
If $E_{\rm{max}} = E_0$, namely if the state negativity lies in the same phase space region as the highest energy, this threshold is constant in temperature and given by $\epsilon_{\star} = 3/7$. 
Otherwise, there is a threshold temperature value $\beta_\star$ given by
\begin{equation}
	\beta_{\star} \coloneqq \frac{1}{E_{\rm{max}} - E_0} \ln2.
\end{equation}
Below the threshold, $0 \leq \beta \leq \beta_\star$, the transition is well defined and the threshold noise level at which it occurs is given by
\begin{equation}
	\epsilon_{\star}(\beta) := 3 - \dfrac{18}{8-e^{(E_{\rm{max}} - E_0)\beta}}.
\end{equation}
This encompasses the case $E_{\rm{max}} = E_0$.
For $\beta > \beta_\star$, we have
\begin{equation*}
	\frac{3-\epsilon}{6-8\epsilon} e^{\beta (E_{\rm{max}} - E_0)} > \frac{3-\epsilon}{6-8\epsilon} 2 \geq \frac{1}{2}2 = 1,
\end{equation*}
so there is no transition and we set $\epsilon_\star = 0$.

The maximum rescaled component can then be expressed as
\begin{equation}
\bmw(\rho_{\rm{S}} | \gamma)_{\rm{max}} =
	\begin{cases}
		(3\Z)^{n} v^n e^{n\beta E_0}, &\epsilon \leq \epsilon_{\star},\ \hspace{3pt}\rm{(C1)}	\\
		(3\Z)^{n} u^n e^{n\beta E_{\rm{max}}}, &\epsilon > \epsilon_{\star}.\ \hspace{5pt}\rm{(C2)} 
	\end{cases}
\end{equation}
Case $\rm{(C1)}$ corresponds to $(i,j,k) = (0,0,0)$, so the multiplicity is $m_{000} = 1$ and the corresponding Wigner components are $\bmw(\rho_{\rm{S}})_{000}, \bmw(\gamma)_{000}$. 
Case $\rm{(C2)}$ corresponds to
\begin{equation}
	(i,j,k) = 
	\begin{cases}
	(0,n,0), &\text{if } E_{\rm{max}} = E_1, \\
	(0,0,n), &\text{if } E_{\rm{max}} = E_2.
	\end{cases}
\end{equation}
and we have $E_{\rm{max}} = E_1$ ($E_{\rm{max}} = E_2$), so the multiplicity is $m_{0n0} = 3^n$ ($m_{00n} = 3^n$) and the corresponding Wigner components are $\bmw(\rho_{\rm{S}})_{0n0}, \bmw(\gamma)_{0n0}$ ($\bmw(\rho_{\rm{S}})_{00n}, \bmw(\gamma)_{00n}$).

The first elbow coordinates can finally be derived as
\begin{equation}\label{eq:first_elb_coords}
	(x_0, L_0) =
	\begin{cases}
		\left( \left(\dfrac{e^{-\beta E_0}}{3\Z_\beta}\right)^n, v^n \right), &\rm{(C1)}	\vspace{10pt}\\
		\left( \left(\dfrac{e^{-\beta E_{\rm{max}}}}{\Z_\beta}\right)^n, (3u)^n \right). &\rm{(C2)} 
	\end{cases}
\end{equation}
\null\\

The Lorenz curves of the initial and target states may each be described by either $\rm{(C1)}$ or $\rm{(C2)}$, depending on the physical parameters $\epsilon, \epsilon', \beta$.
Specifically, we have three scenarios:
\begin{enumerate}
	\item $\rm{(C1)} \rightarrow \rm{(C1)}$ if $E_{\rm{max}} = E_0$ or $E_{\rm{max}} > E_0$, $\beta < \beta_{\star}$ and $\epsilon' < \epsilon  \leq \epsilon_{\star}$.
	\item $\rm{(C2)} \rightarrow \rm{(C1)}$ if $E_{\rm{max}} > E_0$, $\beta < \beta_{\star}$ and $\epsilon' \leq \epsilon_{\star} < \epsilon$.
	\item $\rm{(C2)} \rightarrow \rm{(C2)}$ if $E_{\rm{max}} > E_0$, $\beta < \beta_{\star}$ and $\epsilon_{\star} \leq \epsilon' < \epsilon$ or $E_{\rm{max}} > E_0$, $\beta \geq \beta_{\star}$.
\end{enumerate}
Note that $\rm{(C1)} \rightarrow \rm{(C2)}$ is impossible because it would imply $\beta < \beta_{\star}$ and $\epsilon \leq \epsilon_{\star} \leq \epsilon' < \epsilon$, a contradiction.

In all three scenarios, it is simple to check that the initial state's first elbow is always located to the left (closer to $0$) of the target state's first elbow, $x_0 \leq x_0'$, as proven in~\cref{app:first_elb_loc}, where the prime ($'$) is used to indicate target state coordinates.
Therefore, we can use the first elbow condition,
\begin{equation}\label{eq:first_elb_bound3}
	\frac{L_0}{x_0} \geq \frac{L_0'}{x_0'},
\end{equation}
to compute analytical distillation bounds for the distillation rate $R = R(\epsilon, \epsilon', \beta) \coloneqq n'/n$ in all three possible scenarios.
Involving more elbows gives stricter, but more convoluted necessary distillation constraints.

We substitute coordinates from~\cref{eq:first_elb_coords} appropriately in~\cref{eq:first_elb_bound3} to get the following necessary conditions,
\begin{equation}\label{eq:rate_bounds_proof}
	R \leq
	\begin{cases}
		\dfrac{\ln{\big( 1-\frac{4}{3}\epsilon \big)} + \beta (E_0 - F_\beta)}{\ln{\big( 1-\frac{4}{3}\epsilon' \big)} + \beta (E_0 - F_\beta)},\ &\rm{(C1)} \rightarrow \rm{(C1)}, \vspace{10pt}\\
		\dfrac{\ln{\big( \frac{1}{2}-\frac{1}{6}\epsilon \big)} + \beta (E_{\rm{max}} - F_\beta)}{\ln{\big( 1-\frac{4}{3}\epsilon' \big)} + \beta (E_0 - F_\beta)},\ &\rm{(C2)} \rightarrow \rm{(C1)}, \vspace{10pt}\\
		\dfrac{\ln{\big( \frac{1}{2}-\frac{1}{6}\epsilon \big)} + \beta (E_{\rm{max}} - F_\beta)}{\ln{\big( \frac{1}{2}-\frac{1}{6}\epsilon' \big)} + \beta (E_{\rm{max}} - F_\beta)},\ &\rm{(C2)} \rightarrow \rm{(C2)}.
	\end{cases}
\end{equation}
Equivalently, we require that
\begin{equation}\label{eq:error_bounds_proof}
	\epsilon \leq
	\begin{cases}
		\frac{3}{4} - \frac{3}{4} \left( 1 - \frac{4}{3}\epsilon' \right)^{R} \left( \dfrac{e^{-\beta E_0}}{\Z_\beta} \right)^{1 - R},\ &\rm{(C1)} \rightarrow \rm{(C1)}, \vspace{10pt}\\
		3 - 6 \left( 1 - \frac{4}{3}\epsilon' \right)^{R} \dfrac{e^{-\beta E_{\rm{max}}} e^{R\beta E_0}}{\Z_\beta^{1-R}},\ &\rm{(C2)} \rightarrow \rm{(C1)}, \vspace{10pt}\\
		3 - 6 \left( \frac{1}{2} - \frac{1}{6}\epsilon' \right)^{R} \left( \dfrac{e^{-\beta E_{\rm{max}}}}{\Z_\beta} \right)^{1 - R},\ &\rm{(C2)} \rightarrow \rm{(C2)}.
	\end{cases}
\end{equation}

Substituting $\epsilon' = 0$ in~\cref{eq:rate_bounds_proof} leads to the theorem statement.

\end{proof}

%%%%%%%%%%%%%%%%%%%%%%%%%%%%%%%%%%%%%%%%

\section{Lower magic bounds via majorisation}
\label{sec:lower_bounds}

\nick{Add from notes}

%%%%%%%%%%%%%%%%%%%%%%%%%%%%%%%%%%%%%%%%

\section{Fragments in general resource theories}
\label{sec:general_resources}

\nick{Add from notes}

So far we have introduced the notion of $\sigma$--fragments for any resource theory of magic. 
In this section we briefly generalise this concept to arbitrary resource theories and explain precisely how it connects with resource monotones. 
The busy reader more focussed on magic may skip this section.

State convertibility within a given resource theory is often a hard question to address due to intricacies of the theory structure.
In general, the structure of a theory $\R$ is described by a pre-order $\prec_\R$ and usually resource monotones are employed to reduce this structure into a simple real number ordering.
The subdivisions of magic theories into $\sigma$--fragments suggests a new approach towards investigating state convertibility which retains more structure of the original theory than a measure can.

Monotones reduce the structure of the resource theory $\R$ to a \emph{total} order on the real numbers.
Therefore, two states, even if incomparable in $\R$, are always mapped onto ordered real numbers.
We now generalise this idea of a theory projection that preserves comparability between states. 
\begin{definition}[\textbf{Covariant projection}]\label{def:covproj}
Let $\R = (\F, \O)$ be a resource theory with pre-order $\prec_\R$. 
Then a \emph{covariant resource projection} of $\R$ to a resource theory $\R'$ with pre-order $\prec_{\R'}$, is a pair of mappings $(\Pi_s, \Pi_o)$, where $\Pi_s$ maps quantum states in $\R$ to quantum states in $\R'$, and $\Pi_o$ maps free operations in $\R$ to free operations in $\R'$. 
Moreover, these obey
	\begin{enumerate}
        \item $\Pis(\rho_1) \prec_{\R'} \Pis(\rho_2)$ whenever $\rho_1 \prec_\R \rho_2$;
        \item $\Pio(\E) = \Pio(\E_1) \circ \Pio(\E_2)$ whenever $\E = \E_1 \circ \E_2$.
    \end{enumerate}
We call $\R'$ a \emph{covariant fragment} of $\R$.
\end{definition}

Resource monotones can now be clearly seen as a special case of covariant resource projections.
\begin{proposition}[\textbf{Totally ordered covariant theories}]\label{thm:monoproj}
	Any resource monotone $\M$ of a resource theory $\R$ is a covariant projection for which $\prec_{\R'}$ is a total order. 
	Conversely, any such covariant projection corresponds to a resource monotone $\M$. 
\end{proposition}
\begin{proof}
	Consider a monotone $\M$ in the context of a general resource theory $\R = (\F, \O)$.
	State order is covariantly preserved due to the defining property of a monotone, where the pre-order $\prec_{\R'}$ is simply the total order $\leq$ on $\mathbb{R}$. 
	
	Operational composition is covariantly preserved when we simply choose $\Pio(\E) = 1_\times$, namely the `multiplication by 1' operation on real numbers. 
	The definition of a resource monotone then automatically implies covariance.
	
	Conversely, given any covariant projection of $\R$ for which $\prec_{\R'}$ is a total order, we may map the totally ordered set of elements $\Pis(\rho)$ via an injective, non-decreasing function $f$ into $\mathbb{R}$. 
	Then, $\M(\rho):=f(\Pi_s(\rho))$ provides a numerical value for each $\rho$ that obeys the definition of a monotone.
	
\end{proof}

We can also view $\sigma$--fragments as an example of reducing the structure of a magic theory $\R$ to a subtheory with a tractable pre-order.
However, states which are incomparable in $\R$ remain incomparable and conversions between states which are comparable in $\R$ may no longer be possible.
\begin{definition}[\textbf{Contravariant projection}]\label{def:contraproj}
	Let $\R = (\F, \O)$ be a resource theory with pre-order $\prec_\R$. 
Then a \emph{contravariant resource projection} of $\R$ onto a resource theory $\R'$ with pre-order $\prec_{\R'}$, is a pair of mappings $(\Pi_s, \Pi_o)$, where $\Pi_s$ maps quantum states in $\R$ onto quantum states in $\R'$, and $\Pi_o$ maps free operations in $\R$ onto free operations in $\R'$. 
Moreover, these obey
	\begin{enumerate}
        \item $\rho_1 \prec_\R \rho_2$ whenever $\Pis(\rho_1) \prec_{\R'} \Pis(\rho_2)$;
        \item $\E = \E_1 \circ \E_2$ whenever $\Pio(\E) = \Pio(\E_1) \circ \Pio(\E_2)$.
    \end{enumerate}
We call $\R'$ a \emph{contravariant fragment} of $\R$.
\end{definition}
The use of covariant and contravariant in~\cref{def:covproj,def:contraproj} refers to the direction of implication between the two pre-orders and operation compositions\footnote{Note that strictly these are not projections in the sense of $\Pi^2 = \Pi$, but are instead morphisms. 
Here our use of the term projection is motivated by the idea that one one generally loses information about $\R$ under the mapping.}.

%%%%%%%%%%%%%%%%%%%%%%%%%%%%%%%%%%%%%%%%

\section{Conclusion}
\label{sec:conc}

\nick{Summary}

%%%%%%%%%%%%%%%%%%%%%%%%%%%%%%%%%%%%%%%%

\bibliography{bib}
%\bibliographystyle{apsrev4-2}

%%%%%%%%%%%%%%%%%%%%%%%%%%%%%%%%%%%%%%%%

\appendix
\newpage
\section{Properties of Wigner distributions}
\label{app:wigner}

Here, we present basic properties of the phase-point operators and the Wigner distribution that are used throughout the paper.

\begin{proposition}\label{thm:aproperties}
    For any dimension $d$, the phase-point operators satisfy:
    \begin{enumerate}
        \item[(i)]\label{en:a1} Hermiticity and unitarity: $A_{\bmx}^\dagger = A_{\bmx} = A_{\bmx}^{-1}$;
	    \item[(ii)]\label{en:a2} Closure under transposition: $A_{(x, p)}^T = A_{(x, -p)}$;
	    \item[(iii)]\label{en:a3} Unit trace for odd $d$: $\tr[A_{\bmx}] = 1$;
	    \item[(iv)]\label{en:a4} Completeness relation: $\sum_{\bmz \in \cal{P}_d} A_{\bmz} = d\id$;
	    \item[(i)]\label{en:a5} Orthogonality: $\tr[A_{\bmx}^\dagger A_{\bm{x'}}] = d \delta_{\bmx,\bm{x'}}$.
	\end{enumerate}
\end{proposition}
All properties follow from the definition in~\cref{eq:ax} along with properties of the displacement operator $D_{\bmx}$ and can be found in the literature, e.g.~\cite{cit:veitch,Vourdas_2004,cit:gross3}

\begin{proposition}\label{thm:wstate}
  The Wigner distribution of a state $\rho \in \cal{B}(\cal{H}_d)$ is
  \begin{enumerate}
    \item[(i)]\label{en:w1} Real valued: $\W{\rho} \in \mathbb{R}^{d^2}$;
    \item[(ii)]\label{en:w2} Normalised: $\sum_{\bmz \in \cal{P}_d} \W[\bmz]{\rho}=1$;
    \item[(iii)]\label{en:w3} Bounded: $\abs{\W[\bmx]{\rho}} \leq \frac{1}{d}$.
    \item[(iv)]\label{en:w4} Additive under mixing: \vspace{2pt}\\
    $\W[\bmx]{p\rho_1 + (1-p)\rho_2} = p\W[\bmx]{\rho_1} + (1-p)\W[\bmx]{\rho_2}$;
    \item[(v)]\label{en:w5} Multiplicative under tensor products: \vspace{2pt}\\
    $\W[\bmx_A \oplus \bmx_B]{\rho_A \otimes \rho_B} = \W[\bmx_A]{\rho_A}\W[\bmx_B]{\rho_B}$.
	\end{enumerate}
\end{proposition}
\begin{proof}
	Proof of all properties can be found in the literature~\cite{cit:veitch,Vourdas_2004,cit:gross3,Wang_2019} except for property (iii) which we prove here.
	
Let $\{\lambda_i\}_{i \in \mathbb{Z}_d}$ be the (non-negative) eigenvalues of $\rho$, summing to 1.
Let $\{\alpha_{\bmx,i}\}_{i \in \mathbb{Z}_d}$ be the eigenvalues of $A_{\bmx}$. For any $\bmx, \alpha_{\bmx,i} \in \{-1, 1\}$, due to the hermiticity and unitarity of the phase-point operators. 
Then,
\begin{align}
	\abs{W_{\rho}(\bmx)} &= \frac{1}{d}\abs{\tr[A_{\bmx} \rho]} \leq \frac{1}{d} \abs{\sum_i \alpha_{\bmx,i} \lambda_i} \nonumber\\ &\leq \frac{1}{d}\sum_i \lambda_i = \frac{1}{d}.
\end{align}
The first inequality follows from Theorem 1 of~\cite{cit:mirsky} for the trace of complex matrices, while the second is the Cauchy-Schwarz inequality.
\end{proof}

\begin{proposition}
    \label{thm:wchannel}
    The Wigner distribution of a $\cptp$ operation $\E: \cal{B}(\cal{H}_{d_A}) \mapsto \cal{B}(\cal{H}_{d_B})$ is
    \begin{enumerate}
        \item[(i)]\label{en:wo1} Real-valued: $\W{\E} \in \mathbb{R}^{d^2} \times \mathbb{R}^{d^2}$;
        \item[(ii)]\label{en:wo2} Normalised: $\sum_{\bmz \in \cal{P}_{d_B}} \W[\bmz|\bmx]{\E} = 1$ \\ 
        for any $\bmx \in \cal{P}_{d_A}$;
        \item[(iii)]\label{en:wo3} Bounded: $\abs{\W[\bmy|\bmx]{\E}} \leq \frac{d_A}{d_B}$;
	    \item[(iv)]\label{en:wo4} Transitive: $\W[\bmy]{\E(\rho)} = \sum_{\bmz \in \cal{P}_{d_A}} \W[\bmy|\bmz]{\E} \W[\bmz]{\rho}$ for any $\bmy \in \cal{P}_{d_B}$.
    \end{enumerate}
\end{proposition}
If $d_A = d_B$, and in particular if operation $\E$ maps a Hilbert space onto itself, then the stochasticity condition $\abs{\W[\bmy|\bmx]{\E}} \leq 1$ is satisfied.
\begin{proof}
	Proof of all properties are provided by Wang \textit{et al.}~\cite{Wang_2019} except for property (iii) which is a direct consequence of the definition of $\W{\E}$ and the corresponding property (iii) in~\cref{thm:wstate}.
\end{proof}

%%%%%%%%%%%%%%%%%%%%%%%%%%%%%%%%%%%%%%%%

\section{Properties of majorization}
\label{app:major}
	
We now give the following equivalent formulations of $d$--majorization.

\begin{proposition}\label{prop:rmajor}
Given $\bmx, \bmy, \r \in \mathbb{R}^n$, such that the components of $\r$ are positive, the following statements are equivalent:
  \begin{enumerate}
    \item[(i)] $\bmy = A\bmx$ and $\r = A\r$ for a stochastic map $A$;
    \item[(ii)]\label{en:tm3} $\sum\limits_{i=1}^n \abs{x_i - r_i t} \leq \sum\limits_{i=1}^n \abs{y_i - r_i t}$ for all $t \in \mathbb{R}$;
    \item[(iii)] $L_{\bmx|\r}(t) \leq L_{\bmy|\r}(t)$ for $t\in [0,1)$ and \vspace{5pt}\\ $L_{\bmx|\r}(1) = L_{\bmy|\r}(1)$.
  \end{enumerate}
\end{proposition}
The proofs for these can be found in~\cite{cit:marshall,cit:bhatia,cit:nielsen,cit:lostaglio} and references therein.

The following result is used in the text to relate relative majorization of quasi-distributions to their Lorenz curves.
\begin{proposition}\label{lemma:Lorenz_linearity}
	Let $\p$ be a quasi-probability distribution and let $\r$ be a probability distribution with strictly non-zero components. Let $a > 0$ and $b \in \mathbb{R}$ then $L_{a\p + b \r | \r} (x) = a L_{\p |\r}(x) + b x$.
\end{proposition}
\begin{proof} 
	The Lorenz curve of $a\p + b \r$ relative to $\r$ passes through $(0,0)$ and the points $(\sum_{i=1}^k{r_{\pi(i)}}, \sum_{i=1}^k(a \p + b \r)_{\pi(i)})$ where $\pi$ is the permutation that puts $(a p_i/r_i + b)$ in non-increasing order. Since $a > 0$, the permutation $\pi$ puts  $(p_i/r_i)$ in non-increasing order too. We thus have
\begin{align*}
&\left( \sum_{i=1}^kr_{\pi(i)}, \sum_{i=1}^k(a \p + b \r)_{\pi(i)} \right) = \\ 
&\left( \sum_{i=1}^k r_{\pi(i)},a \sum_{i=1}^k  p_{\pi(i)} + b\sum_{i=1}^k r_{\pi(i)} \right) \nonumber,
\end{align*}
therefore the value of the Lorenz function at each potential elbow point $x_k = \sum_{i=1} ^kr_{\pi(i)}$ is given by
\begin{align}
&L_{a \p +b \r|\r} (x_k) = a L_{\p|\r} (x_k) + b L_{\r|\r}(x_k) = \nonumber\\
&a L_{\p|\r} (x_k) + b x_k,
\end{align}
so we have $L_{a\p  + b\r|\r} (x) = a L_{\p |\r}(x) + b x$ for any $x \in [0,1]$ due to linearity.
\end{proof}

\begin{theorem*}
	Given a magic state $\rho$, the maximum $L_\star$ of its Lorenz curve $\lc{\rho}{\sigma}(x)$ is independent of the $\sigma$--fragment and equal to $1+\sn{\rho}$. Moreover, the majorization constraint is stronger than mana in every fragment.
\end{theorem*}
\begin{proof}
	We denote the Wigner distributions of the states compactly as vectors $\bmw(\rho) \equiv W_\rho(x)$ and $\bmw(\sigma) \equiv W_\sigma(x)$.
	We choose the component indexing so that the rescaled distribution 
	\begin{equation}
		\widetilde{\bmw}(\rho|\sigma) \coloneqq \left(\frac{w(\rho)_1}{w(\sigma)_1}, \dots, \frac{w(\rho)_{d^2}}{w(\sigma)_{d^2}} \right)^T,
	\end{equation}
	is sorted, $\widetilde{\bmw} = \widetilde{\bmw}^\downarrow$.
	Note that all components of $\bmw(\sigma)$ are positive, so $\widetilde{w}_i \geq 0$ if and only if $w(\rho)_i \geq 0$ for any $i=1,\dots,d^2$.
	
	Let $i_\star$ be the index of the smallest non-negative component of $\widetilde{\bmw}^\downarrow$.
	Then, $w(\rho)_i < 0$ if and only if $i > i_\star$, so the maximum of Lorenz curve $\lc{\rho}{\sigma}(x)$ takes the value 
	\begin{equation}
		\lc{\rho}{\sigma}(x_{i_\star}) = \sum_{i=1}^{i_\star} w(\rho)_i,
	\end{equation}
	and is achieved at
	\begin{equation}\label{eq:maxloc}
		x_{i_\star} \coloneqq \sum_{i=1}^{i_\star} w(\sigma)_i.
	\end{equation}

	The location of the maximum ($x=x_{i_\star}$) varies from fragment to fragment, but its value is independent of $\sigma$,
	\begin{align}
	L_\star &:=	\lc{\rho}{\sigma}(x_{i_\star}) 
		= \sum\limits_{\bmx: \W[\bmx]{\rho} \geq 0} \W[\bmx]{\rho} \nonumber \\
		&= 1 + \sn{\rho}.
	\end{align}
	
Since the magic monotone mana is a monotonic function of sum-negativity, $\rm{mana}(\rho) \coloneqq \ln{(2\hspace{1pt}\sn{\rho}+1)}$, we see that mana corresponds precisely to the peak of the Lorenz curve $L_{\rho|\sigma}(x)$. Therefore, mana is one of $d^{2n}$ constraints, so majorization is strictly a stronger constraint in any fragment.
\end{proof}



%%%%%%%%%%%%%%%%%%%%%%%%%%%%%%%%%%%%%%%%

\section{Technical properties of $\sigma$--fragments}
\label{app:frag}

In this section, we discuss some technical aspects of general $\sigma$--fragments.

We first prove a result on the independence of the Lorenz curve constraints, stated in~\cref{sec:major_frag}.
\begin{proposition}\label{thm:elbows}
	Let $\rho, \tau$ be two quantum states with Lorenz curves $\lc{\rho}{\sigma}(x), \lc{\tau}{\sigma}(x)$ in the $\sigma$--fragment.
	
	Let $t$ be the number of elbows of $\lc{\tau}{\sigma}(x)$ at locations $x_1, \dots, x_t$.
	
	Then, $\lc{\rho}{\sigma}(x) \geq \lc{\tau}{\sigma}(x)$ for all $x \in [0,1]$ iff $\lc{\rho}{\sigma}(x_{i}) \geq \lc{\tau}{\sigma}(x_{i})$ for all $i =1,\dots,t$.
\end{proposition}
\begin{proof}	
	$\lc{\rho}{\sigma}(x) \geq \lc{\tau}{\sigma}(x)$ for all $x \in [0,1]$ trivially implies $\lc{\rho}{\sigma}(x_{i}) \geq \lc{\tau}{\sigma}(x_{i})$ for all $i = 1,\dots,n'$.
	
	Conversely, assume that $\lc{\rho}{\sigma}(x_{i}) \geq \lc{\tau}{\sigma}(x_{i})$ for all $i = 1,\dots,r$.
	First, let $x_0 = 0$ and $x_{n'+1} = 1$, so that $\lc{\rho}{\sigma}(x_0) = \lc{\tau}{\sigma}(x_0) = 0$ and $\lc{\rho}{\sigma}(x_{n'+1}) = \lc{\tau}{\sigma}(x_{n'+1}) = 1$.
	Hence, we can extend the set of elbows $E$ to $E' = E \cup \{x_0, x_{n'+1}\}$.
	
	Pick two consecutive locations $x_{i}, x_{i+1}$ in $E'$ and consider the line segment $\ell_\tau(x)$ connecting points $(x_{i}, \lc{\tau}{\sigma}(x_{i}))$ and $(x_{i+1}, \lc{\tau}{\sigma}(x_{i+1}))$ as well as the line segment $\ell_\rho(x)$ connecting points $(x_{i}, \lc{\rho}{\sigma}(x_{i}))$ and $(x_{i+1}, \lc{\rho}{\sigma}(x_{i+1}))$.
	This is illustrated in~\cref{fig:elbows_proof}.
\begin{figure}[h]
    \centering
    \includegraphics[scale=0.5]{figs/elbows_proof.pdf}
    \caption{\textbf{Illustration of~\cref{thm:elbows}}.
    }
    \label{fig:elbows_proof}
\end{figure}

	Due to concavity of $\lc{\rho}{\sigma}$, it is clear that for all $x \in [x_{i}, x_{i+1}]$, we have $\lc{\rho}{\sigma}(x) \geq \ell_\rho(x) \geq \ell_\tau(x) = \lc{\tau}{\sigma}(x)$.
	This argument can be made in all intervals $[x_{i}, x_{i+1}]$ with $i=0,\dots,n'$, so the proof is complete.
\end{proof}
The above theorem can be of practical importance in reducing the necessary distillation constraints derived via majorization in $\sigma$--fragments.

\begin{proposition}\label{thm:frag_app}
    Let $\R = (\F, \O)$ be a magic theory and $\sigma, \sigma' \in \F$. The following statements hold:
    \begin{enumerate}
        \item No $\sigma$--fragment of $\R$ is empty.
        \item If a free operation leaves two states invariant, then it also leaves their mixtures invariant, 
        \begin{equation*}
            \O_{\sigma} \cap \O_{\sigma'} \subseteq \O_{p\sigma + (1-p)\sigma'}\ \text{for any}\ p \in [0,1].
        \end{equation*}
    \end{enumerate}
\end{proposition}
\begin{proof}$ $\vspace{-12pt}\\

\begin{enumerate}
    \item The identity channel $1_{\rm{C}}$ belongs to every $\sigma$--fragment, as $1_{\rm{C}} \in \O$ and $1_{\rm{C}}\sigma = \sigma$ for all $\sigma \in \F$.
    
    \item Let $\E \in \O_{\sigma} \cap \O_{\sigma'}$.
    Then $\E \in \cptp$ and corresponds to stochastic Wigner distribution $\W{\E}$ such that $\W{\E} \W{\sigma} = \W{\sigma}$ and $\W{\E} \W{\sigma'} = \W{\sigma'}$.
    Then, $\W{\E} \W{p\sigma + (1-p)\sigma'} = \W{p\sigma + (1-p)\sigma'}$ for any $p \in [0,1]$ due to the additive property~\ref{en:w4} of the Wigner distribution, implying that state $p\sigma + (1-p)\sigma'$ is also left invariant by $\E$.
\end{enumerate}
\vspace{-20pt}
\end{proof}

Any free state $\sigma \in \F$ corresponds to a $d^2$--dimensional probability distribution $\W{\sigma}$ and any free operation $\E \in \O$ corresponds to a $d^2 \times d^2$ stochastic matrix (or conditional probability distribution) $\W{\E}$.
Note that these mappings are one-to-one due to the orthogonality of the phase-point operators as an operator basis.
Note further that free states $\F$ are mapped onto a \emph{strict subset} of the set of probability distributions.
As a counterexample, the sharp $d^2$--dimensional probability distribution $(1, 0, \dots, 0)$ does not correspond to any qudit Wigner distribution because of the boundedness condition in~\cref{thm:wstate}.
Similarly, not all stochastic matrices correspond to completely positive operations.
As an example, consider the permutation matrix
\begin{equation}
    \Pi_X = \begin{psmallmatrix}
        0 & 1 & 0 & 0 & 0 \\
        0 & 0 & 0 & 0 & 1 \\
        0 & 0 & 0 & 1 & 0 \\
        1 & 0 & 0 & 0 & 0 \\
        0 & 0 & 1 & 0 & 0
    \end{psmallmatrix} \otimes \begin{psmallmatrix}
        0 & 0 & 1 & 0 & 0 \\
        0 & 0 & 0 & 0 & 1 \\
        0 & 0 & 0 & 1 & 0 \\
        1 & 0 & 0 & 0 & 0 \\
        0 & 1 & 0 & 0 & 0    
    \end{psmallmatrix} \in {\rm{S}}_5({\W{\frac{1}{5}\id}}).
\end{equation}
It preserves the uniform distribution $\W{\frac{1}{5}\id}$, but it does not correspond to any positive (hence quantum) operation.

%%%%%%%%%%%%%%%%%%%%%%%%%%%%%%%%%%%%%%%%

\section{Lorenz curves in the unital fragment}
\label{app:lcsu_technical}

\subsection{Binomial distributions and error bounds}\label{app:phi}
Consider an experiment consisting of $n$ trials of throwing a $p$--coin, that is a coin with probability $p$ of landing on one side and $1-p$ of landing on the other.
We express the sums over an even number $m$ of successful trials $\Phi_+$ and an odd number $m$ of successful trials $\Phi_-$,
\begin{align}	
	\Phi_+(m; n, p) &\coloneqq \sum\limits_{\ell=0}^{m/2} \binom{n}{2\ell} p^{2\ell} (1-p)^{n-2\ell}, \nonumber\\ 
	&\text{for even integers } m\in[0,n], \label{eq:fp_app} \\
	\Phi_-(m; n, p) &\coloneqq \sum\limits_{\ell=1}^{(m-1)/2} \binom{n}{2\ell+1} p^{2\ell+1} (1-p)^{n-(2\ell+1)}, \nonumber\\ 
	&\text{for odd integers }m\in[0,n]. \label{eq:fn_app}
\end{align}
Note that index $m$ only takes even (odd) values when labelling $\Phi_+$ ($\Phi_-$).
In~\cref{app:lcsu_coord}, we will use $\Phi_+$ and $\Phi_-$ to express the elbow coordinates of Lorenz curves in the unital fragment.

We also define the classical entropy of a $p$--coin and the classical relative entropy between a $p$--coin and a $q$--coin,
\begin{align}
	S(p) &\coloneqq -p\log{p} -(1-p)\log{(1-p)}, \label{eq:ent}\\
	\ent{p}{q} &\coloneqq p \log{\frac{p}{q}} + (1-p) \log{\frac{1-p}{1-q}}. \label{eq:ent_rel}
\end{align}
They are symmetric in the sense that $S(p) = S(1-p)$ and $\ent{p}{q} = \ent{1-p}{1-q}$.

A useful result is the entropic bound on a combination~\cite{cit:ash}.
\begin{lemma}\label{lem:comb_bounds}
	For all $\ell\in [1,n-1]$,
	\begin{align}
		&\left[ 8\ell\left(1-\frac{\ell}{n}\right) \right]^{-\frac{1}{2}} 2^{n S\left(\frac{\ell}{n}\right)} \leq \binom{n}{\ell} \leq \\
		&\left[ 2\pi \ell\left(1-\frac{\ell}{n}\right) \right]^{-\frac{1}{2}} 2^{n S\left(\frac{\ell}{n}\right)}.
	\end{align}
\end{lemma}
The proof provided in~\cite{cit:ash} proceeds with direct calculation for the edge cases $\ell = 1,2, n-1, n-2$ and use Stirling's approximation for the remaining cases.
We can use~\cref{lem:comb_bounds} to provide strict upper and lower bounds on the functions $\Phi_+, \Phi_-$.

\subsection{Theory on bounding the core functions}
\nick{Should we really keep this section? It is standard calculus and the bounds are really loose.}
Here we present a more manageable method of bounding the core functions $\Phi_{\pm}$, which however results in looser bounds. 
We can rewrite the functions as
\begin{equation}
	\Phi_{\pm}(m; n, a) = \frac{1}{2}(\Phi(m; n, a) \pm (1+a)^{-n} S(m; n, a)),
\end{equation}
where we have substituted $a = p/(1-p)$. 
$\Phi$ is the standard cumulative function
\begin{equation}
	\Phi(m; n, a) = (1+a)^{-n} \sum_{k=0}^m \binom{n}{k} a^k,
\end{equation}
and the remainder term is
\begin{equation}
	S(m; n, a) \coloneqq \sum_{k=0}^m \binom{n}{k} (-a)^k.
\end{equation}
We have the following asymptotic bounds on the behaviour of $\Phi$~\cite{cit:ash},
\begin{lemma}\label{lem:phi_bounds}
	Given fixed $n>0$ and $p$, $\Phi$ satisfies the following bounds:
	\begin{align*}
		\begin{split}
		&\text{1. } \Phi(m; n, p) \geq \left[ 8m\left(1-\frac{m}{n}\right) \right]^{-\frac{1}{2}} 2^{-n\ent{\frac{m}{n}}{p}}, \\
		&\hspace{14pt} m\in [1,n-1] \\
		&\text{2. } \Phi(m; n, p) \geq 1 - 2^{-n\ent{\frac{m+1}{n}}{p}},\ m\in [np+1,n-2] \\
		&\text{3. } \Phi(m; n, p) \leq 1 - \left[ 8(m+1)\left(1-\frac{m+1}{n}\right) \right]^{-\frac{1}{2}}\times \\
		&\hspace{14pt} 2^{-n\ent{\frac{m+1}{n}}{p}},\ m\in [0,n-2]
		\end{split}
		\\
		&\text{4. } \Phi(m; n, p) \leq 2^{-n\ent{\frac{m}{n}}{p}},\ m\in [0,np]
	\end{align*}
\end{lemma}

We would like some theory that estimates the value of $S(m; n, a)$ for different parameter regimes. 
We can consider the function $f(x) = (1+x)^n$ and note that $S(m; n, a)$ is the $m$'th partial sum of this expansion at the point $x=-a$.

The truncated Maclaurin series of a general function $f(x)$ is
\begin{equation}
	f(x) = f(0) + x f'(0) + \dots \frac{x^m}{m!}f^{(m)}(0) + R_m(x)
\end{equation}
with a remainder term
\begin{align}
	R_m (x)&= \int_{0}^x dt f^{(m+1)}(t) \frac{(x-t)^m}{m!} \\
	&= \frac{x^{m+1}}{(m+1)!} f^{(m+1)}(x_*),
\end{align}
where in the second expression, $x_*$ is an implicit point that lies between $0$ and $x$ that comes from the Mean Value Theorem.

Applying this to the function $f(x) = (1+x)^n$ gives
\begin{equation}
	(1+x)^n = \sum_{k=0}^m \binom{n}{k} x^k + R_m.
\end{equation}
Evaluating at $x=-a$ gives
\begin{equation}
	S(m; n, a) = (1-a)^n - R_m(-a),
\end{equation}
where the key remainder term is given by
\begin{align}
	R_m(-a) &= \int_0^{-a} dt f^{(m+1)}(t) \frac{(-a-t)^m}{m!} \\
&= \frac{(-a)^{m+1}}{(m+1)!} f^{(m+1)}(x_*).
\end{align}
We can also compute the derivative $f^{(m+1)}(x)$ explicitly,
\begin{equation}
	f^{(m+1)}(x) = (m+1)!\binom{n}{m+1}(1+x)^{n-m-1}.
\end{equation}
Therefore, we have that
\begin{align}
	R_m(-a) &= (-1)^{m+1}(m+1)\binom{n}{m+1}\times \nonumber\\
	&\hspace{12pt} \int_{-a}^0 dt (1+t)^{n-m-1}(a+t)^m \\
&= \binom{n}{m+1}(-a)^{m+1}(1+x_*)^{n-m-1},
\end{align}
where in the latter expression $x_* \in [-a,0]$. 
Note that the first integral expression can be estimated via the Cauchy-Schwarz or the H{\"o}lder inequality. 
Therefore, we can either work with an explicit form with an unknown (but bounded) parameter $x_*$, or we can use the integral form and provide concrete estimates on it value.

A very simple estimate, based on $x_*$ lying in the interval $[-a,0]$ gives
\begin{equation}
(1-a)^{n-m-1} \leq \frac{R_m(-a)}{\binom{n}{m+1}(-a)^{m+1}} \leq 1,
\end{equation}
which in turn leads to the following bounds on $\Phi_+(m; n, a)$:
\begin{align}
	\hspace{-2cm}2\Phi_+(m; n, a) \leq\ &\Phi(m; n, a) + (1-a)^n - \frac{(-a)^{m+1}}{m!} \times \nonumber\\
	 &\binom{n}{m+1} (1-a)^{n-m-1} \text{ and} \\
	2\Phi_+(m; n, a) \geq\ &\Phi(m; n, a) + (1-a)^n - \frac{(-a)^{m+1}}{m!} \binom{n}{m+1} .
\end{align}

\subsection{Lorenz curve coordinates in the unital fragment}\label{app:lcsu_coord}
The Wigner distribution of the $n$--copy qutrit maximally mixed state $\left(\id/3\right)^{\otimes n}$ is the uniform probability distribution over the phase space, consisting of $9^n$ components equal to $9^{-n}$.
The Wigner distribution of the 1-copy $\epsilon$--noisy Strange state $\rho_{\rm{S}}(\epsilon)$ in the unital fragment consists of some permutation of a single negative component
\begin{equation}
	- v(\epsilon) \coloneqq - \left( \frac{1}{3} -\frac{4}{9}\epsilon \right),
\end{equation} 
and $8$ positive components
\begin{equation}
	u(\epsilon) \coloneqq \frac{1}{6} -\frac{1}{18}\epsilon.
\end{equation}
where in the unital fragment we need the condition $0 \leq \epsilon < 3/4$, so that the state contains some Wigner negativity ($-v < 0$).
It is also clear that $v \geq u$ in the interval $0 \leq \epsilon \leq 3/7$, while $u > v$ in the interval $3/7 < \epsilon < 3/4$.

The Wigner distribution of the $n$--copy $\epsilon$--noisy Strange state $\rho_{\rm{S}}(\epsilon)^{\otimes n}$ in the unital fragment is given by the convolution $\W{\rho_{\rm{S}}(\epsilon)^{\otimes n}} = W_{\rho_{\rm{S}}(\epsilon)}^{\otimes n}$.
In general, $\rho_{\rm{S}}(\epsilon)^{\otimes n}$ contains $n + 1$ distinct components, labelled $0,\dots, n$.
We present the distinct Wigner components of $\rho_{\rm{S}}(\epsilon)^{\otimes n}$ along with their multiplicites in~\cref{tab:lcsu}.
Note that LHS (RHS) refers to elbow coordinates $i$ on the left of and including (right of) the Lorenz curve maximum, stated precisely as
\begin{align}
&\text{LHS: } 0 \leq i \leq \left\lfloor \frac{n}{2} \right\rfloor \text{ and} \\
&\text{RHS: } \left\lfloor \frac{n}{2} \right\rfloor +1 \leq i \leq n.
\end{align}
\begin{table}[h]
  \def\arraystretch{1.5}
  \centering
  \begin{tabular}{c|c|c|r|r}
    \multicolumn{3}{c|}{Case} & \multicolumn{1}{c}{$m_{i}(n, \epsilon)$} & \multicolumn{1}{|c}{$w_{i}(n, \epsilon)$} \\[0.5ex]\hline
    \multirow{4}{*}{\raisebox{-4ex}{\rotatebox[origin=c]{90}{$0\leq \epsilon < \frac{3}{7}$}}} & \hspace{0.8ex}\multirow{2}{*}{\raisebox{-1ex}{\rotatebox[origin=c]{90}{$n$ even}}}\hspace{0.8ex} & LHS & $8^{2i}\binom{n}{2i}$ & $\left( \frac{1}{6} - \frac{1}{18}\epsilon \right)^{2i}\left( -\frac{1}{3} + \frac{4}{9}\epsilon \right)^{n-2i}$ \\
    & & RHS & $8^{n-2i}\binom{n}{2i}$ & $\left( \frac{1}{6} - \frac{1}{18}\epsilon \right)^{n-2i}\left( -\frac{1}{3} + \frac{4}{9}\epsilon \right)^{2i}$ \\ \cline{2-5}
    & \multirow{2}{*}{\raisebox{-2ex}{\rotatebox[origin=c]{90}{$n$ odd}}} & LHS & $8^{2i+1}\binom{n}{2i+1}$ & $\left( \frac{1}{6} - \frac{1}{18}\epsilon \right)^{2i+1}\left( -\frac{1}{3} + \frac{4}{9}\epsilon \right)^{n-2i-1}$ \\
    & & RHS & $8^{n-2i-1}\binom{n}{2i+1}$ & $\left( \frac{1}{6} - \frac{1}{18}\epsilon \right)^{n-2i-1}\left( -\frac{1}{3} + \frac{4}{9}\epsilon \right)^{2i+1}$ \\ \hline
    \multirow{4}{*}{\raisebox{-4ex}{\rotatebox[origin=c]{90}{$\frac{3}{7}\leq \epsilon < \frac{3}{4}$}}} & \multirow{2}{*}{\raisebox{-1ex}{\rotatebox[origin=c]{90}{$n$ even}}} & LHS & $8^{n-2i}\binom{n}{2i}$ & $\left( \frac{1}{6} - \frac{1}{18}\epsilon \right)^{n-2i}\left( -\frac{1}{3} + \frac{4}{9}\epsilon \right)^{2i}$ \\
    & & RHS & $8^{2i}\binom{n}{2i}$ & $\left( \frac{1}{6} - \frac{1}{18}\epsilon \right)^{2i}\left( -\frac{1}{3} + \frac{4}{9}\epsilon \right)^{n-2i}$ \\ \cline{2-5}
    & \multirow{2}{*}{\raisebox{-2ex}{\rotatebox[origin=c]{90}{$n$ odd}}} & LHS & $8^{n-2i}\binom{n}{2i}$ & $\left( \frac{1}{6} - \frac{1}{18}\epsilon \right)^{n-2i}\left( -\frac{1}{3} + \frac{4}{9}\epsilon \right)^{2i}$ \\
    & & RHS & $8^{2i}\binom{n}{2i}$ & $\left( \frac{1}{6} - \frac{1}{18}\epsilon \right)^{2i}\left( -\frac{1}{3} + \frac{4}{9}\epsilon \right)^{n-2i}$ \\ \hline
  \end{tabular}
  \caption{Wigner components $w_{i}(n, \epsilon)$ of $\rho_{\rm{S}}(\epsilon)^{\otimes n}$ along with their multiplicities $m_{i}(n, \epsilon)$, with $0 \leq i \leq n$.
  The expressions change depending on the noise level $\epsilon$, the parity of the number of copies $n$ and whether the index $i$ is lower or higher than the index of the Lorenz curve maximum (LHS vs RHS).
  Multiplication $2i$ is considered modulo $(n+1)$.}
  \label{tab:lcsu}
\end{table}

Every Lorenz curve in the unital fragment contains $n$ elbows, which, along with the boundary points $(x_{-1}, L_{-1}) = (0,0)$ and $(x_{n}, L_{n}) = (1,1)$, are labelled by 
\begin{equation*}
\{(x_{i}, L_{i})\}_{i=-1,0,\dots,n}.
\end{equation*}
The maximum is the $\lfloor n/2 \rfloor$-th elbow and its coordinates are calculated by collecting all the positive Wigner components,
\begin{align}
	x_{\lfloor n/2 \rfloor} &= \frac{1}{2}\left(1 + \left(\frac{7}{9}\right)^n\right), \\
	L_{\lfloor n/2 \rfloor} &= \frac{1}{2}\left (1 + \left(\frac{15 - 8\epsilon}{9}\right)^n \right).
\end{align}
%\sum_{j: even}^n a^j \binom{n}{j} = \frac{1}{2} [ (1+a)^n + (1-a)^n ]

Expressions for all the elbow coordinates follow from summing up the Wigner components in decreasing order.
In~\cref{tab:lcsu_coord_elb_app}, we present the elbow coordinates of the $n$-copy, $\epsilon$--noisy Strange state Lorenz curve in the unital fragment for any combination of parameters $n, \epsilon$.
\begin{table}[h]
  \def\arraystretch{1.5}
  \centering
  \begin{tabular}{c|c|c|r|r}
\multicolumn{3}{c|}{\multirow{2}{*}{Case}} & \multicolumn{1}{c|}{$x_{i}$} & \multicolumn{1}{c}{$L_{i}$} \\
    \multicolumn{3}{c|}{} & \multicolumn{1}{c|}{$x_{i} - x_{\lfloor n/2 \rfloor}$} & \multicolumn{1}{c}{$L_{i} - L_{\lfloor n/2 \rfloor}$} \\[0.5ex]\hline 
    \multirow{4}{*}{\raisebox{-5ex}{\rotatebox[origin=c]{90}{$0\leq \epsilon < \frac{3}{7}$}}} & \hspace{0.8ex}\multirow{2}{*}{\raisebox{-3ex}{\rotatebox[origin=c]{90}{$n$ even}}}\hspace{0.8ex} & LHS & $\Phi_+\left(2i;n,\frac{8}{9}\right)$ & $\left( \frac{5}{3} - \frac{8}{9}\epsilon\ \right)^n \Phi_+\left(2i;n,\frac{12-4\epsilon}{15-8\epsilon}\right)$ \\
    & & RHS & $\Phi_-\left(2i;n,\frac{1}{9}\right)$ & $- \left( \frac{5}{3} - \frac{8}{9}\epsilon\ \right)^n\Phi_-\left(2i;n,\frac{3-4\epsilon}{15-8\epsilon}\right)$ \\ \cline{2-5}
    & \multirow{2}{*}{\raisebox{-3ex}{\rotatebox[origin=c]{90}{$n$ odd}}} & LHS & $\Phi_-\left(2i;n,\frac{8}{9}\right)$ & $\left( \frac{5}{3} - \frac{8}{9}\epsilon\ \right)^n \Phi_-\left(2i;n,\frac{12-4\epsilon}{15-8\epsilon}\right)$ \\
    & & RHS & $\Phi_-\left(2i;n,\frac{1}{9}\right)$ & $- \left( \frac{5}{3} - \frac{8}{9}\epsilon\ \right)^n\Phi_-\left(2i;n,\frac{3-4\epsilon}{15-8\epsilon}\right)$ \\ \hline
    \multirow{4}{*}{\raisebox{-5ex}{\rotatebox[origin=c]{90}{$\frac{3}{7}\leq \epsilon < \frac{3}{4}$}}} & \multirow{2}{*}{\raisebox{-3ex}{\rotatebox[origin=c]{90}{$n$ even}}} & LHS & $\Phi_+\left(2i;n,\frac{1}{9}\right)$ & $\left( \frac{5}{3} - \frac{8}{9}\epsilon\ \right)^n \Phi_+\left(2i;n,\frac{3-4\epsilon}{15-8\epsilon}\right)$ \\
    & & RHS & $\Phi_-\left(2i;n,\frac{8}{9}\right)$ & $- \left( \frac{5}{3} - \frac{8}{9}\epsilon\ \right)^n\Phi_-\left(2i;n,\frac{12-4\epsilon}{15-8\epsilon}\right)$ \\ \cline{2-5}
    & \multirow{2}{*}{\raisebox{-3ex}{\rotatebox[origin=c]{90}{$n$ odd}}} & LHS & $\Phi_+\left(2i;n,\frac{1}{9}\right)$ & $\left( \frac{5}{3} - \frac{8}{9}\epsilon\ \right)^n \Phi_+\left(2i;n,\frac{3-4\epsilon}{15-8\epsilon}\right)$ \\
    & & RHS & $\Phi_+\left(2i;n,\frac{8}{9}\right)$ & $- \left( \frac{5}{3} - \frac{8}{9}\epsilon\ \right)^n\Phi_+\left(2i;n,\frac{12-4\epsilon}{15-8\epsilon}\right)$ \\ \hline
  \end{tabular}
  \caption{Lorenz curve elbow coordinates in the unital fragment.
  The coordinate expressions depend on the noise level $\epsilon$, the parity of the number of copies $n$ and the location of the elbow relative to the maximum (LHS vs RHS).
  Multiplication $2i$ is considered modulo $(n+1)$.
  For completeness, note that $(x_{-1}, L_{-1}) \coloneqq (0,0)$ is not included in the table.
  }
  \label{tab:lcsu_coord_elb_app}
\end{table}

We can get explicit expressions for all $9^{n}$ points of the Lorenz curve $\lc{\rho_{\rm{S}}(\epsilon)^{\otimes n}}{(\id/3)^{\otimes n}}$, in terms of the elbow coordinates:
\begin{align}
    x_{ij} &= \left( 1-\frac{j}{m_{i}} \right) x_{i-1} + \frac{j}{m_{i}} x_{i}, \label{eq:x}\\
    L_{ij} &= \left( 1-\frac{j}{m_{i}} \right) L_{i-1} + \frac{j}{m_{i}} L_{i} \label{eq:l}
\end{align}
for $j = 1,\dots,m_{i}$ and $i=0,\dots,n$, where multiplicities $m_i = m_i(n, \epsilon)$ are given in~\cref{tab:lcsu}.

Consider the state 
\begin{equation*}
\rho_{\rm{S}}(\epsilon')^{\otimes n'} \otimes \left( \frac{1}{3}\id \right)^{\otimes (n-n')},
\end{equation*}
where tensoring with the maximally mixed state keeps the Lorenz curve unchanged, but increases the resolution of (the uniformly distributed) points.
The new point coordinates are given by:
\begin{align}
    &x_{ijk} = \left( 1-p_{ijk}\right) x_{i-1} + p_{ijk} x_{i} \label{eq:lcsu_xcoord}\\
    &L_{ijk} = \left( 1-p_{ijk} \right) L_{i-1} + p_{ijk} L_{i}, \label{eq:lcsu_lcoord}\\
    &\text{where } p_{ijk} = \frac{k + (j-1)9^{n-n'}}{9^{n-n'} m_{i}} \nonumber\\
    &\text{for } i=0,\dots,n',\ j = 1,\dots,m_{i}(n', \epsilon') \text{ and } k = 1,\dots,9^{n-n'}. \nonumber
\end{align}

We can unify the indices, by introducing a single index
\begin{equation}
    I(i,j,k) \coloneqq k + \left[ (j-1) + \sum_{\ell=0}^{i-1} m_{\ell}(n', \epsilon') \right]9^{n-n'},
\end{equation}
so that $I=1,2,\dots, 9^{n}$.
The elbow coordinates correspond to 
\begin{equation}
	I(i, m_{i}(n', \epsilon'), 9^{n-n'}) = \sum_{\ell=0}^{i} m_{\ell}(n', \epsilon'),\ i= 0,\dots,n'.
\end{equation}
The index function $I$ is bijective, i.e.
\begin{equation}
	(i,j,k) = (i',j',k') \text{ iff } I(i,j,k) = I(i',j',k').
\end{equation}

%%%%%%%%%%%%%%%%%%%%%%%%%%%%%%%%%%%%%%%%

\section{Technical details for the derivation of distillation bounds from Lorenz curves}
\label{app:lcst_technical}

\subsection{First and last elbow constraints}
\label{app:elb_constraints}
Here we prove two simple majorization constraints, one arising by considering only the ascending part of the Lorenz curves between the origin $(0,0)$ and the first elbow and the other by considering only the descending part of the curves between the last elbow and the endpoint $(1,1)$.
\begin{proposition}\label{prop:first_elb}
	Consider a magic state process $\rho \longrightarrow \tau$ with input and output Lorenz curves $\lc{\rho}{\sigma}(x), \lc{\tau}{\sigma}(x)$ in $\R_\sigma$ and denote by $X_0, X'_0$ the first elbow locations of the input and output curves respectively.
	
	Then, given any coordinates $(x_0, L_0)$ and $(x'_0, L'_0)$ on the input and output Lorenz curves, where $x_0 \leq X_0$ and $x'_0 \leq X'_0$, the process is possible in $\R_\sigma$ only if
\begin{equation}\label{eq:first_elb_bound1}
	\frac{L_0}{x_0} \geq \frac{L_0'}{x_0'}.
\end{equation}
\end{proposition}
\begin{proof}
Since both pairs of coordinates are located between $(0,0)$ and the first elbow of their respective curves, we can derive the bound via a simple interpolation on the line segment connecting the origin and the appropriate first elbow.

First assume that $x_0 < x'_0$ and consider the Lorenz curve constraint at $x = x_0$,
\begin{equation}
	\lc{\rho}{\sigma}(x_0) \geq \lc{\tau}{\sigma}(x_0).
\end{equation}
We can find the output state Lorenz curve coordinate $L'_\star$ at location $x = x_0$ by interpolating between the origin and the output state's first elbow, 
\begin{equation}
	L'_\star = \frac{x_0}{x'_0}L'_0.
\end{equation}
The process is then possible only if $L_0 \geq L_\star'$ which is a rearrangement of~\cref{eq:first_elb_bound1}.

If instead, $x_0 \geq x'_0$, consider the Lorenz curve constraint at $x = x'_0$,
\begin{equation}
	\lc{\rho}{\sigma}(x'_0) \geq \lc{\tau}{\sigma}(x'_0).
\end{equation}
We now need to find the input state Lorenz curve coordinate $L_\star$ at location $x = x'_0$ by interpolating between the origin and the input state's first elbow, 
\begin{equation}
	L_\star = \frac{x'_0}{x_0}L_0.
\end{equation}
The process is then possible only if $L_\star \geq L'_0$ which is again a rearrangement of~\cref{eq:first_elb_bound1}.
\end{proof}

\begin{proposition}\label{prop:last_elb}
	Consider a magic state process $\rho \longrightarrow \tau$ with input and output Lorenz curves $\lc{\rho}{\sigma}(x), \lc{\tau}{\sigma}(x)$ in $\R_\sigma$ and denote by $X_E, X'_E$ the last elbow locations of the input and output curves respectively.
	
	Then, given any coordinates $(x_E, L_E)$ and $(x'_E, L'_E)$ on the input and output Lorenz curves, where $x_E \geq X_E$ and $x'_E \geq X'_E$, the process is possible in $\R_\sigma$ only if
\begin{equation}\label{eq:last_elb_bound1}
	\frac{L_E - 1}{1-x_E} \geq \frac{L'_E - 1}{1-x_E'}.
\end{equation}
\end{proposition}
\begin{proof}
Since both pairs of coordinates are located between the last elbow of their respective curves and $(1,1)$, we can derive the bound via a simple interpolation on the line segment connecting the endpoint and the appropriate last elbow.

First assume that $x_E > x'_E$ and consider the Lorenz curve constraint at $x = x_E$,
\begin{equation}
	\lc{\rho}{\sigma}(x_E) \geq \lc{\tau}{\sigma}(x_E).
\end{equation} 
We can find the output state Lorenz curve coordinate $L'_\star$ at location $x = x_E$ by interpolating between the endpoint $(1,1)$ and the output state's last elbow, 
\begin{equation}
	L'_\star = 1 + \frac{1-x_E}{1-x'_E} (L'_E - 1)
\end{equation}
The process is then possible only if $L_E \geq L_\star'$ which is a rearrangement of~\cref{eq:last_elb_bound1}.

If instead, $x_E \leq x'_E$, consider the Lorenz curve constraint at $x = x'_E$,
\begin{equation}
	\lc{\rho}{\sigma}(x'_E) \geq \lc{\tau}{\sigma}(x'_E).
\end{equation} 
We now need to find the input state Lorenz curve coordinate $L_\star$ at location $x = x'_E$ by interpolating between the endpoint $(1,1)$ and the input state's last elbow, 
\begin{equation}
	L_\star = 1 + \frac{1-x'_E}{1-x_E} (L_E - 1).
\end{equation}
The process is then possible only if $L_\star \geq L'_E$ which is again a rearrangement of~\cref{eq:last_elb_bound1}.
\end{proof}

\subsection{Component-multiplicity pairs}
\label{app:cmpairs}
In general, a $1$--copy $d$--dimensional state $\rho$ is described exactly by its $d^2$--dimensional Wigner distribution $\W{\rho}$. 
The distribution $\W{\rho}$ is usually defined on the phase space, but it can be convenient to define it using vector notation. 
In particular, we introduce a component vector $\bmw(\rho) = (w_i)_{i=1,\dots,D}$ and a multiplicity vector $\bmm(\rho) = (m_i)_{i=1,\dots,D}$, where $D \leq d^2$ which together form a set of component-multiplicity pairs $\{(w_i, m_i)\}_{i=1,\dots,D}$.
\begin{definition}
	Consider a distribution $W$ and a positive integer $D \leq {\rm{dim}}\hspace{1pt}W$. 
	We call the set of ordered pairs $\{(w_i, m_i)\}_{i=1,\dots,D}$ a \emph{complete set of component-multiplicity pairs}, if $W$ contains $m_i$ components $w_i$ and $\sum_{i=0}^D m_i = d^2$.
\end{definition}
Therefore, such a set describes each component of $\W{\rho}$ exactly once.
As an example, two complete sets of pairs for the Strange state are $\{( -1/3, 1), ( 1/6, 8)\}$ and $\{(-1/3, 1), (1/6, 2), (1/6, 3), (1/6, 3)\}$.
The latter corresponds to the phase space split in~\cref{fig:pd_split}

Consider two states $\rho_A, \rho_B$ with Wigner distributions $\W{\rho_A}, \W{\rho_B}$ described respectively by complete sets of component-multiplicity pairs 
\begin{equation}
	\{(w_i, m_i)\}_{i=1,\dots,D_A} \text{ and } \{(w_j', m_j')\}_{j=0,\dots,D_B}.
\end{equation}
The multiplicative property of the Wigner distribution over a composite phase space $\cal{P}_{d_A} \times \cal{P}_{d_B}$ shown in~\cref{thm:wstate},
\begin{equation}
	\W[\bmx_A \oplus \bmx_B]{\rho_A \otimes \rho_B} = \W[\bmx_A]{\rho_A}\W[\bmx_B]{\rho_B},
\end{equation}
implies that the distribution $\W{\rho_A \otimes \rho_B}$ is $d_A^2 d_B^2$--dimensional and contains components of the form $w_i w_j'$. 
Therefore, the set $\{(w_i w_j', m_i m_j')\}$ with $i=1,\dots,D_A$ and $j=1,\dots,D_B$ is a complete set of component-multiplicity pairs for the distribution of the composite system $\W{\rho_A \otimes \rho_B}$.
This is true because all components are of the form $w_i w_j'$ and 
\begin{equation*}
	\sum_{i=1}^{D_A}\sum_{j=1}^{D_B} m_i m_j' = \sum_{i=1}^{D_A} m_i \sum_{j=1}^{D_B} m_j' = d_A^2 d_B^2.
\end{equation*}

Note that the rescaled distribution is also multiplicative,
\begin{align}
	&\widetilde{\rm{W}}_{\rho_A \otimes \rho_B | \gamma_A \otimes \gamma_B}(\bmx_A \oplus \bmx_B) = \frac{\W[\bmx_A \oplus \bmx_B]{\rho_A \otimes \rho_B}}{\W[\bmx_A \oplus \bmx_B]{\gamma_A \otimes \gamma_B}} = \nonumber \\
	&\frac{\W[\bmx_A]{\rho_A}\W[\bmx_B]{\rho_B}}{\W[\bmx_A]{\gamma_A}\W[\bmx_B]{\gamma_B}} = \widetilde{\rm{W}}_{\rho_A | \gamma_A}(\bmx_A)\widetilde{\rm{W}}_{\rho_B  | \gamma_B}(\bmx_B),
\end{align}
so a complete set of component-multiplicity pairs can be obtained for this distribution in the same fashion as for usual Wigner distributions.

Given a state $\rho$ and a complete set of component-multiplicity pairs describing its Wigner distribution $\W{\rho}$, we now provide a method of computing the components (and multiplicities) of the $n$--copy distribution $\W{\rho}^{\otimes n}$.
\begin{lemma}\label{lem:ncopycomponents}
	Let $W$ be a distribution defined by a complete set of component-multiplicity pairs $\{(w_i, m_i)\}_{i=1,\dots,D}$ with $D \leq {\rm{dim}}\hspace{1pt}W$ and consider the distribution $W^{\otimes n}$ obtained by taking the $n$-fold (Kronecker) product $W \otimes \dots \otimes W$ between $n$ copies of $W$.
	
	Denote by $C_D^n \coloneqq \{\bmk\}$ the set of all vectors $\bmk \coloneqq (k_1, \dots, k_D)$ with non-negative integer components that sum to $n$, i.e.
	\begin{equation*}
	0 \leq k_1, \dots, k_D \leq n \text{ and } k_1 + \dots + k_D = n.
	\end{equation*}
	
	Then, $W^{\otimes n}$ admits a complete set of component-multiplicity pairs $\{(W_{\bmk}, M_{\bmk})\}_{\bmk \in C_D^n}$, where
\begin{align}
	M_{\bmk} &= \frac{n!}{k_1!\dots k_D!} \prod\limits_{i=1}^D {m_i}^{k_i}, \label{eq:M}\\
	W_{\bmk} &= \prod\limits_{i=1}^D {w_i}^{k_i}. \label{eq:W}
\end{align}
\end{lemma}
\begin{proof}
	We proceed by induction.
	
	Assume $n = 1$.
	Let $\bmk_i$ be the vector with its $i$-th component equal to 1 and 0's elsewhere.
	The set $C_D^1$ consists of all vectors of this form, i.e. 
\begin{equation*}
	C_D^1 = \{ \bmk_i \}_{i=1,\dots,D}
\end{equation*}
	It is also true by direct calculation that
\begin{equation*}
	\left( W_{\bmk_i}, M_{\bmk_i} \right) = (w_i, m_i).
\end{equation*}
Therefore, $\{ (W_{\bmk}, M_{\bmk}) \}_{\bmk \in C_D^1}$ is a complete set of component-multiplicity pairs for $W$.

	Assume that $\{(W_{\bmk}, M_{\bmk})\}_{\bmk \in C_D^n}$ as given in~\cref{eq:M,eq:W} is a complete set of component-multiplicity pairs for the $n$--copy distribution $W^{\otimes n}$.
	By construction, the distribution $W^{\otimes (n+1)} = W^{\otimes n} \otimes W$ is multiplicative, so it admits the complete set of component multiplicity pairs
\begin{equation}
	\{(W_{\bmk} w_i, M_{\bmk} m_i)\},\ \bmk \in C_D^n \text{ and } i=1,\dots,D.
\end{equation}
	
	Consider the component sum of the distribution $W^{\otimes (n+1)}$,
\begin{align*}
	&\sum_{\bmk \in C_D^n}\sum_{i=1}^D M_{\bmk} m_i W_{\bmk} w_i = \sum_{\bmk \in C_D^n} M_{\bmk}W_{\bmk} \sum_{i=1}^D m_i w_i =\\
	&\sum_{\bmk \in C_D^n} \frac{n!}{k_1!\dots k_D!} \prod\limits_{i=1}^D {m_i}^{k_i}{w_i}^{k_i} \sum_{i=1}^D m_i w_i =\\
	&\left( \sum_{i=1}^D m_i w_i \right)^n \left( \sum_{i=1}^D m_i w_i \right) = \left( \sum_{i=1}^D m_i w_i \right)^{n+1} =\\
	&\sum_{\bmq \in C_D^{n+1}} M_{\bmq}W_{\bmq},
\end{align*}
where in the last expression, vectors $\bmq = (q_1, \dots, q_D)$ have non-negative integer components that sum to $(n+1)$ and 
\begin{align*}
	M_{\bmq} &= \frac{(n+1)!}{q_1!\dots q_D!} \prod\limits_{i=1}^D {m_i}^{q_i},\\
	W_{\bmq} &= \prod\limits_{i=1}^D {w_i}^{q_i}.
\end{align*}
We have used the multinomial theorem to proceed between lines 2-3 and lines 3-4 of the derivation.

We have achieved a regrouping of the distribution components.
Every component $W_{\bmq}$ is of the form $W_{\bmk} w_i$ with $q_i = k_i + 1$ and $q_j = k_j$ for $j\neq i$ and 
\begin{align*}
	\sum_{\bmq \in C_D^{n+1}}  \hspace{-6pt} M_{\bmq} =  \hspace{-10pt} \sum_{\bmq \in C_D^{n+1}} \frac{(n+1)!}{q_1!\dots q_D!} \prod\limits_{i=1}^D {m_i}^{q_i} = 
	\left( \sum_{i=1}^D m_i \right)^{n+1} \hspace{-10pt} = d^{n+1},
\end{align*}
which is the dimension of $W^{\otimes (n+1)}$.

Therefore, $\{ (W_{\bmq}, M_{\bmq}) \}_{\bmq \in C_D^{n+1}}$ is a complete set of component-multiplicity pairs for $W^{\otimes n}$, completing the proof.
\end{proof}

%Index vector $\bmk$ has $D-1$ independent components and in the proof of our main theorem in~\cref{sec:stab} we have $D=4$, so we simplify the notation by writing the component and multiplicity vectors of the $n$--copy state distributions as $m_{ijk}, w(\rho_{\rm{S}})_{ijk}, w(\sigma)_{ijk}$ and $w(\rho_{\rm{S}}|\sigma)_{ijk}$, where $i,j,k$ are the 3 independent index components.

\newpage
\subsection{Free energy dependent magic distillation bounds}\label{free-energy-bound-proof}
\label{app:main_proof}
Here we prove~\cref{thm:free-energy}, which bounds magic distillation rates in terms of free energies.

\begin{theorem*}
	Consider a magic distillation protocol on qutrits that transforms $n$ copies of an $\epsilon$--noisy Strange state into $m$ copies of an $\epsilon'$--noisy Strange state, with depolarising errors $\epsilon' \leq \epsilon \leq 3/7$. 
	
	Let $T =(k\beta)^{-1}$ be any temperature for the physical system and let $H= \sum_{k \in \mathbb{Z}_3} E_k |E_k\>\<E_k|$ be the Hamiltonian of each qutrit subsystem in its eigen-decomposition.
Assume that for $n,m \rightarrow \infty$ the protocol's channel generates negligible correlations on the equilibrium state $\tau$, and so $\tau^{\otimes n} \longrightarrow \tau^{\prime \otimes m}$ for $n,m \gg 1$. We write the state $\tau'$ as $\tau' = e^{-\beta H'}/\Z'$ for some Hermitian $H'$.

Given this, there are local Clifford changes of basis $C_1,C_2$, such that $\rho_1 := C_1 \rho_S(\epsilon) C_1^\dagger$ and $\rho_2 := C_2 \rho_S(\epsilon') C_2^\dagger$ for which the protocol gives
\begin{equation}
\rho_1^{\otimes n} \longrightarrow \rho_2^{ \otimes m}.
\end{equation}
Moreover, the magic distillation rate $R = m/n$ for the protocol is bounded by the expression
\begin{equation}\label{eq:rate_bounds_proof}
	R \leq \dfrac{\ln{\big( 1-\frac{4}{3}\epsilon \big)} + \beta (\phi - F)}{\ln{\big( 1-\frac{4}{3}\epsilon' \big)} + \beta (\phi' - F')},
\end{equation}
where $F$ is the free energy of $\tau$,  and 
\begin{equation}
	\phi = -\beta^{-1} \log \zeta
\end{equation}
with $\zeta$ given by the equations
\begin{align}
	\zeta &= \sum_{k\in \mathbb{Z}_3} \alpha_k e^{-\beta E_k}, \nonumber\\
	\alpha_k &= \sum_{r \in \mathbb{Z}_3} \braket{E_k}{-r}\braket{r}{E_k}.
\end{align}
The primed variables are defined similarly for the output system.
\end{theorem*}


\begin{proof}	
For the sake of clarity, we write $\rho_n \coloneqq \rho_S(\epsilon)^{\otimes n}$, $\rho'_m \coloneqq \rho_S(\epsilon')^{\otimes m}$, $\tau_n \coloneqq \tau^{\otimes m}$ and $\tau'_m \coloneqq \tau'^{\otimes m}$. 

To establish the distillation bound we consider the distillation protocol that gives $\E( \rho_n) = \rho'_m$ for the magic states. 
We then consider that the protocol transforms the reference equilibrium state as $\E( \tau_n) = \tau'_m$.
Since $\tau$ and $\tau'$ are assumed to be full rank stabilizer states, they have a strictly positive Wigner distribution, while, in contrast, the input and output magic states generally have quasi-probability Wigner distributions. 
For any such protocol we therefore have that
\begin{equation}
	( W_{\rho_n}(\bmx), W_{\tau_n}(\bmx) ) \succ ( W_{\rho'_m}(\bmx), W_{\tau'_m}(\bmx) ),
\end{equation}
or, equivalently, in terms of the relevant Lorenz curves,
\begin{equation}
	L_{\rho_n |\tau_n}(x) \ge L_{\rho'_m |\tau'_m}(x) \mbox{ for all } x.
\end{equation}

We define the rescaled Wigner distribution $W_{\rho | \tau}(\x) \coloneqq W_\rho(\x)/W_\tau(\x)$, which is always well-defined since $\tau$ is full-rank. 
Due to the multiplicative property of the Wigner distribution, the rescaled distribution is also multiplicative in the sense that
\begin{equation}
	W_{\rho \otimes \rho' | \tau \otimes \tau'} (\x_1 \oplus \x_2) = W_{\rho | \tau}(\x_1)W_{\rho' | \tau'}(\x_2),
\end{equation}
for any states $\rho, \rho'$ and any full-rank stabilizer states $\tau, \tau'$.
Therefore we have that
\begin{align}
	W_{\rho_n |\tau_n} (\x) &= \prod_{i=1}^n W_{\rho|\tau}(\x_i)\\
	W_{\rho_n } (\x) &= \prod_{i=1}^n W_\rho(\x_i)
\end{align}
where $\x = \oplus_{i=1}^n \x_i \in \mathbb{Z}_3^n$ is the phase space point for the full system in terms of those of the individual subsystems.

The points defining the Lorenz curve $L_{\rho_n |\tau_n}(x)$ are obtained from sorting the components of $W_{\rho_n |\tau_n}(\x)$ in non-increasing order and then computing the partial sums of $W_{\rho_n |\tau_n}(\pi(\x))$ where $\pi$ is the permutation that realises the sorting. 
However, similarly to the unital fragment analysis, we aim to use the constraint that is obtained by considering the line segment connecting the origin to the first elbow of both Lorenz curves.

The Wigner distribution of a single noisy Strange state consists of 1 negative component $W_{\rho}(0,0) = -v(\epsilon)$ and 8 positive components $W_{\rho}(\bmx) = u(\epsilon)$ for $\bmx \neq \bmo$.
The Wigner distribution of the full-rank, stabilizer equilibrium state $\tau$ is $W_{\tau}(\bmx) > 0$ for all $\bmx \in \mathbb{Z}_3^2$.

Assume that the smallest component of the distribution $W_\tau(\x)$ is at $x=\x_\star$.
We then make use of the freedom to apply Clifford unitaries in order to perform the  Clifford operation
\begin{equation}
	\tau \longrightarrow X^{-i}Z^{-j}\ \tau\ Z^{j}X^{i},
\end{equation}
which permutes the smallest component to the origin of the phase space, $W_{\tau}(i,j) \longrightarrow W_{\tau}(0,0)$.
This operation would inevitably affect the magic distillation protocol, but since we consider the state distillation modulo Clifford operations, we are allowed to perform the reverse Clifford operation keeping the magic distillation process as is. \nick{statement here is inaccurate - incorporate $C_1, C_2$ in the proof.}

The rescaled distribution components are in general given by
\begin{equation}
	W_{\rho_n|\tau_n} = \left(\frac{-v}{W_{\tau}(0,0)}\right)^{i_{\bmo}} \prod_{\bmx \neq \bmo} \left(\frac{u}{W_{\tau}(\bmx)}\right)^{i_{\bmx}},
\end{equation}
where the (integer) indices obey the following conditions:
\begin{align}
&0 \leq i_{\bmx} \leq n \mbox{ for all } \bmx \in \mathbb{Z}_3^2, \nonumber \\
&\sum_{\bmx \in \mathbb{Z}_3^2} i_{\bmx} = n.
\end{align}
It is now easy to find the largest rescaled component.
Firstly, note that $n$ is even, so we require that $i_{\bmo} \in \{0,2,\dots,n\}$ for the component to be positive.
Then, we have that $v \geq u$ because $\epsilon \leq 3/7$, and we have already ensured that $W_{\tau}(0,0) \leq W_{\tau}(\bmx)$ for all $\bmx \in \mathbb{Z}_3^2$.
Therefore, the largest rescaled component occurs when $i_{\bmo} = n$ and $i_{\bmx} = 0$ for $\bmx \neq \bmo$ and is equal to $(v/W_{\tau}(0,0))^n$.
Accordingly, the coordinates of the first Lorenz curve point after the origin are given by
\begin{equation}
	(x_0, L_0) = ((W_{\tau}(0,0))^n, v^n)
\end{equation}

Given a Hamiltonian decomposition,
\begin{equation}
	H = \sum_{k \in \mathbb{Z}_3} E_k\ketbra{E_k}{E_k},
\end{equation}
we can re-express the coordinate location, by writing
\begin{align}
	W_{\tau}(0,0) &= \frac{1}{3\Z}\tr\left[ A_{0,0}e^{-\beta H} \right] \nonumber\\
	&= \frac{e^{\beta F}}{3} \tr\left[ \sum_{r \in \mathbb{Z}_3} \ketbra{-r}{r}\sum_{k \in \mathbb{Z}_3} e^{-\beta E_k}\ketbra{E_k}{E_k} \right] \\
	&= \frac{e^{\beta F}}{3} \sum_{r \in \mathbb{Z}_3} \alpha_k e^{-\beta E_k}
	= \frac{e^{\beta F}}{3} \zeta \\
	&= \frac{e^{\beta (F - \phi)}}{3},
\end{align}
where $\alpha_k, \zeta$ and $\phi$ are as defined in the statement of the theorem.
Finally, the coordinates can be expressed as
\begin{equation}
	(x_0, L_0) = \left( \frac{e^{n\beta (F - \phi)}}{3^n}, v(\epsilon)^n \right).
\end{equation}
In the same manner, we can derive the first point coordinates of the output Lorenz curve as
\begin{equation}
	(x'_0, L'_0) = \left( \frac{e^{n\beta (F' - \phi')}}{3^n}, v(\epsilon')^n \right).
\end{equation}

If the largest rescaled component of a state is distinct with no multiplicities, then these coordinates correspond to the first elbow of the corresponding Lorenz curve, whereas if it appears multiple times, then the coordinates derived correspond to a point on the interior of the line segment connecting the origin to the first elbow.
In both cases, the distillation bound remains the same, as is clear by its derivation in~\cref{app:elb_constraints}, and it is given by $L_0 / x_0 \geq L'_0 / x'_0$, which can be directly rearranged into the stated bound of the theorem.
\end{proof}
	
\subsection{Deriving distillation bounds from the last elbow}
\label{sec:last_elb}

Using a similar analysis, we can also derive upper bounds from comparison of the last point coordinates of the Lorenz curve, which now corresponds to the smallest rescaled component.

To this end, assume that the largest component of the equilibrium state distribution is $W_{\tau}(i,j)$.
We now perform the Clifford operation
\begin{equation}
	\tau \longrightarrow X^{1-i}Z^{-j}\ \tau\ Z^{j}X^{i-1},
\end{equation}
which permutes the largest component such that $W_{\tau}(i,j) \longrightarrow W_{\tau}(1,0)$.
Since we have that $u \leq v$ because $\epsilon \leq 3/7$, and we have already ensured that $W_{\tau}(1,0) \geq W_{\tau}(\bmx)$ for all $\bmx \in \mathbb{Z}_3^2$.
Therefore, the largest rescaled component occurs when $i_{(1,0)} = n$ and $i_{\bmx} = 0$ for $\bmx \neq \bmo$ and is equal to $(v/W_{\tau}(1,0))^n$.
Accordingly, the coordinates of the first Lorenz curve point after the origin are given by $(x_E, L_E) = ((W_{\tau}(1,0))^n, u(\epsilon)^n)$ for the input state and $(x'_E, L'_E) = ((W_{\tau'}(1,0))^m, u(\epsilon')^m)$ for the output state.

We now redefine the quantities $\alpha_k$, so that again $\phi = -\beta^{-1} \log \zeta$, where $\zeta= \sum_{k\in \mathbb{Z}_3} \alpha_k e^{-\beta E_k}$ and $\alpha_k$ now given by
\begin{align}
	\alpha_k = \sum_{r \in \mathbb{Z}_3} \braket{E_k}{1-r}\braket{1+r}{E_k}.
\end{align}
This allows us to rewrite $W_{\tau}(1,0) = {e^{\beta (F - \phi)}}/{3}$ and by using the last elbow constraint 
\begin{equation*}
	\frac{L_E - 1}{1-x_E} \geq \frac{L'_E - 1}{1-x_E'},
\end{equation*}
derived in~\cref{app:elb_constraints}, we can get a new bound expression.

We note that it is also possible to perform a different Clifford transformation and get a different location $(W_{\tau}(\bmx))^n$, with $\bmx \neq \bmo$, since components $(W_{\rho}(\bmx))^n = u$ are all the same as long as $\bmx \neq \bmo$.
The effect of this is to alter the expressions for the quantities $\alpha_k$, but not the resulting bound expression.

\nick{This section on the last elbow bound is a little weak?}

\subsection{OLD VERSION OF MAIN THEOREM}

\ddd{[Do we need to keep any of the details in this section? If not then delete it entirely.]}
\nick{This proof offers: 1. the exact derivation of our main contour plot, justifying the quantities $\epsilon_\star$ and $\beta_\star$ 2. a method for lifting the ``Clifford change in basis'' assumption - Can be trimmed down to the case where $H=H'$. What do you think?}
\begin{theorem}
	Consider a magic distillation protocol transforming noisy Strange state
	\begin{equation*}
		\rho_S(\epsilon)^{\otimes n} \longrightarrow \E(\rho_S(\epsilon)^{\otimes n})=\rho_S(\epsilon')^{\otimes m} 
	\end{equation*}
	with $n, m \gg 1$.

Let each qutrit have a Hamiltonian $H$ with stabilizer eigenstates and energies $E_0, E_1, E_2$, and define $E_{\rm max} = \max\{E_0, E_1, E_2\}$ and $E_s$ the eigenvalue of the eigenstate that \nick{overlaps} the negative component of $\rho_S(\epsilon)$ in the Wigner representation. Let $T =(k\beta)^{-1}$ be any characteristic temperature for the physical system in the state $\tau^{\otimes n}= (e^{-\beta H}/\Z)^{\otimes n}$, with free energy $F$. Assume that for $n,m \gg 1$ the channel $\E$ generates negligible correlations on $\tau^{\otimes n}$, and so $\E(\tau^{\otimes n}) = \sigma^{\otimes m}$ for some state $\sigma$.

Define $\beta_\star = (k T_\star)^{-1}$ through the relation
\begin{equation}
	E_{\rm max} - E_s \eqqcolon kT_\star \ln 2,
\end{equation}
and define a threshold noise,
\begin{equation}
	\epsilon_{\star}(\beta) \coloneqq 
	\begin{cases}
		3 - \dfrac{9}{4-2^{\beta/\beta_\star - 1}}, &\text{ for } \beta \leq \beta_\star \\
		0, &\text{ for } \beta > \beta_\star.
	\end{cases}
\end{equation}
Then, the distillation rate $R = m/n$ of the magic protocol is bounded as:
\begin{equation}\label{eq:rate_bounds_proof}
	R \leq
	\begin{cases}
		\dfrac{\ln{\big( 1-\frac{4}{3}\epsilon \big)} + \beta (E_s - F)}{\ln{\big( 1-\frac{4}{3}\epsilon' \big)} + \beta (E'_s - F')},\ &\epsilon \leq \epsilon_\star, \epsilon' \leq \epsilon'_\star, \vspace{10pt}\\
		\dfrac{\ln{\big( 1-\frac{4}{3}\epsilon \big)} + \beta (E_s - F)}{\ln{\big( \frac{1}{2}-\frac{1}{6}\epsilon' \big)} + \beta (E'_{\rm{max}} - F')},\ &\epsilon \leq \epsilon_\star, \epsilon' > \epsilon'_\star, \vspace{10pt}\\
		\dfrac{\ln{\big( \frac{1}{2}-\frac{1}{6}\epsilon \big)} + \beta (E_{\rm{max}} - F)}{\ln{\big( 1-\frac{4}{3}\epsilon' \big)} + \beta (E'_s - F')},\ &\epsilon > \epsilon_\star, \epsilon' \leq \epsilon'_\star, \vspace{10pt}\\
		\dfrac{\ln{\big( \frac{1}{2}-\frac{1}{6}\epsilon \big)} + \beta (E_{\rm{max}} - F)}{\ln{\big( \frac{1}{2}-\frac{1}{6}\epsilon' \big)} + \beta (E'_{\rm{max}} - F')},\ &\epsilon > \epsilon_\star, \epsilon' > \epsilon'_\star,
	\end{cases}
\end{equation}
where $F'$ is the free energy of the state $\sigma = e^{-\beta H'}/\Z'$ and other primed quantities are defined in the same way as the corresponding unprimed quantities. 
\end{theorem}
We note, firstly, that the specific numerical factors in $\epsilon_\star$ are a result of our choice of magic state. 
\begin{proof}
To establish the distillation bounds we consider the distillation protocol that gives
\begin{equation}
	\E( \rho_S(\epsilon)^{\otimes n}) = \rho_S(\epsilon')^{\otimes m},
\end{equation}
for the magic state. We then consider how the protocol transforms the reference equilibrium state as
\begin{equation}
	\E( \tau^{\otimes n}) = \sigma^{\otimes m}.
\end{equation}
Since $\tau$ and $\sigma$ are assumed to be full rank stabilizer states they have a strictly positive Wigner distribution, while, in contrast, the input and output magic states will generally have quasi-probability distributions for their Wigner functions. For any such protocol we therefore have that
\begin{equation}
	(W_{\rho_n} (x), W_{\tau_n}(x) ) \succ  (W_{\rho'_m} (x), W_{\sigma_m}(x) ),
\end{equation}
where, for the sake of clarity, we write $\rho_n \coloneqq \rho_S(\epsilon)^{\otimes n}$, $\rho'_m \coloneqq \rho_S(\epsilon')^{\otimes m}$ and $\tau_n \coloneqq \tau^{\otimes m}$, $\sigma_m \coloneqq \sigma^{\otimes m}$. We then have that
\begin{equation}
	L_{\rho_n |\tau_n}(x) \ge L_{\rho'_m |\sigma_m}(x) \mbox{ for all } x.
\end{equation}
We must therefore compute the Lorenz curve data for $\rho_n$ relative to $\tau_n$, and compare with the Lorenz curve of the output state $\rho_m'$ relative to $\sigma_m$.

We define $W_{\rho | \tau}(\x) \coloneqq W_\rho(\x)/W_\tau(\x)$, which is always well-defined since $\tau$ is full-rank. Now we show in~\cref{app:wigner} that
\begin{equation}
W_{\rho_1\otimes \rho_2 | \tau_1 \otimes \tau_2} (\x_1 \oplus \x_2) = W_{\rho_1 | \tau_1}(\x_1)W_{\rho_2 | \tau_2}(\x_2),
\end{equation}
and also,
\begin{equation}
W_{\rho_1\otimes \rho_2} (\x_1 \oplus \x_2) = W_{\rho_1}(\x_1)W_{\rho_2}(\x_2),
\end{equation}
for any states $\rho_1, \rho_2$ and any full-rank stabilizer states $\tau_1, \tau_2$. \ddd{[Notation clashing here. Frame result outside as its own lemma -- we gotta check everything.]} \nick{Will frame in~\cref{app:wigner} and fix notation in the next commit}
Therefore we have that
\begin{align}
W_{\rho_n |\tau_n} (\x) &= \prod_{i=1}^n W_{\rho|\tau}(\x_i)\\
W_{\rho_n } (\x) &= \prod_{i=1}^n W_\rho(\x_i)
\end{align}
where $\x = \oplus_{i=1}^n \x_i$ is the phase space point for the full system in terms of those of the individual subsystems.

The points defining the Lorenz curve $L_{\rho_n |\tau_n}(x)$ are obtained from first sorting the components of $W_{\rho_n |\tau_n}(\x)$ in non-increasing order and then computing the partial sums of $W_{\rho_n |\tau_n}(\pi(\x))$ where $\pi$ is the permutation that realises the sorting. Therefore, we first look at the values of $W_{\rho|\tau}(\x_i)$ for the single-copy case.

The equilibrium state at inverse temperature $\beta$ on a single qutrit is given by $\tau = e^{-\beta H} / \Z$. Moreover we have that $\tau$ is a full-rank stabilizer state, where $\beta \geq 0$ and $H = E_0 \ketbra{\varphi_0} + E_1 \ketbra{\varphi_1} + E_2 \ketbra{\varphi_2}$ is an eigendecomposition of $H$.
The state $\tau$ can now be written as 
\begin{equation}
	\tau = \frac{e^{-\beta E_0}}{\Z} \ketbra{\varphi_0} + \frac{e^{-\beta E_1}}{\Z} \ketbra{\varphi_1} + \frac{e^{-\beta E_2}}{\Z} \ketbra{\varphi_2},
\end{equation}
where the eigenstates $\{\ket{\varphi_k}\<\varphi_k|\}$ are pure, orthonormal stabiliser states, which can be represented in terms of generalized Paulis. To make our analysis simpler, we perform a change of basis that does not affect the Wigner negativity of the problem. We let $C$ be the unitary transforming each $|\varphi_k\>\<\varphi_k|$ to $|k\>\<k|$. Since the Clifford group is the normalizer of the Heisenberg-Weyl group, $C$ is a Clifford unitary. Therefore, $C$ maps $\tau$ to another stabilizer state that is diagonal in the computational basis, and we can assume without loss of generality that $\tau$ is diagonal in $\{|0\>,|1\>, |2\>\}$. However, this choice means that the location of the negative Wigner component $-v(\epsilon)$ of the Strange state will be permuted on the discrete phase space. We denote by $E_s$ the eigenvalue of $H$ where the associated eigenvector has Wigner distribution overlapping the negative component of the magic state $C\rho_S(\epsilon)C^\dagger$.  This is unique, since the eigenstates form an orthonormal basis.

The Wigner distribution of state $\tau$ is then given by
\begin{align}
	\W[\bmx]{\tau} &= \sum\limits_{k=0}^2 \frac{e^{-\beta E_k}}{\Z}W_{\ketbra{k}}(x, p) \nonumber\\
	&= \sum\limits_{k=0}^2 \frac{e^{-\beta E_k}}{\Z} \delta_{x, k} = \frac{e^{-\beta E_x}}{3\Z},
\end{align}
where $x$ labels one of the three vertical lines in the phase space.
The rescaled Wigner distribution $W_{\rho|\tau}(\x)$ is then easily computed. It has $9$ components, but several of these come with multiplicities. In total, there are four distinct values on the phase space, as illustrated in~\cref{fig:pd_split}.
\begin{figure}[h]
    \centering
    \includegraphics[scale=0.45]{figs/pd_split_thermal.pdf}
    \caption{\textbf{Qutrit phase space regions for $W_{\rho | \tau}(\x)$.}
    Here, the negative component of the magic state overlaps the Wigner distribution of $|0\>$. The rescaled distribution attains a single value in each of the four regions, proportional to the value depicted in the region, see~\cref{eq:bmw_rescaled}.
    }
    \label{fig:pd_split}
\end{figure}

We now denote by $\w(\rho), \w(\rho|\tau)$ the unique values occurring in $W_\rho(\x), W_{\rho|\tau}(\x)$ respectively and $\bmm$ the vector of associated multiplicities of each value in $W_\rho(\x)$. The component values and multiplicities of the relevant distributions in the four distinct regions are given by
\begin{align}
	\w(\rho) &\coloneqq (-v, u, u, u), \\
		\bmm &\coloneqq (1,2,3,3), \\
	\bmw(\tau) &\coloneqq \frac{1}{3\Z} \left( e^{-\beta E_0}, e^{-\beta E_0}, e^{-\beta E_1}, e^{-\beta E_2} \right), \\
	\bmw(\rho_{\rm{S}} | \tau) &\coloneqq 3\Z \left( -v e^{\beta E_0}, u e^{\beta E_0}, u e^{\beta E_1}, u e^{\beta E_2} \right). \label{eq:bmw_rescaled}
\end{align}

Using this notation, the values and multiplicities of the $n$--copy distribution $\bmw(\rho_n |\tau_n)$ are computed in~\cref{lem:ncopycomponents} in~\cref{app:cmpairs}. The values are given by 
\begin{align}\label{eq:ncopy_w_rescaled}
	[\w(\rho_n | \tau_n)]_{ijk} &= (3\Z)^{n} (-v)^{n-\alpha} u^{\alpha} e^{\beta (n-\alpha)E_s} e^{\beta ( i E_0 + j E_1 + k E_2 )},
\end{align}
where the indices $i,j,k$ are non-negative integers that obey the constraint $\alpha \coloneqq i+j+k \leq n$.
The multiplicity of this above value is $m_{ijk}$ with
\begin{equation}
	m_{ijk} = \frac{n!}{i!j!k!(n-\alpha)!} 2^i 3^j 3^k.
\end{equation}
The associated components of $\w(\rho_n)$ are given by
\begin{align}
	[\w(\rho_n)]_{ijk} &= (-v)^{n-\alpha} u^{\alpha}, \label{eq:ncopy_wrho}\\
	[\w(\tau)]_{ijk} &= (3\Z)^{-n} e^{-\beta (n-\alpha)E_s} e^{-\beta ( i E_0 + j E_1 + k E_2 )}. \label{eq:ncopy_wsigma}
\end{align}

In order to construct the $n$--copy Lorenz curve $L_{\rho_n|\tau_n}(x)$ we need to order the components of the distribution, $w(\rho_{\rm{S}} | \tau)_{ijk}$ in decreasing order, and identify the sequence of indices that give us $W_{\rho_n}(\pi(\x))$.

Generally this is complex, but in order to obtain distillation bounds it is sufficient to determine the location of the first elbow $(x_0, L_0)$ of $L_{\rho_n|\tau_n}(\x)$. To do so, we compute the largest component 
\begin{equation}
	w_{\rm max} \coloneqq \max_{i,j,k} [\w(\rho_n | \tau_n)]_{ijk},
\end{equation}
and determine the indices at which this occurs.
Putting in the values we obtain
\begin{align}
	&(3\Z)^{-n}w_{\rm max} = \nonumber\\
	&\max\limits_{i,j,k}\Big\{ (-v)^{n-\alpha} u^{\alpha} e^{\beta (n-\alpha)E_s} e^{\beta ( i E_0 + j E_1 + k E_2 )} \Big\}, \label{eq:max_slope}
\end{align}
where $0 \leq i,j,k \leq n$ and $\alpha \coloneqq i+j+k \leq n$.
Now for $0 \leq \epsilon \leq 3/7$, we have $v \geq u$. Since we assume that $n$ is even, we need the sum $\alpha = i+j+k$ to be even too, so that the expression is positive. 

Given an even value for $\alpha$, the term $v^{n-\alpha} u^{\alpha} e^{-\beta (n-\alpha)E_s}$ is fixed, so the expression is maximised by setting the coefficient of the highest energy $E_{\rm{max}}$ equal to $\alpha$.
Hence, we have
\begin{align}
	&w_{\rm max} = \nonumber\\
	&(3\Z)^{n} v^n e^{n\beta E_s}\max\limits_{\substack{\alpha = 0,2, \\ \dots,n-2,n}}{\Big\{ \left( \frac{u}{v} e^{\beta (E_{\rm{max}} - E_s)} \right)^{\alpha} \Big\}}.
\end{align}
If the expression $\frac{u(\epsilon)}{v(\epsilon)} e^{\beta (E_{\rm{max}} - E_s)}$ is less than $1$ then the maximum occurs at $\alpha=0$, otherwise it occurs at $\alpha = n$. For a fixed state $\tau$, this transition is determined by the value of the depolarising noise parameter $\epsilon$ of the noisy magic state. The transition occurs at $\epsilon = \epsilon_\star$ where
\begin{equation}\label{eq:noise_transition}
	\frac{u(\epsilon_\star)}{v(\epsilon_\star)} e^{\beta (E_{\rm{max}} - E_s)} = \frac{3-\epsilon_\star}{6-8\epsilon_\star} e^{\beta (E_{\rm{max}} - E_s)} = 1.
\end{equation}
If $E_{\rm{max}} = E_s$, namely if the state negativity lies in the same phase space region as the highest energy, this threshold is constant in temperature and given by $\epsilon_{\star} = 3/7$. However, the condition that $\epsilon_\star \ge 0$ also implies a constraint on the effective temperature of the stabilizer state. Specifically, there is a threshold temperature value $\beta_\star$ given by
\begin{equation}
	\beta_{\star} \coloneqq \frac{1}{E_{\rm{max}} - E_s} \ln2,
\end{equation}
such that for the regime $0 \leq \beta \leq \beta_\star$ a transition noise level $\epsilon_\star$ exists, and for $\beta > \beta_\star$ no such transition exists, so we choose $\epsilon_\star = 0$. 
Therefore, the transition value for the noise is given by
\begin{equation}
	\epsilon_{\star}(\beta) \coloneqq 
	\begin{cases}
		3 - \dfrac{9}{4-2^{\beta/\beta_\star - 1}}, &\text{ for } \beta \leq \beta_\star \\
		0, &\text{ for } \beta > \beta_\star.
	\end{cases}
\end{equation}
The quantity $w(\rho_{\rm{S}} | \sigma)_{\rm{max}}$ is now given by
\begin{equation*}
w_{\rm max} =
	\begin{cases}
		(3\Z)^{n} v^n e^{n\beta E_s}, &\mbox{if }\epsilon \leq \epsilon_{\star},\ \hspace{3pt}\rm{(C1)}\\
		(3\Z)^{n} u^n e^{n\beta E_{\rm{max}}}, &\mbox{if }\epsilon > \epsilon_{\star}.\ \hspace{5pt}\rm{(C2)} 
	\end{cases}
\end{equation*}
Case $\rm{(C1)}$ corresponds to $(i,j,k) = (0,0,0)$, so the multiplicity is $m_{000} = 1$, while
Case $\rm{(C2)}$ corresponds to
\begin{equation}
	(i,j,k) = 
	\begin{cases}
	(0,n,0), &\text{if } E_{\rm{max}} = E_1, \\
	(0,0,n), &\text{if } E_{\rm{max}} = E_2,
	\end{cases}
\end{equation}
so the multiplicity in both cases is $3^n$.

Using that $F = -\beta^{-1} \log \Z$, the first elbow coordinates in the two cases are now given by
\begin{equation}\label{eq:first_elb_coords}
	(x_0, L_0) =
	\begin{cases}
		\left(\frac{1}{3^n} e^{-n\beta (E_s - F)}, v^n \right), &\epsilon \leq \epsilon_\star \vspace{10pt}\\
		\left( e^{-n\beta (E_{\rm{max}}-F)}, (3u)^n \right). &\epsilon > \epsilon_\star
	\end{cases}
\end{equation}

\ddd{[Hmmm this needs an additional assumption on the output state to re-run the same analysis. Very annoying.]}\nick{What additional assumption beyond that the output state can be written as a Gibbs state for some $H'$?}
Similarly, considering the output magic state with respect to state $\sigma'$, the image of equilibrium state $\sigma$ under the magic protocol, we get output Lorenz curve coordinates,
\begin{equation}\label{eq:transformed_first_elb_coords}
	(x'_0, L'_0) =
	\begin{cases}
		\left(\frac{1}{3^{n'}} e^{-n\beta (E'_s - F')}, v(\epsilon')^{n'} \right), &\epsilon' \leq \epsilon'_\star \vspace{10pt}\\
		\left( e^{-n'\beta (E'_{\rm{max}}\hspace{-2.5pt}-F')}, (3u(\epsilon'))^{n'} \right), &\epsilon' > \epsilon'_\star
	\end{cases}
\end{equation}
There are four combinations of coordinates, depending on the noise parameters $\epsilon, \epsilon'$ for the input and output states.
In each of these combinations, we simply use the first elbow constraint, as described in~\cref{app:elb_constraints}, which leads to the bounds in the statement of the theorem.
\end{proof}






























\end{document}