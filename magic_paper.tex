\documentclass[pra,
aps,
twocolumn,
superscriptaddress,
groupedaddress,
nofootinbib,
reprint
]{revtex4-1}

% PACKAGES
\usepackage{amsmath,amsfonts, amssymb, amsthm}
\usepackage{bm, bbm, physics, mathtools}
\usepackage{graphicx, subfigure, multirow, makecell}
\usepackage{xcolor, enumerate}
\usepackage{xifthen, hyperref}
\usepackage[capitalise]{cleveref}

\hypersetup{
	colorlinks=true,  
	linkcolor=blue,   
	citecolor=blue,   
	urlcolor=blue     
}

\newcommand{\crefrangeconjunction}{--}
\creflabelformat{figure}{(#2#1#3)}

% COMMENT NOTATION
\newcommand{\nick}[1]{{\color{red}[#1]}}
\newcommand{\ddd}[1]{\textcolor{blue}{#1}}

% ENVIRONMENTS
\newtheorem{theorem}{Theorem}
\newtheorem{proposition}[theorem]{Proposition}
\newtheorem{lemma}[theorem]{Lemma}
\newtheorem{definition}[theorem]{Definition}
\newtheorem{example}{Example}

% REFERENCES
\iffalse
\renewcommand{\eqref}[1]{Eq.~(\ref{#1})}
\newcommand{\figref}[1]{Fig.~(\ref{#1})}
\newcommand{\tabref}[1]{Tab.~(\ref{#1})}
\newcommand{\secref}[1]{Section~(\ref{#1})}
\newcommand{\appref}[1]{Appendix~(\ref{#1})}
\newcommand{\defref}[1]{Definition~\ref{#1}}
\newcommand{\lemref}[1]{Lemma~\ref{#1}}
\newcommand{\thmref}[1]{Theorem~\ref{#1}}
\fi

% SYMBOL DEFINITIONS
\renewcommand{\cal}[1]{\mathcal{#1}}

\newcommand{\reals}{\mathbb{R}}
\newcommand{\id}{\mathbbm{1}}
\newcommand{\idc}{1_{\rm{C}}}
\newcommand{\supf}{\mathfrak{c}}
\newcommand{\floor}[1]{\left\lfloor #1 \right\rfloor}
\newcommand{\ent}[2]{S\left( #1 \middle\vert\middle\vert #2 \right)}
\newcommand{\ents}{{\ent{\frac{m}{n}}{p}}}
\renewcommand{\tr}{{\rm{tr}}}
\renewcommand{\det}{{\rm{det}}}


\newcommand{\spanv}[1]{
    {{\rm{span}}\left\{#1\right\}}
}
\newcommand{\conv}[1]{
    {{\rm{conv}}#1}
}
\newcommand{\orb}[1]{
    {{\rm{orb}}(#1)}
}
\newcommand{\sn}[1]{
    {{\rm{sn}}\left(#1\right)}
}
\newcommand{\mana}[1]{
    {{\rm{mana}}\left(#1\right)}
}
\newcommand{\lc}[2]{
	{{\rm{L}}_{#1|#2}}
}

\newcommand{\bmx}{\bm{x}}
\newcommand{\bmy}{\bm{y}}
\newcommand{\bmz}{\bm{z}}
\newcommand{\bmu}{\bm{u}}
\newcommand{\bmo}{\bm{0}}
\newcommand{\bmd}{\bm{d}}
\newcommand{\bma}{\bm{a}}
\newcommand{\bmw}{\bm{w}}
\newcommand{\bmm}{\bm{m}}
\newcommand{\bmk}{\bm{k}}
\newcommand{\bmq}{\bm{q}}
\newcommand{\bmg}{\bm{g}}

\newcommand{\spd}[1][]{
    \ifthenelse{\isempty{#1}}{
    {{\rm{Sp}}(2, \zd)} }{
    {{\rm{Sp}}(2, \zd[#1])}}
}
\newcommand{\stoch}[1][]{
    \ifthenelse{\isempty{#1}}{
    {{\rm{S}}_d(\bmd)} }{
    {{\rm{S}}_d(#1)}}
}
\newcommand{\stochw}[1][]{
    \ifthenelse{\isempty{#1}}{
    {{\rm{S}}_{d^2}(\W{\sigma})} }{
    {{\rm{S}}_{d^2}(#1)}}
}
\makeatletter
\def\W{\@ifnextchar[{\@with}{\@without}}
\def\@with[#1]#2{ 
    {{\rm{W}}_{#2}\left(#1\right)} }
\def\@without#1{ 
    {{\rm{W}}_{#1}} }
\makeatother

\newcommand{\T}{\cal{T}}
\newcommand{\Z}{\cal{Z}}
\newcommand{\C}{\cal{C}}
\newcommand{\E}{\cal{E}}
\newcommand{\J}{\cal{J}}
\newcommand{\R}{\cal{R}}
\newcommand{\D}{\cal{D}}
\newcommand{\M}{\cal{M}}
\newcommand{\F}{\cal{F}}
\renewcommand{\O}{\cal{O}}

\newcommand{\Fmax}{\F_{\rm{max}}}
\newcommand{\Omax}{\O_{\rm{max}}}
\newcommand{\Rmax}{\R_{\rm{max}}}
\newcommand{\Pis}{\Pi_{\rm{s}}}
\newcommand{\Pio}{\Pi_{\rm{o}}}

\newcommand{\cptp}{{\rm{CPTP}}}
\newcommand{\cpos}{{\rm{CP}}}
\newcommand{\so}{{\rm{SO}}}
\newcommand{\stab}{{\rm{STAB}}}
\newcommand{\spo}{{\rm{SPO}}}
\newcommand{\cspo}{{\rm{CSPO}}}
\newcommand{\rcu}{{\rm{RCU}}}
\newcommand{\tho}{{\rm{TO}}}
\newcommand{\cpwp}{{\rm{CPWPO}}}
\newcommand{\ru}{{\rm{RU}}}



%%%%%% DJ DEFINITIONS %%%%


\def\>{\rangle}
\def\<{\langle}
\def\K{ {\mathcal K} }
\def\E{ {\mathcal E} }
\def\P{ {\mathcal P} }
\def\H{ {\mathcal H} }
\def\M{ {\mathcal M} }

\def\U {{\mathcal U}}
\def\R {{\mathcal R}}
\def\G {{\mathcal G}}
\def\N{ {\mathcal N} }
\def\F{ {\mathcal F} }
\def\A{ {\mathcal A} }
\def\B{ {\mathcal B} }
\def\O{ {\mathcal O} }
\def\P{ {\mathcal P} }
\def\D{ {\mathcal D} }
\def\T{ {\mathcal T} }
\def\I{ \mathbbm{1} }
\def\tr{ \mbox{tr} }
\def\non{ \nonumber\\}
\def\plus{ |+\> }
\def\minus{|-\> }
\def\plusdag{ \<+| }
\def\minusdag{ \<-| }
\def\diag{ \mathrm{diag}}

\def\r{\boldsymbol{r}}
\def\w{\boldsymbol{w}}
\def\x{\boldsymbol{x}}
\def\y{\boldsymbol{y}}
\def\z{\boldsymbol{z}}
\def\t{\boldsymbol{t}}
\def\p{\boldsymbol{p}}
\def\q{\boldsymbol{q}}


%%%%%%%%%%%%%%%%

\begin{document}

\begin{abstract}
Magic states are essential for achieving universality in fault-tolerant schemes.
Magic resource theories attempt to quantify magic via monotones and thus describe the manipulation of magic states.
Here we introduce the concept of majorization fragments as a more generalised projection of such theories in discrete odd dimensions which allows for more powerful results than what monotones can provide.
Fragments naturally link the symmetries of a gate sequence with conditions on the convertibility between states.
We demonstrate the power of fragments by providing exact conditions for the convertibility of single-copy qutrit magic states as well as stricter distillation bounds than the established mana monotone in any odd dimension.
\end{abstract}

\preprint{APS/123-QED}

\title{Thermodynamic fragments in magic state theories}

\author{Nikolaos Koukoulekidis}
	\email{nk2314@imperial.ac.uk}
	\affiliation{Department of Physics, Imperial College London, London SW7 2AZ, UK}
\author{David Jennings}
	\affiliation{School of Physics and Astronomy, University of Leeds, Leeds, LS2 9JT, UK}
	\affiliation{Department of Physics, Imperial College London, London SW7 2AZ, UK}

\date{\today}
\maketitle

%%%%%%%%%%%%%%%%%%%%%%%%%%%%%%%%%%%%%%%%

\section{Introduction}
\label{sec:intro}

\begin{enumerate}
    \item Fault-tolerance~\cite{cit:raussendorf, cit:gross2, cit:markov, cit:gross, cit:nest, cit:nest2, cit:vidal, cit:fujii, cit:gottesman}
    \item Magic~\cite{cit:veitch, cit:veitch2, cit:wang, cit:wang2, cit:howard, cit:campbell, cit:gross3, cit:gross4}
    \item Bringing in majorization~\cite{cit:cwiklinski, cit:lostaglio2, cit:lostaglio, cit:gour, cit:janzing, cit:brandao2, cit:gour2, cit:marshall, cit:nielsen}
    \item Section breakdown
\end{enumerate}

%%%%%%%%%%%%%%%%%%%%%%%%%%%%%%%%%%%%%%%%

\section{Magic Resource Theories}
\label{sec:magic}

\subsection{Introduction}

Magic states are necessary for achieving universal quantum computation within fault-tolerant schemes.
Identifying magic as a resource for quantum universality has led to several theories which try to provide a framework for its quantification and manipulation \nick{CITE}.
The main question that such a theory attempts to answer is:
\begin{center}
    \emph{Given two magic states $\rho$ and $\rho'$ is there a free operation that can convert $\rho$ to $\rho'$?}
\end{center}

We are interested in all resource theories of magic $\R = (\F, \O)$ in which free operations cannot generate any amount of resource. 
Further denote by $\D$ the set of states considered under the theory, that is the union of free and resource states.
The structure of such theory is described by a partial order \nick{CITE}, hereinafter called a \emph{pre-order}, $\prec_{\R}$ between states.
We write $\rho' \prec_{\R} \rho$ iff there exists $\E \in \O$ such that $\E(\rho) = \rho'$.
Naturally, states may be incomparable under the given theory, meaning that there exists no free operation that converts one to the other.
We further call $\R' = (\F', \O')$ a \emph{subtheory} of $\R$ iff $\F' \subseteq \F$ and $\O' \subseteq \O$. 
The above notation will be used for general resource theories as well.

Formally, the no resource generation condition on the theories translates into two assumptions:
\begin{enumerate}[I]
    \item Free operations send free states into free states, $\E: \F \mapsto \F$, for all $\E \in \O$;
    \item Resource theory $\R$ is a completely free state preserving theory, in the sense that for any $d$-dimensional ancilla system and all free operations $\E$, $(\id_d \otimes \E ) \sigma \in \F$ whenever $\sigma \in \F$.
\end{enumerate}
The first assumption simply states that resources cannot be generated for free and is a minimal requirement for a resource theory. 
An immediate consequence is that if statistical mixing is included in $\O$, then the resource theory is convex.
Convex resource theories have attracted a lot of attention recently \nick{CITE} and include the magic theories discussed in~\cref{sec:prev}.
The second assumption implies that resources cannot be generated even when ancillas are allowed \nick{example of T state generation on Bell state by Campbell}.

Monotones are often used \nick{CITE} to address the question of state convertibility, although such approaches are usually generic.
We highlight the defining property of a monotone here, as it is useful in several sections that follow.
\begin{definition}[\textbf{Resource monotone}]\label{def:mono}
    Let $\R = (\F, \O)$ be a resource theory.
    A resource monotone $\M$ is a projection from the set of quantum states of the theory onto the real line, so that $\M$ is monotonically decreasing under free operations,
    \begin{equation}
        \M(\rho_1) \leq \M(\rho_2)\ \text{whenever}\ \rho_1 \prec_{\R} \rho_2.
    \end{equation}
\end{definition}
The monotonicity condition reflects the no resource generating property of free operations, so that monotones respect the pre-order $\prec_\R$ of the theory.
A monotone of any general resource theory is a projection of the theory onto the non-negative real numbers, collapsing the pre-order of the theory to the total order defined on the real line.
Our contribution is the introduction of a generalised notion of \emph{resource projection} which maps a general resource theory onto a subtheory which in principle still retains a partial structure (as opposed to the real line).
Applying this notion on existing magic theories highlights the hidden stochasticity that governs magic state conversions.
We show that a magic theory can be subdivided into \emph{fragments} \nick{expand}

\subsection{Previous work}\label{sec:prev}

The stabilizer theory \nick{CITE} is the first theory to introduce the idea of magic and it is discussed in sufficient detail for our purposes in~\cref{sec:so}. 
It comprises of the so-called ``stabilizer'' states ($\stab$) and operations ($\so$), while non-stabilizer (resource) states are called magic.
The stabilizer operations can be expressed in terms of a Stinespring dilation as 
\begin{equation}
    \E(\rho) = \tr_E [U(\rho \otimes \sigma_E)U^\dagger],
\end{equation} 
for an ancilla stabilizer state $\sigma_E$. 
The motivation of the theory stems from the fact that stabilizer operations are experimentally straightforward to implement and they can be used to detect and correct errors on the stabilizer states due to their construction \nick{CITE}.
The Gottesman-Knill theorem however indicates that stabilizer operations need to be supplemented with magic states in order to achieve universality, justifying the term ``magic''.

Generalisations of the stabilizer theory appear in the literature intending to include broader classes of operations \nick{CITE}.
The class of stabilizer preserving operations ($\spo$) is defined as the set of $\cptp$ maps that send stabilizer states into stabilizer states~\cite{cit:ahmadi}.
An important subclass of $\spo$ is the set of completely stabilizer preserving operations ($\cspo$), which intuitively cannot induce ``non-stabilizerness'' even when applied to only part of a quantum state, i.e. operations $\E$ such that $(\id_d \otimes \E ) \sigma \in \stab$ for all positive dimensions $d$ whenever $\sigma \in \stab$.

Even though non-stabilizerness is a necessary resource for universality, it has been proven insufficient for magic state distillation~\cite{cit:bravyi, cit:campbell}.
In fact, all states with non-negative Wigner distributions have been proven to be efficiently classically simulable in~\cite{cit:mari}, a result that serves as a generalization of the Gottesman-Knill theorem.
The Wigner distribution of a state in odd prime dimensions is discussed rigorously in~\cref{sec:wigner} and arises as the unique quasi-probability representation of quantum theory that identifies non-contextuality exactly with the states that are efficiently classically simulable~\cite{cit:howard2, cit:veitch2}.
In this framework, the stabilizer states are the only pure states represented with non-negative distributions~\cite{cit:gross3}. 
However, there exist mixed states with non-negative Wigner distributions that are not mixtures of stabilizer states~\cite{cit:gross}.
Therefore, stabilizer-preserving theories have been extended to a theory that preserves state ``Wigner positivity''~\cite{cit:wang}, formally defined in~\cref{sec:wigner} for odd prime dimensions.
Informally, it can be considered as the maximal theory of magic $\Rmax = (\Fmax, \Omax)$, where free states have non-negative Wigner distributions and free operations completely preserve this property.

\ddd{Things we must emphasize:
\begin{enumerate}
\item Perhaps a nice lead-in question: ``What happens if we view stabilizer states as thermodynamic equilibrium states and magic as a form of free energy?''
\item We can tackle more `physicsy' questions like: how much magic can be distilled via free operations with some given fixed-point structure? E.g. 
\item This allows a diagnostic on the kind of operations needed to do good distillation. I.e. what fixed point structure should they have?
\item We go beyond magic monotones and replace a monotone with a $\sigma$-fragment.
\item We can get both upper and lower bounds on magic distillation.
\end{enumerate}
}
%%%%%%%%%%%%%%%%%%%%%%%%%%%%%%%%%%%%%%%%

\section{Phase Space formalism}
\label{sec:ps}

\subsection{Stabilizer Theory}\label{sec:so}

Let $\{\ket{k}\}_{k \in \zd}$ be the standard computational basis for an arbitrary fault-tolerant scheme, defined over the finite field $\zd = \{ 0, 1, \dots,d-1 \}$, with $d$ an odd prime. 
Since the field has character $d$, addition and multiplication on the field are always considered modulo $d$.
The Hilbert space of any system associated with this scheme is $\hd \coloneqq \spanv{\ket{k}: k \in \zd}$.

The generalised Pauli matrices $X, Z$ can be defined by their respective roles as shift and phase operators,
\begin{align}
    X \ket{k} &= \ket{k + 1} \label{eq:xpauli}\\
	Z \ket{k} &= \omega^k \ket{k}, \label{eq:zpauli}
\end{align}
where $\omega \coloneqq e^{2\pi i/d}$ is the $d$-th root of unity. 

The Hilbert space $\hd$ is associated with a phase space $\pd \coloneqq \zd \times \zd$, where every point $\bmx \coloneqq (x_0, x_1)$ corresponds to a displacement operator, defined as
\begin{equation}\label{eq:ddef}
    D_{\bmx} \coloneqq \tau^{x_0 x_1} X^{x_0} Z^{x_1},\ \bmx  \in \pd.
\end{equation}
The phase factor $\tau \coloneqq -\omega^{1/2}$ ensures unitarity.
For a system with composite Hilbert space, $\hd = \hd[d_A] \otimes \hd[d_B]$, the displacement operators are defined as
\begin{equation}\label{eq:composited}
    D_{\bmx_A \oplus \bmx_B} \coloneqq D_{\bmx_A} \otimes D_{\bmx_B},
\end{equation}
where $\bmx_A \oplus \bmx_B \coloneqq (x_{A0}, x_{B0}, x_{A1}, x_{B1}) \in \pd[d_A] \times \pd[d_B]$.

The displacement operators, form a group under matrix multiplication modulo phases,
\begin{equation}\label{eq:gp}
    \gp \coloneqq \{ \tau^k D_{\bmz}: k \in \zd, \bmz \in \pd\}.
\end{equation}
The Clifford unitaries $\cd$ can then be defined as the normaliser of this group, \nick{Reformulate for copies of qudits: $\gp,\ \cd \rightarrow \gp^n,\ \cd^n$. $C-SUM$s live in $\cd^2$}
\begin{equation}
    \cd \coloneqq \{ U \in {\rm{SU}}(d): U\gp U^\dagger = \gp \}.
\end{equation}
The pure stabilizer states are then the orbit of the Clifford unitaries over a computational basis state,
\begin{equation}\label{eq:stab}
    \stab_{{\rm{pure}}} \coloneqq \{ U\ketbra{0}{0}U^\dagger: U \in \cd \}.
\end{equation}

The free states of the stabilizer theory are mixtures of pure stabilizers, 
\begin{equation}
    \stab = \conv\ \stab_{{\rm{pure}}}.
\end{equation}
The free operations of the stabilizer theory is the set of stabilizer operations $\so$ defined as any composition of:
\begin{enumerate}
    \item Preparation in computational basis;
    \item Random Clifford unitaries $\rcu$, i.e. operations $\E$ such that 
    \begin{equation}
        \E(\rho) = \sum_i p_i U_i \rho U_i^\dagger,\ U_i \in \cd;
\end{equation}
    \item Measurement in computational basis.
\end{enumerate}

\subsection{Wigner Distribution}\label{sec:wigner}

We can define the phase-point operators,
\begin{align}
	A_{\bmx} \coloneqq \frac{1}{d} \sum_{\bmz \in \pd} \omega^{\bmx \wedge \bmz} D_{\bmz},\ \bmx \in \pd. \label{eq:ax}
\end{align}
\nick{$\wedge$ has not be defined} They form an orthogonal Hermitian operator basis.
Therefore, any quantum state $\rho \in \cal{B}(\hd)$ can be expressed as a linear combination of the phase-point operators,
\begin{equation}
    \rho = \sum_{\bmz \in \pd} \W[\bmz]{\rho} A_{\bmz},
\end{equation}
where the coefficient vector $\W{\rho}$ is the Wigner distribution of state $\rho$,
\begin{equation}\label{eq:wstate}
    \W[\bmx]{\rho} \coloneqq \frac{1}{d}\tr[A_{\bmx} \rho].
\end{equation}
It is in fact a real, bounded, $d^2$-dimensional quasi-probability distribution over $\pd$ as shown in~\cref{app:wigner}. 

The Wigner distributions of different types of qutrit states are illustrated in~\cref{fig:wstate_examples}.
\begin{figure}%
    \centering
    \subfigure[][]{%
    \label{fig:maxmix}%
    \includegraphics[height=2cm]{figs/maxmixed.pdf}
    %\caption{Maximally mixed state $\frac{1}{3}\id$}%
    }\hspace{8pt}%
    \subfigure[][]{%
    \label{fig:zero}%
    \includegraphics[height=2cm]{figs/zerostate.pdf}
    %\caption{Zero state $\ketbra{0}{0}$}%
    }\\
    \subfigure[][]{%
    \label{fig:bound}%
    \includegraphics[height=2cm]{figs/boundstate.pdf}
    %\caption{Bound state}%
    }\hspace{8pt}%
    \subfigure[][]{%
    \label{fig:strange}%
    \includegraphics[height=2cm]{figs/strangestate.pdf}
    %\caption{Strange state $\ketbra{S}{S}$}%
    }
    \caption{\textbf{Qutrit Wigner distributions of varying magic.} 
    \subref{fig:maxmix} Maximally mixed state $\frac{1}{3}\id$; \subref{fig:zero} Stabilizer zero state $\ketbra{0}{0}$; \subref{fig:bound} A non-stabilizer Wigner-positive state; \subref{fig:strange} Magic strange state $\ket{{\rm{S}}} = \frac{1}{\sqrt{2}}(\ket{1} - \ket{2})$ \nick{CITE}.
    \nick{Explain what a magic state is in intro}
    }%
    \label{fig:wstate_examples}
\end{figure}

We can exploit the channel-state duality and use the normalised Choi-Jamio\l{}kowski state 
\begin{equation}\label{eq:cj}
    \frac{1}{d_A}\J(\E) \coloneqq \frac{1}{d_A}(\id \otimes \E) \sum_{i,j} \ket{ii}\bra{jj}
\end{equation}
to extend the definition of the Wigner distribution to quantum $\cptp$ operations $\E: \cal{B}(\hd[d_A]) \mapsto \cal{B}(\hd[d_B])$, 
\begin{align}\label{eq:woperation}
    \W[\bmy|\bmx]{\E} 
    &\coloneqq d_A^2 \W[\bm{\bar x} \oplus \bmy]{\frac{1}{d_A}\J(\E)} \\
    &= \frac{1}{d_B} \tr_B[A_{\bmy} \E(A_{\bmx})],
\end{align}
where $\bm{\bar x} \coloneqq (x_0, -x_1)$.

The specific form of~\cref{eq:woperation} is chosen so that Wigner distributions of operations act as transition matrices for Wigner distributions of states, $\W{\E(\rho)} = \W{\E}\W{\rho}$.
In particular, $\cptp$ operations that map between density operators of equal dimensions and have non-negative Wigner distributions correspond to stochastic matrices, as shown in~\cref{app:wigner}

The single-qudit Hadamard gate $H$ and phase gate $S$ generate the $d$-dimensional Clifford group $\cd$. \nick{CITE}
Their Wigner distributions are given by permutation matrices,
\begin{align}
    H &\coloneqq \frac{1}{\sqrt{d}}\sum_{j,k} \omega^{jk} \ketbra{j}{k}, \W[\bmy|\bmx]{H} = \delta_{y_0, -x_1}\delta_{y_1, x_0};\label{eq:H}\\
    S &\coloneqq \sum_k \tau^{k(k+1)} \ketbra{k}{k}, \W[\bmy|\bmx]{S} = \delta_{y_0, x_0}\delta_{y_1, x_0 + x_1 + 2^{-1}}.\label{eq:S}
\end{align}

%%%%%%%%%%%%%%%%%%%%%%%%%%%%%%%%%%%%%%%%

\section{Stochastic structure of magic theories}
\label{sec:struc}

\subsection{Magic fragments}\label{sec:magfrag}

Equipped with the definitions of the Wigner distribution in odd prime dimensions, we can formally recast the maximal magic theory $\Rmax$ into a stochasticity setting.
The free states correspond to proper probability distributions 
\begin{equation}
    \Fmax \coloneqq \{ \rho: \W[\bmz]{\rho} \geq 0 \text{ for all } \bmz \in \pd\}
\end{equation}

The free operations should send the set of free states $\Fmax$ into itself and completely preserve the non-negativity of the states, in the sense that $\E \in \Omax$ iff $(\id_d \otimes \E ) \sigma \in \stab$ for all odd prime dimensions $d$ whenever $\sigma \in \Fmax$.
It is shown by Wang \textit{et al.}~\cite{cit:wang} that $\Omax$ coincides with the set of operations $\E$ that correspond to stochastic Wigner distributions, 
\begin{equation}
    \Omax = \{ \E: \W[\bmy|\bmx]{\E} \geq 0 \text{ for all } \bmx, \bmy \in \pd\}.
\end{equation}

Any magic theory $\R = (\F, \O)$ is a subtheory of $\Rmax$ as explained in~\cref{sec:intro}, and as such it falls under this new stochasticity setting. 
For technical simplicity in what follows we assume that $\F$ is a closed set, and note that $\F_{\rm{ max}}$ is itself a closed set, since it is specified by a finite set of linear constraints of the form $\tr[ L\rho] \geq 0$ with operators $L \in \cal{B}(\cal{H})$ ensuring that the state is positive and Wigner-positive.


Given this context we now define the following key notion, that is central to our analysis.
\begin{definition}[\textbf{$\boldsymbol\sigma$-fragment}]\label{def:sigmafrag}
   Given a resource theory of magic $\R = (\F, \O)$, the \emph{$\sigma$--fragment of $\R$} is the resource theory $\R_\sigma = (\F, \O_\sigma)$, where the free operations are restricted to the ones that leave $\sigma$ invariant, namely
    \begin{equation}
        \O_\sigma \coloneqq \{ \E \in \O: \E(\sigma) = \sigma \}.
    \end{equation}
\end{definition}

With this basic notion defined, we now show that any resource theory of magic can be faithfully subdivided into $\sigma$--fragments, in such a way that any problem of interconversion in the parent magic theory $\R$ can be analysed across the different fragments.

\begin{theorem}
    Let $\R = (\F, \O)$ be a theory of magic.
    Every operation in $\O$ leaves at least one free state invariant,
  \begin{equation}
\O = \bigcup\limits_{\sigma \in \F} \O_\sigma.
\end{equation}
Therefore, $\rho \longrightarrow \tau$ in $\R$ if and only if $\rho \longrightarrow \tau$ in a $\sigma$--fragment of $\R$.
\end{theorem}
\begin{proof}
    Suppose $\E$ is in a $\sigma$--fragment $\O_\sigma$.
    Then it is also in $\O$, hence $\bigcup\limits_{\sigma \in \F} \O_\sigma \subseteq \O$. 
    
    Conversely, suppose $\E$ is in $\O$. 
    The free states are a closed set that is mapped one-to-one to a closed subset $\cal{S}$ of the $(d^2 - 1)$-dimensional probability simplex.
    $\cal{S}$ is convex, since any combination of free states is also free and the Wigner distribution is linear.
    Therefore, $\cal{S}$ is convex and compact as a closed convex subset of the bounded compact probability simplex.
    
    We can now view $\W{\E}$ as a stochastic, continuous mapping from $\cal{S}$ to itself, thus Brouwer's fixed point theorem \nick{CITE} implies that there exists a probability distribution $d_{\bmz}$ for some $ \bmz \in \pd$ that is a fixed point of $\W{\E}$.
    This corresponds to a free state $\sigma \coloneqq \sum_{\bmz \in \pd} d_{\bmz} A_{\bmz} \in \F$. 
    Therefore $\E \in \O_\sigma$, and so $\O = \bigcup\limits_{\sigma \in \F} \O_\sigma$. 
    
    The state interconversion result follows immediately.
\end{proof}

The zoo of all magic operation classes is summarised in ~\cref{fig:zoo}.
Completely positive-Wigner-preserving operations~\cite{cit:wang} form the operation class $\Omax$.
Therefore, $\sigma$--fragments cover this theory of magic exactly and any magic subtheory is contained within this cover.
In particular, the stabilizer operations $\so$ are contained within $\Omax$.
\begin{figure}[t]
    \centering
        \includegraphics[scale=0.47
        ]{figs/operations.pdf}
    \caption{\textbf{Decomposition of a magic theory $\R$ intro $\sigma$--fragments.} 
	Examples of magic theories ($\so$: Stabilizer operations, $\Omax$: Completely positive-Wigner-preserving operations, $\rcu$: Random Clifford Unitaries -- subclass of $\so$) involve operations denoted by the two yellow regions, with every other established magic theory  between them.
    We introduce $\sigma$--fragments $\O_\sigma$ defined for all free states $\sigma$ that cover $\Omax$. 
    Each $\O_\sigma$ is extensible to a set of stochastic maps outside the $\cptp$ operations.
    Within each $\sigma$--fragment $\bmd$-majorization can be used allowing for an intricate, but tractable approach towards the study of magic state interconversion.
    }
    \label{fig:zoo}
\end{figure}

The subdivision of magic theories into $\sigma$--fragments is powerful because the pre-order $\prec_{\R_\sigma}$ of every $\sigma$--fragment is described by well-behaved majorization tools, as we establish in~\cref{sec:major}.

\subsection{Majorization of quasi-probabilities in $\sigma$--fragments}\label{sec:major}

Majorization is a collection of powerful tools that has recently found many applications in quantum information theory \nick{CITE}.
It can describes the \nick{disorder / non-uniformity} of distributions that undergo stochastic transformations.

To formally state majorization results, we first denote by $\stoch$ the set of $(d \times d)$ stochastic matrices that preserve the probability vector $\bmd$. \nick{Should we introduce notation directly in the magic setting?}
Specifically, for any $S \in \stoch$, all matrix elements are non-negative, all rows sum to $1$ and $S\bmd = \bmd$.
\iffalse
\begin{enumerate}
    \item $S_{ij} \geq 0$ for all $i, j \in \zd$;
    \item $\sum\limits_{j=1}^n S_{ij} = 1$ for all $i \in \zd$;
    \item $S\bmd = \bmd$.
\end{enumerate}
\fi
The set $\stoch$ forms a group under matrix multiplication for all $\bmd$ with positive components.

Majorization finds an important application on quantum thermodynamics in the absence of coherence.
The use of majorization in this setting provides useful intuition for our purposes.
At any given temperature $\beta^{-1}$, the thermal state $\gamma_\beta$ is thermodynamically the most disordered state. 
Thermal operations are defined as operations that cannot extract energy from the Gibbs state, $\E(\gamma_\beta) = \gamma_\beta$.
Convertibility between states via thermal operations is equivalent to a stochasticity condition on the energy level populations of the states \nick{CITE}.
Roughly, the statement is that there exists a thermal operation $\E$ such that $\tau = \E(\rho)$ if and only if there exists a a matrix $S \in \stoch$ such that $\bm{q} = S\bm{p}$, where $\bm{q}, \bm{p}$ and $\bmd$ and the energy level population vectors of $\tau, \rho, \gamma_\beta$ respectively.

Drawing intuition from this setting, we can define majorization as follows.
\begin{definition}[\textbf{$\boldsymbol\bmd$-majorization}]\label{def:dmajor}
    Given $\bmx, \bmy, \bmd \in \reals^d$, such that the components of $\bmd$ are positive, $\bmy$ is said to $\bmd$-majorize $\bmx$, iff there exists a matrix $S \in \stoch$ such that $\bmx = S\bmy$.
\end{definition}
We denote this pre-order by $\bmx \prec_{\bmd} \bmy$.
If $\bmd = \frac{1}{d}\bm{1}$, the $d$-dimensional uniform distribution, then $\stoch$ is the set of doubly stochastic matrices and we retrieve the familiar notion of majorization in entanglement theory. \nick{CITE}

The pre-order $\prec_{\R_\sigma}$ of the $\sigma$--fragment $\R_\sigma = (\F, \O_\sigma)$ between $d$-dimensional states corresponds to the majorization pre-order $\prec_{\W{\sigma}}$ between their $d^2$-dimensional Wigner distributions.
For simplicity we shall merge the notation into $\prec_\sigma$, as there is little risk of confusion.


\begin{theorem}\label{thm:sigmamajor}
    Let $\R = (\F, \O)$ be a theory of magic. Then $\rho \longrightarrow \tau$ is possible \nick{only if}  $\W{\tau} \prec_\sigma \W{\rho}$ within a $\sigma$--fragment.
\end{theorem}
\begin{proof}
Suppose we can convert $\rho$ into $\tau$ in the magic theory. 
Thus there is some $\O_\sigma$ and some $\E \in \O_\sigma$ such that $\E(\rho) = \tau$, and $\E(\sigma)=\sigma$. 
Therefore, the Wigner distribution of this free operation satisfies $\W{\E} \in \stochw$ and $\W{\E}\W{\rho} = \W{\tau}$. 
Since $\sigma$ is \nick{full-rank} and free, its Wigner distribution is strictly positive in all components, so it directly follows from~\cref{def:dmajor} that $\W{\tau} \prec_{\sigma} \W{\rho}$.
\end{proof}
This result can be understood as an extension of the idea of a magic monotone, where we replace $\M(\tau) \leq \M(\rho)$ with $\W{\tau} \prec_\sigma \W{\rho}$. 
However the majorization constraints can be used to place upper bounds on magic state distillation in a way that allows one to incorporate the physics of the allowed operations -- this enables us to bound how much magic can be distilled via quantum operations that, for example, preserve the equilibrium state of the system, or via operations that are symmetric about the $Z$-axis of the Bloch sphere.

This approach can also provide \emph{lower bounds} on distillation, however now more structure about the specific free operations must be included. 
\nick{This is more involved and we discuss this later in the paper.
Also mention relative majorization somewhere}

\subsection{Majorization computations in $\sigma$--fragments}
A numerically efficient reformulation of $\bmd$-majorization is provided by Lorenz curves.
Let the vector $\bmu^\downarrow$ denote the vector $\bmu \in \reals^d$ with its components arranged in non-increasing order.
\begin{definition}[\textbf{Lorenz curve}]\label{def:lc}
    Let $\bmz, \bmd \in \reals^d$, where the components of $\bmd$ are positive with $D = \sum_{i=1}^d d_i$ and denote by $\tilde{\bmz} \coloneqq (z_1/d_1, \dots, z_d/d_d)^T$ the vector of ratios between the corresponding components of $\bmz$ and $\bmd$.
    
    Finally, denote by $\pi: \zd \mapsto \zd$ the permutation that sorts $\tilde{\bmz}$, $(\tilde{\bmz}^\downarrow)_i = z_{\pi(i)}$ for all $i=1,\dots,d$.
    
    Consider the piecewise linear curve obtained by joining the points $\{(0,0)\} \cup \{ (x_k, \lc{\bmz}{\bmd}(k)) \}_{k=1,\dots,d}$, where
    \begin{equation}\label{eq:lorenz}
        (x_k, \lc{\bmz}{\bmd}(k)) \coloneqq \left( \frac{1}{D}\sum_{i=1}^k d_{\pi(i)}, \sum_{i=1}^k z_{\pi(i)} \right) \in \mathbb{R}^2.
    \end{equation}
    We define the set of points on this curve, $\lc{\bmz}{\bmd}(x),\ x \in [0,1]$, as the \emph{Lorenz curve} of vector $\bmz$ with respect to $\bmd$.
\end{definition}
Components $x_k$ are rescaled by $D$ so that comparison of curves with unequal dimensions is possible.
In fact, the Lorenz curves $\lc{\bmz}{\bmd}$ and $\lc{\bmz \otimes \bmd}{\bmd \otimes \bmd}$, where $\otimes$ denotes the Kronecker product, are identical.
Furthermore, a Lorenz curve $\lc{\bmz}{\bmd}(x)$ is always concave in $x$, since it consists of $d$ line segments each with slope $(\tilde{\bmz}^\downarrow)_i$ for $i=1,\dots,d$ which is non-increasing by definition.
Finally, points on the interior of the Lorenz curve connecting line segments of different slopes are called \emph{elbows}.

A vector $\bmy$ \emph{$\bmd$--majorizes} another vector $\bmx$ if and only if the Lorenz curve $\lc{\bmy}{\bmd}$ lies above Lorenz curve $\lc{\bmx}{\bmd}$, thus reducing $\bmd$--majorization into a finite set of inequalities.
\begin{theorem}\label{thm:dmajor}
    Let $\bmx, \bmy, \bmd \in \reals^d$, such that the components of $\bmd$ are positive. 
    Then, $\bmx \prec_{\bmd} \bmy$ if and only if $\lc{\bmx}{\bmd}(x) \leq \lc{\bmy}{\bmd}(x)$ for all $x \in [0,1]$ with strict equality at $x=1$.
\end{theorem}
A restatement of the theorem including more equivalent conditions and a proof are provided in~\cref{app:major}.
An example of comparison between different Lorenz curves is illustrated in~\cref{fig:lctoy}.
\begin{figure}
    \centering
    \includegraphics[height=5cm]{figs/lctoy.pdf}
    \caption{Example of different Lorenz curves for quasi-probability vectors under $\bmd$-majorization.
    Vectors $\bmy$ and $\bmd$ are simply probability distributions.
    The curve corresponding to vector $\bmd$ is always the line segment connecting $(0,0)$ and $(1,1)$, so that any other Lorenz curve lies above it, for example $\bmx \prec_{\bmd} \bmd$.
    Curves $L_k(\bmx|\bmd)$ and $L_k(\bmy|\bmd)$ intersect, so $\bmx \nprec_{\bmd} \bmy$ as well as $\bmy \nprec_{\bmd} \bmx$.
    \nick{Recast in terms of magic - replace with~\cref{fig:test}?}
    }
    \label{fig:lctoy}
\end{figure}

It is now straightforward to construct Lorenz curves for Wigner distributions in any $\sigma$--fragment.
As we have seen in~\cref{thm:sigmamajor}, for any full-rank free state $\sigma$ we have that $\W{\sigma}$ is a strictly positive full-rank probability distribution, and so one can define a corresponding notion of $d$--majorization on \emph{quasi}--distributions.
We write $\lc{\rho}{\sigma}(x)$ for the Lorenz curve of $\W{\rho}$ with respect to $\W{\sigma}$.

We now establish important properties of Lorenz curves that are independent of the $\sigma$--fragment they are defined in.
\begin{proposition}\label{thm:lccont}
	The Lorenz curve $\lc{\rho}{\sigma}(x)$ of a state $\rho$ with respect to state $\sigma$ is uniformly continuous in $\sigma$. 
	\nick{In the sense that if $\sigma$ and $\sigma'$ are $\delta$-close with respect to a state norm, then the curves $\lc{\rho}{\sigma}(x)$ and $\lc{\rho}{\sigma'}(x)$ are $\epsilon$-close at all $x \in [0,1]$.
	I think this clears up the narrative quite a bit? - Should we keep and prove this result?} \ddd{[Is a nice result...hmm let's wait and see. It might distract from the main message.]}
\end{proposition}
The proof of this important result is in \nick{the appendix}.
The consequence of this statement is that infinitesimal changes in state $\sigma$ only result in infinitesimal shifts of the $\sigma$--fragment.
Therefore, Lorenz curves and majorization analysis in general \nick{behave nicely in the presence of / are robust under} imperfections in the experimental implementation of quantum operations.
In particular, any $\sigma_0$--fragment defined by a non full-rank state $\sigma_0$ can be approximated by a $\sigma$--fragment defined by a full-rank $\sigma$ which is arbitrarily close to $\sigma_0$, such that $\sigma = (1-\epsilon)\sigma_0 + \epsilon\frac{1}{d}\id$, for some small $\epsilon > 0$.
Such $\sigma_0$--fragments may include important stabilizer operations such that the replacement channel defined as $\E(\rho) = \sigma_0$ for all states $\rho$.
For simplicity, we restrict analysis to $\sigma$-fragments, where $\sigma$ is full-rank.

Normalisation of the Wigner distribution \nick{REF} ensures that for all quantum states $\rho$, $\lc{\rho}{\sigma}(x) \geq 0$ and $\lc{\rho}{\sigma}(1) = 1$.
As a consequence, checking whether a state conversion $\rho \longrightarrow \tau$ is possible as per~\cref{thm:dmajor}, reduces to the condition $\lc{\rho}{\sigma}(x) \leq \lc{\tau}{\sigma}(x)$ for all $x \in [0,1]$.
We stress that $0 \leq \lc{\rho}{\sigma}(x) \leq 1$ for all $x \in [0,1]$ if and only if $\rho$ is a positive Wigner state.

The area $\A_\sigma(\rho)$ between the curve $L_{\rho|\sigma}$ and the line $y=1$ is a resource monotone in the $\sigma$--fragment. 
This is clear because for any state conversion $\rho \longrightarrow \tau$ for some $\E \in \O_\sigma$, the Lorenz curve $L_{\tau|\sigma}$ is lower than $L_{\rho|\sigma}$, hence $\A_\sigma(\E(\rho)) \leq \A_\sigma(\rho)$.
The exact form of the monotone is given in~\cref{app:areamono}

\ddd{[relate to mana?]} \nick{the area depends on several positive Wigner components and cannot be written as a function of mana annoyingly - should we include the area? Have a look at~\cref{app:areamono}}

%%%%%%%%%%%%%%%%%%%%%%%%%%%%%%%%%%%%%%%%%

\section{Magic state interconversion and Lorenz curve comparison}
\label{sec:distill}

Any quantum circuit aiming at a given magic state conversion $\rho \longrightarrow \tau$ possesses certain symmetries according to~\cref{thm:frag} that allow us to study the conversion within only certain $\sigma$-fragments.
As a simple example, the dephasing channel
\begin{equation}\label{eq:dephase}
	\Delta(\rho) = \sum_{k \in \zd} \ketbra{k}{k}\rho\ketbra{k}{k}
\end{equation}
removes coherent phases in the computational basis and therefore leaves exactly all mixtures of computational basis states invariant.
A model circuit consisting only of such dephasing channels can be fully analysed in the $\sigma$--fragments for the pure computational basis states $\sigma$ \nick{as seen in a theorem in the appendix}.

Lorenz curves provide an efficient method of numerically checking if a certain state conversion is impossible within a $\sigma$--fragment by exploiting the equivalence with $d$--majorization stated in~\cref{thm:dmajor}.
Such methods are often more conclusive than magic monotones as we discuss in section~\cref{sec:scmana}.

\subsection{Majorization in magic theories}\label{sec:scmana}
We now discuss majorization features common to all fragments, before specialising to particular $\sigma$--fragments and how analysis in them proceeds.

As discussed above, we know that every magic interconversion problem can be analysed across all $\sigma$--fragments that reflect symmetries of the circuit and moreover the pre-order in each such fragment is exactly specified by $d$--majorization.

Consider a general magic state interconversion in $\O_\sigma$, 
\begin{equation}
	\rho \xrightarrow{\E \in \O_\sigma} \tau.
\end{equation}

Magic theories utilise monotone bounds of the form
\begin{equation}\label{eq:majbound}
    \M(\rho) \geq \M(\tau),
\end{equation}
to determine whether the interconversion is allowed.
Monotone bounds hold independently of $\sigma$--fragments, therefore they provide no understanding with regard to the type of operations that allow the interconversion.

New bounds can be obtained in the $\sigma$--fragment by pointwise Lorenz curve comparison of the initial and target states,
\begin{equation}\label{eq:majbound}
    \lc{\rho}{\sigma}(x) \geq \lc{\tau}{\sigma}(x),\ x\in [0,1].
\end{equation}
This inequality is parametrised by $x$ and if it is not satisfied at any point $x$, then the Lorenz curves intersect and the conversion is impossible.
We can refine the independent constraints stemming from this inequality and consider it at only the locations of the target state elbows.
\begin{theorem}\label{thm:elbows}
	Let $\rho, \tau$ be two quantum states with Lorenz curves $\lc{\rho}{\sigma}(x), \lc{\tau}{\sigma}(x)$ in the $\sigma$-fragment.
	
	Let $n'$ be the number of elbows of $\lc{\tau}{\sigma}(x)$ and $E \coloneqq \{x_1, \dots, x_n'\}$ be their locations.
	
	Then, $\lc{\rho}{\sigma}(x) \geq \lc{\tau}{\sigma}(x)$ for all $x \in [0,1]$ iff $\lc{\rho}{\sigma}(x_{i}) \geq \lc{\tau}{\sigma}(x_{i})$ for all $i =1,\dots,n'$.
\end{theorem}
A proof is provided in~\cref{app:lc_constraints}.
Any of the $t$ inequalities in the theorem provides a valid contraint leading to some upper distillation bound, and the more one cosiders, the stricter the bound is.

\nick{We can skip mana and just discuss sum-negativity}

One of the most fundamental and commonly used magic monotones is the \emph{mana} of a state \nick{CITE}, defined as
\begin{equation}
    \mana{\rho} \coloneqq \ln{(2\sn{\rho}+1)},
\end{equation}
where the \emph{sum-negativity} ($\rm{sn}$) \nick{CITE} is the sum of the negative components in $\W{\rho}$,
\begin{equation}
    \sn{\rho} \coloneqq \sum\limits_{\bmx: \W[\bmx]{\rho} < 0} \abs{\W[\bmx]{\rho}}.
\end{equation}
Mana is an additive\footnote{It satisfies the condition $\mana{\rho_1 \otimes \rho_2} = \mana{\rho_1} + \mana{\rho_2}$ which is practical is distillation settings.} magic monotone, so it provides a necessary condition for magic state interconvertibility.

Here, we show that its properties can be viewed as majorization-based, and independent of the particular $\sigma$--fragment one works in.
In fact, our current setting makes it apparent that the mana provides a weaker condition than majorization for all magic state interconversions.

We first show that the Lorenz curve maximum of state $\rho$ is independent of the $\sigma$--fragment and directly related to its sum-negativity.
\begin{lemma}\label{lem:lcmax}
	Given a quantum state $\rho$, the maximum of its Lorenz curve $\lc{\rho}{\sigma}$ is independent of the $\sigma$--fragment in which it is defined and equal to $1+\sn{\rho}$.
\end{lemma}
\begin{proof}
	For notational simplicity, denote $(\W{\rho})_i = \W[\bmx]{\rho},\ i = 1,\dots, d^2$ and similarly for $\W{\sigma}$.
	We choose the index mapping so that the vector of component ratios, 
	\begin{equation}
		\tilde{\bm{w}} \coloneqq \left(\frac{(\W{\rho})_1}{(\W{\sigma})_1}, \dots, \frac{(\W{\rho})_{d^2}}{(\W{\sigma})_{d^2}} \right)^T,
	\end{equation}
	is sorted in the sense that $\tilde{\bm{w}} = \tilde{\bm{w}}^\downarrow$.
	Note that all components of $\W{\sigma}$ are positive, so $(\tilde{\bm{w}})_i \geq 0$ if and only if $(\W{\rho})_i \geq 0$ for any $i=1,\dots,d^2$.
	
	Let $i_\star$ be the index of the smallest non-negative component of $\tilde{\bm{w}}^\downarrow$.
	Then, $(\W{\rho})_i < 0$ if and only if $i > i_\star$, so the maximum of Lorenz curve $\lc{\rho}{\sigma}(x)$ takes the value 
	\begin{equation}
		\lc{\rho}{\sigma}(x_{i_\star}) = \sum_{i=1}^{i_\star} (\W{\rho})_i,
	\end{equation}
	and is achieved at
	\begin{equation}\label{eq:maxloc}
		x_{i_\star} \coloneqq \sum_{i=1}^{i_\star} (\W{\sigma})_i.
	\end{equation}

	The location of the maximum ($x=x_{i_\star}$) varies from fragment to fragment, but its value is independent of $\sigma$,
	\begin{equation}
		\lc{\rho}{\sigma}(x_{i_\star})
		= \sum\limits_{\bmx: \W[\bmx]{\rho} \geq 0} \W[\bmx]{\rho}
		= 1 + \sn{\rho}.
	\end{equation}
	
\end{proof}

We can therefore view mana as just one feature of the Lorenz curve, namely its maximum value. 
This implies the following result.
\begin{theorem}\label{thm:bounds}
    Given a magic state conversion $\rho \longrightarrow \tau$, the majorization condition is stronger than the mana condition in every $\sigma$--fragment.
\end{theorem}
\begin{proof}
    The maximum of the Lorenz curve of a state $\rho$ is independent of the $\sigma$--fragment due to~\cref{lem:lcmax}, and can be expressed as an increasing function of mana,
    \begin{equation}
        \max_{x\in[0,1]}{\lc{\rho}{\sigma}(x)} = 1 + \sn{\rho} = \frac{1}{2} \left( 1 + e^\mana{\rho} \right).
    \end{equation}
    Therefore, the majorization bound
    \begin{equation}
    	\lc{\rho}{\sigma}(x) \geq \lc{\tau}{\sigma}(x),\ x\in[0,1]
    \end{equation}
    implies the order $\max_{x\in[0,1]}{\lc{\rho}{\sigma}(x)} \geq \max_{x\in[0,1]}{\lc{\tau}{\sigma}(x)}$, equivalent to the mana condition $\mana{\rho} \geq \mana{\tau}$.
\end{proof}

Denote an $n$-copy, $\epsilon$-noisy magic state $\rho$ in the $\sigma$--fragment by
\begin{equation}\label{eq:dist}
    \rho(\epsilon)^{\otimes n} = \left[ (1 - \epsilon) \rho + \epsilon \sigma \right]^{\otimes n},
\end{equation}
and consider the general magic state distillation (MSD) / purification process
\begin{equation}
		\rho(\epsilon)^{\otimes n} \xrightarrow{\E \in \O_\sigma} \rho(\epsilon')^{\otimes n'},
\end{equation}
where it is converted into a $n'$-copy, $\epsilon'$-noisy state, such that the number of copies has decreased ($n' < n$), but the state is less noisy ($\epsilon' < \epsilon$).

The more noisy the initial state is, the more copies needed to reach the target. 
We make this idea precise by defining the \emph{noise level threshold} as
\begin{equation}\label{eq:ethresh}
	\begin{split}
	\epsilon_{\star}(n, n', \epsilon', \sigma) \coloneqq &\sup{\{\epsilon:\ \lc{\rho(\epsilon)^{\otimes n}}{\sigma}}(x) \geq \\
	&{\lc{\rho(\epsilon')^{\otimes n'}}{\sigma}(x),\ x\in[0,1]\}}.
	\end{split}
\end{equation}
The MSD process~(\ref{eq:dist}) is impossible for all $\epsilon > \epsilon_{\star}(n, n', \epsilon', \sigma)$.

\nick{Can also define a distillation bound
\begin{equation}\label{eq:ethresh}
	n_{\star}(\epsilon, \epsilon', \sigma) \coloneqq \min{\{n: \lc{\rho_{\rm{S}}n}{\sigma} \geq \lc{\rho_{\rm{S}}(\epsilon')}{\sigma}\}}.
\end{equation}
}

In general, the Wigner components of an $n$-copy state $\rho^{\otimes n}$ can be calculated directly from $\W{\rho}$ by convolution of the distribution with itself,
\begin{equation}
	\W{\rho^{\otimes n}} = \rm{W}_{\rho}^{\otimes n},
\end{equation}
where $\otimes$ can be interpreted as the usual Kroenecker product in the last expression.
Furthermore, single-copy Lorenz curves are simply additive in noise, 
\begin{equation}
\lc{(1-\epsilon)\rho + \epsilon\sigma}{\sigma} = (1-\epsilon)\lc{\rho}{\sigma} + \epsilon\lc{\sigma}{\sigma},
\end{equation}
for any $\sigma, \rho$.
This is not true for higher number of copies.

An important example is the purificiation process~\cref{eq:dist} of the qutrit Strange state $\ketbra{\rm{S}}$,
\begin{equation}
    \rho_{\rm{S}}(\epsilon) = (1 - \epsilon) \ket{\rm{S}}\bra{\rm{S}} + \epsilon \sigma,
\end{equation}
in $\O_\sigma$, where $\ket{\rm{S}}$ is defined in~\cref{fig:strange}. 
For $\epsilon=0$, this state is a qutrit of maximal sum-negativity \nick{CITE} and therefore acts as the ideal distiallation target, analogous to a Bell state in bipartite entangelement theory.

In~\cref{fig:lcs}, we illustrate the Lorenz curves of pure and noisy $n$-copy Strange states for $n$ up to $4$ in different $\sigma$-fragments.
\begin{figure}%
    \centering
    \subfigure[][]{%
    \label{fig:lcs_maxmix}%
    \includegraphics[height=3cm]{figs/lcs_maxmix.pdf}
    %\caption{Maximally mixed state $\frac{1}{3}\id$}%
    }\hspace{1pt}%
    \subfigure[][]{%
    \label{fig:lcs_thermal}%
    \includegraphics[height=3cm]{figs/lcs_thermal.pdf}
    %\caption{Zero state $\ketbra{0}{0}$}%
    }\\
    \subfigure[][]{%
    \label{fig:lcs_zero}%
    \includegraphics[height=3cm]{figs/lcs_zero.pdf}
    %\caption{Bound state}%
    }\hspace{1pt}%
    \subfigure[][]{%
    \label{fig:lcs_one}%
    \includegraphics[height=3cm]{figs/lcs_one.pdf}
    %\caption{Strange state $\ketbra{S}{S}$}%
    }
    \caption{\textbf{Lorenz curves of Strange state copies.} Lorenz curves of $\rho_{\rm{S}}n$ for $n=1,2,3,4$.
    Solid lines represent pure Strange states ($\epsilon = 0$); dashed lines represent noisy Strange states ($\epsilon = 0.1$).
    \subref{fig:lcs_maxmix} Unital fragment; \subref{fig:lcs_thermal} Thermal fragment with $H = (0,1,2)$ and $\beta = 0.5$; \subref{fig:lcs_zero} Fragment $\O_{\ketbra{0}{0}}$ (corresponds to red curve in~\cref{fig:stab_distill}); \subref{fig:lcs_one} Fragment $\O_{\ketbra{1}{1}}$ (corresponds to blue curve in~\cref{fig:stab_distill}).
    The elbows of a curve are the non-differentiable points that form ``angles'' between consecutive line segments.
    }%
    \label{fig:lcs}
\end{figure}

In the following sections, we address the question of state interconversions in various $\sigma$--fragments, including concrete examples of the Strange state distillation process. \nick{adapt}

\section{Magic in unital and thermal fragments}\label{sec:unital}
The unital fragment encompasses the circuits which preserve the maximally mixed state ($\id/d$) and so it includes many important families of circuits.

MSD circuits in principle consist of bulk sequences of random Clifford unitaries ($\rcu$)\nick{CITE}, depicted in~\cref{fig:zoo}.
Operations in $\rcu$ can be expressed as
\begin{equation}
    \E(\rho) = \sum_i p_i U_i \rho U_i^\dagger,\ U_i \in \cd.
\end{equation}
Depending on the symmetries of such operations, a Clifford sequence may belong in other $\sigma$--fragments as well.
In this case, the majorization condition~(\ref{eq:majbound}) needs to be checked in all the $\sigma$-fragments that reflect the symmetries of the operation sequence.

In general, noisy circuits are well-described by the unital fragment.
To see this, consider incorporating noisy channels in the circuit, for example dephasing channels as in~\cref{eq:dephase} defined in different bases.
This process destroys the circuit symmetries, except for the invariance of the maximally mixed state.
Dephasing and bit-flip error channels are examples of the many error-inducing channels that respect the unital symmetry. \nick{expand}

In~\cref{fig:unital_distill}, we examine the Strange state distillation process of~\cref{eq:dist} with $\epsilon' = 0.05$ which describes all circuits that respect the unital symmetry.
Thresholds provided by Lorenz curve comparison are always much stricter than mana thresholds.
\nick{Explain asymptotic behaviour?}
\begin{figure}[h]
    \centering
    \includegraphics[scale=0.5]{figs/unital_distill.pdf}
    \caption{\textbf{Noise thresholds in unital fragment.} Mana and Lorenz curve noise thresholds for the Strange state purification process in~\cref{eq:dist} with $\epsilon' = 0.05$.
    Lorenz curve comparison consistently gives stricter bounds as proven in~\cref{thm:bounds}.
    The line $\epsilon = \frac{3}{4}$ indicates the threshold noise beyond which the Strange state no longer contains negativities. 
    \nick{Redo figure}
    }
    \label{fig:unital_distill}
\end{figure}

\iffalse
% Copies plot
\begin{figure}
    \centering
    \includegraphics[scale=0.5]{figs/unital_distill_copies.pdf}
    \caption{\textbf{Copies required for Strange state distillation.} Mana and Lorenz curve copy thresholds for the Strange state purification process in~\cref{eq:sdist} with $\epsilon_\tau = 0.05$.
    Lorenz curve comparison consistently provides higher low bounds on the number of noisy Strange state copies required.
    }
    \label{fig:unital_distill_copies}
\end{figure}
\fi

\null\newpage

\subsection{Distillation rate bounds in the unital fragment}\label{sec:lcsu}

We would now like to study the constraints coming solely from majorization in the unital fragment. 
We consider the problem of purifying $n$ copies of a noisy strange state $\rho_{\rm{S}}n$ into a smaller number of copies $n'$ of a less noisy strange state $\rho_{\rm{S}}(\epsilon')^{\otimes n'}$, with $\epsilon' \le \epsilon$ and $n' \leq n$. 
Since the state $\frac{1}{3} \id$ is free, the interconversion problem is unaffected if we tensor in copies of it. More precisely the associated Lorenz curve in invariant under tensoring in arbitrary many copies of this free state.
This turns out to simplify the Lorenz curve analysis, and so we focus on the process,
\begin{equation}\label{eq:sudist}
	\rho_{\rm{S}}^{\otimes n} \longrightarrow \rho_{\rm{S}}(n', \epsilon', n - n') \coloneqq  \rho_{\rm{S}}(\epsilon')^{\otimes n'} \otimes \left( \frac{1}{3}\id \right)^{\otimes (n-n')},
\end{equation}
where all copies $n, n', n - n'$ are even.
Since the initial state can be written in the form $\rho_{\rm{S}}(n, \epsilon, 0)$, it suffices to construct the Lorenz curve $\lc{\rho_{\rm{S}}}{\id/3}(x)$ for the generic noisy strange state $\rho_{\rm{S}}(n, \epsilon ,n')$.
This curve is defined at $9^n$ points between $0$ and $1$.
The exact expressions for the coordinates of these points can take $8$ different forms, depending on whether the noise level $\epsilon$ is greater than or less than $\frac{3}{7}$, the parity of the number of copies $n$ is even/odd and the location relative to the curve peak is on the left hand side (LHS -- including the maximum) / right hand side (RHS).
The full details are provided in~\cref{app:lcsu_technical}.

Here we focus on the case of Lorenz curves of even copies $n, n'$ and low noise levels ($\epsilon \leq 3/7$).
The part of the curve on the LHS of its peak is constructed by the positive components of the state Wigner distribution that take values $w_i$, with associated multiplicities $m_i$,
\begin{align}
	w_i &= \left( \frac{1}{6} - \frac{1}{18}\epsilon \right)^{2i}\left( -\frac{1}{3} + \frac{4}{9}\epsilon \right)^{n-2i}, \\
	m_i &= 8^{2i}\binom{n}{2i},
\end{align}
where $i=0,\dots,\frac{n}{2}$.
From this, it is readily seen that $\lc{\rho}{\id/3}(x)$ has elbows on the LHS of its peak located at points $x=x_i$, where
\begin{equation}
	x_i = \Phi_+ \left( 2i; n , \frac{8}{9} \right),
\end{equation}
with $i = 0,1, \dots , n$ and where $\Phi_+(i,n,p)$ is an even-power cumulant distribution function given by
\begin{equation}
	\Phi_+(2i; n, p) \coloneqq \sum_{\ell=0}^{i} \binom{n}{2\ell} p ^{2\ell} (1-p)^{n-2\ell}.
\end{equation}
We provide technical details on the function $\Phi_+$ in the appendices.

The value of the Lorenz curve at an elbow is given by
\begin{equation}
	\lc{\rho}{\id/3}(x_{i}) \equiv L_{i} =  \left( \frac{5}{3} - \frac{8}{9}\epsilon\ \right)^{n} \Phi_+\left(2i;n,\frac{12-4\epsilon}{15-8\epsilon}\right),
\end{equation}
and we define $(x_{-1}, L_{-1}) = (0,0)$.

\iffalse
The coordinates of any point on the Lorenz curve LHS can be expressed as convex sums of the neighbouring elbows,
\begin{align}
    x_{ijk} &= \left( 1-p_{ijk}\right) x_{i-1} + p_{ijk} x_{i} \label{eq:lcsu_xcoord}\\
    L_{ijk} &= \left( 1-p_{ijk} \right) L_{i-1} + p_{ijk} L_{i}, \label{eq:lcsu_lcoord}
\end{align}
where we use a three-index parametrisation for the coordinates, with $i$ the index of the elbow that follows, $j = 1,\dots,m_i$ and $k=1,\dots,9^{2(n-n')}$.
Therefore, for the initial state $\rho_{\rm{S}}^{\otimes 2n}$ we have
\begin{equation}
	p_{ijk} = \frac{j}{8^{2i}\binom{2n}{2i}},
\end{equation}
with $i=0,\dots,n,\ j(i) = 1,\dots,8^{2i}\binom{2n}{2i}$ and $k = 1$.
For the target state $\rho_{\rm{S}}(\epsilon')^{\otimes 2n'} \otimes \left( \frac{1}{3}\id \right)^{\otimes 2(n-n')}$,
\begin{equation}
	p'_{i'j'k'} = \frac{k' + (j'-1)9^{2(n-n')}}{9^{2(n-n')} 8^{2i'}\binom{2n'}{2i'}}
\end{equation}
with $i'=0,\dots,n',\ j'(i') = 1,\dots,8^{2i'}\binom{n'}{2i'}$ and $k' = 1,\dots,9^{2(n-n')}$.
\fi

Lorenz curve comparison provides $9^n$ necessary constraints for the process~\cref{eq:sudist}.
\cref{thm:elbows} shows that they can be reduced to $n'$ independent constraints.
However, in principle we can pick any location $x_\star \in [0,1]$ at which the Lorenz curves are defined and we then get a necessary distillation constraint, 
\begin{equation}
	\lc{\rho_{\rm{S}}(n, \epsilon, 0)}{\frac{\id}{3}}(x_\star) \geq \lc{\rho_{\rm{S}}(n', \epsilon', n-n')}{\frac{\id}{3}}(x_\star).
\end{equation}

We choose $x_\star = x_0 = \Phi_+(0, n, \frac{8}{9}) = \frac{1}{9^{n}}$, the location of the first elbow of the initial state $\rho_{\rm{S}}(n, \epsilon, 0)$.
The initial state Lorenz curve coordinate at this location is
\begin{equation}
	L_\star = L_0 = \left( \frac{5}{3} - \frac{8}{9}\epsilon\ \right)^{n} \Phi_+\left(0;n,\frac{12-4\epsilon}{15-8\epsilon}\right) = \left(\frac{3-4\epsilon}{9}\right)^n
\end{equation}

The first elbow of the target state is located at $x'_0 = \frac{1}{9^{n'}} > x_\star$, so in order to find the target state Lorenz curve coordinate $L'_\star$ at location $x_\star$, we can interpolate between the origin and the first target state elbow, 
\begin{equation}\label{eq:constraint_deriv}
	\begin{split}
	L'_\star &= \frac{x_\star}{x'_0}L'_0 = \frac{1}{9^{n-n'}}\left( \frac{5}{3} - \frac{8}{9}\epsilon'\ \right)^{n'}\Phi_+\left(0;n',\frac{12-4\epsilon'}{15-8\epsilon'}\right) \\
	&= \frac{(3-4\epsilon')^{n'}}{9^{n}}.
	\end{split}
\end{equation}
The process in~\cref{eq:sudist} is therefore possible only if $L_\star \geq L'_\star$ or equivalently,
\begin{align}
	\frac{n}{n'} &\geq \frac{\ln{(3-4\epsilon')}}{\ln{(3-4\epsilon)}}, \text{ or}\\
	\epsilon &\leq \frac{3}{4} - \frac{1}{4}(3-4\epsilon')^{\frac{n'}{n}}.
\end{align}
As an example, the process with $(n, n') = (12, 4)$ and $\epsilon' = 0.05$ yields $\epsilon \leq 0.398$.
Numerical comparison of the curves with varying $\epsilon$ suggests that we require $\epsilon \leq 0.373$ for the process to be possible.

Moreover, for the problem of distilling pure magic states where $\epsilon'=0$ we obtain an upper bound on the unital fragment distillation rate $R=n'/n$ given by
\begin{equation}
R \leq \frac{\ln (3-4 \epsilon)}{\ln 3},
\end{equation}
which holds for all finite $n$. 
In comparison, the bound from mana gives 
\begin{equation}
	R_{\rm{mana}} \leq \frac{\ln \left(\frac{5}{3} - \frac{8}{9}\epsilon \right)}{\ln \frac{5}{3}},
\end{equation}
a threshold which is looser than the one provided by majorization for all $\epsilon > 0$ as illustrated in~\cref{fig:distill_bounds}.
For example, for $\epsilon = 0.4$ we obtain $R\le 0.31 $ while the mana bound gives $R_{\rm{mana}} \le 0.53$.
\begin{figure}[h]
    \centering
    \includegraphics[scale=0.5]{figs/distill_bounds.pdf}
    \caption{\textbf{Distillation rate in unital fragment.} Distillation rates obtained by mana and majorization are plotted for $\epsilon' = 0$, up to noise levels $\epsilon_{\rm{max}} = \frac{3}{7}$.
    Majorization consistently provides stricter rates as proven in~\cref{thm:bounds}.
    }
    \label{fig:distill_bounds}
\end{figure}

\ddd{[Going forward:
\begin{enumerate}
\item Does the ``last elbow'' give the same bound? Is it as simple?
\item Numerically is this about as good as it gets from majorization?
\item How does this rate compare with existing literature?
\end{enumerate}}

\newpage
\section{Magic bounds in arbitrary stabilizer fragments}\label{sec:thermal}

We now generalise the approach taken in the previous section and consider bounds on magic distillation for an arbitrary stabilizer fragment, $\R_\sigma$ where $\sigma$ is any quantum state $\sigma \in \rm{STAB}$. In other words we consider those bounds on distillation that apply when the free operations have $\sigma$ as a fixed point.

These bounds can be interpreted in two different ways: on one hand they can be viewed as a family of upper bounds parameterized by a stabilizer state $\sigma$, on the other we can associate $\sigma$ to actual hardware limitations or to biased noise models in which it is an equilibrium state of some kind. Without loss of generality we can always write $\sigma = \frac{1}{\Z} e^{-\beta H}$ for some $\beta \ge 0$ and Hamiltonian $H$ (either effective or actual)\footnote{Technicalities arise for the case where $\sigma$ is not full rank, but this can be still described via $ \beta \rightarrow 0$ limiting process.}.

Given this context, we can now state the following result, which is dependent on the free energy $F_\beta$ of a Gibbs state $\sigma$ where $ \beta F_\beta =  \beta\tr[H \sigma] - S(\sigma)$. We also fix the noisy Strange state to be of the form 
\begin{equation}
	\rho_{\rm{S}}(\epsilon) \coloneqq (1-\epsilon)\ketbra{\rm{S}} + \epsilon \frac{1}{3}\id,
\end{equation}
but more general states can be handled in exactly the same manner.
\begin{theorem}Let $\sigma$ be a qutrit stabilizer state, given by $\sigma = \frac{1}{\Z} e^{-\beta H}$ with $H$ having eigenvalues $E_0 \le E_1 \le E_2$, and where $\beta$ is an (effective) inverse temperature for the state. Then any distillation rate $R$ in the $\sigma$-fragment is bounded as follows.
If $\beta \leq \beta_{\star}$ and $\epsilon' < \epsilon  \leq \epsilon_{\star}$,
\begin{equation}
	R \leq 1 + \frac{\ln{\left( 1 - \frac{4}{3}\epsilon \right)}}{\beta (E_0 - F_\beta)},
\end{equation}
Or if $\beta \leq \beta_{\star}$ and $\epsilon' \leq \epsilon_{\star} < \epsilon$
\begin{equation}
	R \le 1 + \frac{\ln{\left(1-\frac{1}{3}\epsilon \right)} - (E_2 - E_0)(\beta_{\star} - \beta)}{\beta (E_0 - F_\beta)},
\end{equation}
Or finally if $\beta > \beta_{\star}$
\begin{equation}
	R \leq  1+ \frac{\ln{\left(1-\frac{1}{3}\epsilon \right)}}{-\ln{2} + \beta (E_2 - F_\beta)},
\end{equation}
where
\begin{equation}
	\beta_\star \coloneqq \frac{1}{E_2 - E_0} \ln{2}
\end{equation}
is a threshold temperature and for $\beta \leq \beta_\star$,
\begin{equation}
	\epsilon_{\star}(\beta) \coloneqq 3 - \dfrac{18}{8-e^{(E_2 - E_0)\beta}}.
\end{equation}
is a threshold error rate.
\end{theorem}

Consider a qutrit system with maximally Wigner-negative magic state $\ket{{\rm{S}}} = \frac{1}{\sqrt{2}}(\ket{1} - \ket{2})$.
It can be written in density operator form as
\begin{align}
	\ketbra{\rm{S}} = -\frac{1}{3} A_{\bmo} + \sum_{\bmx \neq \bmo} \frac{1}{6} A_{\bmx},
\end{align}
where $\bmx$ ranges in the $3$-dimensional phase space $\pd[3]$.
This is the pure Strange state visualised in~\cref{fig:strange}.
We allow for isotropic (uniform) noise, so that the system state is a convex sum of the pure Strange state and the maximally mixed state,
\begin{equation}
	\rho_{\rm{S}}(\epsilon) \coloneqq (1-\epsilon)\ketbra{\rm{S}} + \epsilon \frac{1}{3}\id,
\end{equation}
where we choose to restrict the noise level, $0 \leq \epsilon \leq \frac{3}{7}$, so that the Wigner distribution of the state always contains a negative component which is greater than the remaining 8 positive components.

In particular, the Wigner distribution of the state can be written as 
\begin{equation}
	\W[\bmx]{\rho_{\rm{S}}(\epsilon)} = (1-\epsilon)\W[\bmx]{\ketbra{\rm{S}}} + \epsilon\W[\bmx]{\frac{1}{3}\id},
\end{equation}
so we get positive components
\begin{equation}
	u(\epsilon) = \frac{1}{6} -\frac{1}{18}\epsilon
\end{equation}
at the 8 phase space points $\bmx \in \pd[3] \setminus \{\bmo\}$ and a negative component
\begin{equation}
	- v(\epsilon) = - \left( \frac{1}{3} -\frac{4}{9}\epsilon \right)
\end{equation}
at the origin $\bmx = \bmo$.

We are interested in studying Strange state distillation in general thermal fragments $\O_{\gamma_\beta}$ where the thermal state is
\begin{equation}
	\gamma_\beta \coloneqq \frac{e^{-\beta H}}{\Z_\beta},
\end{equation}
with partition function $\Z_\beta = \tr \big[e^{-\beta H} \big]$, defined in terms of some Hamiltonian $H$ and temperature $\beta^{-1}$.

To this end, consider the general qutrit Hamiltonian
\begin{equation}
	H = E_0 \ketbra{0} + E_1 \ketbra{1} + E_2 \ketbra{2},
\end{equation}
where the energy eigenbasis coincides with the computational basis.
We have that energy eigenvalues $E_i \geq 0$, but impose no order between them.
We denote the highest energy eigenvalue as
\begin{equation}
	E_{\rm{max}} \coloneqq \max{\{E_0, E_1, E_2\}}.
\end{equation}

The thermal state associated with this Hamiltonian is
\begin{align}
	\gamma_\beta &= \frac{e^{-\beta H}}{\Z_\beta} = \frac{e^{-\beta \sum\limits_{k=0}^2 E_k \ketbra{k}}}{\Z_\beta} \\
	&= \frac{e^{-\beta E_0}}{\Z_\beta}\ketbra{0} + \frac{e^{-\beta E_1}}{\Z_\beta}\ketbra{1} + \frac{e^{-\beta E_2}}{\Z_\beta}\ketbra{2},
\end{align}
where $\Z_\beta = e^{-\beta E_0} + e^{-\beta E_1} + e^{-\beta E_2}$.

The Wigner distribution of state $\gamma_\beta$ can be seen as the ensemble average of the distributions of the computational basis states,
\begin{align}
	\W[\bmx]{\gamma_\beta} &= \sum\limits_{k=0}^2 \frac{e^{-\beta E_k}}{\Z_\beta}\W[\bmx]{\ketbra{k}} \\
	&= \sum\limits_{k=0}^2 \frac{e^{-\beta E_k}}{\Z_\beta} \delta_{x_0, k} = \frac{e^{-\beta E_{x_0}}}{3\Z_\beta},
\end{align}
for all $\bmx \in \pd[3]$. 
All Wigner components are strictly positive, therefore the pre-order $\prec_{\gamma_\beta}$ is well defined.

\subsubsection{Component-multiplicity pairs}

\nick{Probably will move this subsection to the earlier section where we discuss phase spaces in general}

In general, a $1$-copy $d$-dimensional state is defined exactly by its $d^2$ Wigner distribution $W$. 
The distribution $W$ can always be described by a Wigner-component vector $\bmw = (w_i)_{i=1,\dots,D}$ and a multiplicity vector $\bmm = (m_i)_{i=1,\dots,D}$, where $D \leq d^2$ and they form component-multiplicity pairs denoted by $\{(w_i, m_i)\}_{i=1,\dots,D}$.
\begin{definition}
	Consider a $d^2$-dimensional distribution $W$ and a positive integer $D$. 
	We call the set of order pairs $\{(w_i, m_i)\}_{i=1,\dots,D}$ a \emph{complete set of component-multiplicity pairs}, if $W$ contains $m_i$ components $w_i$ and $\sum_{i=0}^D m_i = d^2$.
\end{definition}
Therefore, such a set that describes a distribution $W$ contains in its description each component of $W$ exactly once.
As an example, two complete sets of pairs for the Strange state are $\{( -1/3, 1), ( 1/6, 8)\}$ and $\{(-1/3, 1), (1/6, 2), (1/6, 3), (1/6, 3)\}$.

Consider two states $\rho_A, \rho_B$ with Wigner distributions $\W{\rho_A}, \W{\rho_B}$ described respectively by complete sets of component-multiplicity pairs 
\begin{equation}
	\{(w_i, m_i)\}_{i=1,\dots,D_A} \text{ and } \{(w_j', m_j')\}_{j=0,\dots,D_B}.
\end{equation}
The multiplicative property of the Wigner distribution over a composite phase space $\pd[d_A] \times \pd[d_B]$,
\begin{equation}
	\W[\bmx_A \oplus \bmx_B]{\rho_A \otimes \rho_B} = \W[\bmx_A]{\rho_A}\W[\bmx_B]{\rho_B},
\end{equation}
implies that the distribution $\W{\rho_A \otimes \rho_B}$ is $d_A^2 d_B^2$-dimensional and contains components of the form $w_i w_j'$. 
Therefore, the set $\{(w_i w_j', m_i m_j')\}$ with $i=1,\dots,D_A$ and $j=1,\dots,D_B$ is a complete set of component-multiplicity pairs for the distribution of the composite system $\W{\rho_A \otimes \rho_B}$.
This is true because all components are of the form $w_i w_j'$ and 
\begin{equation*}
	\sum_{i=1}^{D_A}\sum_{j=1}^{D_B} m_i m_j' = \sum_{i=1}^{D_A} m_i \sum_{j=1}^{D_B} m_j' = d_A^2 d_B^2.
\end{equation*}

Given a state $\rho$ and a complete set of component-multiplicity pairs describing its Wigner distribution $\W{\rho}$, we now provide a method of computing the components (and multiplicities) of the $n$-copy distribution $\W{\rho}^{\otimes n}$.
\begin{lemma}\label{lem:ncopycomponents}
	Let $W$ be a $d$-dimensional distribution defined by a complete set of component-multiplicity pairs $\{(w_i, m_i)\}_{i=1,\dots,D}$ with $D \leq d$ and consider the distribution $W^{\otimes n}$ obtained by taking the Kronecker product $W \otimes \dots \otimes W$ between $n$ copies of $W$.
	
	Denote by $C_D^n \coloneqq \{\bmk\}$ the set of all vectors $\bmk \coloneqq (k_1, \dots, k_D)$ with non-negative integer components that sum to $n$, i.e.
	\begin{equation*}
	0 \leq k_1, \dots, k_D \leq n \text{ and } k_1 + \dots + k_D = n.
	\end{equation*}
	
	Then, $W^{\otimes n}$ admits a complete set of component-multiplicity pairs $\{(W_{\bmk}, M_{\bmk})\}_{\bmk \in C_D^n}$, where
\begin{align}
	M_{\bmk} &= \frac{n!}{k_1!\dots k_D!} \prod\limits_{i=1}^D {m_i}^{k_i}, \label{eq:M}\\
	W_{\bmk} &= \prod\limits_{i=1}^D {w_i}^{k_i}. \label{eq:W}
\end{align}
\end{lemma}
\begin{proof}
	We proceed by induction.
	
	Assume $n = 1$.
	Let $\bmk_i \coloneqq (0,\dots,0,1,0,\dots,0)$ be a vector with its $i$-th component equal to 1 and 0's elsewhere.
	The set $C_D^1$ consists of all vectors of this form, i.e. 
\begin{equation*}
	C_D^1 = \{ \bmk_i \}_{i=1,\dots,D}
\end{equation*}
	It is also true by direct calculation that
\begin{equation*}
	\left( W_{\bmk_i}, M_{\bmk_i} \right) = (w_i, m_i).
\end{equation*}
Therefore, $\{ (W_{\bmk}, M_{\bmk}) \}_{\bmk \in C_D^1}$ is a complete set of component-multiplicity pairs for $W$.

	Assume that $\{(W_{\bmk}, M_{\bmk})\}_{\bmk \in C_D^n}$ as given in~\cref{eq:M,eq:W} is a complete set of component-multiplicity pairs for the $n$-copy distribution $W^{\otimes n}$.
	By construction, the distribution $W^{\otimes (n+1)} = W^{\otimes n} \otimes W$ is multiplicative, so it admits the complete set of component multiplicity pairs
\begin{equation}
	\{(W_{\bmk} w_i, M_{\bmk} m_i)\},\ \bmk \in C_D^n \text{ and } i=1,\dots,D.
\end{equation}
	
	Consider the component sum of the distribution $W^{\otimes (n+1)}$,
\begin{align*}
	&\sum_{\bmk \in C_D^n}\sum_{i=1}^D M_{\bmk} m_i W_{\bmk} w_i = \sum_{\bmk \in C_D^n} M_{\bmk}W_{\bmk} \sum_{i=1}^D m_i w_i =\\
	&\sum_{\bmk \in C_D^n} \frac{n!}{k_1!\dots k_D!} \prod\limits_{i=1}^D {m_i}^{k_i}{w_i}^{k_i} \sum_{i=1}^D m_i w_i =\\
	&\left( \sum_{i=1}^D m_i w_i \right)^n \left( \sum_{i=1}^D m_i w_i \right) = \left( \sum_{i=1}^D m_i w_i \right)^{n+1} =\\
	&\sum_{\bmq \in C_D^{n+1}} M_{\bmq}W_{\bmq},
\end{align*}
where in the last expression, vectors $\bmq = (q_1, \dots, q_D)$ have non-negative integer components that sum to $(n+1)$ and 
\begin{align*}
	M_{\bmq} &= \frac{(n+1)!}{q_1!\dots q_D!} \prod\limits_{i=1}^D {m_i}^{q_i},\\
	W_{\bmq} &= \prod\limits_{i=1}^D {w_i}^{q_i}.
\end{align*}
We have used the multinomial theorem to proceed between lines 2-3 and lines 3-4 of the derivation.

We have achieved a regrouping of the distribution components.
Every component $W_{\bmq}$ is of the form $W_{\bmk} w_i$ with $q_i = k_i + 1$ and $q_j = k_j$ for $j\neq i$ and 
\begin{align*}
	\sum_{\bmq \in C_D^{n+1}}  \hspace{-6pt} M_{\bmq} =  \hspace{-10pt} \sum_{\bmq \in C_D^{n+1}} \frac{(n+1)!}{q_1!\dots q_D!} \prod\limits_{i=1}^D {m_i}^{q_i} = 
	\left( \sum_{i=1}^D m_i \right)^{n+1} \hspace{-10pt} = d^{n+1},
\end{align*}
which is the dimension of $W^{\otimes (n+1)}$.

Therefore, $\{ (W_{\bmq}, M_{\bmq}) \}_{\bmq \in C_D^{n+1}}$ is a complete set of component-multiplicity pairs for $W^{\otimes n}$, completing the proof.
\end{proof}

\ddd{Note the following is relevant: you might have $W_{\bmq} = W_{\bmq'}$ for two different $\bmq$, $\bmq'$. This would make a difference to our calculations for the first elbow. Check this doesn't occur. (If it does, our analysis is still correct, but it simply could be improved).}
\nick{It can occur accidentally, but it doesn't make a difference for first elbow analysis, as we ``bring the top Lorenz curve down till its first elbow touches the bottom Lorenz curve''. Basically, Lorenz curves have fewer elbows, but because the first elbow constraint boils down to a ratio comparison~\cref{eq:first_elb_bound} it doesn't change the analysis.}

\subsubsection{Distillation bounds}

We are now in a position to consider the Strange state distillation process
\begin{equation}\label{eq:msd_thermal}
		\rho_{\rm{S}}(\epsilon)^{\otimes n} \xrightarrow{\E \in \O_{\gamma_\beta}} \rho_{\rm{S}}(\epsilon')^{\otimes n'} \otimes \gamma_\beta^{\otimes (n-n')},
\end{equation}
where we are allowed to tensor in copies of the thermal state $\gamma_\beta$, since it is free in fragment $\O_{\gamma_\beta}$.

Our aim is to obtain a distillation bound for~\cref{eq:msd_thermal} which depends on variables $n, n', \epsilon, \epsilon'$ as well as $\beta$.
In the analysis that follows, we drop obvious variable dependencies for clarity.

Consider a general $\epsilon$-noisy Strange state $\rho_{\rm{S}}$ in a thermal fragment $\O_{\gamma}$.
We can construct the Gibbs-rescaled distribution 
\begin{equation}
	\widetilde{\rm{W}}_{\rho_{\rm{S}}|\gamma}(\bmx) \coloneqq \frac{\W[\bmx]{\rho_{\rm{S}}}}{\W[\bmx]{\gamma}},
\end{equation}
Notice that the Gibbs-rescaled distribution is multiplicative in general,
\begin{align}
	&\widetilde{\rm{W}}_{\rho_A \otimes \rho_B | \gamma_A \otimes \gamma_B}(\bmx_A \oplus \bmx_B) = \frac{\W[\bmx_A \oplus \bmx_B]{\rho_A \otimes \rho_B}}{\W[\bmx_A \oplus \bmx_B]{\gamma_A \otimes \gamma_B}} = \nonumber \\
	&\frac{\W[\bmx_A]{\rho_A}\W[\bmx_B]{\rho_B}}{\W[\bmx_A]{\gamma_A}\W[\bmx_B]{\gamma_B}} = \widetilde{\rm{W}}_{\rho_A | \gamma_A}(\bmx_A)\widetilde{\rm{W}}_{\rho_B  | \gamma_B}(\bmx_B).
\end{align}

We construct component and multiplicity vectors based on the distinct components of the Gibbs-rescaled distribution,
\begin{align}
	\bmm &\coloneqq (1,2,3,3) \\
	\bmw(\rho_{\rm{S}}) &\coloneqq (-v, u, u, u) \\
	\bmw(\gamma) &\coloneqq \frac{1}{3\Z} \left( e^{-\beta E_0}, e^{-\beta E_0}, e^{-\beta E_1}, e^{-\beta E_2} \right) \\
	\bmw(\rho_{\rm{S}} | \gamma) &\coloneqq 3\Z \left( -v e^{\beta E_0}, u e^{\beta E_0}, u e^{\beta E_1}, u e^{\beta E_2} \right) \label{eq:bmw_rescaled}
\end{align}

There are four distinct values of the Gibbs-rescaled distribution on the phase space, with the negative Strange component $-v$ corresponding to energy $E_0$ as illustrated in the phase space split in~\cref{fig:pd_split_thermal}.
The Wigner-component vectors $\bmw(\rho_{\rm{S}}), \bmw(\gamma)$ are then constructed so as to match the four phase space regions.
\begin{figure}[h]
    \centering
    \includegraphics[scale=0.5]{figs/pd_split_thermal.pdf}
    \caption{\textbf{Qutrit phase space regions with different Gibbs-rescaled values.}
    The Gibbs-rescaled distribution attains a unique value in each region $1,X,Y,Z$, given by~\cref{eq:bmw_rescaled} respectively.
    }
    \label{fig:pd_split_thermal}
\end{figure}

For notational convenience, we define the index vector
\begin{equation}
	\bmq \coloneqq (n-\alpha,i,j,k),
\end{equation}
where $i,j,k$ are non-negative integers and
\begin{equation*}
	\alpha \coloneqq i+j+k \leq n.
\end{equation*}
The Wigner components of the $n$-copy distributions are now readily provided by~\cref{lem:ncopycomponents}.
Firstly, the $n$-copy multiplicity is given according to~\cref{eq:M} by
\begin{equation}
	m_{ijk} \coloneqq M_{\bmq} = \frac{n!}{i!j!k!(n-\alpha)!} 2^i 3^j 3^k.
\end{equation}
The distribution values that correspond to the same index $\bmq$ are given according to~\cref{eq:W} by
\begin{align}
	w(\rho_{\rm{S}})_{ijk} &= (-v)^{n-\alpha} u^{\alpha} \\
	w(\gamma)_{ijk} &= (3\Z)^{-n} e^{-\beta (n-\alpha)E_0} e^{-\beta ( i E_0 + j E_1 + k E_2 )} \\
	w(\rho_{\rm{S}} | \gamma)_{ijk} &= (3\Z)^{n} (-v)^{n-\alpha} u^{\alpha} e^{\beta (n-\alpha)E_0} e^{\beta ( i E_0 + j E_1 + k E_2 )}
\end{align}

In order to construct the $n$-copy Lorenz curve of a Strange state in a thermal fragment we need to sort the components of the $n$-copy Gibbs-rescaled distribution, $w(\rho_{\rm{S}} | \gamma)_{ijk}$ in decreasing order.
In particular, to find the coordinates of the first elbow $(x_0, L_0)$, we need to evaluate the greatest component,
\begin{align}
	&\bmw(\rho_{\rm{S}} | \gamma)_{\rm{max}} \coloneqq \label{eq:max_slope}\\
	&(3\Z)^{n} \max\limits_{i,j,k}\Big\{ (-v)^{n-\alpha} u^{\alpha} e^{\beta (n-\alpha)E_0} e^{\beta ( i E_0 + j E_1 + k E_2 )} \Big\}, \nonumber
\end{align}
where $0 \leq i,j,k \leq n$ and $\alpha \coloneqq i+j+k \leq n$.
Notice that for $0 \leq \epsilon \leq 3/7$, we have $v \geq u$. 
We assume that $n$ is even, so we then need the sum $\alpha = i+j+k$ to be even, so that the expression is positive.

Given an even value $\alpha$ for the sum, the term $v^{n-\alpha} u^{\alpha} e^{-\beta (n-\alpha)E_0}$ is fixed, so the expression is maximised by setting the coefficient of the highest energy $E_{\rm{max}}$ equal to $\alpha$.
Hence, we have
\begin{align}
	&\bmw(\rho_{\rm{S}} | \gamma)_{\rm{max}} = \nonumber\\
	&(3\Z)^{n} \max\limits_{\substack{\alpha = 0,2, \\ \dots,n-2,n}}{\Big\{ v^{n-\alpha} u^{\alpha} e^{\beta (n-\alpha)E_0} e^{\beta \alpha E_{\rm{max}}} \Big\}} =  \nonumber\\
	&(3\Z)^{n} v^n e^{n\beta E_0}\max\limits_{\substack{\alpha = 0,2, \\ \dots,n-2,n}}{\Big\{ \left( \frac{u}{v} e^{\beta (E_{\rm{max}} - E_0)} \right)^{\alpha} \Big\}}
\end{align}
If the expression $\frac{u}{v} e^{\beta (E_{\rm{max}} - E_0)}$ is less than $1$ then the maximum occurs at $\alpha=0$, and otherwise the maximum occurs at $\alpha = n$. To determine this transition we set
\begin{equation}
	\frac{u(\epsilon)}{v(\epsilon)} e^{\beta (E_{\rm{max}} - E_0)} = 1.
\end{equation}
This can be rewritten as
\begin{equation}\label{eq:noise_transition}
	\frac{3-\epsilon}{6-8\epsilon} e^{\beta (E_{\rm{max}} - E_0)} = 1.
\end{equation}
The function $(3-\epsilon) / (6-8\epsilon)$ is increasing in $\epsilon$, achieving its minimum value $1/2$ at $\epsilon = 0$ and its maximum value $1$ at $\epsilon = 3/7$.
We want to find in which cases there exists a threshold noise level $\epsilon_\star$ at which the transition in~\cref{eq:noise_transition} occurs.

For the case $E_{\rm{max}} = E_0$, namely when the negative component overlaps the highest energy state distribution, this threshold is constant in temperature and given by $\epsilon_{\star} = 3/7$. 

There is a threshold temperature value $\beta_\star$ given by
\begin{equation}
	\beta_{\star} \coloneqq \frac{1}{E_{\rm{max}} - E_0} \ln2.
\end{equation}
Below the threshold temperature, $0 \leq \beta \leq \beta_\star$, the transition is well defined and the threshold noise level at which it occurs is given by
\begin{equation}
	\epsilon_{\star}(\beta) := 3 - \dfrac{18}{8-e^{(E_{\rm{max}} - E_0)\beta}}.
\end{equation}
This encompasses the case $E_{\rm{max}} = E_0$.

We note that the threshold noise level $\epsilon_\star$ becomes $0$ when $\beta = \beta_\star$, namely the difference between the maximum energy level and the level associated with the state negativity is equal to the energy extracted from a single particle Szilard's engine,
\begin{equation}
	E_{\rm{max}} - E_0 = kT_{\star} \ln2.
\end{equation}

For $\beta > \beta_\star$, we have
\begin{equation*}
	\frac{3-\epsilon}{6-8\epsilon} e^{\beta (E_{\rm{max}} - E_0)} > \frac{3-\epsilon}{6-8\epsilon} 2 \geq \frac{1}{2}2 = 1,
\end{equation*}
so there is no transition and the threshold noise level remains $\epsilon_\star = 0$.

The maximum value of the Gibbs-rescaled distribution finally is
\begin{equation}
\bmw(\rho_{\rm{S}} | \gamma)_{\rm{max}} =
	\begin{cases}
		(3\Z)^{n} v^n e^{n\beta E_0}, &\epsilon \leq \epsilon_{\star},\ \hspace{30pt}\rm{(C1)}	\\
		(3\Z)^{n} u^n e^{n\beta E_{\rm{max}}}, &\epsilon > \epsilon_{\star}.\ \hspace{32pt}\rm{(C2)} 
	\end{cases}
\end{equation}
Case $\rm{(C1)}$ corresponds to $(i,j,k) = (0,0,0)$, while case $\rm{(C2)}$ corresponds to the  the conditions 
\begin{equation}
	(i,j,k) = 
	\begin{cases}
	(0,n,0), &\text{if } E_{\rm{max}} = E_1, \\
	(0,0,n), &\text{if } E_{\rm{max}} = E_2.
	\end{cases}
\end{equation}
Note that if $E_{\rm{max}} = E_0$, we get $\epsilon_{\star} = 3/7$ and $\rm{(C1)}$ holds for all $\epsilon$ and $\beta$.
Assuming $E_{\rm{max}} > E_0$, we get the cases:
\begin{itemize}
	\item $\beta = 0$. Then, $\epsilon_{\star} = 3/7$ and $\rm{(C1)}$ holds for all noise levels $\epsilon$.
	This is the unital fragment.
	\item $0 < \beta \leq \beta_{\star}$. Then, $0 \leq \epsilon_{\star} < 3/7$ and $\rm{(C1)}$ holds for $\epsilon \leq \epsilon_{\star}$, while $\rm{(C2)}$ holds for $\epsilon \geq \epsilon_{\star}$.
	\item $\beta > \beta_{\star}$. Then, $ \epsilon_{\star} = 0$ and $\rm{(C2)}$ holds for all $\epsilon$.
\end{itemize}

In $\rm{(C1)}$, the multiplicity is $m_{000} = 1$ and the corresponding Wigner components are $\bmw(\rho_{\rm{S}})_{000}, \bmw(\gamma)_{000}$.
In $\rm{(C2)}$, we have $E_{\rm{max}} = E_1$ ($E_{\rm{max}} = E_2$), the multiplicity is $m_{0n0} = 3^n$ ($m_{00n} = 3^n$) and the corresponding Wigner components are $\bmw(\rho_{\rm{S}})_{0n0}, \bmw(\gamma)_{0n0}$ ($\bmw(\rho_{\rm{S}})_{00n}, \bmw(\gamma)_{00n}$).
The first elbow coordinate expression can finally be expressed as
\begin{equation}\label{eq:first_elb_coords}
	(x_0, L_0) =
	\begin{cases}
		\left( \left(\dfrac{e^{-\beta E_0}}{3\Z_\beta}\right)^n, v^n \right), &\rm{(C1)}	\vspace{10pt}\\
		\left( \left(\dfrac{e^{-\beta E_{\rm{max}}}}{\Z_\beta}\right)^n, (3u)^n \right). &\rm{(C2)} 
	\end{cases}
\end{equation}

Consider the distillation process in~\cref{eq:msd_thermal},
\begin{equation*}
		\rho_{\rm{S}}(\epsilon)^{\otimes n} \xrightarrow{\E \in \O_{\gamma_\beta}} \rho_{\rm{S}}(\epsilon')^{\otimes n'} \otimes \gamma_\beta^{\otimes (n-n')},
\end{equation*}
with $n > n'$ and $\epsilon > \epsilon'$.
A necessary distillation constraint can be defined at the location $x_\star = x_0$ of the initial state's first elbow, expressed as
\begin{equation}\label{eq:first_elbow_constraint}
	\lc{\rho_{\rm{S}}(\epsilon)^{\otimes n}}{\gamma_\beta}(x_\star) \geq \lc{\rho_{\rm{S}}(\epsilon')^{\otimes n'}}{\gamma_\beta}(x_\star).
\end{equation}

The Lorenz curves of the initial and target states may each be described by either $\rm{(C1)}$ or $\rm{(C2)}$, depending on the physical parameters $\epsilon, \epsilon', \beta$.
Specifically, we can get three scenarios:
\begin{enumerate}
	\item $\rm{(C1)} \rightarrow \rm{(C1)}$ if $E_{\rm{max}} = E_0$ OR $E_{\rm{max}} > E_0$, $\beta < \beta_{\star}$ and $\epsilon' < \epsilon  \leq \epsilon_{\star}$.
	\item $\rm{(C2)} \rightarrow \rm{(C1)}$ if $E_{\rm{max}} > E_0$, $\beta < \beta_{\star}$ and $\epsilon' \leq \epsilon_{\star} < \epsilon$.
	\item $\rm{(C2)} \rightarrow \rm{(C2)}$ if $E_{\rm{max}} > E_0$, $\beta \geq \beta_{\star}$.
\end{enumerate}
Note that $\rm{(C1)} \rightarrow \rm{(C2)}$ is impossible because it would imply $\beta < \beta_{\star}$ and $\epsilon \leq \epsilon_{\star} \leq \epsilon' < \epsilon$, a contradiction.

In all three scenarios, it is simple to check that the first elbow of the initial state is always located closer to $0$ compared to the first elbow of the target state, $x_0 \leq x_0'$:
\begin{enumerate}
	\item $\rm{(C1)} \rightarrow \rm{(C1)}$. 
	It is always true that $e^{-\beta E_0} / \Z_\beta \leq 1$, so
	\begin{align*}
		x_0 &= \left(\dfrac{e^{-\beta E_0}}{3\Z_\beta}\right)^n = \left(\dfrac{e^{-\beta E_0}}{3\Z_\beta}\right)^{n-n'} \left(\dfrac{e^{-\beta E_0}}{3\Z_\beta}\right)^{n'} \\
		&< \left(\dfrac{e^{-\beta E_0}}{3\Z_\beta}\right)^{n'} = x_0'		
	\end{align*}
	\item $\rm{(C2)} \rightarrow \rm{(C1)}$.
	It is always true that
	\begin{align*}
		&\Z_\beta^{n-n'} > (e^{-\beta E_{\rm{max}}})^{n-n'} = e^{-\beta (n-n') E_{\rm{max}}}, \text{ so} \\
		x_0 &= \left(\dfrac{e^{-\beta E_{\rm{max}}}}{\Z_\beta}\right)^n = \dfrac{e^{-\beta(n E_{\rm{max}} - n' E_0)}}{\Z_\beta^{n-n'}} \left(\dfrac{e^{-\beta E_0}}{3\Z_\beta}\right)^{n'} \\
		&< \dfrac{e^{-\beta(n E_{\rm{max}} - n' E_0)}}{e^{-\beta (n-n') E_{\rm{max}}}} \left(\dfrac{e^{-\beta E_0}}{3\Z_\beta}\right)^{n'} \\
		&= e^{-\beta n' (E_{\rm{max}} - E_0)} \left(\dfrac{e^{-\beta E_0}}{3\Z_\beta}\right)^{n'} \\
		&< \left(\dfrac{e^{-\beta E_0}}{3\Z_\beta}\right)^{n'} = x_0'.
	\end{align*}
	\item $\rm{(C2)} \rightarrow \rm{(C2)}$.
	It is always true that $e^{-\beta E_{\rm{max}}} / \Z_\beta \leq 1$, so
	\begin{align*}
		x_0 &= \left(\dfrac{e^{-\beta E_{\rm{max}}}}{\Z_\beta}\right)^n = \left(\dfrac{e^{-\beta E_{\rm{max}}}}{\Z_\beta}\right)^{n-n'} \left(\dfrac{e^{-\beta E_{\rm{max}}}}{\Z_\beta}\right)^{n'} \\
		&\leq \left(\dfrac{e^{-\beta E_{\rm{max}}}}{\Z_\beta}\right)^{n'} = x_0'	
	\end{align*}
\end{enumerate}
We can thus find the target state Lorenz curve coordinate $L'_\star$ at location $x_\star$ by interpolating between the origin and the target state's first elbow, 
\begin{equation}\label{eq:constraint_deriv_temp}
	L'_\star = \frac{x_\star}{x'_0}L'_0.
\end{equation}
We need $L_0 \geq L_\star'$. Therefore, restricting to the first elbow, the necessary distillation condition in~\cref{eq:first_elbow_constraint} simply becomes
\begin{equation}\label{eq:first_elb_bound}
	\frac{L_0}{x_0} \geq \frac{L_0'}{x_0'}.
\end{equation}
Involving more elbows gives stricter, but more convoluted necessary distillation constraints.

We now compute distillation bounds based on the first elbow constraint in the three possible scenarios. \ddd{[Does anything special happen if we have degeneracies in energy? ]}
\nick{If we have degeneracies, basically the Gibbs-rescaled distribution in~\cref{eq:bmw_rescaled} has some equal components, so Lorenz curves have fewer elbows, but because the first elbow constraint boils down to a ratio comparison~\cref{eq:first_elb_bound} it doesn't change the analysis}
We express them in terms of the free energy of thermal state $\gamma_\beta$ denoted by
\begin{equation}
	F_\beta \coloneqq \tr[\gamma_\beta H] + \beta^{-1} \tr[\gamma_\beta \ln{\gamma_\beta}] = -\beta^{-1} \ln{\Z_\beta}.
\end{equation}
Substituting coordinates from~\cref{eq:first_elb_coords} appropriately in~\cref{eq:first_elb_bound}, we get the following necessary conditions:
\begin{enumerate}
	\item $\rm{(C1)} \rightarrow \rm{(C1)}$ if $E_{\rm{max}} = E_0$ OR $E_{\rm{max}} > E_0$, $\beta < \beta_{\star}$ and $\epsilon' < \epsilon  \leq \epsilon_{\star}$.
	
\begin{align}
	\frac{n}{n'} &\geq \frac{\ln{\big( 1-\frac{4}{3}\epsilon' \big)} + \beta (E_0 - F_\beta)}{\ln{\big( 1-\frac{4}{3}\epsilon \big)} + \beta (E_0 - F_\beta)}, \text{ or} \\
	\epsilon &\leq \frac{3}{4} - \frac{3}{4} \left( 1 - \frac{4}{3}\epsilon' \right)^{\frac{n'}{n}} \left( \frac{e^{-\beta E_0}}{\Z_\beta} \right)^{1 - \frac{n'}{n}}	
\end{align}
This leads to an upper bound for the distillation rate $R=n'/n$ of pure Strange states ($\epsilon' = 0$),
\begin{equation}
	R \leq R(\epsilon, \beta) \coloneqq 1 - \frac{\ln{\left( \frac{1}{1 - \frac{4}{3}\epsilon} \right)}}{\beta (E_0 - F_\beta)},
\end{equation}
	
	\item $\rm{(C2)} \rightarrow \rm{(C1)}$ if $E_{\rm{max}} > E_0$, $\beta < \beta_{\star}$ and $\epsilon' \leq \epsilon_{\star} < \epsilon$.
	
\begin{align}
	\frac{n}{n'} &\geq \frac{\ln{\big( 1-\frac{4}{3}\epsilon' \big)} + \beta (E_0 - F_\beta)}{\ln{\big( \frac{1}{2}-\frac{1}{6}\epsilon \big)} + \beta (E_{\rm{max}} - F_\beta)}, \text{ or} \\
	\epsilon &\leq 3 - 6 \left( 1 - \frac{4}{3}\epsilon' \right)^{\frac{n'}{n}} \frac{e^{-\beta E_{\rm{max}}} e^{\frac{n'}{n}\beta E_0}}{\Z_\beta^{1-\frac{n'}{n}}}	
\end{align}
This leads to an upper bound for the distillation rate $R=n'/n$ of pure Strange states ($\epsilon' = 0$),
\begin{equation}
	R \leq R(\beta) \coloneqq 1 - \frac{\ln{\left(\frac{1}{1-\frac{1}{3}\epsilon} \right)} + (E_{\rm{max}} - E_0)(\beta_{\star} - \beta)}{\beta (E_0 - F_\beta)},
\end{equation}

	\item $\rm{(C2)} \rightarrow \rm{(C2)}$ if $E_{\rm{max}} > E_0$, $\beta \geq \beta_{\star}$.
	
\begin{align}
	\frac{n}{n'} &\geq \frac{\ln{\big( \frac{1}{2}-\frac{1}{6}\epsilon' \big)} + \beta (E_{\rm{max}} - F_\beta)}{\ln{\big( \frac{1}{2}-\frac{1}{6}\epsilon \big)} + \beta (E_{\rm{max}} - F_\beta)}, \text{ or} \\
	\epsilon &\leq 3 - 6 \left( \frac{1}{2} - \frac{1}{6}\epsilon' \right)^{\frac{n'}{n}} \left( \frac{e^{-\beta E_{\rm{max}}}}{\Z_\beta} \right)^{1 - \frac{n'}{n}}	
\end{align}
This leads to an upper bound for the distillation rate $R=n'/n$ of pure Strange states ($\epsilon' = 0$),
\begin{equation}
	R \leq R(\beta) \coloneqq 1 - \frac{\ln{\left(\frac{1}{1-\frac{1}{3}\epsilon} \right)}}{-\ln{2} + \beta (E_{\rm{max}} - F_\beta)},
\end{equation}
\end{enumerate}

\begin{figure}[h]
    \centering
    \includegraphics[scale=0.5]{figs/noise_threshold.pdf}
    \caption{\textbf{Noise threshold $\epsilon_\star$ and rate $R(\beta, \epsilon_\star)$ versus $\beta$.}
    At $\beta = 0$ and threshold noise level $\epsilon = \epsilon_\star = 3/7$, the rate is $R(0, \epsilon_\star) = \frac{\ln{(9/7)}}{\ln{3}} \approx 0.23$.
    }
    \label{fig:noise_threshold}
\end{figure}

\begin{figure}[h]
    \centering
    \includegraphics[scale=0.5]{figs/rate_contour.png}
    \caption{\textbf{Rate $R(\beta, \epsilon)$ versus $\beta$ and $\epsilon$.}
    The vertical dashed line indicates the temperature threshold $\beta_\star$ and the diagonal dashed curve indicates the noise threshold $\epsilon_\star$ at every $\beta \leq \beta_\star$.
    }
    \label{fig:rate_contour}
\end{figure}

Consider pure Strange state distillation at high temperatures, $\beta \leq \beta_{\star}$.
Depending on the noise level of the initial state, we have either a $\rm{(C1)} \rightarrow \rm{(C1)}$ or a $\rm{(C2)} \rightarrow \rm{(C1)}$ scenario.

We compare $R(\beta)$ for different temperatures with the mana bound $R_{\rm{mana}}$ in~\cref{fig:distill_bounds_temp}.
At every $\beta > 0$, the rates indicate a transition at $\epsilon_{\star}$.
As $\beta$ increases, the rate becomes stricter.
At the high temperature limit, $\beta \rightarrow 0$, we get $\beta F_\beta \rightarrow -\ln{3}$, and we retrieve the unital fragment bound,
\begin{equation}
	R(\beta) \xrightarrow{\beta \rightarrow 0} \frac{\ln{(3-4\epsilon)}}{\ln{3}}.
\end{equation}

\begin{figure}[h]
    \centering
    \includegraphics[scale=0.5]{figs/distill_bounds_temp.pdf}
    \caption{\textbf{Distillation rate with low noise in thermal fragments.} Distillation rates obtained by majorization $R(\beta)$ for low noise levels $\epsilon \leq \epsilon_{\star}$ are plotted for different temperatures $\beta^{-1}$ such that $0 \leq \beta \leq \beta_{\star}$.
    The mana distillation rate $R_{\rm{mana}}$ is given for comparison.
    The Hamiltonian spectrum is simple harmonic, $(E_0, E_1, E_2) = (0,1,2)$.
    As $\beta$ increases, the range of noise levels for which $R(\beta)$ is valid decreases and the distillation bound becomes stricter.
    }
    \label{fig:distill_bounds_temp}
\end{figure}

Consider pure Strange state distillation at low temperatures, $\beta \rightarrow \infty$.
Independently of the noise level of the initial state, we have a $\rm{(C2)} \rightarrow \rm{(C2)}$ scenario.

We compare $R(\beta)$ for different temperatures with the mana bound $R_{\rm{mana}}$ in~\cref{fig:distill_bounds_lowtemp}.
As $\beta$ decreases, the rate becomes stricter.
At the zero temperature limit, $F_\beta$ tends to zero, so $R(\beta) \xrightarrow{\beta \rightarrow \infty} 1$.

\begin{figure}[h]
    \centering
    \includegraphics[scale=0.5]{figs/distill_bounds_lowtemp.pdf}
    \caption{\textbf{Distillation rate for low temperatures in thermal fragments.} Distillation rates obtained by majorization $R(\beta)$ in the temperature regime $\beta \geq \beta_{\star}$ are plotted for different temperatures $\beta^{-1}$.
    The mana distillation rate $R_{\rm{mana}}$ is given for comparison.
    The Hamiltonian spectrum is simple harmonic, $(E_0, E_1, E_2) = (0,1,2)$.
    \nick{Not-so-good bound for low temperatures, other elbows perform a lot better numerically.}
    }
    \label{fig:distill_bounds_lowtemp}
\end{figure}



%%%%%%%%%%%%%%%%%%%%%%%%%%%%%%%%%%%%%%%%
\null\newpage
\null\newpage

\subsection{Random plots}

We generalise the analysis of unital fragments into any circuit with thermalisation noise parametrised by the temperature $\beta^{-1}$.
In~\cref{fig:thermal_distill}, we examine the majorization bound for the same purification process of~\cref{eq:sdist} with $\epsilon' = 0.05$.
The curves plotted suggest that there exists a fragment $\O_{\gamma_{\beta_{\rm{max}}}}$ which allows for a highest noise threshold at any given number of copies of the initial state.
Adding more copies results in a lower optimal temperature $\beta^{-1}$.

\begin{figure}[h]
    \centering
    \includegraphics[scale=0.5]{figs/thermal_distill.pdf}
    \caption{\textbf{Threshold dependence on temperature in thermal fragments.} Lorenz curve ratios for the Strange state purifying process in~\cref{eq:sdist} with $\epsilon' = 0.05$.
    The peaks of each curve indicate the optimal temperature $\beta_{\rm{max}}^{-1}$ that allows for the highest noise threshold at every given number of initial state copies $k$.
    The line $\epsilon = \frac{3}{4}$ indicates the threshold noise beyond which the Strange state no longer contains negativities.
    }
    \label{fig:thermal_distill}
\end{figure}

There exist $d(d+1)$ pure stabilizer states for a $d$-dimensional system, all of which correspond to a uniform distribion over an affine line in the phase space \nick{CITE}.
As an example, the distribution of the qutrit zero state $\ketbra{0}$ is illustrated in~\cref{fig:zero}.
Any pure stabilizer state $\sigma$ is a rank--$1$ state and contains $d^2 - d$ zeros in its Wigner distribution, so we approximate it by the full--rank state $\sigma' = (1-\epsilon)\sigma + \epsilon \frac{1}{d}\id$ for some infinitesimal $\epsilon > 0$.

We explore the distillation process of~\cref{eq:sdist} in these modified fragments $\O_\sigma'$ in~\cref{fig:stab_distill}.
We repeat the plot for distillation of Norrell states in~\cref{fig:stab_distill_norrell}.


\begin{figure}[h]
    \centering
    \includegraphics[scale=0.5]{figs/stab_distill.pdf}
    \caption{\textbf{Strange state distillation thresholds in stabilizer fragments.} Lorenz curve thresholds for the Strange state purifying process in~\cref{eq:sdist} with $\epsilon' = 0.05$.
    All $12$ stabilizer fragments produce two distinct trends: ($k=0$ - red curve) the stabilizer states that have non-zero value at the origin of the phase space lie low and ($k=1,2$ - blue curve) the stabilizer states that have a zero value at the origin of the phase space lie high.
    This is specific to the Strange state.
    For high enough copies, these Lorenz curves surpass the unital noise threshold $\epsilon_{\star} = \frac{3}{4}$.
    }
    \label{fig:stab_distill}
\end{figure}

\begin{figure}[h]
    \centering
    \includegraphics[scale=0.5]{figs/stab_distill_norrell.pdf}
    \caption{\textbf{Norrell state distillation thresholds in stabilizer fragments.} Lorenz curve thresholds for the Strange state purifying process in~\cref{eq:sdist} with $\epsilon' = 0.05$.
    For high enough copies, certain Lorenz curves surpass the unital noise threshold $\epsilon_{\star} = \frac{3}{4}$.
    }
    \label{fig:stab_distill_norrell}
\end{figure}

%%%%%%%%%%%%%%%%%%%%%%%%%%%%%%%%%%%%%%%%
\null\newpage
\null\newpage
\section{Lower bounds in $\sigma$-fragments}
\ddd{[Am throwing down some rough material, to be polished later.]}

In order to obtain lower bounds one must now make precise what the free operations actually are in the  theory, beyond the condition of preserving Wigner-positivity.

We first consider the unital fragment, and consider the transformations possible using Clifford unitaries, and convex mixtures of Clifford unitaries. We denote the Clifford group as $\C$, and given some quantum state $\rho$ the accessible states in the unital fragment are given by $\E(\rho) = \sum_{g \in \C} p(g) U(g) \rho U(g)^\dagger$.

\subsection{Symplectic majorization of magic state Wigner distributions.}
We now exploit the the group structure, which leads to a more general notation majorization that has been extensively studied in the classical context, but to our knowledge has not yet been used in quantum information theory.

\begin{definition} Given a group $G$ that acts on a vector space $V$, we say that $\bmx$ \emph{$G$-majorises} $\bmy$ precisely when $\y \in H(\x)$, where $H(\x)$ is the convex hull of $\{g \x : g \in G\}$. We denote this as $\y \prec_G \x$.
\end{definition}

Now any quantum state $\rho$ has a Wigner distribution $W_\rho(\x)$. Given a Clifford unitary $\U \in \C$ we have that its representation in the Heisenberg-Weyl frame is given by
\begin{equation}
W_{\U(\rho)} (\x) = [\mbox{\ddd{fill in}}].
\end{equation}

This group action corresponds to the action of the affine symplectic group on the discrete phase space $\P_d$. To proceed we can study the discrete translational action, and the symplectic action separately.

\subsection{Cyclic majorization conditions.}
We can consider for $G$, the cyclic group of order $N$, which is described by $G= \< g | g^N = e\>$. in Terms of the Wigner distributions, this arises for the discrete lattice translations arising from displacement operators.

A set of necessary and sufficient conditions for cyclic majorization has been obtained and is given as follows. Suppose $\x, \y$ are two real vectors in $\mathbb{R}^n$. Let $\Delta$ be the elementary shift operator, defined by
\begin{equation}
\Delta \x = \Delta (x_1, x_2, \dots, x_n) := (x_n, x_1, x_2, \dots, x_{n-1}),
\end{equation}
and from this it is clear that $\Delta$ generates a representation of the abelian group $(\mathbb{Z}_n,+)$.
We now say that $\x$ \emph{cyclically majorizes} $\y$, written $\x \succ_C \y$, if and only if 
\begin{equation}
\y = \sum_{k=0}^{n-1} p_k \Delta^k \x,
\end{equation}
for some probability distribution $p=(p_k)$. This means that $\y$ lies in the convex hull of the orbit of $\x$ under cyclic shifts. Thus, we may write this condition as $\y = L(\p) \x$, where
\begin{equation}
L(\p) :=  \sum_{k=0}^{n-1} p_k \Delta^k,
\end{equation}
is a linear operator, as a function of an unknown distribution $\p$. We now have the following key identity for linear operators of this form:
\begin{equation}
L(\p)\x = QL(\x) \p,
\end{equation}
where $Q$ is a permutation matrix sending $e_0:=(1,0,0,\dots ,0)$ to itself, and otherwise sending $e_k$ to $e_{n-1-k}$, where $e_k=  (0,0,\dots, 0,1,0,\dots, 0)$ is the $k$'th basis vector. We thus have
\begin{align}
\x \succ_C \y &\Leftrightarrow \y = L(\p) \x \mbox{ for some dist. } \p. \nonumber \\
&\Leftrightarrow \y = QL(\x) \p \mbox{ for some dist. } \p. \nonumber \\
&\Leftrightarrow [QL(\x)]^{-1}\y = \p \mbox{ for some dist. } \p. \nonumber \\
&\Leftrightarrow [QL(\x)]^{-1}\y = \p  \ge \mathbf{0} \mbox{ and normalized}.
\end{align}
This implies that to check whether $\x \succ_C \y$ it suffices to compute the components of the left-hand side and ensure non-negativity (I suspect it is normalised if $\x$ and $\y$ are normalised).

Note though that by using Fourier analysis we can simplify this in terms of computational demands to checking if the vector
\begin{equation}
\p = C(\x') Q\y,
\end{equation}
has non-negative components, where
\begin{equation}
\x' := n\F_n [\F_n\x]^{-1},
\end{equation}
with $\F_n$ being the discrete Fourier transform, and the bracket term $[\F_n \x]^{-1}$ denotes the vector obtained by inverting the components of $\F \x$ individually. So the recipe for cyclic majorization is:
\begin{enumerate}
\item Given inputs $\x$ and $\y$.
\item Compute the vector $\x'$ via two Fourier transforms and $n$ inversions.
\item Compute the circulant matrix $C(\x')$.
\item Compute $\C(\x')Q \y$ and check if all components are non-negative.
\end{enumerate}•


\subsection{Symplectic majorization for qutrit magic}
The qutrit system provides a good illustration of the techniques. In this case the symplectic group $SL(2,\mathbb{Z}_3)$ is isomorphic (up to $\pm \I$) the symmetry group of the tetrahedron, which in turn is isomorphic to $S_4$, the permutation group on 4 symbols. Therefore we expect in this case that symplectic majorization on $\P_3$ corresponds to a restricted form of majorization.


\subsection{Fundamental Regions of a group $G$}
\ddd{[This is theory for the appendices, and also to help flesh out the theory of $G$ majorization. It looks a bit technical, but the core idea is simple enough once you get it.]}
Given a group $G$ that acts on a vector space $V$, we now have the following concept.
\begin{definition} A \emph{fundamental region} $F$ of $G$ in $V$ is any open set $F$ such that $F \cap gF = \varnothing$ for $g \ne e$ and moreover 
\begin{equation}
V = \bigcup_{g} \overline{g F}.
\end{equation}
\end{definition}
Here, the overline denotes the closure of a set. Loosely speaking, we can view $F$ as obtained by quotienting the vector space $V$ via the group action. We then have the following key theorem, which is proved in [CITE].
\begin{theorem} If $G$ has a fundamental region $F$ that is unique, modulo actions of the group $F\rightarrow gF$, then $\bar{F}$ is a closed, convex cone and for any $\x, \y \in \bar{F}$ we have
\begin{equation}
\y \prec_G \x \Leftrightarrow \mathbf{a}\cdot \y \le \mathbf{a} \cdot \x, \mbox{ for all } \mathbf{a} \in \bar{F}.
\end{equation}
\end{theorem}
If the cone $\bar{F}$ is finitely generated, namely 
\begin{equation}
\bar{F} = \mbox{cone}( \mathbf{c}_1, \dots, \mathbf{c}_N),
\end{equation}
for some finite set of vectors, then the majorization condition reduces to checking a \emph{finite} set of inequalities, namely checking $\mathbf{c}_k \cdot \y \le \mathbf{c}_k \cdot \x$ for $k=1, \dots , N$.

Note now that any group action $G$ on a vector space $V$ always has a fundamental region. Consider any $\x \in V$ such that $g \x \ne \x$ unless $g =e$. For this we define
\begin{equation}
K = \{ \mathbf{a} \in V : \sup_g \left [ (g\mathbf{a})\cdot \x \right ] = \mathbf{a} \cdot \x \},
\end{equation}
then $F = K_{\mbox{\tiny int}}$ is a fundamental region of $G$ in $V$, where $ K_{\mbox{\tiny int}}$ is the interior of the set $K$. In simple terms, this fundamental region corresponds to the set of vectors that are `close' to $\x$, in the sense that any non-trivial group action $\mathbf{a} \rightarrow g \mathbf{a}$ on them moves them away from $\x$ with respect the inner product. See [CITE] for a proof of this statement.

\subsection{Computing the fundamental region $F$ for the qutrit}
Firstly, note that we actually have that $SL(2, \mathbb{Z}_3)$ is represented by $U(g)$ on $\P_3$, and so we should use that notation for precision.

Since the symplectic group for $d=3$ is a reflection group, it turns out that this guarantees that an essentially unique fundamental region exists, and so we can reduce to a \emph{finite} set of majorization conditions (I think 5 in this case). Note that the $K$ defined above is always a closed, convex cone. This means it should be `easy' to determine the fundamental region. Reflection groups are Coexter groups, and I believe this set $K$ is essentially a Weyl chamber in that language. Anyhow, let's not get distracted by abstraction.


The concrete recipe going forward:
\begin{enumerate}
\item Pick your favorite $\x \in P_3$ that is \emph{not} stabilized by $g\ne e$. I.e. a vector that moves under all non-trivial group actions.
\item For each $g_k$ look for the extremal cases of $\y$ such that have constant inner product with $\x$. This corresponds to 
\begin{equation}
\left [ (U(g_k) -\I) \y \right ] \cdot \x = 0 \mbox{ for }k=1, \dots ,24.
\end{equation}
Or equivalently,
\begin{equation}
\y \cdot A^T\x = 0,
\end{equation}
where $A := U(g_k) -\I$.
\item Write down the matrix equation for each $g_k$.
\item Each of these conditions defines a hyperplane $H_k$ in $\P_3$ corresponding to the boundary of $K$.
\item We possibly don't need to range over all 24 group elements....but am not sure on this.
\item From this, we should be able to extract a generating set of vectors.
\end{enumerate}

\subsection{Toy example to see how the algorithm works}
Okay, let's see how it works in a simple case. Let's consider the case of $V= \mathbb{R}^2$ and $G=S_2=\<g|g^2=e\>$ with the group action $g.(x,y) = (y,x)$ that swaps the components of the vector. We should get standard majorization out of this $G$--majorization.

First we find an $\x$ that transforms non-trivially under non-trivial group actions. The vector $\x=(1,0)$ does this. To construct a fundamental region we now look at the equation 
\begin{equation}
[g.(x,y) - (x,y)] \cdot (1,0) = 0
\end{equation}
and solve for $x,y$. This becomes,
\begin{equation}
(y-x, x-y)\cdot(1,0)=0,
\end{equation}
which implies the hyperplane (line!) $y=x$ is the \emph{boundary} of the fundamental region. Since we decided to start with $(1,0)$ we can take the region to the right of this line, and so:
\begin{equation}
F = \{(x,y) : x>y \}
\end{equation}
where we note that we use the strict inequality to get the interior. The region $\bar{F}$ is a half-space, but this is actually a cone, and can be written as
\begin{equation}
F= \mbox{cone}( (1,1),(-1,-1), (5,0)).
\end{equation}
The first two vectors give the boundary, and we just need one other vector inside $F$ to generate it fully, chosen arbitrarily to be $(5,0)$. This last one is needed since $(1,1)$ and $(-1,-1)$ are linearly dependent! Therefore
\begin{align}
\mathbf{c}_1 &= (1,1) \nonumber \\
\mathbf{c}_2 &= (-1,-1) \nonumber \\
\mathbf{c}_3 &= (5,0).
\end{align}
The $S_2$--majorization ordering is then given by
\begin{equation}
\y \prec \x \Leftrightarrow \mathbf{c}_k \cdot \y \le  \mathbf{c}_k \cdot \x,
\end{equation}
for $k=1,2,3$ whenever $\x,\y \in F$. Note that is gives the ordering for all vectors in the space, since any general vector can be transformed into $F$ via a group action, since $F$ is a fundamental region. Indeed mapping an $\x$ into $F$ is simply $\x \rightarrow \x^\downarrow$, the sorting maneuver!
The generating vectors $(1,1)$ and $(-1,-1)$ imply that both 
\begin{equation}
y_1 + y_2 \le x_1 +x_2,
\end{equation}
and
\begin{equation}
y_1 + y_2 \ge x_1 +x_2,
\end{equation}
which becomes the majorization condition
\begin{equation}
y_1+y_2 = x_1 + x_2.
\end{equation}
Easy! The final condition from $\mathbf{c}_3=(1,0)$ is then the condition that
\begin{equation}
5y_1 \le 5x_1 \Rightarrow y_1 \le x_1,
\end{equation}
which is the final majorization condition. Thus we have shown how the familiar majorization structure corresponds to the theory of $G$--majorization, and the cone ordering structure that comes from fundamental regions.

\subsection{Another toy example -- non-standard majorization this time}
Okay, let's look at another option for $V=\mathbb{R}^2$. Now consider the group $G=\mathbb{Z}_2 \times \mathbb{Z}_2$ that can be represented as $G=\<g_1,g_2 | g^1=e, g_2 = e, g_1g_2 = g_2g_1\>$, and acts on vectors as $g_1.(x,y) = (-x,y), g_2.(x,y)=(x,-y)$. In other words the group just flips vectors about X/Y axes.

A vector that moves under the non-trivial group actions is $(1,1)$, so we use this one and solve for $[g.(x,y) - (x,y)] \cdot (1,1) = 0$. We have for $g = g_1$ the equation
\begin{equation}
[(-x,y) - (x,y)]\cdot (1,1) = 0,
\end{equation}
which implies the line $x=0$.

For $g=g_2$ we have
\begin{equation}
[(x,-y) - (x,y)]\cdot (1,1) = 0,
\end{equation}
which implies the line $y=0$.

For $g=g_1g_2$ we have
\begin{equation}
[(-x,-y) - (x,y)]\cdot (1,1) = 0,
\end{equation}
which implies the line $x+y=0$. The open region bounded by these three conditions is the set
\begin{equation}
F := \{ (x,y) : x>0, y>0\}.
\end{equation}
So the positive quadrant of $\mathbb{R}^2$ is a fundamental region for $G$, which makes sense since via sign flips we can map this onto the whole plane, once we close the set. This is again a convex cone and $\bar{F} = \mbox{cone}((1,0), (0,1))$. Thus given any two vectors $\x,\y$ we first act with $G$ to flip their signs so that their components are all positive, $\x \rightarrow \x^\star\ge \mathbf{0}$, then $\y \prec_G \x$ if and only if $y_1 \le x_1$ and $y_2 \le x_2$. This implies that 
\begin{equation}
\y \prec_G \x \Leftrightarrow \y^\star \le \x^\star. 
\end{equation}
So the $G=\mathbb{Z}_2 \times \mathbb{Z}_2$ majorization gives the component-wise ordering on vectors once you sort them. It is clear that this extends to any $G=\mathbb{Z}_2^{\times n}$.
\subsection{A more complex example $S_3$ on $\mathbb{R}^3$.}
So the symmetric group $S_n$ always gives majorization. But the symmetric group has $|S_n| = n!$ elements to it, which is super-exponentially big in $n$. We clearly don't have to check $n!$ equations, since the majorization conditions involve just $n+1$ inequalities, and so it must be that the basic generating conditions of the group suffice to determine the fundamental region of $G$.

Let's look at $V = \mathbb{R}^3$ and $G=S_3$ acting on $\x = (x_i)$ as $g.\x = (x_{g^{-1}}(i))$. A vector that transforms non-trivially under all non-trivial actions is $(2,1,0)$. Let's look at the transpositions, and the boundary planes they give first.

For $g= (1\,\,2)$ we have the equation
\begin{equation}
[(y,x,z) - (x,y,z)] \cdot (2,1,0) = 0,
\end{equation}
which implies the plane $2(y-x)+ x-y = 0$, namely,
\begin{equation}
H_1 = \{(x,y,z) : y=x\}.
\end{equation}
The other choices of transpositions work the same and give the planes
\begin{align}
H_2 &= \{(x,y,z) : y=z\} \\
H_3 &= \{(x,y,z) : x=z\}.
\end{align}
Note that their intersection gives the line $\{(x,y,z) : x=y=z\}$, which is the line of uniform vectors at the bottom of the majorization pre-order.

What information do the $3$-cycles give? Okay, let's now look at $g=( 1 \,\, 2 \,\, 3)$. The core equation for this becomes
\begin{equation}
[(z,x,y) - (x,y,z)] \cdot (2,1,0) = 0,
\end{equation}
and so gives
\begin{align}
2(z-x) + (x-y) &= 0 \nonumber \\
\Rightarrow x+y &=2z. 
\end{align}
Namely the plane
\begin{equation}
H_4 = \{(x,y,z) :  x+y =2z\}.
\end{equation}The remaining $3$--cycle $(3\,\,2\,\,1)$ gives
\begin{align}
H_5 &= \{(x,y,z) : y+z = 2x\}.
\end{align}
Note that $|S_3| = 6$, but the identity element $g=e$ gives no constraint, and so there are exactly $5$ bounding planes. Note also that the line $\{ (t,t,t)\} $ is again the intersection of all the planes.

These 5 planes bound a cone containing $(2,1,0)$, namely the fundamental region $F$. What's a good algorithm to obtain a generating set of vectors for this cone?

Firstly, since the line $\{(t,t,t)\}$ being the intersection of all the planes, is definitely on the boundary of $F$, this means we must have $\mathbf{c}_1 = (1,1,1)$ and $\mathbf{c}_2 = (-1,-1,-1)$ in our generating set. Perhaps a good algorithm is to compute intersections, obtain generating vectors for these intersections of the hyperplanes, $H_{ij}$ and then let $K_i$ and $K_j$ be the half-spaces in which $(2,1,0)$ resides. Now the intersection of two cones, is itself a cone so we can just keep intersecting the half-space/cones. Thus, we compute
\begin{align}
H_{12} &:= H_1 \cap H_2 \\
&=\{ (t,t,t) : t\in \mathbb{R}\}=:L.
\end{align}
Similarly, $H_{13} = H_{23}= L$. What about the other intersections? Well $H_{14} = H_{15}=L$. In fact all intersections are the same! 
\begin{equation}
H_{ij} = H_i \cap H_j = L \mbox{ for all } i,j.
\end{equation}
The fundamental region is given by the intersection of $K_1$ and $K_2$, or any other two half-spaces. This is generated by $\mathbf{c}_1$ and $\mathbf{c}_2$ above, and any two independent vectors in the intersection of these half-spaces. We can choose 
\begin{align}
\mathbf{c}_3 &= (1,1,0) \\
\mathbf{c}_4 &= (1,0,0),
\end{align}
note that $\mathbf{c}_3 \in H_1$ and $\mathbf{c}_4 \in H_2$, while $\mathbf{c}_3 \in K_2$ and $\mathbf{c}_4 \in K_1$ and so this works correctly. We thus have
\begin{equation}
\bar{F} = \mbox{cone}(\mathbf{c}_1 ,\mathbf{c}_2 ,\mathbf{c}_3 ,\mathbf{c}_4 ).
\end{equation}
The first two inequalities give the condition that
\begin{equation}
x_1+x_2+x_3 = y_1+y_2+y_3,
\end{equation}
while the remaining two conditions give the inequalities
\begin{align}
y_1 &\le x_1 \\
y_1+y_2 &\le x_1 +x_2.
\end{align}
This is the standard majorization relation. Geometrically, the cone $\bar{F}$ is actually a funny one to visualise: it consists of a half-line radiating out from $(0,0,0)$ to $(-1, -1, -1)$, together with the triangular ``infinite pyramid'' generated by positive combinations of $(1,1,1), (1,1,0), (1,0,0)$. But we can also take positive linear combinations of $(-1,-1,-1)$ and e.g. $(1,1,0)$ and so it is still a bit tricky to visualise.
\subsection{Symplectic majorization for $d =5$ and beyond.}
For the case $d=5$ we also have a particularly simple group structure, where now the symplectic group $SL(2, \mathbb{Z}_5)$ is isomorphic to the symmetry group of the icosahedron. This is another reflection group and therefore we can again obtain the finite set of majorization conditions to describe it. Here the order of the group is $60$ and so we probably shouldn't make Nick compute this by hand.
\newpage
\section{Extension to general quantum resource theories}
\label{sec:frag}

In the previous section we introduced the notion of $\sigma$--fragments for any resource theory of magic. In this section we pause to generalise this concept to an arbitrary resource theory and explain precisely how it connects with resource monotones. The busy reader more focussed on magic may skip this section. \nick{restate}

State convertibility within a given resource theory is often a hard question to address due to the intricate structure of the theory.
In general, the structure of a theory $\R$ is described by a pre-order $\prec_\R$ and usually resource monotones are employed to reduce this structure into a simple real number ordering\nick{CITE}.
The subdivisions of magic theories into $\sigma$--fragments suggests a new approach towards investigating state convertibility which retains more structure of the origin theory than a measure can.

Monotones reduce the structure of the resource theory $\R$ to a \emph{total} order on the real numbers.
Therefore, two states, even if incomparable in $\R$, are always mapped onto ordered real numbers.
We now generalise this idea of a theory projection that preserves comparability between states. 
\begin{definition}[\textbf{Covariant projection}]\label{def:covproj}
Let $\R = (\F, \O)$ be a resource theory with pre-order $\prec_\R$. 
Then a \emph{covariant resource projection} of $\R$ to a resource theory $\R'$ with pre-order $\prec_{\R'}$, is a pair of mappings $(\Pi_s, \Pi_o)$, where $\Pi_s$ maps quantum states in $\R$ to quantum states in $\R'$, and $\Pi_o$ maps free operations in $\R$ to free operations in $\R'$. 
Moreover, these obey
	\begin{enumerate}
        \item $\Pis(\rho_1) \prec_{\R'} \Pis(\rho_2)$ whenever $\rho_1 \prec_\R \rho_2$;
        \item $\Pio(\E) = \Pio(\E_1) \circ \Pio(\E_2)$ whenever $\E = \E_1 \circ \E_2$.
    \end{enumerate}
We call $\R'$ a \emph{covariant fragment} of $\R$.
\end{definition}

Resource monotones can now be clearly seen as a special case of covariant resource projections.
\begin{proposition}[\textbf{Totally ordered covariant theories}]\label{thm:monoproj}
	Any resource monotone $\M$ of a resource theory $\R$ is a covariant projection for which $\prec_{\R'}$ is a total order. 
	Conversely, any such covariant projection corresponds to a resource monotone $\M$. 
\end{proposition}
\begin{proof}
	Consider a monotone $\M$ in the context of a general resource theory $\R = (\F, \O)$.
	State order is covariantly preserved due to the defining property of a monotone, stated in~\cref{def:mono}, where the pre-order $\prec_{\R'}$ is simply the total order $\leq$ on $\mathbb{R}$. 
	
	Operational composition is covariantly preserved when we simply choose $\Pio(\E) = 1_\times$, namely the `multiplication by 1' operation on real numbers. 
	The definition of a resource monotone then automatically implies covariance.
	
	Conversely, given any covariant projection of $\R$ for which $\prec_{\R'}$ is a total order, we may map the totally ordered set of elements $\Pis(\rho)$ via an injective, non-decreasing function $f$ into $\mathbb{R}$. 
	Then, $\M(\rho):=f(\Pi_s(\rho))$ provides a numerical value for each $\rho$ that obeys the definition of a monotone.
	
\end{proof}

We can also view $\sigma$--fragments as an example of reducing the structure of a magic theory $\R$ to a subtheory with a tractable pre-order.
However, states which are incomparable in $\R$ remain incomparable and conversions between states which are comparable in $\R$ may no longer be possible.
\begin{definition}[\textbf{Contravariant projection}]\label{def:contraproj}
	Let $\R = (\F, \O)$ be a resource theory with pre-order $\prec_\R$. 
Then a \emph{contravariant resource projection} of $\R$ onto a resource theory $\R'$ with pre-order $\prec_{\R'}$, is a pair of mappings $(\Pi_s, \Pi_o)$, where $\Pi_s$ maps quantum states in $\R$ onto quantum states in $\R'$, and $\Pi_o$ maps free operations in $\R$ onto free operations in $\R'$. 
Moreover, these obey
	\begin{enumerate}
        \item $\rho_1 \prec_\R \rho_2$ whenever $\Pis(\rho_1) \prec_{\R'} \Pis(\rho_2)$;
        \item $\E = \E_1 \circ \E_2$ whenever $\Pio(\E) = \Pio(\E_1) \circ \Pio(\E_2)$.
    \end{enumerate}
We call $\R'$ a \emph{contravariant fragment} of $\R$.
\end{definition}
The use of covariant and contravariant in~\cref{def:covproj,def:contraproj} refers to the direction of implication between the two pre-orders and operation compositions\footnote{Note that strictly these are not projections in the sense of $\Pi^2 = \Pi$, but are instead morphisms. 
Here our use of the term projection is motivated by the idea that one one generally loses information about $\R$ under the mapping.}.

\nick{Perhaps scrap definition of contravariant projection and define contravariant fragment as a subtheory $\R'$ where there exists a (injective) covariant projection from $\R'$ to parent theory $\R$. That would make every subtheory $\R' = (\F', \O')$ a contravariant fragment if $\O'$ is closed under composition, this includes $\sigma$--fragments.}

\begin{proposition}\label{thm:subproj}
    Let $\R = (\F, \O)$ be a resource theory, and let $\O' \subseteq \O$ be a non-empty subset of the free operations that is closed under composition, and moreover $\O'$ is the largest such subset, in the sense that for any $\E_1 \not \in \O'$ and any $\E_2 \in \O'$ we have that both $\E_1 \circ \E_2$ and $\E_2 \circ \E_1$ are not in $\O'$. Then $\R' = (\F, \O')$ of $\R$ defines a contravariant fragment of $\R$.    
\end{proposition}
\begin{proof}
    We first define $\Pis (\rho) = \rho$ for all $\rho$. It is clear that since $\O'$ is a subset of $\O$ any operation in $\O'$ will map the set of free states into itself. Moreover the identity channel $id$ is necessarily in $\O'$, due to the maximality assumption. For $\Pio$ we let $\Pis(\E) = \E$ if $\E \in \O'$ and otherwise $\Pis(\E) =id$. Now consider $\Pio (\E_1 \circ \E_2)$. Either the triple $\{\E_1, \E_2, \E_1\circ \E_2\}$ are all in $\O'$ or they are all outside of $\O'$. For the former case $\Pio(\E_1 \circ \E_2) = \E_1 \circ \E_2 = \Pio(\E_1) \circ \Pio(\E_2)$, while for the latter $\Pio(\E_1 \circ \E_2) = id = \Pio(\E_1) \circ \Pio(\E_2)$, which proves that compositions are respected under the map. Finally, $\rho \prec_{R'} \sigma$ implies there exists $\E \in \O' \subseteq \O$ such that $ \E(\rho) = \sigma$, and since $\E \in \O$ this implies $\rho \prec _{\R} \sigma$, as required, which completes the proof.
    
\end{proof}
\nick{Suppose $\E_2$ is the replacement channel $\E_2(\rho) = \sigma$. This is a stabilizer operation. Then $\E_2 \circ \E_1 \in \O_\sigma$ even if $\E_1 \notin \O_\sigma$. Now suppose $\E$ is a Hadamard unitary, then $id = \E \circ \E_{reverse}$ but $id \in \O_{\ketbra{0}}$ while $\E, \E_{reverse} \notin \O_{\ketbra{0}}$.} \ddd{[Ah ok, agreed. This is fine for now. Let's not spend ages trying to generalise this so as to include the majorization fragments. It's just good to explore the possibilities a bit, to illustrate the non-trivial aspects.]}
As an immediate corollary of~\cref{thm:subproj}, a $\sigma$--fragment of any magic theory $\R$ is a contravariant fragment of $\R$.


\begin{proposition}
	Let $\R = (\F,\O)$ be a resource theory, and let $\D \in \O$ be a free operation, which is reversible by $\D_{\rm{rev}} \in \O$, so that $\D_{\rm{rev}} \circ \D = \idc$.
	
	Then, we can define a contravariant projection of $\R$, by acting on all quantum states with $\D$.
\end{proposition}
\begin{proof}
	We show that the theory $\R' = (\F', \O)$, with $\F' = \{\D(\rho):\rho \in \F \}$, is a contravariant fragment of $\R$.
	
	Let $\Pis$ map every state $\rho$ to $\D(\rho)$ and suppose $\D(\rho_1) \prec \D(\rho_2)$. 
	Then, there exists $\E \in \O$ such that $\rho_1 = (\D_{\rm{rev}} \circ \E \circ \D) (\rho_2)$, so $\rho_1 \prec \rho_2$.
	
	Finally, let $\Pio$ map every free operation to itself, so that composition of operations is trivially preserved.
	
\nick{If $\D$ is a recovery map, so that $\D \circ \D_{\rm{rev}} = \idc$, then this is a covariant projection instead.

If $\D$ is not reversible, this mapping is in general NOT contravariant (consider the replacement map $\D(\rho) = \frac{1}{d}\id$ for a strange state and stabilizer state - surely there is such a counterexample in thermodynamics theory if we consider a highly coherent state and one with the same energy population but no coherences.} \ddd{[Ok. Again, let's not spend time worrying about this now. It's clear there is various fine-print to these cases...but they're not essential to our work so let's put this on hold.]}
\end{proof}

Important examples of resource fragments appear in several established resource theories. \nick{Need to check if the thermodynamics example works, include magic theories as fragments of $\Rmax$, include Nielsen's bipartite entanglement.} \ddd{[Don't worry about these things now -- let's get the computations section improved]}

\newpage

%%%%%%%%%%%%%%%%%%%%%%%%%%%%%%%%%%%%%%%%

\section{Conclusion}
\label{sec:conc}

\begin{enumerate}
    \item Introduced fragments
    \item Identify symmetries of the setup
    \item Combined single-shot thermodynamics with magic 
    \item Can we solve other cases exactly? (apart from single qutrit)
\end{enumerate}

%%%%%%%%%%%%%%%%%%%%%%%%%%%%%%%%%%%%%%%%

\bibliography{bib}
%\bibliographystyle{apsrev4-2}

%%%%%%%%%%%%%%%%%%%%%%%%%%%%%%%%%%%%%%%%

\appendix
\newpage
\section{Properties of Wigner distributions}
\label{app:wigner}

Here, we present basic properties of the phase-point operators and the Wigner distribution that are used throughout the paper.

\begin{proposition}\label{thm:aproperties}
    For any dimension $d$, the phase-point operators satisfy:
    \begin{enumerate}
        \item[(i)]\label{en:a1} Hermiticity and unitarity: $A_{\bmx}^\dagger = A_{\bmx} = A_{\bmx}^{-1}$;
	    \item[(ii)]\label{en:a2} Closure under transposition: $A_{(x, p)}^T = A_{(x, -p)}$;
	    \item[(iii)]\label{en:a3} Unit trace for odd $d$: $\tr[A_{\bmx}] = 1$;
	    \item[(iv)]\label{en:a4} Completeness relation: $\sum_{\bmz \in \cal{P}_d} A_{\bmz} = d\id$;
	    \item[(i)]\label{en:a5} Orthogonality: $\tr[A_{\bmx}^\dagger A_{\bm{x'}}] = d \delta_{\bmx,\bm{x'}}$.
	\end{enumerate}
\end{proposition}
All properties follow from the definition in~\cref{eq:ax} along with properties of the displacement operator $D_{\bmx}$ and can be found in the literature, e.g.~\cite{cit:veitch,Vourdas_2004,cit:gross3}

\begin{proposition}\label{thm:wstate}
  The Wigner distribution of a state $\rho \in \cal{B}(\cal{H}_d)$ is
  \begin{enumerate}
    \item[(i)]\label{en:w1} Real valued: $\W{\rho} \in \mathbb{R}^{d^2}$;
    \item[(ii)]\label{en:w2} Normalised: $\sum_{\bmz \in \cal{P}_d} \W[\bmz]{\rho}=1$;
    \item[(iii)]\label{en:w3} Bounded: $\abs{\W[\bmx]{\rho}} \leq \frac{1}{d}$.
    \item[(iv)]\label{en:w4} Additive under mixing: \vspace{2pt}\\
    $\W[\bmx]{p\rho_1 + (1-p)\rho_2} = p\W[\bmx]{\rho_1} + (1-p)\W[\bmx]{\rho_2}$;
    \item[(v)]\label{en:w5} Multiplicative under tensor products: \vspace{2pt}\\
    $\W[\bmx_A \oplus \bmx_B]{\rho_A \otimes \rho_B} = \W[\bmx_A]{\rho_A}\W[\bmx_B]{\rho_B}$.
	\end{enumerate}
\end{proposition}
\begin{proof}
	Proof of all properties can be found in the literature~\cite{cit:veitch,Vourdas_2004,cit:gross3,Wang_2019} except for property (iii) which we prove here.
	
Let $\{\lambda_i\}_{i \in \mathbb{Z}_d}$ be the (non-negative) eigenvalues of $\rho$, summing to 1.
Let $\{\alpha_{\bmx,i}\}_{i \in \mathbb{Z}_d}$ be the eigenvalues of $A_{\bmx}$. For any $\bmx, \alpha_{\bmx,i} \in \{-1, 1\}$, due to the hermiticity and unitarity of the phase-point operators. 
Then,
\begin{align}
	\abs{W_{\rho}(\bmx)} &= \frac{1}{d}\abs{\tr[A_{\bmx} \rho]} \leq \frac{1}{d} \abs{\sum_i \alpha_{\bmx,i} \lambda_i} \nonumber\\ &\leq \frac{1}{d}\sum_i \lambda_i = \frac{1}{d}.
\end{align}
The first inequality follows from Theorem 1 of~\cite{cit:mirsky} for the trace of complex matrices, while the second is the Cauchy-Schwarz inequality.
\end{proof}

\begin{proposition}
    \label{thm:wchannel}
    The Wigner distribution of a $\cptp$ operation $\E: \cal{B}(\cal{H}_{d_A}) \mapsto \cal{B}(\cal{H}_{d_B})$ is
    \begin{enumerate}
        \item[(i)]\label{en:wo1} Real-valued: $\W{\E} \in \mathbb{R}^{d^2} \times \mathbb{R}^{d^2}$;
        \item[(ii)]\label{en:wo2} Normalised: $\sum_{\bmz \in \cal{P}_{d_B}} \W[\bmz|\bmx]{\E} = 1$ \\ 
        for any $\bmx \in \cal{P}_{d_A}$;
        \item[(iii)]\label{en:wo3} Bounded: $\abs{\W[\bmy|\bmx]{\E}} \leq \frac{d_A}{d_B}$;
	    \item[(iv)]\label{en:wo4} Transitive: $\W[\bmy]{\E(\rho)} = \sum_{\bmz \in \cal{P}_{d_A}} \W[\bmy|\bmz]{\E} \W[\bmz]{\rho}$ for any $\bmy \in \cal{P}_{d_B}$.
    \end{enumerate}
\end{proposition}
If $d_A = d_B$, and in particular if operation $\E$ maps a Hilbert space onto itself, then the stochasticity condition $\abs{\W[\bmy|\bmx]{\E}} \leq 1$ is satisfied.
\begin{proof}
	Proof of all properties are provided by Wang \textit{et al.}~\cite{Wang_2019} except for property (iii) which is a direct consequence of the definition of $\W{\E}$ and the corresponding property (iii) in~\cref{thm:wstate}.
\end{proof}

%%%%%%%%%%%%%%%%%%%%%%%%%%%%%%%%%%%%%%%%

\section{Properties of majorization}
\label{app:major}
	
We now give the following equivalent formulations of $d$--majorization.

\begin{proposition}\label{prop:rmajor}
Given $\bmx, \bmy, \r \in \mathbb{R}^n$, such that the components of $\r$ are positive, the following statements are equivalent:
  \begin{enumerate}
    \item[(i)] $\bmy = A\bmx$ and $\r = A\r$ for a stochastic map $A$;
    \item[(ii)]\label{en:tm3} $\sum\limits_{i=1}^n \abs{x_i - r_i t} \leq \sum\limits_{i=1}^n \abs{y_i - r_i t}$ for all $t \in \mathbb{R}$;
    \item[(iii)] $L_{\bmx|\r}(t) \leq L_{\bmy|\r}(t)$ for $t\in [0,1)$ and \vspace{5pt}\\ $L_{\bmx|\r}(1) = L_{\bmy|\r}(1)$.
  \end{enumerate}
\end{proposition}
The proofs for these can be found in~\cite{cit:marshall,cit:bhatia,cit:nielsen,cit:lostaglio} and references therein.

The following result is used in the text to relate relative majorization of quasi-distributions to their Lorenz curves.
\begin{proposition}\label{lemma:Lorenz_linearity}
	Let $\p$ be a quasi-probability distribution and let $\r$ be a probability distribution with strictly non-zero components. Let $a > 0$ and $b \in \mathbb{R}$ then $L_{a\p + b \r | \r} (x) = a L_{\p |\r}(x) + b x$.
\end{proposition}
\begin{proof} 
	The Lorenz curve of $a\p + b \r$ relative to $\r$ passes through $(0,0)$ and the points $(\sum_{i=1}^k{r_{\pi(i)}}, \sum_{i=1}^k(a \p + b \r)_{\pi(i)})$ where $\pi$ is the permutation that puts $(a p_i/r_i + b)$ in non-increasing order. Since $a > 0$, the permutation $\pi$ puts  $(p_i/r_i)$ in non-increasing order too. We thus have
\begin{align*}
&\left( \sum_{i=1}^kr_{\pi(i)}, \sum_{i=1}^k(a \p + b \r)_{\pi(i)} \right) = \\ 
&\left( \sum_{i=1}^k r_{\pi(i)},a \sum_{i=1}^k  p_{\pi(i)} + b\sum_{i=1}^k r_{\pi(i)} \right) \nonumber,
\end{align*}
therefore the value of the Lorenz function at each potential elbow point $x_k = \sum_{i=1} ^kr_{\pi(i)}$ is given by
\begin{align}
&L_{a \p +b \r|\r} (x_k) = a L_{\p|\r} (x_k) + b L_{\r|\r}(x_k) = \nonumber\\
&a L_{\p|\r} (x_k) + b x_k,
\end{align}
so we have $L_{a\p  + b\r|\r} (x) = a L_{\p |\r}(x) + b x$ for any $x \in [0,1]$ due to linearity.
\end{proof}

\begin{theorem*}
	Given a magic state $\rho$, the maximum $L_\star$ of its Lorenz curve $\lc{\rho}{\sigma}(x)$ is independent of the $\sigma$--fragment and equal to $1+\sn{\rho}$. Moreover, the majorization constraint is stronger than mana in every fragment.
\end{theorem*}
\begin{proof}
	We denote the Wigner distributions of the states compactly as vectors $\bmw(\rho) \equiv W_\rho(x)$ and $\bmw(\sigma) \equiv W_\sigma(x)$.
	We choose the component indexing so that the rescaled distribution 
	\begin{equation}
		\widetilde{\bmw}(\rho|\sigma) \coloneqq \left(\frac{w(\rho)_1}{w(\sigma)_1}, \dots, \frac{w(\rho)_{d^2}}{w(\sigma)_{d^2}} \right)^T,
	\end{equation}
	is sorted, $\widetilde{\bmw} = \widetilde{\bmw}^\downarrow$.
	Note that all components of $\bmw(\sigma)$ are positive, so $\widetilde{w}_i \geq 0$ if and only if $w(\rho)_i \geq 0$ for any $i=1,\dots,d^2$.
	
	Let $i_\star$ be the index of the smallest non-negative component of $\widetilde{\bmw}^\downarrow$.
	Then, $w(\rho)_i < 0$ if and only if $i > i_\star$, so the maximum of Lorenz curve $\lc{\rho}{\sigma}(x)$ takes the value 
	\begin{equation}
		\lc{\rho}{\sigma}(x_{i_\star}) = \sum_{i=1}^{i_\star} w(\rho)_i,
	\end{equation}
	and is achieved at
	\begin{equation}\label{eq:maxloc}
		x_{i_\star} \coloneqq \sum_{i=1}^{i_\star} w(\sigma)_i.
	\end{equation}

	The location of the maximum ($x=x_{i_\star}$) varies from fragment to fragment, but its value is independent of $\sigma$,
	\begin{align}
	L_\star &:=	\lc{\rho}{\sigma}(x_{i_\star}) 
		= \sum\limits_{\bmx: \W[\bmx]{\rho} \geq 0} \W[\bmx]{\rho} \nonumber \\
		&= 1 + \sn{\rho}.
	\end{align}
	
Since the magic monotone mana is a monotonic function of sum-negativity, $\rm{mana}(\rho) \coloneqq \ln{(2\hspace{1pt}\sn{\rho}+1)}$, we see that mana corresponds precisely to the peak of the Lorenz curve $L_{\rho|\sigma}(x)$. Therefore, mana is one of $d^{2n}$ constraints, so majorization is strictly a stronger constraint in any fragment.
\end{proof}



%%%%%%%%%%%%%%%%%%%%%%%%%%%%%%%%%%%%%%%%

\section{Technical properties of $\sigma$--fragments}
\label{app:frag}

In this section, we discuss some technical aspects of general $\sigma$--fragments.

We first prove a result on the independence of the Lorenz curve constraints, stated in~\cref{sec:major_frag}.
\begin{proposition}\label{thm:elbows}
	Let $\rho, \tau$ be two quantum states with Lorenz curves $\lc{\rho}{\sigma}(x), \lc{\tau}{\sigma}(x)$ in the $\sigma$--fragment.
	
	Let $t$ be the number of elbows of $\lc{\tau}{\sigma}(x)$ at locations $x_1, \dots, x_t$.
	
	Then, $\lc{\rho}{\sigma}(x) \geq \lc{\tau}{\sigma}(x)$ for all $x \in [0,1]$ iff $\lc{\rho}{\sigma}(x_{i}) \geq \lc{\tau}{\sigma}(x_{i})$ for all $i =1,\dots,t$.
\end{proposition}
\begin{proof}	
	$\lc{\rho}{\sigma}(x) \geq \lc{\tau}{\sigma}(x)$ for all $x \in [0,1]$ trivially implies $\lc{\rho}{\sigma}(x_{i}) \geq \lc{\tau}{\sigma}(x_{i})$ for all $i = 1,\dots,n'$.
	
	Conversely, assume that $\lc{\rho}{\sigma}(x_{i}) \geq \lc{\tau}{\sigma}(x_{i})$ for all $i = 1,\dots,r$.
	First, let $x_0 = 0$ and $x_{n'+1} = 1$, so that $\lc{\rho}{\sigma}(x_0) = \lc{\tau}{\sigma}(x_0) = 0$ and $\lc{\rho}{\sigma}(x_{n'+1}) = \lc{\tau}{\sigma}(x_{n'+1}) = 1$.
	Hence, we can extend the set of elbows $E$ to $E' = E \cup \{x_0, x_{n'+1}\}$.
	
	Pick two consecutive locations $x_{i}, x_{i+1}$ in $E'$ and consider the line segment $\ell_\tau(x)$ connecting points $(x_{i}, \lc{\tau}{\sigma}(x_{i}))$ and $(x_{i+1}, \lc{\tau}{\sigma}(x_{i+1}))$ as well as the line segment $\ell_\rho(x)$ connecting points $(x_{i}, \lc{\rho}{\sigma}(x_{i}))$ and $(x_{i+1}, \lc{\rho}{\sigma}(x_{i+1}))$.
	This is illustrated in~\cref{fig:elbows_proof}.
\begin{figure}[h]
    \centering
    \includegraphics[scale=0.5]{figs/elbows_proof.pdf}
    \caption{\textbf{Illustration of~\cref{thm:elbows}}.
    }
    \label{fig:elbows_proof}
\end{figure}

	Due to concavity of $\lc{\rho}{\sigma}$, it is clear that for all $x \in [x_{i}, x_{i+1}]$, we have $\lc{\rho}{\sigma}(x) \geq \ell_\rho(x) \geq \ell_\tau(x) = \lc{\tau}{\sigma}(x)$.
	This argument can be made in all intervals $[x_{i}, x_{i+1}]$ with $i=0,\dots,n'$, so the proof is complete.
\end{proof}
The above theorem can be of practical importance in reducing the necessary distillation constraints derived via majorization in $\sigma$--fragments.

\begin{proposition}\label{thm:frag_app}
    Let $\R = (\F, \O)$ be a magic theory and $\sigma, \sigma' \in \F$. The following statements hold:
    \begin{enumerate}
        \item No $\sigma$--fragment of $\R$ is empty.
        \item If a free operation leaves two states invariant, then it also leaves their mixtures invariant, 
        \begin{equation*}
            \O_{\sigma} \cap \O_{\sigma'} \subseteq \O_{p\sigma + (1-p)\sigma'}\ \text{for any}\ p \in [0,1].
        \end{equation*}
    \end{enumerate}
\end{proposition}
\begin{proof}$ $\vspace{-12pt}\\

\begin{enumerate}
    \item The identity channel $1_{\rm{C}}$ belongs to every $\sigma$--fragment, as $1_{\rm{C}} \in \O$ and $1_{\rm{C}}\sigma = \sigma$ for all $\sigma \in \F$.
    
    \item Let $\E \in \O_{\sigma} \cap \O_{\sigma'}$.
    Then $\E \in \cptp$ and corresponds to stochastic Wigner distribution $\W{\E}$ such that $\W{\E} \W{\sigma} = \W{\sigma}$ and $\W{\E} \W{\sigma'} = \W{\sigma'}$.
    Then, $\W{\E} \W{p\sigma + (1-p)\sigma'} = \W{p\sigma + (1-p)\sigma'}$ for any $p \in [0,1]$ due to the additive property~\ref{en:w4} of the Wigner distribution, implying that state $p\sigma + (1-p)\sigma'$ is also left invariant by $\E$.
\end{enumerate}
\vspace{-20pt}
\end{proof}

Any free state $\sigma \in \F$ corresponds to a $d^2$--dimensional probability distribution $\W{\sigma}$ and any free operation $\E \in \O$ corresponds to a $d^2 \times d^2$ stochastic matrix (or conditional probability distribution) $\W{\E}$.
Note that these mappings are one-to-one due to the orthogonality of the phase-point operators as an operator basis.
Note further that free states $\F$ are mapped onto a \emph{strict subset} of the set of probability distributions.
As a counterexample, the sharp $d^2$--dimensional probability distribution $(1, 0, \dots, 0)$ does not correspond to any qudit Wigner distribution because of the boundedness condition in~\cref{thm:wstate}.
Similarly, not all stochastic matrices correspond to completely positive operations.
As an example, consider the permutation matrix
\begin{equation}
    \Pi_X = \begin{psmallmatrix}
        0 & 1 & 0 & 0 & 0 \\
        0 & 0 & 0 & 0 & 1 \\
        0 & 0 & 0 & 1 & 0 \\
        1 & 0 & 0 & 0 & 0 \\
        0 & 0 & 1 & 0 & 0
    \end{psmallmatrix} \otimes \begin{psmallmatrix}
        0 & 0 & 1 & 0 & 0 \\
        0 & 0 & 0 & 0 & 1 \\
        0 & 0 & 0 & 1 & 0 \\
        1 & 0 & 0 & 0 & 0 \\
        0 & 1 & 0 & 0 & 0    
    \end{psmallmatrix} \in {\rm{S}}_5({\W{\frac{1}{5}\id}}).
\end{equation}
It preserves the uniform distribution $\W{\frac{1}{5}\id}$, but it does not correspond to any positive (hence quantum) operation.

%%%%%%%%%%%%%%%%%%%%%%%%%%%%%%%%%%%%%%%%

\section{Lorenz curves in the unital fragment}
\label{app:lcsu_technical}

\subsection{Binomial distributions and error bounds}\label{app:phi}
Consider an experiment consisting of $n$ trials of throwing a $p$--coin, that is a coin with probability $p$ of landing on one side and $1-p$ of landing on the other.
We express the sums over an even number $m$ of successful trials $\Phi_+$ and an odd number $m$ of successful trials $\Phi_-$,
\begin{align}	
	\Phi_+(m; n, p) &\coloneqq \sum\limits_{\ell=0}^{m/2} \binom{n}{2\ell} p^{2\ell} (1-p)^{n-2\ell}, \nonumber\\ 
	&\text{for even integers } m\in[0,n], \label{eq:fp_app} \\
	\Phi_-(m; n, p) &\coloneqq \sum\limits_{\ell=1}^{(m-1)/2} \binom{n}{2\ell+1} p^{2\ell+1} (1-p)^{n-(2\ell+1)}, \nonumber\\ 
	&\text{for odd integers }m\in[0,n]. \label{eq:fn_app}
\end{align}
Note that index $m$ only takes even (odd) values when labelling $\Phi_+$ ($\Phi_-$).
In~\cref{app:lcsu_coord}, we will use $\Phi_+$ and $\Phi_-$ to express the elbow coordinates of Lorenz curves in the unital fragment.

We also define the classical entropy of a $p$--coin and the classical relative entropy between a $p$--coin and a $q$--coin,
\begin{align}
	S(p) &\coloneqq -p\log{p} -(1-p)\log{(1-p)}, \label{eq:ent}\\
	\ent{p}{q} &\coloneqq p \log{\frac{p}{q}} + (1-p) \log{\frac{1-p}{1-q}}. \label{eq:ent_rel}
\end{align}
They are symmetric in the sense that $S(p) = S(1-p)$ and $\ent{p}{q} = \ent{1-p}{1-q}$.

A useful result is the entropic bound on a combination~\cite{cit:ash}.
\begin{lemma}\label{lem:comb_bounds}
	For all $\ell\in [1,n-1]$,
	\begin{align}
		&\left[ 8\ell\left(1-\frac{\ell}{n}\right) \right]^{-\frac{1}{2}} 2^{n S\left(\frac{\ell}{n}\right)} \leq \binom{n}{\ell} \leq \\
		&\left[ 2\pi \ell\left(1-\frac{\ell}{n}\right) \right]^{-\frac{1}{2}} 2^{n S\left(\frac{\ell}{n}\right)}.
	\end{align}
\end{lemma}
The proof provided in~\cite{cit:ash} proceeds with direct calculation for the edge cases $\ell = 1,2, n-1, n-2$ and use Stirling's approximation for the remaining cases.
We can use~\cref{lem:comb_bounds} to provide strict upper and lower bounds on the functions $\Phi_+, \Phi_-$.

\subsection{Theory on bounding the core functions}
\nick{Should we really keep this section? It is standard calculus and the bounds are really loose.}
Here we present a more manageable method of bounding the core functions $\Phi_{\pm}$, which however results in looser bounds. 
We can rewrite the functions as
\begin{equation}
	\Phi_{\pm}(m; n, a) = \frac{1}{2}(\Phi(m; n, a) \pm (1+a)^{-n} S(m; n, a)),
\end{equation}
where we have substituted $a = p/(1-p)$. 
$\Phi$ is the standard cumulative function
\begin{equation}
	\Phi(m; n, a) = (1+a)^{-n} \sum_{k=0}^m \binom{n}{k} a^k,
\end{equation}
and the remainder term is
\begin{equation}
	S(m; n, a) \coloneqq \sum_{k=0}^m \binom{n}{k} (-a)^k.
\end{equation}
We have the following asymptotic bounds on the behaviour of $\Phi$~\cite{cit:ash},
\begin{lemma}\label{lem:phi_bounds}
	Given fixed $n>0$ and $p$, $\Phi$ satisfies the following bounds:
	\begin{align*}
		\begin{split}
		&\text{1. } \Phi(m; n, p) \geq \left[ 8m\left(1-\frac{m}{n}\right) \right]^{-\frac{1}{2}} 2^{-n\ent{\frac{m}{n}}{p}}, \\
		&\hspace{14pt} m\in [1,n-1] \\
		&\text{2. } \Phi(m; n, p) \geq 1 - 2^{-n\ent{\frac{m+1}{n}}{p}},\ m\in [np+1,n-2] \\
		&\text{3. } \Phi(m; n, p) \leq 1 - \left[ 8(m+1)\left(1-\frac{m+1}{n}\right) \right]^{-\frac{1}{2}}\times \\
		&\hspace{14pt} 2^{-n\ent{\frac{m+1}{n}}{p}},\ m\in [0,n-2]
		\end{split}
		\\
		&\text{4. } \Phi(m; n, p) \leq 2^{-n\ent{\frac{m}{n}}{p}},\ m\in [0,np]
	\end{align*}
\end{lemma}

We would like some theory that estimates the value of $S(m; n, a)$ for different parameter regimes. 
We can consider the function $f(x) = (1+x)^n$ and note that $S(m; n, a)$ is the $m$'th partial sum of this expansion at the point $x=-a$.

The truncated Maclaurin series of a general function $f(x)$ is
\begin{equation}
	f(x) = f(0) + x f'(0) + \dots \frac{x^m}{m!}f^{(m)}(0) + R_m(x)
\end{equation}
with a remainder term
\begin{align}
	R_m (x)&= \int_{0}^x dt f^{(m+1)}(t) \frac{(x-t)^m}{m!} \\
	&= \frac{x^{m+1}}{(m+1)!} f^{(m+1)}(x_*),
\end{align}
where in the second expression, $x_*$ is an implicit point that lies between $0$ and $x$ that comes from the Mean Value Theorem.

Applying this to the function $f(x) = (1+x)^n$ gives
\begin{equation}
	(1+x)^n = \sum_{k=0}^m \binom{n}{k} x^k + R_m.
\end{equation}
Evaluating at $x=-a$ gives
\begin{equation}
	S(m; n, a) = (1-a)^n - R_m(-a),
\end{equation}
where the key remainder term is given by
\begin{align}
	R_m(-a) &= \int_0^{-a} dt f^{(m+1)}(t) \frac{(-a-t)^m}{m!} \\
&= \frac{(-a)^{m+1}}{(m+1)!} f^{(m+1)}(x_*).
\end{align}
We can also compute the derivative $f^{(m+1)}(x)$ explicitly,
\begin{equation}
	f^{(m+1)}(x) = (m+1)!\binom{n}{m+1}(1+x)^{n-m-1}.
\end{equation}
Therefore, we have that
\begin{align}
	R_m(-a) &= (-1)^{m+1}(m+1)\binom{n}{m+1}\times \nonumber\\
	&\hspace{12pt} \int_{-a}^0 dt (1+t)^{n-m-1}(a+t)^m \\
&= \binom{n}{m+1}(-a)^{m+1}(1+x_*)^{n-m-1},
\end{align}
where in the latter expression $x_* \in [-a,0]$. 
Note that the first integral expression can be estimated via the Cauchy-Schwarz or the H{\"o}lder inequality. 
Therefore, we can either work with an explicit form with an unknown (but bounded) parameter $x_*$, or we can use the integral form and provide concrete estimates on it value.

A very simple estimate, based on $x_*$ lying in the interval $[-a,0]$ gives
\begin{equation}
(1-a)^{n-m-1} \leq \frac{R_m(-a)}{\binom{n}{m+1}(-a)^{m+1}} \leq 1,
\end{equation}
which in turn leads to the following bounds on $\Phi_+(m; n, a)$:
\begin{align}
	\hspace{-2cm}2\Phi_+(m; n, a) \leq\ &\Phi(m; n, a) + (1-a)^n - \frac{(-a)^{m+1}}{m!} \times \nonumber\\
	 &\binom{n}{m+1} (1-a)^{n-m-1} \text{ and} \\
	2\Phi_+(m; n, a) \geq\ &\Phi(m; n, a) + (1-a)^n - \frac{(-a)^{m+1}}{m!} \binom{n}{m+1} .
\end{align}

\subsection{Lorenz curve coordinates in the unital fragment}\label{app:lcsu_coord}
The Wigner distribution of the $n$--copy qutrit maximally mixed state $\left(\id/3\right)^{\otimes n}$ is the uniform probability distribution over the phase space, consisting of $9^n$ components equal to $9^{-n}$.
The Wigner distribution of the 1-copy $\epsilon$--noisy Strange state $\rho_{\rm{S}}(\epsilon)$ in the unital fragment consists of some permutation of a single negative component
\begin{equation}
	- v(\epsilon) \coloneqq - \left( \frac{1}{3} -\frac{4}{9}\epsilon \right),
\end{equation} 
and $8$ positive components
\begin{equation}
	u(\epsilon) \coloneqq \frac{1}{6} -\frac{1}{18}\epsilon.
\end{equation}
where in the unital fragment we need the condition $0 \leq \epsilon < 3/4$, so that the state contains some Wigner negativity ($-v < 0$).
It is also clear that $v \geq u$ in the interval $0 \leq \epsilon \leq 3/7$, while $u > v$ in the interval $3/7 < \epsilon < 3/4$.

The Wigner distribution of the $n$--copy $\epsilon$--noisy Strange state $\rho_{\rm{S}}(\epsilon)^{\otimes n}$ in the unital fragment is given by the convolution $\W{\rho_{\rm{S}}(\epsilon)^{\otimes n}} = W_{\rho_{\rm{S}}(\epsilon)}^{\otimes n}$.
In general, $\rho_{\rm{S}}(\epsilon)^{\otimes n}$ contains $n + 1$ distinct components, labelled $0,\dots, n$.
We present the distinct Wigner components of $\rho_{\rm{S}}(\epsilon)^{\otimes n}$ along with their multiplicites in~\cref{tab:lcsu}.
Note that LHS (RHS) refers to elbow coordinates $i$ on the left of and including (right of) the Lorenz curve maximum, stated precisely as
\begin{align}
&\text{LHS: } 0 \leq i \leq \left\lfloor \frac{n}{2} \right\rfloor \text{ and} \\
&\text{RHS: } \left\lfloor \frac{n}{2} \right\rfloor +1 \leq i \leq n.
\end{align}
\begin{table}[h]
  \def\arraystretch{1.5}
  \centering
  \begin{tabular}{c|c|c|r|r}
    \multicolumn{3}{c|}{Case} & \multicolumn{1}{c}{$m_{i}(n, \epsilon)$} & \multicolumn{1}{|c}{$w_{i}(n, \epsilon)$} \\[0.5ex]\hline
    \multirow{4}{*}{\raisebox{-4ex}{\rotatebox[origin=c]{90}{$0\leq \epsilon < \frac{3}{7}$}}} & \hspace{0.8ex}\multirow{2}{*}{\raisebox{-1ex}{\rotatebox[origin=c]{90}{$n$ even}}}\hspace{0.8ex} & LHS & $8^{2i}\binom{n}{2i}$ & $\left( \frac{1}{6} - \frac{1}{18}\epsilon \right)^{2i}\left( -\frac{1}{3} + \frac{4}{9}\epsilon \right)^{n-2i}$ \\
    & & RHS & $8^{n-2i}\binom{n}{2i}$ & $\left( \frac{1}{6} - \frac{1}{18}\epsilon \right)^{n-2i}\left( -\frac{1}{3} + \frac{4}{9}\epsilon \right)^{2i}$ \\ \cline{2-5}
    & \multirow{2}{*}{\raisebox{-2ex}{\rotatebox[origin=c]{90}{$n$ odd}}} & LHS & $8^{2i+1}\binom{n}{2i+1}$ & $\left( \frac{1}{6} - \frac{1}{18}\epsilon \right)^{2i+1}\left( -\frac{1}{3} + \frac{4}{9}\epsilon \right)^{n-2i-1}$ \\
    & & RHS & $8^{n-2i-1}\binom{n}{2i+1}$ & $\left( \frac{1}{6} - \frac{1}{18}\epsilon \right)^{n-2i-1}\left( -\frac{1}{3} + \frac{4}{9}\epsilon \right)^{2i+1}$ \\ \hline
    \multirow{4}{*}{\raisebox{-4ex}{\rotatebox[origin=c]{90}{$\frac{3}{7}\leq \epsilon < \frac{3}{4}$}}} & \multirow{2}{*}{\raisebox{-1ex}{\rotatebox[origin=c]{90}{$n$ even}}} & LHS & $8^{n-2i}\binom{n}{2i}$ & $\left( \frac{1}{6} - \frac{1}{18}\epsilon \right)^{n-2i}\left( -\frac{1}{3} + \frac{4}{9}\epsilon \right)^{2i}$ \\
    & & RHS & $8^{2i}\binom{n}{2i}$ & $\left( \frac{1}{6} - \frac{1}{18}\epsilon \right)^{2i}\left( -\frac{1}{3} + \frac{4}{9}\epsilon \right)^{n-2i}$ \\ \cline{2-5}
    & \multirow{2}{*}{\raisebox{-2ex}{\rotatebox[origin=c]{90}{$n$ odd}}} & LHS & $8^{n-2i}\binom{n}{2i}$ & $\left( \frac{1}{6} - \frac{1}{18}\epsilon \right)^{n-2i}\left( -\frac{1}{3} + \frac{4}{9}\epsilon \right)^{2i}$ \\
    & & RHS & $8^{2i}\binom{n}{2i}$ & $\left( \frac{1}{6} - \frac{1}{18}\epsilon \right)^{2i}\left( -\frac{1}{3} + \frac{4}{9}\epsilon \right)^{n-2i}$ \\ \hline
  \end{tabular}
  \caption{Wigner components $w_{i}(n, \epsilon)$ of $\rho_{\rm{S}}(\epsilon)^{\otimes n}$ along with their multiplicities $m_{i}(n, \epsilon)$, with $0 \leq i \leq n$.
  The expressions change depending on the noise level $\epsilon$, the parity of the number of copies $n$ and whether the index $i$ is lower or higher than the index of the Lorenz curve maximum (LHS vs RHS).
  Multiplication $2i$ is considered modulo $(n+1)$.}
  \label{tab:lcsu}
\end{table}

Every Lorenz curve in the unital fragment contains $n$ elbows, which, along with the boundary points $(x_{-1}, L_{-1}) = (0,0)$ and $(x_{n}, L_{n}) = (1,1)$, are labelled by 
\begin{equation*}
\{(x_{i}, L_{i})\}_{i=-1,0,\dots,n}.
\end{equation*}
The maximum is the $\lfloor n/2 \rfloor$-th elbow and its coordinates are calculated by collecting all the positive Wigner components,
\begin{align}
	x_{\lfloor n/2 \rfloor} &= \frac{1}{2}\left(1 + \left(\frac{7}{9}\right)^n\right), \\
	L_{\lfloor n/2 \rfloor} &= \frac{1}{2}\left (1 + \left(\frac{15 - 8\epsilon}{9}\right)^n \right).
\end{align}
%\sum_{j: even}^n a^j \binom{n}{j} = \frac{1}{2} [ (1+a)^n + (1-a)^n ]

Expressions for all the elbow coordinates follow from summing up the Wigner components in decreasing order.
In~\cref{tab:lcsu_coord_elb_app}, we present the elbow coordinates of the $n$-copy, $\epsilon$--noisy Strange state Lorenz curve in the unital fragment for any combination of parameters $n, \epsilon$.
\begin{table}[h]
  \def\arraystretch{1.5}
  \centering
  \begin{tabular}{c|c|c|r|r}
\multicolumn{3}{c|}{\multirow{2}{*}{Case}} & \multicolumn{1}{c|}{$x_{i}$} & \multicolumn{1}{c}{$L_{i}$} \\
    \multicolumn{3}{c|}{} & \multicolumn{1}{c|}{$x_{i} - x_{\lfloor n/2 \rfloor}$} & \multicolumn{1}{c}{$L_{i} - L_{\lfloor n/2 \rfloor}$} \\[0.5ex]\hline 
    \multirow{4}{*}{\raisebox{-5ex}{\rotatebox[origin=c]{90}{$0\leq \epsilon < \frac{3}{7}$}}} & \hspace{0.8ex}\multirow{2}{*}{\raisebox{-3ex}{\rotatebox[origin=c]{90}{$n$ even}}}\hspace{0.8ex} & LHS & $\Phi_+\left(2i;n,\frac{8}{9}\right)$ & $\left( \frac{5}{3} - \frac{8}{9}\epsilon\ \right)^n \Phi_+\left(2i;n,\frac{12-4\epsilon}{15-8\epsilon}\right)$ \\
    & & RHS & $\Phi_-\left(2i;n,\frac{1}{9}\right)$ & $- \left( \frac{5}{3} - \frac{8}{9}\epsilon\ \right)^n\Phi_-\left(2i;n,\frac{3-4\epsilon}{15-8\epsilon}\right)$ \\ \cline{2-5}
    & \multirow{2}{*}{\raisebox{-3ex}{\rotatebox[origin=c]{90}{$n$ odd}}} & LHS & $\Phi_-\left(2i;n,\frac{8}{9}\right)$ & $\left( \frac{5}{3} - \frac{8}{9}\epsilon\ \right)^n \Phi_-\left(2i;n,\frac{12-4\epsilon}{15-8\epsilon}\right)$ \\
    & & RHS & $\Phi_-\left(2i;n,\frac{1}{9}\right)$ & $- \left( \frac{5}{3} - \frac{8}{9}\epsilon\ \right)^n\Phi_-\left(2i;n,\frac{3-4\epsilon}{15-8\epsilon}\right)$ \\ \hline
    \multirow{4}{*}{\raisebox{-5ex}{\rotatebox[origin=c]{90}{$\frac{3}{7}\leq \epsilon < \frac{3}{4}$}}} & \multirow{2}{*}{\raisebox{-3ex}{\rotatebox[origin=c]{90}{$n$ even}}} & LHS & $\Phi_+\left(2i;n,\frac{1}{9}\right)$ & $\left( \frac{5}{3} - \frac{8}{9}\epsilon\ \right)^n \Phi_+\left(2i;n,\frac{3-4\epsilon}{15-8\epsilon}\right)$ \\
    & & RHS & $\Phi_-\left(2i;n,\frac{8}{9}\right)$ & $- \left( \frac{5}{3} - \frac{8}{9}\epsilon\ \right)^n\Phi_-\left(2i;n,\frac{12-4\epsilon}{15-8\epsilon}\right)$ \\ \cline{2-5}
    & \multirow{2}{*}{\raisebox{-3ex}{\rotatebox[origin=c]{90}{$n$ odd}}} & LHS & $\Phi_+\left(2i;n,\frac{1}{9}\right)$ & $\left( \frac{5}{3} - \frac{8}{9}\epsilon\ \right)^n \Phi_+\left(2i;n,\frac{3-4\epsilon}{15-8\epsilon}\right)$ \\
    & & RHS & $\Phi_+\left(2i;n,\frac{8}{9}\right)$ & $- \left( \frac{5}{3} - \frac{8}{9}\epsilon\ \right)^n\Phi_+\left(2i;n,\frac{12-4\epsilon}{15-8\epsilon}\right)$ \\ \hline
  \end{tabular}
  \caption{Lorenz curve elbow coordinates in the unital fragment.
  The coordinate expressions depend on the noise level $\epsilon$, the parity of the number of copies $n$ and the location of the elbow relative to the maximum (LHS vs RHS).
  Multiplication $2i$ is considered modulo $(n+1)$.
  For completeness, note that $(x_{-1}, L_{-1}) \coloneqq (0,0)$ is not included in the table.
  }
  \label{tab:lcsu_coord_elb_app}
\end{table}

We can get explicit expressions for all $9^{n}$ points of the Lorenz curve $\lc{\rho_{\rm{S}}(\epsilon)^{\otimes n}}{(\id/3)^{\otimes n}}$, in terms of the elbow coordinates:
\begin{align}
    x_{ij} &= \left( 1-\frac{j}{m_{i}} \right) x_{i-1} + \frac{j}{m_{i}} x_{i}, \label{eq:x}\\
    L_{ij} &= \left( 1-\frac{j}{m_{i}} \right) L_{i-1} + \frac{j}{m_{i}} L_{i} \label{eq:l}
\end{align}
for $j = 1,\dots,m_{i}$ and $i=0,\dots,n$, where multiplicities $m_i = m_i(n, \epsilon)$ are given in~\cref{tab:lcsu}.

Consider the state 
\begin{equation*}
\rho_{\rm{S}}(\epsilon')^{\otimes n'} \otimes \left( \frac{1}{3}\id \right)^{\otimes (n-n')},
\end{equation*}
where tensoring with the maximally mixed state keeps the Lorenz curve unchanged, but increases the resolution of (the uniformly distributed) points.
The new point coordinates are given by:
\begin{align}
    &x_{ijk} = \left( 1-p_{ijk}\right) x_{i-1} + p_{ijk} x_{i} \label{eq:lcsu_xcoord}\\
    &L_{ijk} = \left( 1-p_{ijk} \right) L_{i-1} + p_{ijk} L_{i}, \label{eq:lcsu_lcoord}\\
    &\text{where } p_{ijk} = \frac{k + (j-1)9^{n-n'}}{9^{n-n'} m_{i}} \nonumber\\
    &\text{for } i=0,\dots,n',\ j = 1,\dots,m_{i}(n', \epsilon') \text{ and } k = 1,\dots,9^{n-n'}. \nonumber
\end{align}

We can unify the indices, by introducing a single index
\begin{equation}
    I(i,j,k) \coloneqq k + \left[ (j-1) + \sum_{\ell=0}^{i-1} m_{\ell}(n', \epsilon') \right]9^{n-n'},
\end{equation}
so that $I=1,2,\dots, 9^{n}$.
The elbow coordinates correspond to 
\begin{equation}
	I(i, m_{i}(n', \epsilon'), 9^{n-n'}) = \sum_{\ell=0}^{i} m_{\ell}(n', \epsilon'),\ i= 0,\dots,n'.
\end{equation}
The index function $I$ is bijective, i.e.
\begin{equation}
	(i,j,k) = (i',j',k') \text{ iff } I(i,j,k) = I(i',j',k').
\end{equation}

%%%%%%%%%%%%%%%%%%%%%%%%%%%%%%%%%%%%%%%%

\section{Technical details for the derivation of distillation bounds from Lorenz curves}
\label{app:lcst_technical}

\subsection{First and last elbow constraints}
\label{app:elb_constraints}
Here we prove two simple majorization constraints, one arising by considering only the ascending part of the Lorenz curves between the origin $(0,0)$ and the first elbow and the other by considering only the descending part of the curves between the last elbow and the endpoint $(1,1)$.
\begin{proposition}\label{prop:first_elb}
	Consider a magic state process $\rho \longrightarrow \tau$ with input and output Lorenz curves $\lc{\rho}{\sigma}(x), \lc{\tau}{\sigma}(x)$ in $\R_\sigma$ and denote by $X_0, X'_0$ the first elbow locations of the input and output curves respectively.
	
	Then, given any coordinates $(x_0, L_0)$ and $(x'_0, L'_0)$ on the input and output Lorenz curves, where $x_0 \leq X_0$ and $x'_0 \leq X'_0$, the process is possible in $\R_\sigma$ only if
\begin{equation}\label{eq:first_elb_bound1}
	\frac{L_0}{x_0} \geq \frac{L_0'}{x_0'}.
\end{equation}
\end{proposition}
\begin{proof}
Since both pairs of coordinates are located between $(0,0)$ and the first elbow of their respective curves, we can derive the bound via a simple interpolation on the line segment connecting the origin and the appropriate first elbow.

First assume that $x_0 < x'_0$ and consider the Lorenz curve constraint at $x = x_0$,
\begin{equation}
	\lc{\rho}{\sigma}(x_0) \geq \lc{\tau}{\sigma}(x_0).
\end{equation}
We can find the output state Lorenz curve coordinate $L'_\star$ at location $x = x_0$ by interpolating between the origin and the output state's first elbow, 
\begin{equation}
	L'_\star = \frac{x_0}{x'_0}L'_0.
\end{equation}
The process is then possible only if $L_0 \geq L_\star'$ which is a rearrangement of~\cref{eq:first_elb_bound1}.

If instead, $x_0 \geq x'_0$, consider the Lorenz curve constraint at $x = x'_0$,
\begin{equation}
	\lc{\rho}{\sigma}(x'_0) \geq \lc{\tau}{\sigma}(x'_0).
\end{equation}
We now need to find the input state Lorenz curve coordinate $L_\star$ at location $x = x'_0$ by interpolating between the origin and the input state's first elbow, 
\begin{equation}
	L_\star = \frac{x'_0}{x_0}L_0.
\end{equation}
The process is then possible only if $L_\star \geq L'_0$ which is again a rearrangement of~\cref{eq:first_elb_bound1}.
\end{proof}

\begin{proposition}\label{prop:last_elb}
	Consider a magic state process $\rho \longrightarrow \tau$ with input and output Lorenz curves $\lc{\rho}{\sigma}(x), \lc{\tau}{\sigma}(x)$ in $\R_\sigma$ and denote by $X_E, X'_E$ the last elbow locations of the input and output curves respectively.
	
	Then, given any coordinates $(x_E, L_E)$ and $(x'_E, L'_E)$ on the input and output Lorenz curves, where $x_E \geq X_E$ and $x'_E \geq X'_E$, the process is possible in $\R_\sigma$ only if
\begin{equation}\label{eq:last_elb_bound1}
	\frac{L_E - 1}{1-x_E} \geq \frac{L'_E - 1}{1-x_E'}.
\end{equation}
\end{proposition}
\begin{proof}
Since both pairs of coordinates are located between the last elbow of their respective curves and $(1,1)$, we can derive the bound via a simple interpolation on the line segment connecting the endpoint and the appropriate last elbow.

First assume that $x_E > x'_E$ and consider the Lorenz curve constraint at $x = x_E$,
\begin{equation}
	\lc{\rho}{\sigma}(x_E) \geq \lc{\tau}{\sigma}(x_E).
\end{equation} 
We can find the output state Lorenz curve coordinate $L'_\star$ at location $x = x_E$ by interpolating between the endpoint $(1,1)$ and the output state's last elbow, 
\begin{equation}
	L'_\star = 1 + \frac{1-x_E}{1-x'_E} (L'_E - 1)
\end{equation}
The process is then possible only if $L_E \geq L_\star'$ which is a rearrangement of~\cref{eq:last_elb_bound1}.

If instead, $x_E \leq x'_E$, consider the Lorenz curve constraint at $x = x'_E$,
\begin{equation}
	\lc{\rho}{\sigma}(x'_E) \geq \lc{\tau}{\sigma}(x'_E).
\end{equation} 
We now need to find the input state Lorenz curve coordinate $L_\star$ at location $x = x'_E$ by interpolating between the endpoint $(1,1)$ and the input state's last elbow, 
\begin{equation}
	L_\star = 1 + \frac{1-x'_E}{1-x_E} (L_E - 1).
\end{equation}
The process is then possible only if $L_\star \geq L'_E$ which is again a rearrangement of~\cref{eq:last_elb_bound1}.
\end{proof}

\subsection{Component-multiplicity pairs}
\label{app:cmpairs}
In general, a $1$--copy $d$--dimensional state $\rho$ is described exactly by its $d^2$--dimensional Wigner distribution $\W{\rho}$. 
The distribution $\W{\rho}$ is usually defined on the phase space, but it can be convenient to define it using vector notation. 
In particular, we introduce a component vector $\bmw(\rho) = (w_i)_{i=1,\dots,D}$ and a multiplicity vector $\bmm(\rho) = (m_i)_{i=1,\dots,D}$, where $D \leq d^2$ which together form a set of component-multiplicity pairs $\{(w_i, m_i)\}_{i=1,\dots,D}$.
\begin{definition}
	Consider a distribution $W$ and a positive integer $D \leq {\rm{dim}}\hspace{1pt}W$. 
	We call the set of ordered pairs $\{(w_i, m_i)\}_{i=1,\dots,D}$ a \emph{complete set of component-multiplicity pairs}, if $W$ contains $m_i$ components $w_i$ and $\sum_{i=0}^D m_i = d^2$.
\end{definition}
Therefore, such a set describes each component of $\W{\rho}$ exactly once.
As an example, two complete sets of pairs for the Strange state are $\{( -1/3, 1), ( 1/6, 8)\}$ and $\{(-1/3, 1), (1/6, 2), (1/6, 3), (1/6, 3)\}$.
The latter corresponds to the phase space split in~\cref{fig:pd_split}

Consider two states $\rho_A, \rho_B$ with Wigner distributions $\W{\rho_A}, \W{\rho_B}$ described respectively by complete sets of component-multiplicity pairs 
\begin{equation}
	\{(w_i, m_i)\}_{i=1,\dots,D_A} \text{ and } \{(w_j', m_j')\}_{j=0,\dots,D_B}.
\end{equation}
The multiplicative property of the Wigner distribution over a composite phase space $\cal{P}_{d_A} \times \cal{P}_{d_B}$ shown in~\cref{thm:wstate},
\begin{equation}
	\W[\bmx_A \oplus \bmx_B]{\rho_A \otimes \rho_B} = \W[\bmx_A]{\rho_A}\W[\bmx_B]{\rho_B},
\end{equation}
implies that the distribution $\W{\rho_A \otimes \rho_B}$ is $d_A^2 d_B^2$--dimensional and contains components of the form $w_i w_j'$. 
Therefore, the set $\{(w_i w_j', m_i m_j')\}$ with $i=1,\dots,D_A$ and $j=1,\dots,D_B$ is a complete set of component-multiplicity pairs for the distribution of the composite system $\W{\rho_A \otimes \rho_B}$.
This is true because all components are of the form $w_i w_j'$ and 
\begin{equation*}
	\sum_{i=1}^{D_A}\sum_{j=1}^{D_B} m_i m_j' = \sum_{i=1}^{D_A} m_i \sum_{j=1}^{D_B} m_j' = d_A^2 d_B^2.
\end{equation*}

Note that the rescaled distribution is also multiplicative,
\begin{align}
	&\widetilde{\rm{W}}_{\rho_A \otimes \rho_B | \gamma_A \otimes \gamma_B}(\bmx_A \oplus \bmx_B) = \frac{\W[\bmx_A \oplus \bmx_B]{\rho_A \otimes \rho_B}}{\W[\bmx_A \oplus \bmx_B]{\gamma_A \otimes \gamma_B}} = \nonumber \\
	&\frac{\W[\bmx_A]{\rho_A}\W[\bmx_B]{\rho_B}}{\W[\bmx_A]{\gamma_A}\W[\bmx_B]{\gamma_B}} = \widetilde{\rm{W}}_{\rho_A | \gamma_A}(\bmx_A)\widetilde{\rm{W}}_{\rho_B  | \gamma_B}(\bmx_B),
\end{align}
so a complete set of component-multiplicity pairs can be obtained for this distribution in the same fashion as for usual Wigner distributions.

Given a state $\rho$ and a complete set of component-multiplicity pairs describing its Wigner distribution $\W{\rho}$, we now provide a method of computing the components (and multiplicities) of the $n$--copy distribution $\W{\rho}^{\otimes n}$.
\begin{lemma}\label{lem:ncopycomponents}
	Let $W$ be a distribution defined by a complete set of component-multiplicity pairs $\{(w_i, m_i)\}_{i=1,\dots,D}$ with $D \leq {\rm{dim}}\hspace{1pt}W$ and consider the distribution $W^{\otimes n}$ obtained by taking the $n$-fold (Kronecker) product $W \otimes \dots \otimes W$ between $n$ copies of $W$.
	
	Denote by $C_D^n \coloneqq \{\bmk\}$ the set of all vectors $\bmk \coloneqq (k_1, \dots, k_D)$ with non-negative integer components that sum to $n$, i.e.
	\begin{equation*}
	0 \leq k_1, \dots, k_D \leq n \text{ and } k_1 + \dots + k_D = n.
	\end{equation*}
	
	Then, $W^{\otimes n}$ admits a complete set of component-multiplicity pairs $\{(W_{\bmk}, M_{\bmk})\}_{\bmk \in C_D^n}$, where
\begin{align}
	M_{\bmk} &= \frac{n!}{k_1!\dots k_D!} \prod\limits_{i=1}^D {m_i}^{k_i}, \label{eq:M}\\
	W_{\bmk} &= \prod\limits_{i=1}^D {w_i}^{k_i}. \label{eq:W}
\end{align}
\end{lemma}
\begin{proof}
	We proceed by induction.
	
	Assume $n = 1$.
	Let $\bmk_i$ be the vector with its $i$-th component equal to 1 and 0's elsewhere.
	The set $C_D^1$ consists of all vectors of this form, i.e. 
\begin{equation*}
	C_D^1 = \{ \bmk_i \}_{i=1,\dots,D}
\end{equation*}
	It is also true by direct calculation that
\begin{equation*}
	\left( W_{\bmk_i}, M_{\bmk_i} \right) = (w_i, m_i).
\end{equation*}
Therefore, $\{ (W_{\bmk}, M_{\bmk}) \}_{\bmk \in C_D^1}$ is a complete set of component-multiplicity pairs for $W$.

	Assume that $\{(W_{\bmk}, M_{\bmk})\}_{\bmk \in C_D^n}$ as given in~\cref{eq:M,eq:W} is a complete set of component-multiplicity pairs for the $n$--copy distribution $W^{\otimes n}$.
	By construction, the distribution $W^{\otimes (n+1)} = W^{\otimes n} \otimes W$ is multiplicative, so it admits the complete set of component multiplicity pairs
\begin{equation}
	\{(W_{\bmk} w_i, M_{\bmk} m_i)\},\ \bmk \in C_D^n \text{ and } i=1,\dots,D.
\end{equation}
	
	Consider the component sum of the distribution $W^{\otimes (n+1)}$,
\begin{align*}
	&\sum_{\bmk \in C_D^n}\sum_{i=1}^D M_{\bmk} m_i W_{\bmk} w_i = \sum_{\bmk \in C_D^n} M_{\bmk}W_{\bmk} \sum_{i=1}^D m_i w_i =\\
	&\sum_{\bmk \in C_D^n} \frac{n!}{k_1!\dots k_D!} \prod\limits_{i=1}^D {m_i}^{k_i}{w_i}^{k_i} \sum_{i=1}^D m_i w_i =\\
	&\left( \sum_{i=1}^D m_i w_i \right)^n \left( \sum_{i=1}^D m_i w_i \right) = \left( \sum_{i=1}^D m_i w_i \right)^{n+1} =\\
	&\sum_{\bmq \in C_D^{n+1}} M_{\bmq}W_{\bmq},
\end{align*}
where in the last expression, vectors $\bmq = (q_1, \dots, q_D)$ have non-negative integer components that sum to $(n+1)$ and 
\begin{align*}
	M_{\bmq} &= \frac{(n+1)!}{q_1!\dots q_D!} \prod\limits_{i=1}^D {m_i}^{q_i},\\
	W_{\bmq} &= \prod\limits_{i=1}^D {w_i}^{q_i}.
\end{align*}
We have used the multinomial theorem to proceed between lines 2-3 and lines 3-4 of the derivation.

We have achieved a regrouping of the distribution components.
Every component $W_{\bmq}$ is of the form $W_{\bmk} w_i$ with $q_i = k_i + 1$ and $q_j = k_j$ for $j\neq i$ and 
\begin{align*}
	\sum_{\bmq \in C_D^{n+1}}  \hspace{-6pt} M_{\bmq} =  \hspace{-10pt} \sum_{\bmq \in C_D^{n+1}} \frac{(n+1)!}{q_1!\dots q_D!} \prod\limits_{i=1}^D {m_i}^{q_i} = 
	\left( \sum_{i=1}^D m_i \right)^{n+1} \hspace{-10pt} = d^{n+1},
\end{align*}
which is the dimension of $W^{\otimes (n+1)}$.

Therefore, $\{ (W_{\bmq}, M_{\bmq}) \}_{\bmq \in C_D^{n+1}}$ is a complete set of component-multiplicity pairs for $W^{\otimes n}$, completing the proof.
\end{proof}

%Index vector $\bmk$ has $D-1$ independent components and in the proof of our main theorem in~\cref{sec:stab} we have $D=4$, so we simplify the notation by writing the component and multiplicity vectors of the $n$--copy state distributions as $m_{ijk}, w(\rho_{\rm{S}})_{ijk}, w(\sigma)_{ijk}$ and $w(\rho_{\rm{S}}|\sigma)_{ijk}$, where $i,j,k$ are the 3 independent index components.

\newpage
\subsection{Free energy dependent magic distillation bounds}\label{free-energy-bound-proof}
\label{app:main_proof}
Here we prove~\cref{thm:free-energy}, which bounds magic distillation rates in terms of free energies.

\begin{theorem*}
	Consider a magic distillation protocol on qutrits that transforms $n$ copies of an $\epsilon$--noisy Strange state into $m$ copies of an $\epsilon'$--noisy Strange state, with depolarising errors $\epsilon' \leq \epsilon \leq 3/7$. 
	
	Let $T =(k\beta)^{-1}$ be any temperature for the physical system and let $H= \sum_{k \in \mathbb{Z}_3} E_k |E_k\>\<E_k|$ be the Hamiltonian of each qutrit subsystem in its eigen-decomposition.
Assume that for $n,m \rightarrow \infty$ the protocol's channel generates negligible correlations on the equilibrium state $\tau$, and so $\tau^{\otimes n} \longrightarrow \tau^{\prime \otimes m}$ for $n,m \gg 1$. We write the state $\tau'$ as $\tau' = e^{-\beta H'}/\Z'$ for some Hermitian $H'$.

Given this, there are local Clifford changes of basis $C_1,C_2$, such that $\rho_1 := C_1 \rho_S(\epsilon) C_1^\dagger$ and $\rho_2 := C_2 \rho_S(\epsilon') C_2^\dagger$ for which the protocol gives
\begin{equation}
\rho_1^{\otimes n} \longrightarrow \rho_2^{ \otimes m}.
\end{equation}
Moreover, the magic distillation rate $R = m/n$ for the protocol is bounded by the expression
\begin{equation}\label{eq:rate_bounds_proof}
	R \leq \dfrac{\ln{\big( 1-\frac{4}{3}\epsilon \big)} + \beta (\phi - F)}{\ln{\big( 1-\frac{4}{3}\epsilon' \big)} + \beta (\phi' - F')},
\end{equation}
where $F$ is the free energy of $\tau$,  and 
\begin{equation}
	\phi = -\beta^{-1} \log \zeta
\end{equation}
with $\zeta$ given by the equations
\begin{align}
	\zeta &= \sum_{k\in \mathbb{Z}_3} \alpha_k e^{-\beta E_k}, \nonumber\\
	\alpha_k &= \sum_{r \in \mathbb{Z}_3} \braket{E_k}{-r}\braket{r}{E_k}.
\end{align}
The primed variables are defined similarly for the output system.
\end{theorem*}


\begin{proof}	
For the sake of clarity, we write $\rho_n \coloneqq \rho_S(\epsilon)^{\otimes n}$, $\rho'_m \coloneqq \rho_S(\epsilon')^{\otimes m}$, $\tau_n \coloneqq \tau^{\otimes m}$ and $\tau'_m \coloneqq \tau'^{\otimes m}$. 

To establish the distillation bound we consider the distillation protocol that gives $\E( \rho_n) = \rho'_m$ for the magic states. 
We then consider that the protocol transforms the reference equilibrium state as $\E( \tau_n) = \tau'_m$.
Since $\tau$ and $\tau'$ are assumed to be full rank stabilizer states, they have a strictly positive Wigner distribution, while, in contrast, the input and output magic states generally have quasi-probability Wigner distributions. 
For any such protocol we therefore have that
\begin{equation}
	( W_{\rho_n}(\bmx), W_{\tau_n}(\bmx) ) \succ ( W_{\rho'_m}(\bmx), W_{\tau'_m}(\bmx) ),
\end{equation}
or, equivalently, in terms of the relevant Lorenz curves,
\begin{equation}
	L_{\rho_n |\tau_n}(x) \ge L_{\rho'_m |\tau'_m}(x) \mbox{ for all } x.
\end{equation}

We define the rescaled Wigner distribution $W_{\rho | \tau}(\x) \coloneqq W_\rho(\x)/W_\tau(\x)$, which is always well-defined since $\tau$ is full-rank. 
Due to the multiplicative property of the Wigner distribution, the rescaled distribution is also multiplicative in the sense that
\begin{equation}
	W_{\rho \otimes \rho' | \tau \otimes \tau'} (\x_1 \oplus \x_2) = W_{\rho | \tau}(\x_1)W_{\rho' | \tau'}(\x_2),
\end{equation}
for any states $\rho, \rho'$ and any full-rank stabilizer states $\tau, \tau'$.
Therefore we have that
\begin{align}
	W_{\rho_n |\tau_n} (\x) &= \prod_{i=1}^n W_{\rho|\tau}(\x_i)\\
	W_{\rho_n } (\x) &= \prod_{i=1}^n W_\rho(\x_i)
\end{align}
where $\x = \oplus_{i=1}^n \x_i \in \mathbb{Z}_3^n$ is the phase space point for the full system in terms of those of the individual subsystems.

The points defining the Lorenz curve $L_{\rho_n |\tau_n}(x)$ are obtained from sorting the components of $W_{\rho_n |\tau_n}(\x)$ in non-increasing order and then computing the partial sums of $W_{\rho_n |\tau_n}(\pi(\x))$ where $\pi$ is the permutation that realises the sorting. 
However, similarly to the unital fragment analysis, we aim to use the constraint that is obtained by considering the line segment connecting the origin to the first elbow of both Lorenz curves.

The Wigner distribution of a single noisy Strange state consists of 1 negative component $W_{\rho}(0,0) = -v(\epsilon)$ and 8 positive components $W_{\rho}(\bmx) = u(\epsilon)$ for $\bmx \neq \bmo$.
The Wigner distribution of the full-rank, stabilizer equilibrium state $\tau$ is $W_{\tau}(\bmx) > 0$ for all $\bmx \in \mathbb{Z}_3^2$.

Assume that the smallest component of the distribution $W_\tau(\x)$ is at $x=\x_\star$.
We then make use of the freedom to apply Clifford unitaries in order to perform the  Clifford operation
\begin{equation}
	\tau \longrightarrow X^{-i}Z^{-j}\ \tau\ Z^{j}X^{i},
\end{equation}
which permutes the smallest component to the origin of the phase space, $W_{\tau}(i,j) \longrightarrow W_{\tau}(0,0)$.
This operation would inevitably affect the magic distillation protocol, but since we consider the state distillation modulo Clifford operations, we are allowed to perform the reverse Clifford operation keeping the magic distillation process as is. \nick{statement here is inaccurate - incorporate $C_1, C_2$ in the proof.}

The rescaled distribution components are in general given by
\begin{equation}
	W_{\rho_n|\tau_n} = \left(\frac{-v}{W_{\tau}(0,0)}\right)^{i_{\bmo}} \prod_{\bmx \neq \bmo} \left(\frac{u}{W_{\tau}(\bmx)}\right)^{i_{\bmx}},
\end{equation}
where the (integer) indices obey the following conditions:
\begin{align}
&0 \leq i_{\bmx} \leq n \mbox{ for all } \bmx \in \mathbb{Z}_3^2, \nonumber \\
&\sum_{\bmx \in \mathbb{Z}_3^2} i_{\bmx} = n.
\end{align}
It is now easy to find the largest rescaled component.
Firstly, note that $n$ is even, so we require that $i_{\bmo} \in \{0,2,\dots,n\}$ for the component to be positive.
Then, we have that $v \geq u$ because $\epsilon \leq 3/7$, and we have already ensured that $W_{\tau}(0,0) \leq W_{\tau}(\bmx)$ for all $\bmx \in \mathbb{Z}_3^2$.
Therefore, the largest rescaled component occurs when $i_{\bmo} = n$ and $i_{\bmx} = 0$ for $\bmx \neq \bmo$ and is equal to $(v/W_{\tau}(0,0))^n$.
Accordingly, the coordinates of the first Lorenz curve point after the origin are given by
\begin{equation}
	(x_0, L_0) = ((W_{\tau}(0,0))^n, v^n)
\end{equation}

Given a Hamiltonian decomposition,
\begin{equation}
	H = \sum_{k \in \mathbb{Z}_3} E_k\ketbra{E_k}{E_k},
\end{equation}
we can re-express the coordinate location, by writing
\begin{align}
	W_{\tau}(0,0) &= \frac{1}{3\Z}\tr\left[ A_{0,0}e^{-\beta H} \right] \nonumber\\
	&= \frac{e^{\beta F}}{3} \tr\left[ \sum_{r \in \mathbb{Z}_3} \ketbra{-r}{r}\sum_{k \in \mathbb{Z}_3} e^{-\beta E_k}\ketbra{E_k}{E_k} \right] \\
	&= \frac{e^{\beta F}}{3} \sum_{r \in \mathbb{Z}_3} \alpha_k e^{-\beta E_k}
	= \frac{e^{\beta F}}{3} \zeta \\
	&= \frac{e^{\beta (F - \phi)}}{3},
\end{align}
where $\alpha_k, \zeta$ and $\phi$ are as defined in the statement of the theorem.
Finally, the coordinates can be expressed as
\begin{equation}
	(x_0, L_0) = \left( \frac{e^{n\beta (F - \phi)}}{3^n}, v(\epsilon)^n \right).
\end{equation}
In the same manner, we can derive the first point coordinates of the output Lorenz curve as
\begin{equation}
	(x'_0, L'_0) = \left( \frac{e^{n\beta (F' - \phi')}}{3^n}, v(\epsilon')^n \right).
\end{equation}

If the largest rescaled component of a state is distinct with no multiplicities, then these coordinates correspond to the first elbow of the corresponding Lorenz curve, whereas if it appears multiple times, then the coordinates derived correspond to a point on the interior of the line segment connecting the origin to the first elbow.
In both cases, the distillation bound remains the same, as is clear by its derivation in~\cref{app:elb_constraints}, and it is given by $L_0 / x_0 \geq L'_0 / x'_0$, which can be directly rearranged into the stated bound of the theorem.
\end{proof}
	
\subsection{Deriving distillation bounds from the last elbow}
\label{sec:last_elb}

Using a similar analysis, we can also derive upper bounds from comparison of the last point coordinates of the Lorenz curve, which now corresponds to the smallest rescaled component.

To this end, assume that the largest component of the equilibrium state distribution is $W_{\tau}(i,j)$.
We now perform the Clifford operation
\begin{equation}
	\tau \longrightarrow X^{1-i}Z^{-j}\ \tau\ Z^{j}X^{i-1},
\end{equation}
which permutes the largest component such that $W_{\tau}(i,j) \longrightarrow W_{\tau}(1,0)$.
Since we have that $u \leq v$ because $\epsilon \leq 3/7$, and we have already ensured that $W_{\tau}(1,0) \geq W_{\tau}(\bmx)$ for all $\bmx \in \mathbb{Z}_3^2$.
Therefore, the largest rescaled component occurs when $i_{(1,0)} = n$ and $i_{\bmx} = 0$ for $\bmx \neq \bmo$ and is equal to $(v/W_{\tau}(1,0))^n$.
Accordingly, the coordinates of the first Lorenz curve point after the origin are given by $(x_E, L_E) = ((W_{\tau}(1,0))^n, u(\epsilon)^n)$ for the input state and $(x'_E, L'_E) = ((W_{\tau'}(1,0))^m, u(\epsilon')^m)$ for the output state.

We now redefine the quantities $\alpha_k$, so that again $\phi = -\beta^{-1} \log \zeta$, where $\zeta= \sum_{k\in \mathbb{Z}_3} \alpha_k e^{-\beta E_k}$ and $\alpha_k$ now given by
\begin{align}
	\alpha_k = \sum_{r \in \mathbb{Z}_3} \braket{E_k}{1-r}\braket{1+r}{E_k}.
\end{align}
This allows us to rewrite $W_{\tau}(1,0) = {e^{\beta (F - \phi)}}/{3}$ and by using the last elbow constraint 
\begin{equation*}
	\frac{L_E - 1}{1-x_E} \geq \frac{L'_E - 1}{1-x_E'},
\end{equation*}
derived in~\cref{app:elb_constraints}, we can get a new bound expression.

We note that it is also possible to perform a different Clifford transformation and get a different location $(W_{\tau}(\bmx))^n$, with $\bmx \neq \bmo$, since components $(W_{\rho}(\bmx))^n = u$ are all the same as long as $\bmx \neq \bmo$.
The effect of this is to alter the expressions for the quantities $\alpha_k$, but not the resulting bound expression.

\nick{This section on the last elbow bound is a little weak?}

\subsection{OLD VERSION OF MAIN THEOREM}

\ddd{[Do we need to keep any of the details in this section? If not then delete it entirely.]}
\nick{This proof offers: 1. the exact derivation of our main contour plot, justifying the quantities $\epsilon_\star$ and $\beta_\star$ 2. a method for lifting the ``Clifford change in basis'' assumption - Can be trimmed down to the case where $H=H'$. What do you think?}
\begin{theorem}
	Consider a magic distillation protocol transforming noisy Strange state
	\begin{equation*}
		\rho_S(\epsilon)^{\otimes n} \longrightarrow \E(\rho_S(\epsilon)^{\otimes n})=\rho_S(\epsilon')^{\otimes m} 
	\end{equation*}
	with $n, m \gg 1$.

Let each qutrit have a Hamiltonian $H$ with stabilizer eigenstates and energies $E_0, E_1, E_2$, and define $E_{\rm max} = \max\{E_0, E_1, E_2\}$ and $E_s$ the eigenvalue of the eigenstate that \nick{overlaps} the negative component of $\rho_S(\epsilon)$ in the Wigner representation. Let $T =(k\beta)^{-1}$ be any characteristic temperature for the physical system in the state $\tau^{\otimes n}= (e^{-\beta H}/\Z)^{\otimes n}$, with free energy $F$. Assume that for $n,m \gg 1$ the channel $\E$ generates negligible correlations on $\tau^{\otimes n}$, and so $\E(\tau^{\otimes n}) = \sigma^{\otimes m}$ for some state $\sigma$.

Define $\beta_\star = (k T_\star)^{-1}$ through the relation
\begin{equation}
	E_{\rm max} - E_s \eqqcolon kT_\star \ln 2,
\end{equation}
and define a threshold noise,
\begin{equation}
	\epsilon_{\star}(\beta) \coloneqq 
	\begin{cases}
		3 - \dfrac{9}{4-2^{\beta/\beta_\star - 1}}, &\text{ for } \beta \leq \beta_\star \\
		0, &\text{ for } \beta > \beta_\star.
	\end{cases}
\end{equation}
Then, the distillation rate $R = m/n$ of the magic protocol is bounded as:
\begin{equation}\label{eq:rate_bounds_proof}
	R \leq
	\begin{cases}
		\dfrac{\ln{\big( 1-\frac{4}{3}\epsilon \big)} + \beta (E_s - F)}{\ln{\big( 1-\frac{4}{3}\epsilon' \big)} + \beta (E'_s - F')},\ &\epsilon \leq \epsilon_\star, \epsilon' \leq \epsilon'_\star, \vspace{10pt}\\
		\dfrac{\ln{\big( 1-\frac{4}{3}\epsilon \big)} + \beta (E_s - F)}{\ln{\big( \frac{1}{2}-\frac{1}{6}\epsilon' \big)} + \beta (E'_{\rm{max}} - F')},\ &\epsilon \leq \epsilon_\star, \epsilon' > \epsilon'_\star, \vspace{10pt}\\
		\dfrac{\ln{\big( \frac{1}{2}-\frac{1}{6}\epsilon \big)} + \beta (E_{\rm{max}} - F)}{\ln{\big( 1-\frac{4}{3}\epsilon' \big)} + \beta (E'_s - F')},\ &\epsilon > \epsilon_\star, \epsilon' \leq \epsilon'_\star, \vspace{10pt}\\
		\dfrac{\ln{\big( \frac{1}{2}-\frac{1}{6}\epsilon \big)} + \beta (E_{\rm{max}} - F)}{\ln{\big( \frac{1}{2}-\frac{1}{6}\epsilon' \big)} + \beta (E'_{\rm{max}} - F')},\ &\epsilon > \epsilon_\star, \epsilon' > \epsilon'_\star,
	\end{cases}
\end{equation}
where $F'$ is the free energy of the state $\sigma = e^{-\beta H'}/\Z'$ and other primed quantities are defined in the same way as the corresponding unprimed quantities. 
\end{theorem}
We note, firstly, that the specific numerical factors in $\epsilon_\star$ are a result of our choice of magic state. 
\begin{proof}
To establish the distillation bounds we consider the distillation protocol that gives
\begin{equation}
	\E( \rho_S(\epsilon)^{\otimes n}) = \rho_S(\epsilon')^{\otimes m},
\end{equation}
for the magic state. We then consider how the protocol transforms the reference equilibrium state as
\begin{equation}
	\E( \tau^{\otimes n}) = \sigma^{\otimes m}.
\end{equation}
Since $\tau$ and $\sigma$ are assumed to be full rank stabilizer states they have a strictly positive Wigner distribution, while, in contrast, the input and output magic states will generally have quasi-probability distributions for their Wigner functions. For any such protocol we therefore have that
\begin{equation}
	(W_{\rho_n} (x), W_{\tau_n}(x) ) \succ  (W_{\rho'_m} (x), W_{\sigma_m}(x) ),
\end{equation}
where, for the sake of clarity, we write $\rho_n \coloneqq \rho_S(\epsilon)^{\otimes n}$, $\rho'_m \coloneqq \rho_S(\epsilon')^{\otimes m}$ and $\tau_n \coloneqq \tau^{\otimes m}$, $\sigma_m \coloneqq \sigma^{\otimes m}$. We then have that
\begin{equation}
	L_{\rho_n |\tau_n}(x) \ge L_{\rho'_m |\sigma_m}(x) \mbox{ for all } x.
\end{equation}
We must therefore compute the Lorenz curve data for $\rho_n$ relative to $\tau_n$, and compare with the Lorenz curve of the output state $\rho_m'$ relative to $\sigma_m$.

We define $W_{\rho | \tau}(\x) \coloneqq W_\rho(\x)/W_\tau(\x)$, which is always well-defined since $\tau$ is full-rank. Now we show in~\cref{app:wigner} that
\begin{equation}
W_{\rho_1\otimes \rho_2 | \tau_1 \otimes \tau_2} (\x_1 \oplus \x_2) = W_{\rho_1 | \tau_1}(\x_1)W_{\rho_2 | \tau_2}(\x_2),
\end{equation}
and also,
\begin{equation}
W_{\rho_1\otimes \rho_2} (\x_1 \oplus \x_2) = W_{\rho_1}(\x_1)W_{\rho_2}(\x_2),
\end{equation}
for any states $\rho_1, \rho_2$ and any full-rank stabilizer states $\tau_1, \tau_2$. \ddd{[Notation clashing here. Frame result outside as its own lemma -- we gotta check everything.]} \nick{Will frame in~\cref{app:wigner} and fix notation in the next commit}
Therefore we have that
\begin{align}
W_{\rho_n |\tau_n} (\x) &= \prod_{i=1}^n W_{\rho|\tau}(\x_i)\\
W_{\rho_n } (\x) &= \prod_{i=1}^n W_\rho(\x_i)
\end{align}
where $\x = \oplus_{i=1}^n \x_i$ is the phase space point for the full system in terms of those of the individual subsystems.

The points defining the Lorenz curve $L_{\rho_n |\tau_n}(x)$ are obtained from first sorting the components of $W_{\rho_n |\tau_n}(\x)$ in non-increasing order and then computing the partial sums of $W_{\rho_n |\tau_n}(\pi(\x))$ where $\pi$ is the permutation that realises the sorting. Therefore, we first look at the values of $W_{\rho|\tau}(\x_i)$ for the single-copy case.

The equilibrium state at inverse temperature $\beta$ on a single qutrit is given by $\tau = e^{-\beta H} / \Z$. Moreover we have that $\tau$ is a full-rank stabilizer state, where $\beta \geq 0$ and $H = E_0 \ketbra{\varphi_0} + E_1 \ketbra{\varphi_1} + E_2 \ketbra{\varphi_2}$ is an eigendecomposition of $H$.
The state $\tau$ can now be written as 
\begin{equation}
	\tau = \frac{e^{-\beta E_0}}{\Z} \ketbra{\varphi_0} + \frac{e^{-\beta E_1}}{\Z} \ketbra{\varphi_1} + \frac{e^{-\beta E_2}}{\Z} \ketbra{\varphi_2},
\end{equation}
where the eigenstates $\{\ket{\varphi_k}\<\varphi_k|\}$ are pure, orthonormal stabiliser states, which can be represented in terms of generalized Paulis. To make our analysis simpler, we perform a change of basis that does not affect the Wigner negativity of the problem. We let $C$ be the unitary transforming each $|\varphi_k\>\<\varphi_k|$ to $|k\>\<k|$. Since the Clifford group is the normalizer of the Heisenberg-Weyl group, $C$ is a Clifford unitary. Therefore, $C$ maps $\tau$ to another stabilizer state that is diagonal in the computational basis, and we can assume without loss of generality that $\tau$ is diagonal in $\{|0\>,|1\>, |2\>\}$. However, this choice means that the location of the negative Wigner component $-v(\epsilon)$ of the Strange state will be permuted on the discrete phase space. We denote by $E_s$ the eigenvalue of $H$ where the associated eigenvector has Wigner distribution overlapping the negative component of the magic state $C\rho_S(\epsilon)C^\dagger$.  This is unique, since the eigenstates form an orthonormal basis.

The Wigner distribution of state $\tau$ is then given by
\begin{align}
	\W[\bmx]{\tau} &= \sum\limits_{k=0}^2 \frac{e^{-\beta E_k}}{\Z}W_{\ketbra{k}}(x, p) \nonumber\\
	&= \sum\limits_{k=0}^2 \frac{e^{-\beta E_k}}{\Z} \delta_{x, k} = \frac{e^{-\beta E_x}}{3\Z},
\end{align}
where $x$ labels one of the three vertical lines in the phase space.
The rescaled Wigner distribution $W_{\rho|\tau}(\x)$ is then easily computed. It has $9$ components, but several of these come with multiplicities. In total, there are four distinct values on the phase space, as illustrated in~\cref{fig:pd_split}.
\begin{figure}[h]
    \centering
    \includegraphics[scale=0.45]{figs/pd_split_thermal.pdf}
    \caption{\textbf{Qutrit phase space regions for $W_{\rho | \tau}(\x)$.}
    Here, the negative component of the magic state overlaps the Wigner distribution of $|0\>$. The rescaled distribution attains a single value in each of the four regions, proportional to the value depicted in the region, see~\cref{eq:bmw_rescaled}.
    }
    \label{fig:pd_split}
\end{figure}

We now denote by $\w(\rho), \w(\rho|\tau)$ the unique values occurring in $W_\rho(\x), W_{\rho|\tau}(\x)$ respectively and $\bmm$ the vector of associated multiplicities of each value in $W_\rho(\x)$. The component values and multiplicities of the relevant distributions in the four distinct regions are given by
\begin{align}
	\w(\rho) &\coloneqq (-v, u, u, u), \\
		\bmm &\coloneqq (1,2,3,3), \\
	\bmw(\tau) &\coloneqq \frac{1}{3\Z} \left( e^{-\beta E_0}, e^{-\beta E_0}, e^{-\beta E_1}, e^{-\beta E_2} \right), \\
	\bmw(\rho_{\rm{S}} | \tau) &\coloneqq 3\Z \left( -v e^{\beta E_0}, u e^{\beta E_0}, u e^{\beta E_1}, u e^{\beta E_2} \right). \label{eq:bmw_rescaled}
\end{align}

Using this notation, the values and multiplicities of the $n$--copy distribution $\bmw(\rho_n |\tau_n)$ are computed in~\cref{lem:ncopycomponents} in~\cref{app:cmpairs}. The values are given by 
\begin{align}\label{eq:ncopy_w_rescaled}
	[\w(\rho_n | \tau_n)]_{ijk} &= (3\Z)^{n} (-v)^{n-\alpha} u^{\alpha} e^{\beta (n-\alpha)E_s} e^{\beta ( i E_0 + j E_1 + k E_2 )},
\end{align}
where the indices $i,j,k$ are non-negative integers that obey the constraint $\alpha \coloneqq i+j+k \leq n$.
The multiplicity of this above value is $m_{ijk}$ with
\begin{equation}
	m_{ijk} = \frac{n!}{i!j!k!(n-\alpha)!} 2^i 3^j 3^k.
\end{equation}
The associated components of $\w(\rho_n)$ are given by
\begin{align}
	[\w(\rho_n)]_{ijk} &= (-v)^{n-\alpha} u^{\alpha}, \label{eq:ncopy_wrho}\\
	[\w(\tau)]_{ijk} &= (3\Z)^{-n} e^{-\beta (n-\alpha)E_s} e^{-\beta ( i E_0 + j E_1 + k E_2 )}. \label{eq:ncopy_wsigma}
\end{align}

In order to construct the $n$--copy Lorenz curve $L_{\rho_n|\tau_n}(x)$ we need to order the components of the distribution, $w(\rho_{\rm{S}} | \tau)_{ijk}$ in decreasing order, and identify the sequence of indices that give us $W_{\rho_n}(\pi(\x))$.

Generally this is complex, but in order to obtain distillation bounds it is sufficient to determine the location of the first elbow $(x_0, L_0)$ of $L_{\rho_n|\tau_n}(\x)$. To do so, we compute the largest component 
\begin{equation}
	w_{\rm max} \coloneqq \max_{i,j,k} [\w(\rho_n | \tau_n)]_{ijk},
\end{equation}
and determine the indices at which this occurs.
Putting in the values we obtain
\begin{align}
	&(3\Z)^{-n}w_{\rm max} = \nonumber\\
	&\max\limits_{i,j,k}\Big\{ (-v)^{n-\alpha} u^{\alpha} e^{\beta (n-\alpha)E_s} e^{\beta ( i E_0 + j E_1 + k E_2 )} \Big\}, \label{eq:max_slope}
\end{align}
where $0 \leq i,j,k \leq n$ and $\alpha \coloneqq i+j+k \leq n$.
Now for $0 \leq \epsilon \leq 3/7$, we have $v \geq u$. Since we assume that $n$ is even, we need the sum $\alpha = i+j+k$ to be even too, so that the expression is positive. 

Given an even value for $\alpha$, the term $v^{n-\alpha} u^{\alpha} e^{-\beta (n-\alpha)E_s}$ is fixed, so the expression is maximised by setting the coefficient of the highest energy $E_{\rm{max}}$ equal to $\alpha$.
Hence, we have
\begin{align}
	&w_{\rm max} = \nonumber\\
	&(3\Z)^{n} v^n e^{n\beta E_s}\max\limits_{\substack{\alpha = 0,2, \\ \dots,n-2,n}}{\Big\{ \left( \frac{u}{v} e^{\beta (E_{\rm{max}} - E_s)} \right)^{\alpha} \Big\}}.
\end{align}
If the expression $\frac{u(\epsilon)}{v(\epsilon)} e^{\beta (E_{\rm{max}} - E_s)}$ is less than $1$ then the maximum occurs at $\alpha=0$, otherwise it occurs at $\alpha = n$. For a fixed state $\tau$, this transition is determined by the value of the depolarising noise parameter $\epsilon$ of the noisy magic state. The transition occurs at $\epsilon = \epsilon_\star$ where
\begin{equation}\label{eq:noise_transition}
	\frac{u(\epsilon_\star)}{v(\epsilon_\star)} e^{\beta (E_{\rm{max}} - E_s)} = \frac{3-\epsilon_\star}{6-8\epsilon_\star} e^{\beta (E_{\rm{max}} - E_s)} = 1.
\end{equation}
If $E_{\rm{max}} = E_s$, namely if the state negativity lies in the same phase space region as the highest energy, this threshold is constant in temperature and given by $\epsilon_{\star} = 3/7$. However, the condition that $\epsilon_\star \ge 0$ also implies a constraint on the effective temperature of the stabilizer state. Specifically, there is a threshold temperature value $\beta_\star$ given by
\begin{equation}
	\beta_{\star} \coloneqq \frac{1}{E_{\rm{max}} - E_s} \ln2,
\end{equation}
such that for the regime $0 \leq \beta \leq \beta_\star$ a transition noise level $\epsilon_\star$ exists, and for $\beta > \beta_\star$ no such transition exists, so we choose $\epsilon_\star = 0$. 
Therefore, the transition value for the noise is given by
\begin{equation}
	\epsilon_{\star}(\beta) \coloneqq 
	\begin{cases}
		3 - \dfrac{9}{4-2^{\beta/\beta_\star - 1}}, &\text{ for } \beta \leq \beta_\star \\
		0, &\text{ for } \beta > \beta_\star.
	\end{cases}
\end{equation}
The quantity $w(\rho_{\rm{S}} | \sigma)_{\rm{max}}$ is now given by
\begin{equation*}
w_{\rm max} =
	\begin{cases}
		(3\Z)^{n} v^n e^{n\beta E_s}, &\mbox{if }\epsilon \leq \epsilon_{\star},\ \hspace{3pt}\rm{(C1)}\\
		(3\Z)^{n} u^n e^{n\beta E_{\rm{max}}}, &\mbox{if }\epsilon > \epsilon_{\star}.\ \hspace{5pt}\rm{(C2)} 
	\end{cases}
\end{equation*}
Case $\rm{(C1)}$ corresponds to $(i,j,k) = (0,0,0)$, so the multiplicity is $m_{000} = 1$, while
Case $\rm{(C2)}$ corresponds to
\begin{equation}
	(i,j,k) = 
	\begin{cases}
	(0,n,0), &\text{if } E_{\rm{max}} = E_1, \\
	(0,0,n), &\text{if } E_{\rm{max}} = E_2,
	\end{cases}
\end{equation}
so the multiplicity in both cases is $3^n$.

Using that $F = -\beta^{-1} \log \Z$, the first elbow coordinates in the two cases are now given by
\begin{equation}\label{eq:first_elb_coords}
	(x_0, L_0) =
	\begin{cases}
		\left(\frac{1}{3^n} e^{-n\beta (E_s - F)}, v^n \right), &\epsilon \leq \epsilon_\star \vspace{10pt}\\
		\left( e^{-n\beta (E_{\rm{max}}-F)}, (3u)^n \right). &\epsilon > \epsilon_\star
	\end{cases}
\end{equation}

\ddd{[Hmmm this needs an additional assumption on the output state to re-run the same analysis. Very annoying.]}\nick{What additional assumption beyond that the output state can be written as a Gibbs state for some $H'$?}
Similarly, considering the output magic state with respect to state $\sigma'$, the image of equilibrium state $\sigma$ under the magic protocol, we get output Lorenz curve coordinates,
\begin{equation}\label{eq:transformed_first_elb_coords}
	(x'_0, L'_0) =
	\begin{cases}
		\left(\frac{1}{3^{n'}} e^{-n\beta (E'_s - F')}, v(\epsilon')^{n'} \right), &\epsilon' \leq \epsilon'_\star \vspace{10pt}\\
		\left( e^{-n'\beta (E'_{\rm{max}}\hspace{-2.5pt}-F')}, (3u(\epsilon'))^{n'} \right), &\epsilon' > \epsilon'_\star
	\end{cases}
\end{equation}
There are four combinations of coordinates, depending on the noise parameters $\epsilon, \epsilon'$ for the input and output states.
In each of these combinations, we simply use the first elbow constraint, as described in~\cref{app:elb_constraints}, which leads to the bounds in the statement of the theorem.
\end{proof}






























\end{document}